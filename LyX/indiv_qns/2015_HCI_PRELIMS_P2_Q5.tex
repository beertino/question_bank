\item \textbf{{[}HCI/PRELIM/9597/2015/P2/Q5{]} }

HC university has various campuses around the city and Wilson Parking
is responsible for all the university car parks.

At each car park:

A car arriving triggers a sensor (S1) and a fixed fee (F) is paid
into a machine. This allows a barrier (B1) to be lifted and the car
to enter the car park. When a car leaves the car park it passes over
another sensor (S2) and another barrier (B2) is lifted. 

Each car park has a maximum number of spaces for cars (M) and when
this maximum is reached a \textquotedblleft FULL\textquotedblright{}
sign is illuminated at the entrance and the barrier (B1) will not
rise. The car park is closed at least once a day for cleaning purposes. 
\begin{enumerate}
\item Write an algorithm which will control the barriers and which will
keep a total (T) of the fees paid. \hfill{}{[}8{]}

There are 100 car parks, each of which is identified by a number between
1 and 100. 

At the end of each month, the total fees paid for that month (T) is
collected from each of the car parks as an integer value. 

All data are stored in an array Parks(). 
\begin{itemize}
\item For car park x, Parks(x, 1) to Parks(x, 12) contains the totals for
the twelve months of the year.
\item Parks(x, 13) contain the annual total fees collected for each car
park.
\end{itemize}
\item Using Parks(x,y) to identify individual values in the array, write
an algorithm which can be used to produce the annual totals once the
twelve monthly totals have been input to the array, and the grand
annual total for all the 100 carparks. \hfill{}{[}5{]}
\item When implementing the algorithm into program code, they should be
written to display clarity. Describe \textbf{three} features of the
final program code for this implementation that would achieve this
goal. \hfill{}{[}3{]}
\end{enumerate}