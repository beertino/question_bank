\item \textbf{{[}JPJC/PRELIM/9569/2020/P2/Q2{]} }

In linear queue data structure, elements are inserted until the queue
becomes full. However, after the queue becomes full, new elements
cannot be inserted until all the existing elements are removed from
the queue. Although there are empty spaces in the queue, they remain
unused. This is a disadvantage of a linear queue. 

After inserting all the elements into a linear queue: 
\noindent \begin{center}
\begin{tabular}{|c|c|c|c|c|}
\hline 
\textquotedblleft John\textquotedblright{} & \textquotedblleft Amy\textquotedblright{}  & \textquotedblleft Chetan\textquotedblright{}  & \textquotedblleft Xin Xin\textquotedblright{}  & \textquotedblleft Evan\textquotedblright{}\tabularnewline
\hline 
\multicolumn{1}{c}{$\overset{\uparrow}{\text{Front}}$} & \multicolumn{1}{c}{} & \multicolumn{1}{c}{} & \multicolumn{1}{c}{} & \multicolumn{1}{c}{$\overset{\uparrow}{\text{Rear}}$}\tabularnewline
\end{tabular}
\par\end{center}

Linear queue is still considered full after elements have been dequeued: 
\noindent \begin{center}
\begin{tabular}{|c|c|c|c|c|}
\hline 
\textquotedblleft John\textquotedblright{} & \textquotedblleft Amy\textquotedblright{} & \textquotedblleft Chetan\textquotedblright{} & \textquotedblleft Xin Xin\textquotedblright{} & \textquotedblleft Evan\textquotedblright{}\tabularnewline
\hline 
\multicolumn{1}{c}{} & \multicolumn{1}{c}{} & \multicolumn{1}{c}{} & \multicolumn{1}{c}{$\overset{\uparrow}{\text{Front}}$} & \multicolumn{1}{c}{$\overset{\uparrow}{\text{Rear}}$}\tabularnewline
\end{tabular}
\par\end{center}

To overcome this disadvantage, a circular queue data structure may
be implemented. The next element added to the queue will be stored
at index 0 
\noindent \begin{center}
<INSERT\_IMAGE\_HERE> 
\par\end{center}

\begin{center}
\begin{tabular}{|l|l|l|}
\hline 
\multicolumn{3}{|c|}{\texttt{Queue}}\tabularnewline
\hline 
\multicolumn{3}{|c|}{Attributes}\tabularnewline
\hline 
\texttt{\hspace{0.01\columnwidth}}Identifier & \texttt{\hspace{0.01\columnwidth}}Data Type & \texttt{\hspace{0.05\columnwidth}}Description\tabularnewline
\hline 
\texttt{Items} & \texttt{ARRAY{[}0:4{]} OF STRING} & Stores the elements of queue\tabularnewline
\hline 
\texttt{Front} & \texttt{INTEGER} & Index of the first item added to the queue.\tabularnewline
\hline 
\texttt{Rear} & \texttt{INTEGER} & Index of the last item added to the queue.\tabularnewline
\hline 
\multicolumn{3}{|c|}{\texttt{Methods}}\tabularnewline
\hline 
Identifier & \multicolumn{2}{l|}{Description}\tabularnewline
\hline 
\texttt{Constructor()} & \multicolumn{2}{l|}{Instantiates a \texttt{Queue} object.}\tabularnewline
\hline 
\texttt{IsEmpty()} & \multicolumn{2}{l|}{Returns \texttt{TRUE} if the queue is empty and \texttt{FALSE} otherwise.}\tabularnewline
\hline 
\texttt{IsFull()} & \multicolumn{2}{l|}{Returns \texttt{TRUE} if the queue is full and \texttt{FALSE} otherwise.}\tabularnewline
\hline 
\texttt{Enqueue(STRING)} & \multicolumn{2}{l|}{Inserts a new item to the queue. Displays a suitable message if the
queue is full.}\tabularnewline
\hline 
\texttt{Dequeue():STRING} & \multicolumn{2}{l|}{Returns the item removed from the queue or \textquotedblleft \texttt{NONE}\textquotedblright{}
if the queue is empty.}\tabularnewline
\hline 
\texttt{Display()} & \multicolumn{2}{l|}{Outputs items from the front to the rear of the queue.}\tabularnewline
\hline 
\end{tabular}
\par\end{center}

\begin{center}
\begin{tabular}{|l|l||l|}
\hline 
\multicolumn{3}{|c|}{\texttt{CircularQueue}}\tabularnewline
\hline 
\multicolumn{3}{|c|}{\texttt{Methods}}\tabularnewline
\hline 
Identifier & \multicolumn{2}{l|}{Description}\tabularnewline
\hline 
\texttt{Constructor()} & \multicolumn{2}{l|}{Instantiates a \texttt{CircularQueue} object.}\tabularnewline
\hline 
\texttt{IsFull()} & \multicolumn{2}{l|}{Returns \texttt{TRUE} if the queue is full and \texttt{FALSE} otherwise.Overrides
the method in parent class.}\tabularnewline
\hline 
\texttt{Enqueue(STRING)} & \multicolumn{2}{l|}{Inserts a new item to the queue. Displays a suitable message if the
queue is full.Overrides the method in parent class.}\tabularnewline
\hline 
\texttt{Dequeue():STRING} & \multicolumn{2}{l|}{Returns the item removed from the queue or \textquotedblleft \texttt{NONE}\textquotedblright{}
if the queue is empty.Overrides the method in parent class.}\tabularnewline
\hline 
\texttt{Display()} & \multicolumn{2}{l|}{Outputs items from the front to the rear of the queue.Overrides the
method in parent class.}\tabularnewline
\hline 
\end{tabular}
\par\end{center}

\subsection*{Task 2.1}

Implement the classes \texttt{Queue} and \texttt{CircularQueue} with
object-oriented programming. The first item added to an empty queue
is stored at index 0. The attributes of each object is reinitialised
when the queue becomes empty.\hfill{} {[}20{]} 

\subsection*{Task 2.2 }

There are two printers in the General Office. One of the printers
implements a linear queue while the other implements a circular queue. 

Write the code to instantiate a \texttt{Queue} object and a \texttt{CircularQueue}
object. Test your code, on both queues, using the following steps: 
\begin{enumerate}
\item[i.]  Enqueue five users in the order given in the diagram. 
\item[ii.]  Dequeue twice.
\item[iii.]  Enqueue \textquotedblleft Mohan\textquotedblright .
\item[iv.]  Display the queue. 
\end{enumerate}
Download your program code for Task 2 as 

\texttt{TASK2\_<your class>\_<your name>.ipynb} \hfill{}{[}6{]}