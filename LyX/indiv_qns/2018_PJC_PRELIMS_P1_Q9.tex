\item \textbf{{[}PJC/PRELIM/9597/2018/P1/Q9{]} }

PJ Mall plans to create a database to store data on its shops. It
rents out shops to tenants who run their business.
\begin{itemize}
\item Each \emph{tenant} is to provide information on its \emph{company
name}, \emph{director of company}, \emph{company address}, \emph{contact
number}, and \emph{retail type}. 
\item There is a \emph{start} and \emph{end} date for every rental.
\item Each \emph{shop} rented by the tenant consists of one or more unit
spaces. 
\item Each unit space is located at a particular \emph{level} and has a\emph{
unit number}. 
\item There are 3 categories of unit space. Each \emph{category} has its
own \emph{size} and \emph{rental rate}. 
\noindent \begin{center}
\begin{tabular}{|c|c|c|}
\hline 
Category & Size (square feet) & Rental rate (\$ per square feet)\tabularnewline
\hline 
A & Less than 200 & 40\tabularnewline
\hline 
B & 200 -- 2000 & 30\tabularnewline
\hline 
C & More than 2000 & 20\tabularnewline
\hline 
\end{tabular} 
\par\end{center}

\end{itemize}
Here are some tenants who run their business in PJ mall: 
\noindent \begin{center}
\begin{tabular}{|c|c|c|c|}
\hline 
Company Name & Level & Unit Number & Retail Type\tabularnewline
\hline 
Bata & 2 & 03 -- 04 & Footwear\tabularnewline
\hline 
Challenger & 2 & 06 -- 08 & Technology\tabularnewline
\hline 
Coldwear & 3 & 08 & Fashion\tabularnewline
\hline 
Esprit & 3 & 09 -- 10 & Fashion\tabularnewline
\hline 
Giant & 1 & 01 -- 12 & Supermarket\tabularnewline
\hline 
Hi Tea & 1 & 14 & Food \& Beverage\tabularnewline
\hline 
PappaRich & 2 & 11 -- 13 & Food \& Beverage\tabularnewline
\hline 
$\dots$ & $\dots$ & $\dots$ & $\dots$\tabularnewline
\hline 
\end{tabular}
\par\end{center}
\begin{enumerate}
\item A solution is to create a relational database which requires a number
of tables to store data for this application. 
\begin{enumerate}
\item Draw the E-R diagram showing the tables and the relationships between
them. \hfill{} {[}5{]}
\item A table description can be described as 

\texttt{TableName(Attribute1, Attribute2, Attribute3, \dots \dots ) }

The primary key is indicated by underlining one or more attributes.
Write table descriptions for the tables in part \textbf{(i)}. \hfill{}
{[}6{]}
\end{enumerate}
\item Describe \textbf{two} advantages of using a relational database for
storing data on its shops rather than a customised software. \hfill{}
{[}4{]}
\end{enumerate}