\item \textbf{{[}HCI/PRELIM/9569/2021/P1/Q6{]}}

Check digit is one technique of data validation. 
\begin{enumerate}
\item[(i)]  Give \textbf{two} other techniques of data validation.\hfill{}
{[}2{]}
\item[(ii)]  With \textbf{one} example of data verification, explain the difference
between data verification and data validation. \hfill{}{[}3{]}
\end{enumerate}
A student ID consists of 5 digits and a check digit. 
\begin{enumerate}
\item[(iii)]  One way to calculate the check digit is to use the unit\textquoteright s
digit of the sum of all 5 digits. For example, suppose the 5 digits
are 50879. Since 5 + 0 + 8 + 7 + 9 = 29, the check digit is 9, and
the student ID is 508799. 

Explain, with \textbf{two} examples, why this method is inadequate.
\hfill{}{[}2{]}
\end{enumerate}
The check digit is calculated from the 5 digits using the modulus
11 system. It can be digits \texttt{0 - 9 }or character \texttt{'X'}. 
\begin{enumerate}
\item[(iv)]  Showing your working, determine the check digit for 30526. \hfill{}{[}3{]}
\item[(v)]  Write an algorithm to check if a student ID is valid. \hfill{}{[}5{]}
\item[(vi)]  A function is designed to read a student ID and determine if it
is valid. State the data types of its input parameter and justify.
\hfill{}{[}2{]}
\end{enumerate}