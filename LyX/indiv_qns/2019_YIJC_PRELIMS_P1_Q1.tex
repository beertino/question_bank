\item \textbf{{[}YIJC/PRELIM/9597/2019/P1/Q1{]} }

According to some researches done on children below the age of 16,
it was found that the height of a boy, measured in centimetres (cm),
should lie within the normal range with: 
\noindent \begin{center}
minimum height = 5.3 \texttimes{} Age + 71 
\par\end{center}

\noindent \begin{center}
maximum height = 6.2 \texttimes{} Age + 87 
\par\end{center}

The text file HEIGHTDATA.TXT contains 20 entries of the data in the
following format: 
\noindent \begin{center}
<Name>, <Age>, <Height> 
\par\end{center}

\subsection*{Task 1.1: }

Write program code to:
\begin{itemize}
\item read the entire contents of \texttt{HEIGHTDATA.TXT}. 
\item determine if the boy\textquoteright s height lies within the normal
range. 
\item display the contents using the format given below 
\end{itemize}
\noindent %
\begin{minipage}[t]{0.5\columnwidth}%
Example run of the program: 

\textbf{Input File: }

\texttt{Ali,6,105 }

\texttt{Bob,10,145 }

\texttt{Charlie,15,185} %
\end{minipage}%
\fbox{\begin{minipage}[t]{0.5\columnwidth}%
The output generated from the input file would be: 

\texttt{\textbf{Name ~~~Age Height Within normal range }}

\texttt{Ali ~~~~6 ~~105 ~~~Yes}

\texttt{Bob ~~~~10 ~145 ~~~Yes }

\texttt{Charlie 15 ~185 ~~~No}%
\end{minipage}}

\subsection*{Evidence 1.1: }

Your program code. \hfill{}{[}6{]}

\subsection*{Evidence 1.2: }

Screenshot of the output.\hfill{} {[}2{]}

During data entry, some of the data may have been wrongly entered
with transposition errors. In the case of Charlie\textquoteright s,
his height should have been 158 cm but was wrongly transposed and
entered as 185 cm. 

\subsection*{Task 1.2: }

Write program code to: 
\begin{itemize}
\item determine the correct height for those entries outside the normal
range. 
\item display the amended contents using the format given below. 
\end{itemize}
In cases where there are more than one possible or no possible height,
print \textquoteleft Re-enter data\textquoteright .

\noindent %
\begin{minipage}[t]{0.5\columnwidth}%
Example run of the program:

\textbf{Input File:}

\texttt{Ali,6,105}

\texttt{Bob,10,145 }

\texttt{Charlie,15,185 }

\texttt{Ethan,7,131 }

\texttt{Rick,13,415 }%
\end{minipage}%
\fbox{\begin{minipage}[t]{0.5\columnwidth}%
The output generated from the input file would be:

\texttt{\textbf{Name ~~~Age Height Corrected Height}}

\texttt{Charlie 15 ~185 ~~~158}

\texttt{Ethan ~~7 ~~131 ~~~113 }

\texttt{Rick ~~~13 ~415 ~~~Re-enter data}%
\end{minipage}}

Example run of the program: 

\subsection*{Evidence 1.3: }

Your program code. \hfill{}{[}6{]}

\subsection*{Evidence 1.4: }

Screenshot of the output. \hfill{}{[}1{]}