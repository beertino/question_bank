\item \textbf{{[}HCI/PRELIM/9569/2020/P2/Q3{]} }

A bakery currently keeps records on paper of all its products it sells
in the shop. It decided to trial a database to manage its products.
It is expected that the database should be normalized. The following
information of the bakery is recorded: 
\begin{itemize}
\item \texttt{ProductCode} -- Unique code of the item 
\item \texttt{Name} -- Name of the item 
\item \texttt{Type} -- The type of product -- \texttt{'Cake'}, \texttt{'Loaf'},
\texttt{'Bun'} 
\item \texttt{Location} -- The location at which the product is made --
\texttt{'North'}, \texttt{'South'}, \texttt{'East'}, \texttt{'West'}
Kitchen 
\item \texttt{Price} -- The selling price of the product 
\end{itemize}
For cakes, the following additional information is recorded: 
\begin{itemize}
\item \texttt{ServingSize} -- the estimated number of servings per cake 
\item \texttt{Shape} -- Shape of the cake -- \texttt{'Square'}, \texttt{'Circle'},
\texttt{'Roll'} 
\end{itemize}
For loaves, the following additional information is recorded: 
\begin{itemize}
\item \texttt{Weight} -- weight of loaf in gram 
\end{itemize}
For buns, the following additional information is recorded:
\begin{itemize}
\item \texttt{PiecesPerPackage} -- number of pieces per package
\end{itemize}
The information is to be stored in four different tables: 

\texttt{Product }

\texttt{Cake }

\texttt{Loaf }

\texttt{Bun}

\subsection*{Task 3.1 }

Create a SQL file named \texttt{Task3\_1\_<your\_name>\_<centre number>\_<index
number>.sql} to show the SQL code to create the database \texttt{bakery.db}
with the four tables. The table, \texttt{Product}, must use the \texttt{ProductCode}
as its \textbf{primary key}. The other tables must refer to the \texttt{ProductCode}
as a \textbf{foreign key}.

Save your SQL code as 

\texttt{Task3\_1\_<your\_name>\_<centre number>\_<index number>.sql}
\hfill{}{[}4{]}

\subsection*{Task 3.2 }

The files \texttt{CAKES.TXT}, \texttt{LOAVES.TXT} and \texttt{BUNS.TXT}
contain information about the bakery\textquoteright s cakes, loaves
and buns respectively for insertion into the bakery database. Each
row in the three files is a comma-separated list of information about
a single product. 

For \texttt{CAKES.TXT}, the information about each cake is given in
the following order: 
\noindent \begin{center}
\texttt{ProductCode, Name, Location, Price, ServingSize, Shape }
\par\end{center}

For \texttt{LOAVES.TXT}, the information about each loaf is given
in the following order: 
\noindent \begin{center}
\texttt{ProductCode, Name, Location, Price, Weight }
\par\end{center}

For \texttt{BUNS.TXT}, the information about each bun is given in
the following order:
\noindent \begin{center}
\texttt{ProductCode, Name, Location, Price, PiecesPerPackage}
\par\end{center}

Write a Python program to insert all information from the three files
into the bakery database, bakery.db. Run the program. 

Save your program code as \texttt{Task3\_2\_<your\_name>\_<centre
number>\_<index number>.py} \hfill{}{[}6{]}

\subsection*{Task 3.3 }

Write SQL code to show the \texttt{ProductCode}, \texttt{Name}, \texttt{Location},
\texttt{Price} and the \texttt{ServingSize} of each cake with \texttt{'Circle'}
Shape. Run the query. 

Save this code as 

\texttt{Task3\_3\_<your\_name>\_<centre number>\_<index number>.sql
}\hfill{}{[}4{]}

\subsection*{Task 3.4 }

The bakery wants to filter the products by \texttt{Location} and display
the results in a web browser. 

Write a Python program and the necessary files to create a web application
that: 
\begin{itemize}
\item receives a \texttt{Location} string from a HTML form, then 
\item creates and returns a HTML document that enables the web browser to
display a table tabulating the details of the product based on the
\texttt{Location} in ascending order of \texttt{Price}. The table
will display the following columns: \texttt{Name}, \texttt{Type},
\texttt{Price}.
\end{itemize}
Save your Python program as 

\texttt{Task3\_4\_<your\_name>\_<centre number>\_<index number>.py }

with any additional files/sub-folders as needed in a folder named
Task3\_4\_<your\_name>\_<centre number>\_<index number>

Run the web application. Save the output of the program when \texttt{'North'}
is entered as the \texttt{Location} as 

\texttt{Task3\_4\_<your\_name>\_<centre number>\_<index number>.html}
\hfill{}{[}10{]}