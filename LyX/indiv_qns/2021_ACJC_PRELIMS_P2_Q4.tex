\item \textbf{{[}ACJC/PRELIM/9569/2021/P2/Q4{]} }

\noindent A company specialising in bento boxes wishes to trial a
relational database management system to manage its data. It is expected
that the database should be normalised to third normal form (3NF).

\noindent The company owns four kiosks. For each of the kiosks, the
following information is to be recorded in the table \texttt{Kiosk}: 
\begin{itemize}
\item \texttt{KioskID} -- the unique integer assigned to the kiosk 
\item \texttt{Location} -- the area where the kiosk is located 
\item \texttt{Rating} -- the average rating of the kiosk between 0.0 and
5.0 inclusive
\end{itemize}
\noindent The company offers eight different types of bento boxes.
Some of them may contain egg, nut, seafood or a combination of them.
For each of the bento boxes, the following information is to be recorded
in the table \texttt{BentoBox}:
\begin{itemize}
\item \texttt{BentoName} -- the unique name of the bento box 
\item \texttt{ProductionCost} -- the cost incurred in producing the bento
box in dollars and cents 
\item \texttt{ContainEgg} -- an integer 0 for not containing egg and 1
for containing egg 
\item \texttt{ContainNut} -- an integer 0 for not containing nut and 1
for containing nut 
\item \texttt{ContainSeafood} -- an integer 0 for not containing seafood
and 1 for containing seafood 
\end{itemize}
\noindent Each of the four kiosks sells all eight bento boxes at different
mark-up prices. Another table \texttt{KioskBento} is needed to record
the following information:
\begin{itemize}
\item \texttt{KioskID} -- the unique integer assigned to the kiosk 
\item \texttt{BentoName} -- the unique name of the bento box 
\item \texttt{SellPrice} -- the price at which the bento box is sold at
the kiosk in dollars and cents
\end{itemize}

\subsection*{Task 4.1}

\noindent Create an SQL file called \texttt{TASK4\_1\_<your name>\_<centre
number>\_<index number>.sql} to show the SQL code to create the database
\texttt{bento\_company.db} with the three tables. \hfill{}{[}5{]}

\noindent Save your SQL code as \texttt{TASK4\_1\_<your name>\_<centre
number>\_<index number>.sql}

\subsection*{Task 4.2}

\noindent The files \texttt{KIOSK.txt} and \texttt{BENTOBOX.txt} contain
information about the company\textquoteright s kiosks and bento boxes
respectively for insertion into the database. Each row in the two
files is a comma-separated list of information. 

\noindent For \texttt{KIOSK.txt}, information about each kiosk is
given in the following order: \texttt{KioskID}, \texttt{location},
\texttt{rating}

\noindent For \texttt{BENTOBOX.txt}, information about each bento
box is given in the following order: \texttt{BentoName}, \texttt{ProductionCost},
\texttt{ContainEgg}, \texttt{ContainNut}, \texttt{ContainSeafood}

\noindent The mark-up price for each kiosk has been set as follows:
\begin{itemize}
\item \texttt{KioskID = 1} sells each bento box at a price that is \$2.60
higher than the production cost
\item \texttt{KioskID = 2} sells each bento box at a price that is \$2.90
higher than the production cost 
\item \texttt{KioskID = 3} sells each bento box at a price that is \$2.40
higher than the production cost 
\item \texttt{KioskID = 4} sells each bento box at a price that is \$3.10
higher than the production cost
\end{itemize}
\noindent Write program code to insert all the required information
into the database \texttt{bento\_company.db}. \hfill{}{[}6{]}

\noindent Save your program code as \texttt{TASK4\_2\_<your name>\_<centre
number>\_<index number>.py}

\noindent Run your program.

\subsection*{Task 4.3}

\noindent The company wishes to create a form to display the bento
boxes sold at a particular kiosk and their prices in a web browser.
The form should allow customers to indicate egg, nut and seafood allergies,
if any, and filter out the bento boxes that they cannot consume.

\noindent Write a Python program and the necessary files to create
a web application that:
\begin{itemize}
\item receives input from a HTML form that includes:
\begin{itemize}
\item a text box to enter the \texttt{location} of the kiosk 
\item three checkboxes to indicate egg, nut and seafood allergies, if any
\end{itemize}
\item returns a HTML document to display only the bento boxes that the customers
can consume based on the allergies indicated, if any, and their prices
for the given \texttt{location}
\end{itemize}
\noindent Input validation is not required. \hfill{}{[}10{]}

\noindent Save your Python program as \texttt{TASK4\_3\_<your name>\_<centre
number>\_<index number>.py}

\noindent with any additional files / sub-folders as needed in a folder
named \texttt{TASK4\_3\_<your name>\_<centre number>\_<index number>}

\noindent Run and test the web application using the following input:
\begin{itemize}
\item \texttt{'Woodlands'} entered as the \texttt{location} 
\item checkboxes indicating egg and seafood allergies ticked \hfill{}{[}2{]}
\end{itemize}
\noindent Save the output of the program as \texttt{TASK4\_3\_<your
name>\_<centre number>\_<index number>.html }