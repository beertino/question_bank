\item \textbf{{[}YIJC/PRELIM/9569/2021/P1/Q3{]} }

A stack is a last-in-first-out (LIFO) abstract data type (ADT) in
which all the elements are inserted and removed from one end. 

It is common to either use a linked list or an array to implement
a stack
\begin{itemize}
\item In the linked list implementation, a root pointer points to the top
of a stack and a data structure. The data structure contains the value
of the data and a pointer pointing to the next node in the stack. 
\item In the array implementation, a fixed size array is used to store the
elements. 
\end{itemize}
The basic stack operations of \texttt{push()}, \texttt{pop()} and
\texttt{peek()} are provided in both implementations. 
\begin{itemize}
\item \texttt{push()} is used to insert an element into the stack. 
\item \texttt{pop()} removes an element from the stack and returns the value
of the element. 
\item \texttt{peek()} returns the value of the element at the top of the
stack without removing it. 
\end{itemize}
\begin{enumerate}
\item Describe what an abstract data type is and how it benefits the user.\hfill{}
{[}3{]} 
\item State one advantage and one disadvantage of implementing the stack
ADT using a linked list. \hfill{}{[}2{]} 
\item State one advantage and one disadvantage of implementing the stack
ADT using an array.\hfill{} {[}2{]} 
\item Describe how a \texttt{push()} operation is done in a stack ADT which
is implemented using a linked list. \hfill{}{[}3{]} 
\item Describe how the number of elements within a stack can be counted
using only the basic stack operations provided. \hfill{}{[}3{]} 
\end{enumerate}
The following program code uses a Python list as an array with the
built-in functions\texttt{ <list>.insert()} and \texttt{<list>.pop()}. 

\noindent\begin{minipage}[t]{1\columnwidth}%
\texttt{01 def add(seq): }

\texttt{02 \qquad{}stk = {[}{]} }

\texttt{03 \qquad{}def InsertOne(item): }

\texttt{04 \qquad{}\qquad{}if stk=={[}{]} or item < stk{[}0{]}:}

\texttt{05 \qquad{}\qquad{}\qquad{}stk.insert(0,item) }

\texttt{06 \qquad{}else: }

\texttt{07 \qquad{}\qquad{}temp = stk.pop(0) }

\texttt{08 \qquad{}\qquad{}InsertOne(item) }

\texttt{09 \qquad{}\qquad{}stk.insert(0,temp) }

\texttt{10 \qquad{}for ele in seq: }

\texttt{11 \qquad{}\qquad{}InsertOne(ele) }

\texttt{12 \qquad{}return stk }%
\end{minipage}
\begin{enumerate}
\item[(f)] {}
\begin{enumerate}
\item The above is an example of implementing a stack using an array. State
and explain how the program code adds all the elements in the sequence
\texttt{seq} into the stack. \hfill{}{[}2{]} 
\item Modify the above program code such that it uses only the basic stack
ADT operations provided. \hfill{}{[}4{]}
\end{enumerate}
\end{enumerate}