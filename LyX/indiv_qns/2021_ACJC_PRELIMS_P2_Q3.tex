\item \textbf{{[}ACJC/PRELIM/9569/2021/P2/Q3{]} }

\noindent A community centre is required to keep COVID-19 vaccination
records of its members in a NoSQL database. The CoviDie vaccine, which
requires two doses to be taken at least 21 days apart, has been secured
for all members of this particular community centre. 

\noindent As everyone needs to be vaccinated before the end of 2021,
we shall only consider the year 2021, which is a non-leap year. The
table below shows the number of days available in the twelve months
of 2021.
\noindent \begin{center}
\begin{tabular}{|c|c|c|c|c|c|c|}
\hline 
Month  & 01  & 02  & 03  & 04  & 05  & 06\tabularnewline
\hline 
Days  & 31  & 28  & 31  & 30  & 31 & 30\tabularnewline
\hline 
\end{tabular}
\par\end{center}

\noindent \begin{center}
\begin{tabular}{|c|c|c|c|c|c|c|}
\hline 
Month  & 07  & 08  & 09 & 10  & 11  & 12\tabularnewline
\hline 
Days & 31 & 31 & 30 & 31 & 30 & 31\tabularnewline
\hline 
\end{tabular}
\par\end{center}

\subsection*{Task 3.1}

\noindent Write a function \texttt{second\_dose\_date(date)} that:
\begin{itemize}
\item Takes a string value \texttt{date} in the format \texttt{YYYYMMDD},
where \texttt{YYYY} represents the year, \texttt{MM} represents the
month and \texttt{DD} represents the day 
\item determines the date that is 21 days after the input date 
\item returns the result date in the format \texttt{YYYYMMDD}
\end{itemize}
\noindent Assume that the result date does not go beyond \texttt{20211231}.
\hfill{}{[}3{]}

\noindent Test the function using the following three calls:
\begin{itemize}
\item \texttt{second\_dose\_date('20210105') }
\item \texttt{second\_dose\_date('20210212') }
\item \texttt{second\_dose\_date('20210919')} \hfill{}{[}3{]}
\end{itemize}
\noindent Save your program code as

\noindent \texttt{TASK3\_1\_<your name>\_<centre number>\_<index number>.py}

\subsection*{Task 3.2}

\noindent The list of members under the management committee of the
community centre is stored in the text file \texttt{VACCINATION.txt}.
Some members have not taken the vaccination at all, some others have
only taken the first dose, while the rest have taken both doses.

\noindent Each line of the text file is of the following format: \texttt{\_id,name,date\_first\_dose,date\_second\_dose,remarks}
\begin{itemize}
\item \texttt{\_id} is a unique integer ID assigned to the member 
\item \texttt{name} is the name of the member 
\item \texttt{date\_first\_dose} and \texttt{date\_second\_dose}, if any,
represent the date of the first dose and the date of the second dose
respectively in the format \texttt{YYYYMMDD} 
\item \texttt{remarks}, if any, shows the pre-existing condition of the
member
\end{itemize}
\noindent Write program code to insert the data from \texttt{VACCINATION.txt}
into a NoSQL database \texttt{community\_centre} under the collection
\texttt{management\_committee}. The program should clear the collection
\texttt{management\_committee} if it exists inside the database. \hfill{}{[}7{]}

\noindent Save your program code as T\texttt{ASK3\_2\_<your name>\_<centre
number>\_<index number>.py}

\subsection*{Task 3.3}

\noindent The community centre needs a program to check the vaccination
status of its members. The program should also allow for the downloading
of vaccination certificates for members who are fully vaccinated,
i.e. they have taken the two doses.

\noindent Write program code to: 
\begin{itemize}
\item prompt the user to input a member ID, and keep prompting until the
user keys in numeric character(s) 
\item if the member ID is available in the NoSQL database, perform either
one of the following: 
\begin{itemize}
\item if the member is fully vaccinated, write the vaccination certificate
to an output text file and update the record in the NoSQL database
by including a field and an appropriate value to indicate that the
certificate has been downloaded 
\item if the member has only taken the first dose, output a message to show
the date from which the member can take the second dose 
\item if the member has not taken the vaccination at all, output a message
to tell that the member should take the first dose as soon as possible 
\end{itemize}
\item otherwise, if the member ID is not available in the NoSQL database,
display an appropriate message and terminate the program
\end{itemize}
\noindent The format of the vaccination certificate is as follows.

\noindent \texttt{VACCINATION CERTIFICATE}

\noindent \texttt{Name: <name> }

\noindent \texttt{Vaccine type: CoviDie }

\noindent \texttt{Date of first dose: <date\_first\_dose> }

\noindent \texttt{Date of second dose: <date\_second\_dose>} \hfill{}{[}8{]}

\noindent Save your program code as \texttt{TASK3\_3\_<your name>\_<centre
number>\_<index number>.py}

\noindent Test your program for the member with \texttt{member\_id
= 24}. \hfill{}{[}2{]}

\noindent The output text file should be saved as \texttt{TASK3\_3\_<your
name>\_<centre number>\_<index number>.txt}