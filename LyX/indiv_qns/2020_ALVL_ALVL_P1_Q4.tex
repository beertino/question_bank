\item \textbf{{[}ALVL/9569/2020/P1/Q4{]} }

A computer on LAN A wants to send data to a computer on a remote LAN
B. 

The internet is used to provide a data path between the two LANs. 
\begin{enumerate}
\item {}
\begin{enumerate}
\item State \textbf{two} ways that a particular device can be identified
on a LAN. \hfill{}{[}2{]}
\item State \textbf{two} reasons why LANs need communication protocols.
\hfill{}{[}2{]}
\end{enumerate}
\end{enumerate}
IP is the protocol used to transfer packets of data between hosts
and routers on the intemet. 

The intemet is a packet-switched network. 
\begin{enumerate}
\item[\textbf{(b)}] {}
\begin{enumerate}
\item Explain the term \textbf{packet-switching}. \hfill{}{[}3{]}
\item Describe a disadvantage of packet-switching and how the problem can
be handled. \hfill{}{[}3{]}
\item State how a packet-switched network can cope with a broken cable on
part of the network. \hfill{}{[}2{]}
\end{enumerate}
\end{enumerate}
When using a web browser. most users do not know the IP address of
the server hosting the desired web page. So users enter the domain
name instead. which the browser sends to a local domain name server
(DNS). 
\begin{enumerate}
\item[\textbf{(c)}] Describe the actions that would be carried out by the local DNS on
receiving this request.\hfill{} {[}4{]}
\item[\textbf{(d)}] State the security feature that can be used as a precautionary measure
when sensitive data is sent across a network in each of the following
situations: 
\begin{enumerate}
\item No one other than the intended recipient of the message should be
able to read it.\hfill{} {[}1{]}
\item The intended recipient must be confident that the message is from
the identified sender. \hfill{}{[}1{]}
\end{enumerate}
\end{enumerate}