\item \textbf{{[}ALVL/9597/2016/P1/Q1{]} }

High-level programming languages usually have libraries of commonly
used routines. These include random number generators. 

\subsubsection*{Task 1.1}

Write program code to generate 1000 random integers in the range 1
to 20. 

The program will:
\begin{itemize}
\item Maintain a count of how many times each number is produced 
\item Print out a frequency table. 

Example output:

\begin{tabular}{ll}
Integer & Frequency\tabularnewline
1: & 54\tabularnewline
2: & 48\tabularnewline
3: & 52\tabularnewline
4: & 43\tabularnewline
5: & 48\tabularnewline
6: & 51\tabularnewline
7: & 41\tabularnewline
8: & 48\tabularnewline
9: & 53\tabularnewline
10: & 51\tabularnewline
11: & 45\tabularnewline
12: & 54\tabularnewline
13: & 44\tabularnewline
14: & 40\tabularnewline
15: & 54\tabularnewline
16: & 59\tabularnewline
17: & 47\tabularnewline
18: & 49\tabularnewline
19: & 66\tabularnewline
20: & 53\tabularnewline
\end{tabular}
\end{itemize}

\subsubsection*{Evidence 1}

Your program code. 

Screenshot of the program output.\hfill{}{[}7{]}

Random numbers generated by computers are usually referred to as pseudo-random
numbers because they are generated by executing program code. 

One criterion of a good pseudo-random number generator is that every
number in the range has an equal chance of being generated. This means
if 200 numbers are generated in the range 1 to 10, the expected frequency
value of every number in this range is 20. 

The program code is to be amended to check how well the given pseudo-random
number generator meets this requirement.

\subsubsection*{Task 1.2}

Amend your program code to:
\begin{itemize}
\item Calculate the expected frequency
\item Output this expected frequency
\item Output the difference between the actual and the expected frequency
for each number in the range as a third column of the frequency table.
\end{itemize}

\subsubsection*{Evidence 2}

Your program code.

Screenshot of the program output.\hfill{}{[}5{]}