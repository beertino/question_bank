\item \textbf{{[}HCI/PRELIM/9597/2015/P2/Q3{]} }

Your company is starting the development of relational database for
a patient billing system to be marketed to private medical practices
in Singapore. The system is to be called PATMAN (short for PATient
billing MANager), and is to run as a remote client-server system and
a local area network. 

An initial analysis phase of the project has resulted in the following
description of the relevant data for PATMAN. 
\begin{itemize}
\item A practice has a number of patients and doctors. 
\item Doctors are identified by name.
\item Each patient has a number used to identify the patient called the
OHIP number, a name and an age.
\item Each patient is either a male or female and has a next of kin identified
by name.
\item Each medical procedure paid by the government of Singapore is identified
by a procedure code and has a description and a charging category.
\item Each charging category has a dollar value. 
\item Each patient has a number of billing records, with each billing record
recording the medical procedure, the date on which the procedure was
performed, the examining doctor and some additional comments on the
part of the examining doctor.
\item Billing records are either outstanding or paid in full.
\end{itemize}
\begin{enumerate}
\item Draw an ER diagram that represents the PATMAN data. \hfill{}{[}6{]}
\item Using shorthand notation, what are tables in this relational database?\hfill{}
{[}12{]}
\item Explain why relational database is better than a flat file design?\hfill{}
{[}2{]}
\end{enumerate}