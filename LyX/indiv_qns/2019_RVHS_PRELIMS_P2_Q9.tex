\item \textbf{{[}RVHS/PRELIM/9597/2019/P2/Q9{]} }

The school library has in store different kinds of items which students
can loan out. Information about items are stored and recorded, such
as \texttt{title}, \texttt{description}, and \texttt{damaged}, which
is a Boolean value indicating if the item has any form of physical
damage.

For books in the library, additional information such as the \texttt{author},
\texttt{publisher}, \texttt{ISBN} are recorded.

Digital media resources are also available for students\textquoteright{}
reference. Information such as \texttt{storage media}, \texttt{file
size} and\texttt{ playback time} are stored. 

You are engaged by the school to design an Object-Oriented Programming
solution to manage these items.
\begin{enumerate}
\item Draw a class diagram, with base class \texttt{Item}, showing: 
\begin{itemize}
\item appropriate sub-classes 
\item inheritance
\item the properties required 
\item appropriate methods, including \textbf{constructor} methods, and at
least \textbf{one} pair of \textquoteleft \textbf{get}\textquoteright{}
and \textquoteleft \textbf{set}\textquoteright{} methods for one of
the properties. \hfill{}{[}5{]}
\end{itemize}
\item Using the above example, explain the meaning of the term \textbf{Polymorphism}.
\hfill{}{[}2{]}
\item Using the above example, explain the meaning of the term \textbf{Encapsulation}.\hfill{}
{[}2{]}
\item The school also provides a series of past year examination papers
for the students\textquoteright{} loaning and reference. State how
this would affect the \textbf{class}, \textbf{properties} and \textbf{methods}
in the current example. \hfill{}{[}2{]}
\end{enumerate}