\item \textbf{{[}NYJC/PRELIM/9597/2019/P2/Q6{]} }

A linked list ADT with the following incomplete specification is given
as follows:
\begin{center}
\begin{tabular}{|l|}
\hline 
\texttt{LList}\tabularnewline
\hline 
\texttt{head : Node}\tabularnewline
\hline 
\texttt{constructor()}\tabularnewline
\texttt{addNode(s : Node)}\tabularnewline
\texttt{findmiddle(l : llist)-> INTEGER}\tabularnewline
\hline 
\end{tabular}%
\begin{tabular}{|l|}
\hline 
Node\tabularnewline
\hline 
\texttt{data : INTEGER}\tabularnewline
\texttt{nextPtr : Node}\tabularnewline
\hline 
\texttt{constructor()}\tabularnewline
\texttt{setData(s : INTEGER)}\tabularnewline
\texttt{setnextPtr(x : Node)}\tabularnewline
\texttt{getData(): INTEGER }\tabularnewline
\hline 
\end{tabular}
\par\end{center}
\begin{enumerate}
\item Explain the main difference between an array and a linked list data
structure.\hfill{} {[}2{]}
\item Using pseudo code, write an algorithm to implement \texttt{findmiddle}
that will return the data in the middle of the linked list in one
pass. \hfill{}{[}7{]}
\item State two applications of a linked list. \hfill{}{[}2{]}
\item State two other common methods (including parameters) that should
be included in the \texttt{LList} specification.\hfill{} {[}2{]}
\end{enumerate}