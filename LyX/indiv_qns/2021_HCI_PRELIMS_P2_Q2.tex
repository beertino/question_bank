\item \textbf{{[}HCI/PRELIM/9569/2021/P2/Q2{]}}

FlexiMSG provides messaging services. Information of the messages
are logged into a file. The log records contain the phone numbers
or IP address of the sender, the date which the service is being accessed,
the status indicating whether the message has been sent and the type
of application used. There are two different formats used: 

\texttt{<IP address> <DD/MMM/YYYY> <Status> <App> }

or 

\texttt{<Phone number> <DD/MMM/YYYY> <Status>}

Below is the log records in the file, \texttt{LOG.txt}: 

\noindent %
\noindent\fbox{\begin{minipage}[t]{1\columnwidth - 2\fboxsep - 2\fboxrule}%
\texttt{54.36.149.41 22/Jan/2021 200 WA }

\texttt{188.226.164.216 22/Jan/2021 0 FB }

\texttt{92783423 22/Jan/2021 200 }

\texttt{188.226.164.216 23/Jan/2021 0 FB }

\texttt{88188293 23/Jan/2021 0 }%
\end{minipage}}

\subsubsection*{Task 2.1 }

Write the SQL code to create database \texttt{ServiceLog.db} with
the single table, \texttt{Log}. 

The table will have the following fields of the given SQLite types: 
\begin{itemize}
\item \texttt{LogID} - primary key, an auto-incremented integer 
\item \texttt{Sender} - the client internet address or phone number, text 
\item \texttt{AccessDate} - the access date, text 
\item \texttt{Status} - the status, integer 
\item \texttt{AppType} - application type, text 
\end{itemize}
Save your SQL code as 

\texttt{Task2\_1\_<your name>\_<center number>\_<index number>.sql}
\hfill{}{[}2{]}

\subsubsection*{Task 2.2 }

FlexiMSG wants to use Python programming language and object-oriented
programming to update the information in the log file into the database.

Create the class \texttt{ServiceRecord} that will store the following: 
\begin{itemize}
\item \texttt{Sender} - stored as a string 
\item \texttt{AccessDate} - stored as a string 
\item \texttt{Status} - stored as integer, \texttt{0} or \texttt{200} 
\item \texttt{AppTy}pe - stored as string value \texttt{'WA'} or \texttt{'FB'} 
\end{itemize}
The class has the following methods:
\begin{itemize}
\item \texttt{isSuccess()}- returns a Boolean value to indicate whether
the message has been sent. 
\begin{itemize}
\item returns \texttt{True} if the \texttt{Status} is 200, otherwise returns
\texttt{False} 
\end{itemize}
\item \texttt{getAppType()-} returns a string value to indicate the type
of messaging application. 
\begin{itemize}
\item returns the value of \texttt{AppType} 
\end{itemize}
\end{itemize}
Write program code to read in the information from \texttt{LOG.txt},
creating an instance of the \texttt{ServiceRecord} class for each
record and insert the information into \texttt{ServiceLog.db} database. 

Save your program code as \texttt{Task2\_2\_<your name>\_<center number>\_<index
number>.py} \hfill{}{[}8{]}

\subsubsection*{Task 2.3 }

FlexiMSG wants to publish the database content on a web page. 

Create class \texttt{AppServiceRecord} which inherits from \texttt{ServiceRecord},
such that: 
\begin{itemize}
\item \texttt{getAppType()}- returns the following values based on the value
of \texttt{AppType} 
\begin{itemize}
\item \texttt{WA} - returns \texttt{'WHATSAPP'} 
\item \texttt{FB} - returns \texttt{'FACEBOOK MESSENGER' }
\end{itemize}
\item \texttt{getSuccess()}- returns the following values based on the returned
value of isSuccess() 
\begin{itemize}
\item \texttt{True} - returns \texttt{'SUCCESS'} 
\item \texttt{False} - returns \texttt{'FAILED'} 
\end{itemize}
\end{itemize}
Create class \texttt{SmsServiceRecord} which inherits from \texttt{ServiceRecord},
such that: 
\begin{itemize}
\item \texttt{getAppType()}- always returns \texttt{'SHORT MESSAGE SERVICE' }
\item \texttt{getSuccess()}- returns the following values based on the returned
value of \texttt{isSuccess() }
\begin{itemize}
\item \texttt{True} - returns \texttt{'SUCCESS'} 
\item \texttt{False} - returns \texttt{'FAILED'} 
\end{itemize}
\end{itemize}
Save your program code to \texttt{Task2\_3\_<your name>\_<center number>\_<index
number>.py}\hfill{} {[}4{]}

\subsubsection*{Task 2.4 }

Write a Python program and the necessary files to create a web application
that enables the list of log records to be displayed. 

For each record, the web page should include the: 
\begin{itemize}
\item \texttt{Sender }
\item \texttt{AccessDate }
\item \texttt{AppType} (either \texttt{WHATSAPP},\texttt{ FACEBOOK MESSENGER
}or\texttt{ SHORT MESSAGE SERVICE}) 
\item \texttt{Status} (\texttt{SUCCESS} or \texttt{FAILED}) 
\end{itemize}
Save your program code as 

\texttt{Task2\_4\_<your name>\_<center number>\_<index number>.py }

with any additional files/sub-folders as needed in a folder named 

\texttt{Task2\_4\_<your name>\_<center number>\_<index number>}.\hfill{}
{[}9{]}

Run the web application and save the output of the program as 

\texttt{Task2\_4\_<your name>\_<center number>\_<index number>.html}\hfill{}
{[}1{]}