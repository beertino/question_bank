\item \textbf{{[}RVHS/PRELIM/9597/2019/P2/Q4{]} }

A user-defined linked list has the following operations.

Operations and its descriptions 
\begin{itemize}
\item \texttt{create()} : This function creates an empty linked list.

For example: 

\texttt{lst1 = create()}

A linked list is created and referenced by \texttt{lst1}.
\item \texttt{insert(llst,data)} : This procedure always adds \texttt{data}
as a node at the head of the linked list

For example

\noindent %
\noindent\begin{minipage}[t]{1\columnwidth}%
\texttt{lst1: 'A'->'B'->'C'->'D'->'E'}

\texttt{insert(lst1, 'Z')}

\texttt{lst1: 'Z'->'A'->'B'->'C'->'D'->'E'}%
\end{minipage}
\item \texttt{split(llst,pos)} The function breaks the linked list at position
\texttt{pos} and returns the cut half as a new linked list.

For example: 

\noindent %
\noindent\begin{minipage}[t]{1\columnwidth}%
\texttt{lst1: 'A'->'B'->'C'->'D'->'E'}

\texttt{lst2 = split(lst1, 2)}

\texttt{lst1: 'A'->'B' }

\texttt{lst2: 'C'->'D'->'E'}%
\end{minipage}
\item \texttt{join(llst1,llist2)} This function joins 2 linked lists as
one.

For example: 

\noindent %
\noindent\begin{minipage}[t]{1\columnwidth}%
\texttt{lst1: 'C'->'B' }

\texttt{lst2: 'A'->'D'->'E}

\texttt{join(lst1, lst2)}

\texttt{lst1: 'C'->'B'->'A'->'D'->'E' }

\texttt{lst2: Empty}%
\end{minipage}
\end{itemize}
Using the provided operations of the linked list, write, in pseudo-code
an algorithm that could be used to implement the \texttt{swap} operation.
\emph{Hint: you don\textquoteright t have to use all the operations
provided}.

Operations and its description 
\begin{itemize}
\item \texttt{swap(llst,pos)} : This procedure swaps the data at positions
\texttt{pos} and \texttt{pos + 1}

For example: 

\noindent %
\noindent\begin{minipage}[t]{1\columnwidth}%
\texttt{lst1:'A'->'B'->'C'->'D'->'E' }

\texttt{swap(lst1,3)}

\texttt{lst1:'A'->'B'->'D'->'C'->'E'}%
\end{minipage}
\end{itemize}
You can assume that the operations executed are always valid. For
example, if there is no 5th node, \texttt{swap(lst1,4)} will not be
called. \hfill{}{[}5{]}