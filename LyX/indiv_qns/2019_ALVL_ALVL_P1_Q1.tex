\item \textbf{{[}ALVL/9597/2019/P1/Q1{]} }

A text file, \texttt{TIDES.TXT}. contains the low and high tide information
for a coastal location tor each day of a month. Each line contains
tab-delimited data that shows the date, the time. whether the tide
is high or low and the tide height in metres. 

Each line is in the format: 
\begin{center}
\texttt{YYYY-MM-DD\textbackslash tHH:mm\textbackslash tTIDE\textbackslash tHEIGHT\textbackslash n }
\par\end{center}
\begin{itemize}
\item The date is in the term YYYY-MM-DD, for example. 2019-08-03 is 3rd
August. 2019 
\item The time is in the form HH:mm, for example. 13:47 
\item TIDE is either HIGH or LOW 
\item HEIGHT is a positive number shown to one decimal place 
\item \texttt{\textbackslash t} represents the tab character
\item \texttt{\textbackslash n} represents the newline character 
\end{itemize}
The text file is stored in ascending order of date and time.

\subsubsection*{Task 1.1}

Write program code to:
\begin{itemize}
\item read the tide data from a text file
\item find the highest high tide and print this value
\item find the lowest low tide and print this value.
\end{itemize}
Use \texttt{TIDES.TXT} to test your program code.

\subsubsection*{Evidence 1}

Your program code.

Screenshot of your output. \hfill{}{[}9{]} 

\subsubsection*{Task 1.2}

The tidal range is the difference between the heights of successive
tides; from a high tide to the following low tide or from a low tide
to the following high tide.

Amend your program code to:
\begin{itemize}
\item output the largest tidal range and the date on which the second tide
occurs
\item output the smallest tidal range and the date on which the second tide
occurs.
\end{itemize}
Use \texttt{TIDES.TXT} to test your program code.

\subsubsection*{Evidence 2}

Your program code.

Screenshot of your output. \hfill{}{[}9{]}