\item \textbf{{[}ALVL/9597/2016/P2/Q5{]} }

A programmer implements a linked list of surnames with a start pointer,
\texttt{StartPtr} and two one-dimensional arrays: 
\begin{itemize}
\item Array \texttt{Data} stores the surnames. 
\item Array \texttt{Ptr} stores the link pointers.
\item Both arrays have lower bound 1 and upper bound 3000. 
\end{itemize}
The purpose of procedure \texttt{InsertListItem} is to insert a new
surname to the linked list. 

Assume a function \texttt{NextFree()} is available and returns: 
\begin{itemize}
\item the index position for the array \texttt{Data} at which the new surname
is to be inserted 
\item -1 when the \texttt{Data} array is full. 
\end{itemize}
The programmer designs the algorithm as follows:

\noindent %
\noindent\begin{minipage}[t]{1\columnwidth}%
\texttt{01 PROCEDURE InsertListItem(NewSurname : STRING) }

\texttt{02 \qquad{}IF NextFree() = \textemdash 1 }

\texttt{03 \qquad{}\qquad{}THEN }

\texttt{04 \qquad{}\qquad{}\qquad{}OUTPUT \textquotedbl List is
full\textquotedbl{} }

\texttt{05 \qquad{}\qquad{}ELSE }

\texttt{O6 \qquad{}\qquad{}\qquad{}// input the surname }

\texttt{07 \qquad{}\qquad{}\qquad{}IF StartPtr = 0 }

\texttt{08 \qquad{}\qquad{}\qquad{}\qquad{}THEN }

\texttt{09 \qquad{}\qquad{}\qquad{}\qquad{}\qquad{}StartPtr w
NextFree() }

\texttt{10 \qquad{}\qquad{}\qquad{}\qquad{}\qquad{}Data{[}StartPtr{]}
e NewSurname }

\texttt{11 \qquad{}\qquad{}\qquad{}\qquad{}ELSE }

\texttt{12 \qquad{}\qquad{}\qquad{}\qquad{}\qquad{}// traverse
the linked list to find the position }

\texttt{13 \qquad{}\qquad{}\qquad{}\qquad{}\qquad{}// at which
to insert NewSurname}

$\vdots$

\texttt{\qquad{}\qquad{}\qquad{}\qquad{}ENDIF }

\texttt{\qquad{}\qquad{}ENDIF }

\texttt{\qquad{}ENDPROCEDURE}%
\end{minipage}
\begin{enumerate}
\item Describe the state of the linked list. if the condition \texttt{StartPtr
= 0} in line \texttt{07} is \texttt{True}. {[}1{]} 
\item It is now necessary to complete the design for procedure \texttt{InsertListItem}. 
\begin{enumerate}
\item The pseudocode already uses some variables. 

Copy the table below and complete it to show any extra variables that
you will need to use. 
\begin{center}
\begin{tabular}{|c|c|c|}
\hline 
\textbf{Variable} & \textbf{Data Type} & \textbf{Description}\tabularnewline
\hline 
 &  & \tabularnewline
\hline 
 &  & \tabularnewline
\hline 
 &  & \tabularnewline
\hline 
 &  & \tabularnewline
\hline 
 &  & \tabularnewline
\hline 
\end{tabular}
\par\end{center}

\hfill{}{[}3{]}
\item Write the pseudocode for line \texttt{14} onwards to complete the
procedure. \hfill{}{[}6{]}
\end{enumerate}
\end{enumerate}