\item \textbf{{[}HCI/PRELIM/9569/2020/P1/Q5{]} }
\begin{enumerate}
\item Run-length encoding is a simple data compression technique that can
be effective when repeated values occur at adjacent positions within
a string. Compression is achieved by replacing groups of repeated
values with one copy of the value, followed by the number of times
the value should be repeated. For example, \textquotedblleft \texttt{AAAAABBBAAAB}\textquotedblright{}
would be compressed as \textquotedblleft \texttt{A5B3A3B1}\textquotedblright . 

Write, in \textbf{pseudocode}, a function that implements the run-length
compression technique described above. The function will take a string
argument and return the run-length compressed string. \hfill{}{[}6{]}
\item Using \textbf{pseudocode}, write a detailed algorithm for a function
which will take two string values, called \texttt{P} and \texttt{Q},
and will search the string value \texttt{P} for the first occurrence
of the string value \texttt{Q} within it. The value returned is the
start position of the first occurrence of \texttt{Q} in \texttt{P},
or zero if there is no occurrence. Assume the string index starts
at 1. For example, if \texttt{P} is '\texttt{bananas}' and Q is '\texttt{na}'
then the function would give the result 3, because '\texttt{na}' first
occurs in '\texttt{bananas}' starting at character position three.
{[}6{]}
\end{enumerate}