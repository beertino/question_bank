\item \textbf{{[}DHS/PRELIM/9569/2020/P2/Q1{]} }

On 28 June 2020, nearly 10,000 TraceTogether tokens were distributed
to vulnerable seniors. The TraceTogether token supplements the TraceTogether
mobile app by extending contact tracing to groups in the community
who do not have smart phones and those whose phones do not work well
with the TraceTogether app. 

The TraceTogether token is designed to capture proximity data based
on Bluetooth signals. Every five minutes, it scans to detect other
TraceTogether users on the token or the app. The more 'hits' between
two TraceTogether users, the more likely they are in close proximity
for an extended period of time. Proximity can be estimated by the
strength of the Bluetooth signal. The closer users are to one another,
the stronger the signal and vice versa. 

There are only four types of data contained in the token: 
\begin{itemize}
\item user's randomised ID 
\item randomised ID of any other user in proximity 
\item Bluetooth signal measured using RSSI{*} 
\item timestamp of the encounter. 
\end{itemize}
{*}Received Signal Strength Indicator (RSSI) is a measure of the power
level at the receiver. A more negative number indicates a device is
further away. For example, a value of -20 to -30 indicates a device
is close while a value of -120 indicates it is near the limit of detection. 

It is important to note that these IDs do not refer to NRIC number,
but randomised and anonymised IDs linked to a personal identifier
like a mobile number. Also, no data is extracted unless a user has
tested positive for COVID-19. From there, MOH has a special software
key that can unlock the device and reveal the data for use in contact
tracing. 

A senior is tested positive for COVID-19 and MOH needs to perform
contact tracing. With the user's permission, data from its TraceTogether
token is retrieved and extracted to the file \texttt{TOKEN.txt} and
has the following structure: 

\texttt{UserRandomisedID, OtherRandomisedID, RSSI, Timestamp} 

Timestamp is in the format YYYY-MM-DD HH:MM:SS. 

\subsection*{Task 1.1 }

Prolonged exposure is currently defined as being in close contact
for at least 15 minutes within a single session. For simplicity, you
may assume close contact as a Bluetooth signal strength of greater
than or equal to -30. Generate the list of close contacts' randomised
IDs which MOH needs to perform contact tracing. \hfill{}{[}10{]}

\subsection*{Task 1.2 }

Another 3 seniors with randomised ID 75348257, 45174591 and 02548147
have also been tested as COVD-19 positive. Write a Boolean function
\texttt{is\_close\_contact(rid1, rid2)} to determine if they are close
contacts of 57345286. If yes, your function should also return the
date(s) they are under prolonged exposure with each other, the start
and end times as well as the total time in hours and minutes they
are in close contact. \hfill{}{[}10{]}