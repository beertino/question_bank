\item \textbf{{[}YIJC/PRELIM/9569/2021/P1/Q1{]} }

An iterative function Fn has two parameters, Arr and Object, and returns
an integer. The pseudocode is as follows: 

\noindent\begin{minipage}[t]{1\columnwidth}%
\texttt{01 FUNCTION Fn(Arr: ARRAY, Object: STRING) RETURNS INTEGER }

\texttt{02 \qquad{}Current \textleftarrow{} 1 }

\texttt{03 \qquad{}\qquad{}REPEAT }

\texttt{04 \qquad{}\qquad{}\qquad{}IF Object = Arr{[}Current{]} }

\texttt{05 \qquad{}\qquad{}\qquad{}\qquad{}THEN }

\texttt{06 \qquad{}\qquad{}\qquad{}\qquad{}\qquad{}RETURN Current }

\texttt{07 \qquad{}\qquad{}\qquad{}ENDIF }

\texttt{08 \qquad{}\qquad{}\qquad{}IF Object > Arr{[}Current{]} }

\texttt{09 \qquad{}\qquad{}\qquad{}\qquad{}THEN }

\texttt{10 \qquad{}\qquad{}\qquad{}\qquad{}\qquad{}Next \textleftarrow{}
1 + Current {*} 2 }

\texttt{11 \qquad{}\qquad{}\qquad{}\qquad{}ELSE }

\texttt{12 \qquad{}\qquad{}\qquad{}\qquad{}\qquad{}Next \textleftarrow{}
Current {*} 2 }

\texttt{13 \qquad{}\qquad{}\qquad{}ENDIF }

\texttt{14 \qquad{}\qquad{}\qquad{}Current \textleftarrow{} Next }

\texttt{15 \qquad{}UNTIL Current > LENGTH(Arr) }

\texttt{16 \qquad{}RETURN -1 }

\texttt{17 ENDFUNCTION }%
\end{minipage}A binary search tree is used to store the names of the 12 Chinese
Zodiac animals. The order in which these names were added into the
tree follows the order in the array \texttt{X}. 
\noindent \begin{center}
\begin{tabular}{c|c|cc|c|}
\multicolumn{1}{c}{} & \multicolumn{1}{c}{\texttt{Array X}} &  & \multicolumn{1}{c}{} & \multicolumn{1}{c}{}\tabularnewline
\cline{2-2} \cline{5-5} 
\texttt{X{[}1{]}} & \texttt{'Rat'} &  & \texttt{X{[}7{]}} & \texttt{'Tiger' }\tabularnewline
\cline{2-2} \cline{5-5} 
\texttt{X{[}2{]}} & \texttt{'Monkey'} &  & \texttt{X{[}8{]}} & \texttt{'Dog'}\tabularnewline
\cline{2-2} \cline{5-5} 
\texttt{X{[}3{]}} & \texttt{'Snake'} &  & \texttt{X{[}9{]}} & \texttt{'Horse'}\tabularnewline
\cline{2-2} \cline{5-5} 
\texttt{X{[}4{]}} & \texttt{'Dragon'} &  & \texttt{X{[}10{]}} & \texttt{'Ox'}\tabularnewline
\cline{2-2} \cline{5-5} 
\texttt{X{[}5{]}} & \texttt{'Pig'} &  & \texttt{X{[}11{]}} & \texttt{'Rabbit'}\tabularnewline
\cline{2-2} \cline{5-5} 
\texttt{X{[}6{]}} & \texttt{'Sheep'} &  & \texttt{X{[}12{]}} & \texttt{'Rooster'}\tabularnewline
\cline{2-2} \cline{5-5} 
\end{tabular}
\par\end{center}
\begin{enumerate}
\item Draw the binary tree using the array \texttt{X}. \hfill{}{[}3{]}
\item Complete the trace table templates provided for the following function
calls and state the RETURN value after each function call: 
\noindent \begin{center}
\begin{tabular}{|c|c|c|c|}
\hline 
\texttt{Current } & \texttt{Object = Arr{[}Current{]} } & \texttt{Object > Arr{[}Current{]} } & \texttt{Next }\tabularnewline
\hline 
 &  &  & \tabularnewline
\hline 
 &  &  & \tabularnewline
\hline 
 &  &  & \tabularnewline
\hline 
 &  &  & \tabularnewline
\hline 
\multicolumn{4}{|l|}{\texttt{Output:}}\tabularnewline
\hline 
\end{tabular}
\par\end{center}
\begin{enumerate}
\item Function call: \texttt{Fn(X,'Sheep')} \hfill{}{[}2{]}
\item Function call: \texttt{Fn(X,'Duck')} \hfill{}{[}3{]}
\end{enumerate}
\item Describe the purpose of function \texttt{Fn}. \hfill{}{[}2{]}
\item Explain why this function \texttt{Fn} is more efficient than linear
search.\hfill{} {[}2{]}
\item It is found that one of the Zodiac animals should be named as \textquoteleft Goat\textquoteright{}
instead of \textquoteleft Sheep\textquoteright . Design another array
\texttt{Y} such that the function \texttt{Fn} can be used with the
correct list of the Chinese Zodiac animals. 

Complete the template provided for the array \texttt{Y}.\hfill{}{[}2{]}
\end{enumerate}
\begin{description}
\item [{Template~for~Question~1(b)}]~
\end{description}
\begin{enumerate}
\item[(i)] Function call: \texttt{Fn(X,'Sheep')}

\noindent %
\begin{tabular}{|c|c|c|c|}
\hline 
\texttt{Current} & \texttt{Object = Arr{[}Current{]}} & \texttt{Object > Arr{[}Current{]}} & \texttt{Next}\tabularnewline
\hline 
 &  &  & \tabularnewline
\hline 
 &  &  & \tabularnewline
\hline 
 &  &  & \tabularnewline
\hline 
 &  &  & \tabularnewline
\hline 
\multicolumn{4}{|l|}{\texttt{Output:}}\tabularnewline
\hline 
\end{tabular}
\item[(ii)] Function call: \texttt{Fn(X,'Duck')}
\noindent \begin{flushleft}
\begin{tabular}{|c|c|c|c|}
\hline 
\texttt{Current} & \texttt{Object = Arr{[}Current{]}} & \texttt{Object > Arr{[}Current{]}} & \texttt{Next}\tabularnewline
\hline 
 &  &  & \tabularnewline
\hline 
 &  &  & \tabularnewline
\hline 
 &  &  & \tabularnewline
\hline 
 &  &  & \tabularnewline
\hline 
\multicolumn{4}{|l|}{\texttt{Output:}}\tabularnewline
\hline 
\end{tabular}
\par\end{flushleft}
\end{enumerate}
\begin{description}
\item [{Template~for~Question~1(e)}]~
\end{description}
\noindent \begin{center}
\begin{tabular}{c|c|cc|c|}
\multicolumn{1}{c}{} & \multicolumn{1}{c}{\texttt{Array Y}} &  & \multicolumn{1}{c}{} & \multicolumn{1}{c}{}\tabularnewline
\cline{2-2} \cline{5-5} 
\texttt{Y{[}1{]}} & \texttt{\textcolor{white}{'Rat'}} &  & \texttt{Y{[}7{]}} & \texttt{\textcolor{white}{'Tiger'}}\tabularnewline
\cline{2-2} \cline{5-5} 
\texttt{Y{[}2{]}} & \texttt{\textcolor{white}{'Monkey'}} &  & \texttt{Y{[}8{]}} & \texttt{\textcolor{white}{'Dog'}}\tabularnewline
\cline{2-2} \cline{5-5} 
\texttt{Y{[}3{]}} & \texttt{\textcolor{white}{'Snake'}} &  & \texttt{Y{[}9{]}} & \texttt{\textcolor{white}{'Horse'}}\tabularnewline
\cline{2-2} \cline{5-5} 
\texttt{Y{[}4{]}} & \texttt{\textcolor{white}{'Dragon'}} &  & \texttt{Y{[}10{]}} & \texttt{\textcolor{white}{'Ox'}}\tabularnewline
\cline{2-2} \cline{5-5} 
\texttt{Y{[}5{]}} & \texttt{\textcolor{white}{'Pig'}} &  & \texttt{Y{[}11{]}} & \texttt{\textcolor{white}{'Rabbit'}}\tabularnewline
\cline{2-2} \cline{5-5} 
\texttt{Y{[}6{]}} & \texttt{\textcolor{white}{'Sheep'}} &  & \texttt{Y{[}12{]}} & \texttt{\textcolor{white}{'Rooster'}}\tabularnewline
\cline{2-2} \cline{5-5} 
\end{tabular}
\par\end{center}