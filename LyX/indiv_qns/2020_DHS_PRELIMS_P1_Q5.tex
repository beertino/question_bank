\item \textbf{{[}DHS/PRELIM/9569/2020/P1/Q5{]} }

A student came up with the following Python program to implement a
linked list data structure:

\noindent %
\noindent\begin{minipage}[t]{1\columnwidth}%
\texttt{01\qquad{}class Node: }

\texttt{02\qquad{}\qquad{}def \_\_init\_\_(self, data): }

\texttt{03\qquad{}\qquad{}self.data = data }

\texttt{04\qquad{}\qquad{}self.link = None }

\texttt{05 }

\texttt{06\qquad{}def insert(data): }

\texttt{07\qquad{}\qquad{}global head }

\texttt{08\qquad{}\qquad{}if head == None: \# empty linked list }

\texttt{09\qquad{}\qquad{}\qquad{}head = Node(data) }

\texttt{10\qquad{}\qquad{}else: \# insert to front }

\texttt{11\qquad{}\qquad{}\qquad{}new\_node = Node(data) }

\texttt{12\qquad{}\qquad{}\qquad{}new\_node.link = head }

\texttt{13\qquad{}\qquad{}\qquad{}head = new\_node }

\texttt{14 }

\texttt{15\qquad{}\qquad{}def display(): }

\texttt{16\qquad{}\qquad{}\qquad{}global head }

\texttt{17\qquad{}\qquad{}\qquad{}curr = head }

\texttt{18\qquad{}\qquad{}\qquad{}while curr: }

\texttt{19\qquad{}\qquad{}\qquad{}\qquad{}print(curr.data) }

\texttt{20\qquad{}\qquad{}\qquad{}\qquad{}curr = curr.link }

\texttt{21 }

\texttt{22\qquad{}\# main }

\texttt{23\qquad{}head = None }

\texttt{24\qquad{}insert(\textquotedbl Ali\textquotedbl ) }

\texttt{25\qquad{}insert(\textquotedbl Tom\textquotedbl ) }

\texttt{26\qquad{}insert(\textquotedbl Mary\textquotedbl ) }

\texttt{27\qquad{}display() }%
\end{minipage}
\begin{enumerate}
\item What will be the output of line 27?\hfill{} {[}1{]}
\item Comment on the programming paradigms used and identify any potential
pitfalls in the above program.\hfill{}{[}5{]}
\item Why is OOP appropriate in the implementation of data structures such
as linked lists?\hfill{}{[}2{]}
\item The above program maintains an unordered linked list. Devise an algorithm
to insert to an ordered linked list.\hfill{}{[}5{]}
\item A linked list can be ordered or unordered. Draw a UML class diagram
to illustrate the concepts of encapsulation, inheritance and polymorphism.\hfill{}
{[}5{]}
\item Explain how polymorphism is applied to the \texttt{insert()} rather
than the \texttt{display()} method in this context.\hfill{} {[}3{]}
\end{enumerate}