\item \textbf{{[}DHS/PRELIM/9597/2015/P1/Q4{]} }

You are tasked to maintain a binary search tree for optimal search
efficiency. Each node of the binary search tree contains a unique
identifier (productid) and a reference to a linked list which holds
individual order details for that product. For simplicity, orderid
will suffice. 

\subsection*{Task 4.1 }

Using object-oriented programming, implement the above binary search
tree with its associated class methods (\texttt{insert()}, \texttt{search()}
and \texttt{inorder()}). 

\subsection*{Evidence 21}

Program code for binary search tree with 8 products inserted, 3 of
which has at least 2 orders. \hfill{}{[}8{]}

\subsection*{Evidence 22}

Screenshot of output of inorder traversal (product and order information).
\hfill{} {[}2{]}

\subsection*{Task 4.2 }

Write a support class method most\_popular() to determine the best
selling product(s). 

\subsection*{Evidence 23 }

Program code. \hfill{} {[}4{]}

\subsection*{Evidence 24 }

Screenshot. \hfill{}{[}1{]}

\subsection*{Task 4.3 }

A product is no longer available. Write a class method delete() to
remove this node from the binary search tree, accounting for all possible
cases. 

\subsection*{Evidence 25 }

Program code. \hfill{} {[}5{]}

\subsection*{Evidence 26 }

Screenshots. \hfill{} {[}3{]}

\subsection*{Task 4.4 }

After a series of insertion and deletion, the binary search tree may
become unbalanced and this requires reorganisation to maintain optimal
search efficiency. Write functions \texttt{is\_balanced()} to test
if a binary search tree is balanced, and \texttt{reorg()} to reorganise
an unbalanced binary search tree. Include appropriate annotation.

\subsection*{Evidence 27 }

Program code. \hfill{}{[}7{]}