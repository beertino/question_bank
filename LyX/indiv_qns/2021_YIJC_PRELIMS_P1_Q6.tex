\item \textbf{{[}YIJC/PRELIM/9569/2021/P1/Q6{]} }

A fantasy card game was developed using object-oriented programming
(OOP) to store its cards\textquoteright{} data. A card can either
be a minion or a weapon. Each card has a name, mana cost, health or
durability and attack power. 

In order to play a card, a player must spend a certain amount of mana
as specified in the card\textquoteright s mana cost. When the card
has been played, the player may decide whether to use it to attack
another minion or not. 

A minion may belong to one of the following races: beast, demon, dragon
or elemental. When a minion is attacked, its health would decrease
according to the attacking card\textquoteright s attack power. Once
the health of a minion decreases to zero, it is destroyed and removed
from the game. 

Instead of health, a weapon has durability and it cannot be attacked.
When a weapon is used for an attack, its durability would decrease
by one. Once the durability of a weapon decreases to zero, it is destroyed
and removed from the game. 
\begin{enumerate}
\item Draw a class diagram, showing: 
\begin{itemize}
\item the base class \texttt{CARD}, 
\item any sub-classes and inheritance from the base class, 
\item the properties for the base class and sub-classes, 
\item appropriate methods with at least one getter and one setter method.\hfill{}
{[}7{]}
\end{itemize}
\item In relation to your diagram in part \textbf{(a)}, explain the terms: 
\begin{enumerate}
\item encapsulation, 
\item inheritance, 
\item polymorphism. \hfill{}{[}6{]}
\end{enumerate}
\item Explain why OOP is a preferred programming paradigm in the development
of this game. \hfill{}{[}2{]}
\end{enumerate}