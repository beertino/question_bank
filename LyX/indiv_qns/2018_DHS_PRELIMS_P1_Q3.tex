\item \textbf{{[}DHS/PRELIM/9597/2018/P1/Q3{]} }

A blockchain is a linked list of blocks where each block has the following
structure:
\begin{center}
\begin{tabular}{|l|l|l|}
\hline 
\multicolumn{3}{|c|}{Class\texttt{: Block}}\tabularnewline
\hline 
\multicolumn{3}{|c|}{Attributes}\tabularnewline
\hline 
\texttt{\hspace{0.01\columnwidth}}Identifier & \texttt{\hspace{0.01\columnwidth}}Data Type & \texttt{\hspace{0.05\columnwidth}}Description\tabularnewline
\hline 
\texttt{Data} & \texttt{String} & Block data\tabularnewline
\hline 
\texttt{PrevHash} & \texttt{String} & Hash of previous block\tabularnewline
\hline 
\texttt{CurrHash} & \texttt{String} & Hash of \texttt{Data} and \texttt{PrevHash}\tabularnewline
\hline 
\texttt{Next} & \texttt{Integer} & The next block pointer\tabularnewline
\hline 
\end{tabular}
\par\end{center}

The structure of the blockchain is as follows:
\begin{center}
\begin{tabular}{|l|l|l|}
\hline 
\multicolumn{3}{|c|}{Class\texttt{: BlockChain}}\tabularnewline
\hline 
\multicolumn{3}{|c|}{Attributes}\tabularnewline
\hline 
\texttt{\hspace{0.01\columnwidth}}Identifier & \texttt{\hspace{0.01\columnwidth}}Data Type & \texttt{\hspace{0.05\columnwidth}}Description\tabularnewline
\hline 
\texttt{ChainData} & \texttt{Array{[}1:20{]} of Block} & An array used to store the 20 blocks.\tabularnewline
\hline 
\texttt{Start} & \texttt{Integer} & Index for the genesis block.\tabularnewline
\hline 
\texttt{NextFreeBlock} & \texttt{Integer} & Index for the next available empty block.\tabularnewline
\hline 
\end{tabular}
\par\end{center}

The initial value of \texttt{Start} is 1 and the initial value of
\texttt{NextFreeBlock} is 1.

The first block of the blockchain is called the genesis block and
its \texttt{PrevHash} value is 983.

The blockchain is used to store the achievement data of students in
computing and infocomm programmes. The ensures the integrity and verifiability
of students' portfolios which will be useful in internships, higher
education and career opportunities.

\subsection*{Task 3.1}

Write program code to declare and initialise an empty blockchain of
20 unused blocks. Also write the \texttt{Display} method to show all
contents of the blockchain.

\subsection*{Evidence 11}

Program code. \hfill{}{[}8{]}

\subsection*{Evidence 12 }

Screenshot. \hfill{}{[}2{]}

\subsection*{Task 3.2 }

The following hashing algorithm computes the \texttt{CurrHash} value
of each block: 
\begin{itemize}
\item Compute the sum of ASCII values for the characters in the achievement
data string. 
\item Multiply this sum by the kth prime number, where k is the length of
the achievement data string. 
\item Multiply this with the decimal equivalent of the \texttt{PrevHash}
value of the current block. 
\item Convert this value to its uppercase hexadecimal equivalent. 
\item Prepend the appropriate number of \texttt{'F'} to this result to form
a 23-character resultant string. This will be the current block's
\texttt{CurrHash} value.
\end{itemize}
For example, for the achievement data string
\noindent \begin{center}
\texttt{Splash Awards 2018:Robert Goh,Mary Tan,Choo Ah Beng:First }
\par\end{center}

Its \texttt{CurrHash} value will be \texttt{FFFFFFFFFFFFFFF4D32A036}
(sum of ASCII value {*} 57th prime number {*} \texttt{PrevHash} =
4898 {*} 269 {*} 983)

Write program code for a \texttt{ComputeHash} function to calculate
the \texttt{CurrHash} value of a block. Verify your function with
the following 2 achievement data strings:

\noindent %
\noindent\begin{minipage}[t]{1\columnwidth}%
\texttt{Splash Awards 2018:Robert Goh,Mary Tan,Choo Ah Beng:First}

\texttt{Splash Awards 2018:Lim Ah Huat,Alice Wong,Tan Ah Lian:Honorable
Mention}%
\end{minipage}

\subsection*{Evidence 13 }

Program code. \hfill{}{[}18{]}

\subsection*{Evidence 14}

Screenshot for the 2 achievement data strings. \hfill{}{[}2{]}

\subsection*{Task 3.3 }

Write program code to insert the data in \texttt{ACHIEVEMENTS.TXT}
into the blockchain and display the contents of the updated blockchain.

\subsection*{Evidence 15}

Program code.\hfill{} {[}4{]}

\subsection*{Evidence 16}

Screenshot. \hfill{}{[}1{]}

\subsection*{Task 3.4}

Lim Ah Huat aspires to save the world by studying computer science
in NUS School of Computing. As it is now the toughest course to get
admitted to, he hopes to improve his chances of admission by showcasing
his achievements in the various computing and infocomm related programmes
he has participated in. He claimed that he is a good team player and
is a strong self-directed learner with excellent aptitude for Computing.

How can the university admission panel verify his achievements using
the existing blockchain? You should describe briefly in program comments
this strategy and implement the associated program code.

\subsection*{Evidence 17 }

Program code. \hfill{}{[}5{]}

\subsection*{Evidence 18}

Screenshot.\hfill{} {[}1{]}

\subsection*{Task 3.5 }

Another student Tan Ah Seng claimed that he also has computing or
infocomm related participation, and changes one of the achievement
data string from 

\texttt{Splash Awards 2018:Lim Ah Huat,Alice Wong,Tan Ah Lian:Honorable
Mention}

to

\texttt{Splash Awards 2018:Lim Ah Huat,Alice Wong,Tan Ah Seng:Honorable
Mention}

Using program comments, briefly explain the impact to the blockchain.
Write program code to refute Mr Tan's claim.

\subsection*{Evidence 19 }

Program code. \hfill{}{[}3{]}

\subsection*{Evidence 20 }

Screenshot.\hfill{} {[}1{]}