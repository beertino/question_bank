\item \textbf{{[}DHS/PRELIM/9597/2016/P2/Q2{]} }

The current GovBuy prototype supports two types of reverse auctions:
a standard reverse auction and a sealed bid auction. 

A standard reverse auction is one where bids begin at \$5,000. The
lowest possible bid is \$1, and an auction automatically ends once
a \$1 bid has been submitted. Bidders are allowed to submit multiple
bids throughout the duration of the auction. Bidders will see whether
they are the winning bidder after submitting a bid, and will have
an opportunity to submit a lower bid if the auction is still running.
The GitHub user names of participating bidders are hidden during the
auction. At the end of the auction, all bidders' GitHub accounts and
bid amounts will be unsealed and posted on the GovBuy platform. A
sealed bid auction is a type of reverse auction. Each bidder is allowed
to submit only one bid in an auction. Once a bid is submitted, the
bidder may not submit a second bid for the same auction. The lowest
bidder at the conclusion of the auction will still win the auction.
In the event that one or more bidders have the same bid amount, the
bidder who was first to submit the lowest bid amount will win the
auction. All bids will stay sealed until the end of the auction. A
bidder will not know the amounts other bidders have bidded on the
auction, or how many bids have been submitted. At the end of the auction,
all bidders' GitHub accounts and bid amounts will be unsealed and
posted on the GovBuy platform. 
\begin{enumerate}
\item Illustrate the types of reverse auctions in a class UML diagram. \hfill{}{[}5{]}
\item Using your illustration, explain the following OOP concepts: 
\begin{enumerate}
\item encapsulation \hfill{}{[}2{]}
\item inheritance \hfill{}{[}2{]}
\item polymorphism \hfill{} {[}2{]}
\end{enumerate}
\end{enumerate}