\item \textbf{{[}HCI/PRELIM/9569/2020/P2/Q2{]} }

The file \texttt{COUNTRY1.TXT} stores a list of country names.

\subsection*{Task 2.1 }

Using the country name, a hash address is calculated from a hashing
function as follows: 
\begin{itemize}
\item the ASCII code is calculated for each character (in lowercase) within
the country name 
\item the total of ASCII values is calculated 
\item the total is divided by 30 and the remainder is the hash address for
this country 
\end{itemize}
For example, the hash address for \texttt{Brazil} is 14 

Write program code for the hashing function using the following specification 

\texttt{FUNCTION HashKey (Country: STRING): INTEGER }

This function has a single parameter \texttt{Country} and returns
the hash address as an integer. \hfill{}{[}3{]}

\subsection*{Task 2.2 }

The hash table is implemented as a list with 30 elements. Elements
are written to and read from the hash table using the above hash function
with the country name as the input parameter. 

Write program code which does the following: 
\begin{itemize}
\item Read all country names from \texttt{COUNTRY1.TXT} 
\item Use the function \texttt{HashKey} to calculate the hash address for
each country 
\item Store each country name in the hash table 
\end{itemize}
You must ensure that when a collision occurs, your program design
will deal with this situation by searching sequentially from the calculated
hash address for an empty location and storing the country name at
this empty location. The program will generate an error message if
the hash table is full. \hfill{}{[}7{]}

\subsection*{Task 2.3 }

Write program code for a procedure SearchCountry using the following
specification 

\texttt{PROCEDURE SearchCountry (Country: STRING, HashTable: LIST)}

This procedure has two parameters, \texttt{Country} and \texttt{HashTable},
and does the following: 
\begin{itemize}
\item Calculate the hash address for the country 
\item Locate the country name in the hash table 
\item Report whether or not this country name was found. If found, also
output the address of the hash table where the country was found. 
\end{itemize}
You must ensure that your program can deal with collision when searching.
\hfill{}{[}6{]}

Devise a set of \textbf{three} test cases with the country to be used.
\hfill{} {[}3{]}

\subsection*{Task 2.4 }

The file \texttt{COUNTRY2.TXT} stores a list of countries with their
corresponding numbers of confirmed cases and death cases of COVID19
pandemic on May 16, 2020. 

Each record has the following format: 
\noindent \begin{center}
\texttt{<country>,<no. of confirmed cases>,<no. of death case> }
\par\end{center}

A sample record is 
\noindent \begin{center}
\texttt{Brazil, 233142, 15633 }
\par\end{center}

Death rate for each country is calculated as the number of death cases
/ the number of confirmed cases. This rate is output as a percentage
to 1 decimal place. For the example above, the death rate of Brazil
is $15633/233142=6.7\%$.

Write program code which does the following: 
\begin{itemize}
\item Read the data from \texttt{COUNTRY2.TXT }
\item Implement \textbf{bubble sort} to arrange the countries from highest
to lowest death rate 
\item Generate a text file \texttt{RATE.TXT} to display the list of countries
and their death rate. Each record has the format \texttt{<country>,
<death rate>}. 
\end{itemize}
If two countries have the same death rate, their order in the list
does not matter. \hfill{}{[}9{]}