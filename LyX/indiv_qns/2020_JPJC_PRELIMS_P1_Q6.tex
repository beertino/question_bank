\item \textbf{{[}JPJC/PRELIM/9569/2020/P1/Q6{]} }

Traversal was performed on the binary tree given below. 
\begin{enumerate}
\item List the nodes, in the order, that are visited for, 
\begin{enumerate}
\item in-order traversal, \hfill{}{[}1{]}
\item pre-order traversal, and \hfill{}{[}1{]}
\item post-order traversal.\hfill{}{[}1{]}
\end{enumerate}
\item A binary search tree is considered as an ordered binary tree where
the key values of nodes in the left sub-tree are less than the value
of its parent (root) node's key, and key values of nodes in the right
sub-tree are greater than the value of its parent (root) node\textquoteright s
key. 
\begin{enumerate}
\item Explain how a recursive algorithm can be used to locate a node with
key value \texttt{search\_key} by returning \texttt{TRUE} when \texttt{search\_key}
is found, and \texttt{FALSE} otherwise. \hfill{}{[}4{]}
\item State \textbf{one} advantage of using binary search tree as data structure
over linked list, and describe a situation that would negate this
advantage.\hfill{} {[}2{]}
\end{enumerate}
\end{enumerate}