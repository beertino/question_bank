\item \textbf{{[}ACJC/PRELIM/9569/2021/P1/Q5{]} }

A gym organises various classes and runs a loyalty membership programme
with four tiers: Bronze, Silver, Gold and Diamond 

Upon joining, each member is given a unique membership number and
starts with a Bronze membership. Each member can sign up for multiple
classes at a reduced rate based on the membership tier.

Each class has a unique class name. Some classes are offered at three
different levels: Beginner, Intermediate and Advanced

Each instructor is identified with a unique three-character code and
can take one or more classes.

A relational database is to be created to store data about members,
employees and classes.

Part of the table \texttt{MEMBER}, which is a first attempt at the
database design, is shown below. 

\begin{tabular}{|c|c|c|c|c|}
\hline 
MemberNo  & MemberName  & MemberTier  & ClassName & InstCode\tabularnewline
\hline 
\multirow{3}{*}{5 } & \multirow{3}{*}{Lindy White} & \multirow{3}{*}{Silver } & Body Pump  & WAY \tabularnewline
\cline{4-5} \cline{5-5} 
 &  &  & Yoga (Beginner)  & DAV \tabularnewline
\cline{4-5} \cline{5-5} 
 &  &  & Zumba & ROG\tabularnewline
\hline 
\dots{}  & \dots{}  & \dots{}  & \dots{}  & ...\tabularnewline
\hline 
78  & Derek Davis  & Bronze  & Muay Thai (Beginner)  & CHA \tabularnewline
\hline 
\dots{} & \dots{}  & \dots{} \dots{} \dots{}  & \dots{}  & ...\tabularnewline
\hline 
\multirow{4}{*}{132} & \multirow{4}{*}{John Chua } & \multirow{4}{*}{Diamond } & Circuits (Intermediate)  & JON \tabularnewline
\cline{4-5} \cline{5-5} 
 &  &  & Muay Thai (Intermediate)  & LEX \tabularnewline
\cline{4-5} \cline{5-5} 
 &  &  & Yoga (Advanced) & DAV \tabularnewline
\cline{4-5} \cline{5-5} 
 &  &  & Zumba & ROG\tabularnewline
\hline 
\dots{}  & \dots{} & \dots{}  & \dots{}  & \dots{}\tabularnewline
\hline 
\end{tabular}
\begin{enumerate}
\item The table \texttt{MEMBER} is not normalised.
\begin{enumerate}
\item Describe \textbf{one} potential issue that may be encountered when
the data are maintained in such a non-normalised table. \hfill{}{[}1{]}
\item Explain why the table is not in first normal form (1NF). \hfill{}{[}1{]}
\end{enumerate}
\item A second attempt at the database design gives rise to two tables:

\texttt{MEMBER(MemberNo, MemberName, MemberTier)}

\texttt{MEMBERCLASSES(MemberNo, ClassName, Instructor)}

The primary keys are not shown. 
\begin{enumerate}
\item State what is meant by a \textbf{primary key}. \hfill{}{[}1{]}
\item By referring to the relationship between the tables \texttt{MEMBER}
and \texttt{MEMBERCLASSES}, state how the relationship is implemented.\hfill{}
{[}2{]}
\item Write an SQL query to create the table MEMBER with the appropriate
constraints.\hfill{} {[}4{]}
\end{enumerate}
\item Another attempt at the database design needs to be made to ensure
that all the tables are in third normal form (3NF). 

In addition, the following data need to be recorded in the database:
\begin{itemize}
\item the date on which each member signs up for the gym membership; 
\item the attendance of each member in any classes taken;
\item the original fee, i.e. before discount, of each class; 
\item the name and the salary of each instructor. 
\end{itemize}
\begin{enumerate}
\item State the total number of 3NF tables required and give their names.
\hfill{}{[}1{]}
\item Draw the Entity-Relationship (E-R) diagram to show the 3NF tables
and the relationships between them. \hfill{}{[}4{]}
\item A table description can be written as:

\texttt{TableName(}\texttt{\uline{Attribute1}}\texttt{, Attribute2{*},
Attribute3, ...) }

The primary key is indicated by underlining one or more attributes.
Foreign keys are indicated by using a dashed underline/asterisk. 

Using the information provided, write table descriptions for all the
3NF tables you identified in \textbf{(c)(i)}.\hfill{} {[}8{]}
\end{enumerate}
\item Making backups and archives are performed to prevent the loss of data.

Explain the difference between a backup and an archive.\hfill{} {[}2{]}
\end{enumerate}