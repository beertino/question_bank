\item \textbf{{[}PJC/PRELIM/9597/2015/P2/Q1{]} }

PJ Parcels is a company that specialises in the delivery and collection
of parcels for business customers. To use their services, a customer
must first register for an account. The customer needs to provide
their company name, address, telephone number and the name of a main
contact for any queries. 

As part of the registration process, the customer will have to decide
if they wish to pay monthly on receipt of an invoice, or via credit
card for each delivery made. If paying by credit card, then the card
details are also required. Once these details have been accepted,
the customer will be issued with an account number that they must
quote when contacting the company. 

When a customer requires a parcel to be delivered, they will contact
PJ Parcels to arrange collection. The customer needs to provide details
of where the parcel will be collected from; where it will be delivered
to; how many parcels are to be collected and which type of service
they want, for example, next day delivery. 

Once the collection has been arranged, an Airway Bill will be generated.
The details on this will be used by a Dispatcher to schedule the vehicle
needed for the collection. Each parcel will be given a priority number
by the Dispatcher and those with the highest priority will be collected
first. 

By 12 noon each day, the Dispatcher also needs to generate a delivery
schedule to ensure all the parcels are delivered according to the
service required.

Each Driver has a mobile device with a copy of the Airway Bill; a
person at the delivery address must sign this to say that the parcel
has been delivered. This will flag that the delivery has been completed. 

Once the parcel has been delivered, if the customer pays via credit
card, their card will be debited by the amount required, or if they
pay monthly, then the invoice account will be debited. Once a month,
the Finance Department will generate the invoices for payment. 

If the parcel cannot be delivered for any reason, it will be returned
to the Depot and a card will be left with at the delivery address
with details of how to arrange re-delivery.

The company has decided to replace this manual system with an on-line
computerised system. 

A \textbf{system developer} is employed to carry out the task. The
first task assigned to the system developer is to write a project
proposal. 
\begin{enumerate}
\item One section of the project proposal is the Problem Statement which
lists the problems in the current system. Write the Problem Statement.
\hfill{}{[}4{]}
\item The system developer (who will act as project manager) has drawn up
an initial plan of the work involved: 
\noindent \begin{center}
\begin{tabular}{|c|l|c|}
\hline 
Stage & Activity & Weeks\tabularnewline
\hline 
A & elicit requirements from the intended users, and draw up a specification & 3\tabularnewline
\hline 
B & system analysis & 2\tabularnewline
\hline 
C & system design & 7\tabularnewline
\hline 
D & system development & 5\tabularnewline
\hline 
E & system testing & 4\tabularnewline
\hline 
F & implementation of computer system & 2\tabularnewline
\hline 
G & documentation & 3\tabularnewline
\hline 
H & maintenance & 2\tabularnewline
\hline 
\end{tabular}
\par\end{center}

Task B must follow A. 

Tasks C, D and E can run concurrently, but must follow B. 

Tasks F and G can run concurrently, but cannot start until all three
tasks C, D and E have been completed. 

Task H must follow tasks F and G. 
\begin{enumerate}
\item Draw a Program Evaluation and Review Technique (PERT) chart for this
project. Use week numbers as the time units. \hfill{}{[}4{]}
\item Copy the following table and complete the earliest and latest start
and end time, and the float, for each node. 
\noindent \begin{center}
\begin{tabular}{|c|c|c|c|c|c|c|}
\hline 
Task & Duration & Earliest Start Time & Latest Start Time  & Earliest Finish Time & Latest Finish Time & Float\tabularnewline
\hline 
A & 3 & 0 & 0 & 3 & 3 & 0\tabularnewline
\hline 
B & 2 & 3 & 3 & 5 & 5 & 0\tabularnewline
\hline 
C & 7 &  &  &  &  & \tabularnewline
\hline 
D & 5 &  &  &  &  & \tabularnewline
\hline 
E & 4 &  &  &  &  & \tabularnewline
\hline 
F & 2 &  &  &  &  & 1\tabularnewline
\hline 
G & 3 & 12 &  &  & 15 & 0\tabularnewline
\hline 
H & 2 &  & 15 &  & 17 & 0\tabularnewline
\hline 
\end{tabular}
\par\end{center}
\item State the critical path. \hfill{}{[}1{]}
\item State the minimum time in which the project could be completed. \hfill{}
{[}1{]}
\item Explain dependent stages and concurrent stages. For each type of stage
give an example from this chart. \hfill{}{[}4{]}
\item Draw a Gantt chart showing all eight stages and their dependencies,
allowing for the resource allocations as indicated above. \hfill{}
{[}4{]}
\end{enumerate}
\item Identify \textbf{five} key stages with brief description of the software
development life cycle (SDLC). \hfill{}{[}5{]}
\item Explain at which stage of the SDLC was top-down analysis used, and
why it helps in the solution of complex problems. \hfill{} {[}2{]}
\item The parcel\textquoteright s data from customers entered into the new
system needs to be validated and verified. 

Explain with examples the difference between data validation and data
verification. \hfill{}{[}4{]}
\item Jane is the software testing engineer for this system. Her test strategy
includes beta testing and acceptance testing. 
\begin{enumerate}
\item Describe what is meant by beta testing and how it can be used to test
the program. \hfill{} {[}2{]}
\item Describe what is meant by acceptance testing and how it can be used
to test the program. \hfill{} {[}2{]}
\end{enumerate}
\item Some account clerks spend at least part of their week working from
home. A system analyst will assist in improving their company communication
systems. Explain why it is important to define problem accurately.
\hfill{}{[}2{]}
\item Some customers are worried because so much information is being stored
about their parcels on the server of the company. Describe the fears
that the customers may have and explain what the company can do to
allay those fears. \hfill{} {[}3{]}
\item When data is transmitted across the network it is sent in bytes. The
following bytes of data have been received by a device on the network.
\noindent \begin{center}
01101101 10110100 01101000 10100001 
\par\end{center}

One of the bytes has been corrupted.

State which is the corrupt byte, justifying your choice. \hfill{}
{[}2{]}
\item Explain how transmitting bytes in \textbf{blocks} can allow the receiving
device to selfcorrect errors. \hfill{}{[}2{]}
\item When data is transmitted on a network it can use a number of different
transmission modes. State what is meant by each of the following modes
of data transmission. 
\begin{enumerate}
\item Simplex \hfill{}{[}1{]}
\item Duplex \hfill{} {[}1{]}
\item Half-duplex \hfill{} {[}1{]}
\end{enumerate}
\end{enumerate}