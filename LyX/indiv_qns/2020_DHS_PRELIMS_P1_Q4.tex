\item \textbf{{[}DHS/PRELIM/9569/2020/P1/Q4{]} }

The following diagram shows the contents with some data inserted. 
\begin{enumerate}
\item State two possible insertion orders of data to this BST.\hfill{}
{[}2{]}
\item Generalise how data can be inserted to produce a balanced BST.\hfill{}
{[}3{]}
\item The BST is to be implemented using a 1D array T. Write pseudocode
to show how data can be represented in T with suitable initial values
for an empty BST. \hfill{}{[}4{]}
\item Devise a recursive algorithm to insert to this BST. \hfill{}{[}5{]}
\item Devise a recursive algorithm to check if T is a BST. \hfill{}{[}3{]}
\item Convert the recursive algorithm in part (e) to an iterative one using
a suitable data structure which you should name and justify. \hfill{}{[}5{]}
\item Devise an algorithm to output the items in T that are within a given
range {[}a, b{]} inclusive in ascending order.\hfill{} {[}4{]}
\item Devise an algorithm to output the contents of the leaves of T in descending
order.\hfill{} {[}4{]}
\item Despite its memory overhead, why is recursion often used in BSTs?
\hfill{}{[}3{]}
\item Why is recursion less often used in linked lists?\hfill{} {[}2{]}
\end{enumerate}