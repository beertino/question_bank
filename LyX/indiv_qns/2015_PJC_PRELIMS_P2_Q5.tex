\item \textbf{{[}PJC/PRELIM/9597/2015/P2/Q5{]} }

PJC Enterprise plans to create a computer system to store data on
its employees and the insurance type and coverage for each of them.
An employee may be insured by more than one policy. A solution is
to create a database with three tables: \emph{Employee}, \emph{Insurance}
and \emph{Policy}. 

\emph{Employee} table contains information about its employees. \emph{Insurance}
table gives information on the insurance plan type and the date of
issue of the policy for an employee. \emph{Policy} table contains,
for each type of insurance plan, a description of its coverage and
its monthly cost. 
\begin{enumerate}
\item Draw a fully labelled ER diagram (with attributes) to show how the
tables Employee, Insurance, Policy are related, while keeping data
redundancy to a minimum. \hfill{}{[}5{]}
\item Using examples taken from this application explain what is meant by:
\begin{enumerate}
\item a primary key \hfill{}{[}1{]}
\item a foreign key \hfill{} {[}1{]}
\item a composite key \hfill{}{[}1{]}
\end{enumerate}
\item Write a SQL query to find the monthly cost of Jessie Tan\textquoteright s
insurance. \hfill{}{[}2{]}
\item Explain why using a database for PJC Enterprise, rather than flat
files, results in: 
\begin{enumerate}
\item improved data consistency \hfill{}{[}2{]}
\item prevention of data redundancy \hfill{}{[}2{]}
\item data independence \hfill{} {[}2{]}
\end{enumerate}
\item Explain how the following tools can help employees of PJC Enterprise
make use of data in a database.
\begin{enumerate}
\item query language\hfill{} {[}2{]}
\item report generator \hfill{}{[}2{]}
\end{enumerate}
\end{enumerate}