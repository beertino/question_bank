\item \textbf{{[}PJC/PRELIM/9597/2016/P2/Q8{]} }

Consider the following program in pseudocode, which includes a recursive
function that calculates the power of an integer. 

\noindent %
\noindent\begin{minipage}[t]{1\columnwidth}%
\texttt{010 PROGRAM }

\texttt{020}

\texttt{030 FUNCTION Power(Base: INTEGER, Exponent: INTEGER) RETURN }

\texttt{040 INTEGER }

\texttt{050 \qquad{}IF Exponent = 0 }

\texttt{060 \qquad{}\qquad{}THEN }

\texttt{070 \qquad{}\qquad{}\qquad{}Result <- 1}

\texttt{080 \qquad{}\qquad{}ELSE }

\texttt{090 \qquad{}\qquad{}\qquad{}Result <- Base {*} Power(Base,
Exponent \textendash{} 1)}

\texttt{100 \qquad{}ENDIF }

\texttt{110 \qquad{}RETURN Result }

\texttt{120 ENDFUNCTION}

\texttt{130 }

\texttt{140~// main program}

\texttt{150 DECLARE Answer: INTEGER, Base: INTEGER, Exponent: INTEGER }

\texttt{160 INPUT Base }

\texttt{170 INPUT Exponent }

\texttt{180 Answer <- Power(Base, Exponent) }

\texttt{~~~~OUTPUT Answer }%
\end{minipage}
\begin{enumerate}
\item Trace the execution of the function call \texttt{Power(2,3)}, showing
for each re-entry into the \texttt{Power} function, the values passed
to the function and the results returned.\hfill{} {[}3{]}
\end{enumerate}
The program is executed, starting from line 140.
\begin{enumerate}
\item[(b)]  Explain how the stack content changes during the execution of the
program, with input of 2 for \texttt{Base} (line 150), and 3 for \texttt{Exponent}
(line 160).\hfill{} {[}4{]}
\item[(c)]  Write a pseudocode for a non-recursive version of the \texttt{Power}
function. \hfill{}{[}3{]}
\item[(d)]  State one reason why a non-recursive \texttt{Power} function may
be preferred to a recursive one. \hfill{}{[}1{]}
\end{enumerate}