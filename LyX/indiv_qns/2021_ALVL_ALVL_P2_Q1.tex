\item \textbf{{[}ALVL/9569/2021/P2/Q1{]} }

The IMEI number is a unique value used to identify mobile devices,
such as phones and tablets. The IMEI number has 15 digits: 14 digits
with an extra check digit added to the right-hand Side. 

The check digit is calculated using the following algorithm, on the
left-most 14 digits of an IMEI number: 
\begin{enumerate}
\item starting from the right, the first digit is location number 1 
\item double all digits in the odd numbered positions 
\item sum all the digits, including both the unchanged digits (i.e. those
in the even numbered positions) as well as those doubled (eg. 16 contributes
1 + 6) 
\item the check digit is the value between 0 and 9 that must be added to
the sum to make the result exactly divisible by 10. 
\end{enumerate}
For example, given the 14 digits 14576567654934:

\begin{tabular}{ccccccccccccccc}
{\footnotesize{}Step 1:} & {\footnotesize{}1} & {\footnotesize{}4} & {\footnotesize{}5} & {\footnotesize{}7} & {\footnotesize{}6} & {\footnotesize{}5} & {\footnotesize{}6} & {\footnotesize{}7} & {\footnotesize{}6} & {\footnotesize{}5} & {\footnotesize{}4} & {\footnotesize{}9} & {\footnotesize{}3} & {\footnotesize{}4}\tabularnewline
{\footnotesize{}Step 2:} & {\footnotesize{}1} & {\footnotesize{}8} & {\footnotesize{}5} & {\footnotesize{}14} & {\footnotesize{}6} & {\footnotesize{}10} & {\footnotesize{}6} & {\footnotesize{}14} & {\footnotesize{}6} & {\footnotesize{}10} & {\footnotesize{}4} & {\footnotesize{}18} & {\footnotesize{}3} & {\footnotesize{}8}\tabularnewline
{\footnotesize{}Step 3:} & {\footnotesize{}$1+$} & {\footnotesize{}$(8)+$} & {\footnotesize{}$5+$} & {\footnotesize{}$(1+4)+$} & {\footnotesize{}$6+$} & {\footnotesize{}$(1+0)+$} & {\footnotesize{}$6+$} & {\footnotesize{}$(1+4)+$} & {\footnotesize{}$6+$} & {\footnotesize{}$(1+0)+$} & {\footnotesize{}$4+$} & {\footnotesize{}$(1+8)+$} & {\footnotesize{}$3+$} & {\footnotesize{}$(8)$}\tabularnewline
\end{tabular}

\begin{tabular}{cc}
 & {\footnotesize{}Sum = 68}\tabularnewline
{\footnotesize{}Step 4:} & {\footnotesize{}Check digit = 2}\tabularnewline
\end{tabular}

For each of the sub-tasks, add a comment statement at the beginning
of the code using the hash symbol \textquoteleft \#' to indicate the
sub-task the program code belongs to, for example:

\begin{singlespace}
\noindent \texttt{}%
\begin{tabular}{c|lcccccccccccccccccccccc|}
\cline{2-24} \cline{3-24} \cline{4-24} \cline{5-24} \cline{6-24} \cline{7-24} \cline{8-24} \cline{9-24} \cline{10-24} \cline{11-24} \cline{12-24} \cline{13-24} \cline{14-24} \cline{15-24} \cline{16-24} \cline{17-24} \cline{18-24} \cline{19-24} \cline{20-24} \cline{21-24} \cline{22-24} \cline{23-24} \cline{24-24} 
\texttt{In {[}1{]} :} & \texttt{\#Task 1.1} &  &  &  &  &  &  &  &  &  &  &  &  &  &  &  &  &  &  &  &  &  & \tabularnewline
 & \texttt{Program Code} &  &  &  &  &  &  &  &  &  &  &  &  &  &  &  &  &  &  &  &  &  & \tabularnewline
\cline{2-24} \cline{3-24} \cline{4-24} \cline{5-24} \cline{6-24} \cline{7-24} \cline{8-24} \cline{9-24} \cline{10-24} \cline{11-24} \cline{12-24} \cline{13-24} \cline{14-24} \cline{15-24} \cline{16-24} \cline{17-24} \cline{18-24} \cline{19-24} \cline{20-24} \cline{21-24} \cline{22-24} \cline{23-24} \cline{24-24} 
\multicolumn{1}{c}{} & \texttt{Output:} &  &  &  &  &  &  &  &  &  &  &  &  &  &  &  &  &  &  &  &  &  & \multicolumn{1}{c}{}\tabularnewline
\end{tabular}
\end{singlespace}
\begin{singlespace}

\subsection*{Task 1.1}
\end{singlespace}

Write a function \texttt{task1\_1(input\_value)} that returns an integer.

The function should:
\begin{itemize}
\item validate that the parameter \texttt{input\_value} is either an integer,
or a string containing a valid integer
\item check that it is 14 digits in length
\item return $-1$ if the value received is invalid for any reason
\item calculate the check digit for the given first 14 digits of the IMEI
number
\item return the calculated check digit. \hfill{}{[}6{]}
\end{itemize}
\begin{singlespace}

\subsection*{Task 1.2}
\end{singlespace}

Create \textbf{four} tests that should fully test your function. Ensure
that it validates the inputs accurately and returns the correct expected
result. 

Each of the four tests will be a pair of data items: the input value
and its expected result.

Your four input values should be: 
\begin{itemize}
\item a string containing just a valid integer 
\item a valid integer 
\item a string containing characters that are not numbers 
\item a value of the incorrect length
\end{itemize}
Test your function with your four input values by calling it usrng
the followmg statement:

\texttt{\qquad{}print (task1\_1(input\_value) == expected) }

The four statements should all print \texttt{True}. For example:

\texttt{\qquad{}print(task1\_1(\textquotedbl 14576567654934\textquotedbl )
== 2)}

\textbf{Do not} use the example provided but create your own.\hfill{}{[}4{]}

\subsection*{Task 1.3 }

A full 15-digit IMEI number can be validated by removing the check
digit, calculating the check digit from the remaining 14 digits and
comparing it to the removed digit. 

Write a function \texttt{task1\_3(input\_value)} that returns a Boolean.
The function should: 
\begin{itemize}
\item validate that the parameter \texttt{input\_value} is either an integer,
or a string containing a valid integer 
\item check that it is 15 digits in length 
\item return \texttt{False} if the value received is invalid for any reason 
\item remove the digit on the right-hand side 
\item use your function from Task 1.1 to calculate the check digit from
the remaining 14 digits 
\item return a Boolean representing the comparison between the calculated
check digit and the original removed digit.\hfill{}{[}4{]}
\end{itemize}
Test your function with four input values (these may be based on those
from Task 1.2 or otherwise), two should be correct IMEI numbers, two
should be in error. 

If a value of \texttt{True} is expected, use the following statement: 

\texttt{\qquad{}print(taskl\_3(input\_value)) }

If a value of False is expected to be returned by the comparison,
this can be changed to still output True by using the not keyword,
for example: 

\texttt{\qquad{}print(not task1\_3(\textquotedbl 12345r\textquotedbl )) }

The four statements should all print True. \hfill{}{[}4{]}

Save your Jupyter notebook for Task 1.