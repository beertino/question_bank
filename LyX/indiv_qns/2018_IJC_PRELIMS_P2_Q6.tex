\item \textbf{{[}IJC/PRELIM/9597/2018/P2/Q6{]} }

The Singapore Bowling Federation (SBF) is made up of several affiliate
clubs. To participate in a competition, a competitive bowler is required
to be enrolled in exactly one affiliate club. A relational database
is used by SBF to store data about competition entries and results.
Four tables present in the database are \texttt{CLUB}, \texttt{MEMBER},
\texttt{COMPETITION} and \texttt{COMPETITION-MEMBER}. A new row is
created in the \texttt{COMPETITION-MEMBER} table whenever a competitive
bowler registers for a competition. When the competition results become
available, they are added to the appropriate row.

Each competitive bowler, affiliate club and competition has a unique
identification number.
\begin{enumerate}
\item Explain what a relational database is.\hfill{} {[}2{]}
\item Draw the Entity-Relationship (E-R) diagram to show the relationship
between the four tables that provide for a fully normalised database
design. \hfill{}{[}4{]}
\item A table description can be expressed as:

\texttt{Tablename (}\texttt{\uline{Attribute1}}\texttt{, Attribute2,
Attribute3, \dots ) }

The primary key is indicated by underlining one or more attributes.

Write the table descriptions for the four tables. You should include
at least \textbf{one} other attribute in addition to the primary key.
\hfill{}{[}8{]}
\end{enumerate}
(d) Give \textbf{two} examples of data anomaly and describe how they
may occur if the database that was designed for SBF was not fully
normalised.\hfill{} {[}4{]}