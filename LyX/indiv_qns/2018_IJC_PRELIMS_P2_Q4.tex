\item \textbf{{[}IJC/PRELIM/9597/2018/P2/Q4{]} }

A large file is to be transmitted between two computers over the internet.
\begin{enumerate}
\item Explain how packet switching is used to transmit the file over the
internet. \hfill{}{[}3{]}
\item The bytes that make up a packet is checked by the receiving computer.
Explain how this can be done for:
\begin{enumerate}
\item[I. ] The individual bytes which make up the packet \hfill{}{[}2{]}
\item[II.]  The collection of bytes which makes up the packet. \hfill{}{[}2{]}
\end{enumerate}
\end{enumerate}
Each packet consists of: 
\begin{itemize}
\item A 3-digit number from 100 to 800 
\item Upper case letters 
\item The <space> character 
\item A start character (\$) and an end character (\$). 
\end{itemize}
An example of a packet is:
\noindent \begin{center}
\texttt{\$TBHIKR 565 KUTGW\$ }
\par\end{center}
\begin{enumerate}
\item[(c)]  Explain the difference between data validation and data verification.
\hfill{}{[}2{]}
\item[(d)]  Describe \textbf{three} validation checks that the receiving computer
should perform on each packet received. \hfill{}{[}6{]}
\end{enumerate}