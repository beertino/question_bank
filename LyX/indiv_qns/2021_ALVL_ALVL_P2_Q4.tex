\item \textbf{{[}ALVL/9569/2021/P2/Q4{]} }

A competition has three qualifying rounds. Competitors could score
up to 100 points in each round. Each competitor has a unique ID number.
The competitor's ID, name, and the scores for ac round are held in
a database, \texttt{Task4.db}, provided with this question. 

There are two tables: 
\begin{itemize}
\item \texttt{competitor(}\texttt{\uline{id}}\texttt{, name) }
\item \texttt{scores(}\texttt{\uline{id}}\texttt{, }\texttt{\uline{round}}\texttt{,
score)} 
\end{itemize}
Not every competitor competed in all three rounds, but they did all
compete in round 1. 

\subsection*{Task 4.1 }

Write a Python program and the necessary files to create a web application.
The web application offers the following menu options: 
\noindent \begin{center}
\begin{tabular}{|c|}
\hline 
Round 1 Scores\tabularnewline
Round 2 Scores\tabularnewline
Round 3 Scores\tabularnewline
Mean Scores\tabularnewline
Qualifiers\tabularnewline
\hline 
\end{tabular}
\par\end{center}

Round 1 Scores Round 2 Scores Round 3 Scores Mean Scores Qualifiers 

Save your program code as 

\texttt{Task4\_1\_<your name>\_<centre number>\_<index\_\_number>.py }

With any additional files/subfolders as needed in a folder named 

\texttt{Task4\_l\_<your name>\_<centre number>\_\_<index\_number> }

Run the web application and save the output of the program as 

\texttt{Task4\_1\_\_<your name>\_<centre number>\_<index\_number>.html}
\hfill{}{[}3{]}

\subsection*{Task 4.2}

Write an SQL query that, for 3 given round number, shows:
\begin{itemize}
\item competitor names and their scores for that particular round
\item only those who had a score for that round
\item the scores sorted in descending order.
\end{itemize}
The results of the query should be shown on a web page in a table
that:
\begin{itemize}
\item lists the name of each competitor and their score 
\item has the results shown in descending order of the competitor's score.
\end{itemize}
The resulting web page for each round should be accessed from the
corresponding menu option from Task 4.1.

Save all your SQL code as

\texttt{Task4\_2\_<your name>\_<centre number>\_<index\_number>.sql}

With any additional files/subfolders as needed in a folder named

\texttt{Task4\_2\_<your name>\_<centre number>\_<index\_number>} \hfill{}{[}5{]}

Run the web application and save the output of the program as

\texttt{Task4\_2\_<your name>\_<centre number>\_<index\_number>\_1.html}

\texttt{Task4\_2\_<your name>\_<centre number>\_<index\_\_number>\_2.html}

\texttt{Task4\_2\_<your name>\_<centre number>\_<index\_number>\_3.html}
\hfill{}{[}2{]}

\subsection*{Task 4.3 }

Write a SQL query that shows:
\begin{itemize}
\item competitor name
\item the mean score for each competitor, based on the number of rounds
in which they competed. 
\end{itemize}
The query's results should be Shown on a web page in a table that:
\begin{itemize}
\item lists the name of each competitor and their mean score (to 2 decimal
places)
\item has the results shown in ascending alphabetical order of the competitor's
name.
\end{itemize}
The resulting web page should be accessed from the corresponding menu
option from Task 4.1. 

Save your SQL code as

\texttt{Task4\_3\_<your name>\_<centre number>\_<index\_number>.sql}

With any additional files/subfolders as needed in a folder named

\texttt{Task4\_3\_<your name>\_<centre number>\_<index\_number>} \hfill{}{[}6{]}

Run the web application and save the output of the program as

\texttt{Task4\_3\_<your name>\_<centre number>\_<index\_number>.html}
\hfill{}{[}2{]}

\subsection*{Task 4.4 }

In order to qualify for the final of the competition, competitors
need to score over 250, in total, in the first three rounds. 

Write SQL query that shows:
\begin{itemize}
\item competitor\textquoteright s name
\item their total score
\item whether that competitor has qualified for the final.
\end{itemize}
The results of the query should be shown on a web page in a table
that:
\begin{itemize}
\item lists the names of the competitors, their scores and whether they
qualified
\item has the results shown in descending order of the competitor's score.
\end{itemize}
The resulting web page should be accessed from the corresponding menu
option from Task 4.1.

Save your SQL code as

\texttt{Task4\_4\_<your name>\_<centre number>\_<index\_number>.sql}

With any additional files/subfolders as needed in a folder named

\texttt{Task4\_4\_<your name>\_<centre number>\_<index\_number>} \hfill{}{[}5{]}

Run the web application and save the output of the program as

\texttt{Task4\_4\_<your name>\_<centre number>\_<index\_number>.html}
\hfill{}{[}2{]}