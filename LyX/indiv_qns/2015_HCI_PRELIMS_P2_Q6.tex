\item \textbf{{[}HCI/PRELIM/9597/2015/P2/Q6{]} }

A systems analyst is developing a new computerized admission system
for HC University. 

The project manager identifies the following activities with their
durations and precedence relations: 
\noindent \begin{center}
\begin{tabular}{|c|c|c|}
\hline 
Task & Predecessors & Time(Weeks)\tabularnewline
\hline 
A & - & 3\tabularnewline
\hline 
B & - & 5\tabularnewline
\hline 
C & - & 7\tabularnewline
\hline 
D & A & 8\tabularnewline
\hline 
E & B & 5\tabularnewline
\hline 
F & C & 5\tabularnewline
\hline 
G & E & 4\tabularnewline
\hline 
H & F & 5\tabularnewline
\hline 
I & D & 6\tabularnewline
\hline 
J & G, H & 4\tabularnewline
\hline 
\end{tabular}
\par\end{center}
\begin{enumerate}
\item {}
\begin{enumerate}
\item Draw a Program Evaluation and Review Technique (PERT) chart, show
clearly the early start and late start time of each task, showing
dummy tasks, where necessary. \hfill{}{[}4{]}
\item Explain dependent stages and concurrent stages, giving examples from
your chart. \hfill{}{[}2{]}
\item State the critical path and the minimum time in which the project
can be completed. \hfill{}{[}2{]}
\end{enumerate}
\item Produce a Gantt chart based on the above information. \hfill{}{[}3{]}
\item Give \textbf{one} reason why a Gantt chart may be preferred over a
PERT chart. \hfill{}{[}1{]}
\end{enumerate}
In the current system,
\begin{itemize}
\item Each new student sends a completed form that has their name, date
of birth and the courses that they wish to enroll on. 
\item The date of birth is checked to see whether the student is of the
correct age range for admission to the college. 
\item If the student is too young or too old, a standard rejection letter
is produced. 
\item If the student is of the correct age then each of the courses that
the student has identified are checked on the course file to see whether
they are full or not. 
\item If there is room on a course then the student name is added to the
appropriate course record on the course file. 
\item A standard letter is produced with details of which course(s) the
student has been enrolled on. 
\end{itemize}
\begin{enumerate}
\item[(d)]  
\begin{enumerate}
\item Draw a data flow diagram (DFD) for the current system. \hfill{}{[}6{]}
\item Using examples from your DFD, explain how the diagram helps to inform
a database solution for the new computerized system. \hfill{}{[}4{]}
\item Give \textbf{two} parts of the database design that is not possible
from the DFD. \hfill{}{[}2{]}
\end{enumerate}
\end{enumerate}