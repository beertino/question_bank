\begin{onehalfspace}
\item \textbf{{[}ALVL/9569/2020/P2/Q4{]} }
\end{onehalfspace}

\begin{onehalfspace}
\noindent A school has usd a text file to store data collected about
people who work at the school and students who attend the school.
People who have a teaching role at the school are referred to as 'staff'.
The school decides to transfer this information into a database.

\noindent A web page will then be used to summarise the data. Different
information will be visible on the web page, depending on the type
of person displayed.
\end{onehalfspace}
\begin{onehalfspace}

\subsubsection*{Task 4.1}
\end{onehalfspace}

\begin{onehalfspace}
\noindent Create an SQL file called \texttt{TASK4\_1\_<your name>\_<center
number>\_<index number>.sql} to show the SQL code to create database
\texttt{school.db} with the single table, \texttt{People}.

\noindent The table will have the following fields of the given SQLite
types:
\end{onehalfspace}
\begin{itemize}
\begin{onehalfspace}
\item \texttt{PersonID} - primary key, an auto-incremented integer
\item \texttt{FullName} - the full name of the person, text
\item \texttt{DateOfBirth} - the person's date of birth, text
\item \texttt{ScreenName} - the person's screen name, text
\item \texttt{IsAdult} - a Boolean using 0 for False adn 1 for True, integer.
\end{onehalfspace}
\end{itemize}
\begin{onehalfspace}
\noindent Save your SQL code as

\noindent \texttt{TASK4\_1\_<your name>\_<center number>\_<index number>.sql}\hfill{}{[}4{]}
\end{onehalfspace}
\begin{onehalfspace}

\subsubsection*{Task 4.2}
\end{onehalfspace}

\begin{onehalfspace}
\noindent The school wants to use the Python programming language
and object-oriented programming to help publish the database content
on a web page.

\noindent The class \texttt{Person} will store the following data:
\end{onehalfspace}
\begin{itemize}
\begin{onehalfspace}
\item \texttt{full\_name} - stored as a string
\item \texttt{date\_of\_birth} - initiliased with a string with the format
YYYY-MM--DD
\end{onehalfspace}
\end{itemize}
\begin{onehalfspace}
\noindent The class has two methods defined on it:
\end{onehalfspace}
\begin{itemize}
\begin{onehalfspace}
\item \texttt{is\_adult()} - returns a Boolean value to indicate whether
the person is an adult or not. It:
\end{onehalfspace}
\begin{itemize}
\begin{onehalfspace}
\item subtracts the year of the \texttt{date\_of\_birth} from the year of
today's date
\item returns True if the result is greater than 18, otherwise returns False.
\end{onehalfspace}
\end{itemize}
\begin{onehalfspace}
\item \texttt{screen\_name()} - returns a string which creates an identifier
to be used as a screen name, which should be construted as follows:
\end{onehalfspace}
\begin{itemize}
\begin{onehalfspace}
\item the full name with all spaces and punctuation removed
\item followed by the two-digit month of their birth
\item then the two-digit day of their birth.
\end{onehalfspace}
\end{itemize}
\begin{onehalfspace}
\noindent For example, John Tan, born on the $1^{\text{st}}$ of June
2000 (``\texttt{2000-06-01}''), would have the screen name ``\texttt{JohnTan0601}''
\end{onehalfspace}
\end{itemize}
\begin{onehalfspace}
\noindent Save your program code as

\noindent \texttt{TASK4\_2\_<your name>\_<centre number>\_<index number>.py}\hfill{}{[}7{]}

\noindent The \texttt{Staff} class inherits from \texttt{Person},
such that:
\end{onehalfspace}
\begin{itemize}
\begin{onehalfspace}
\item \texttt{screen\_name()} should be the name followed by ``\texttt{Staff}''
\item \texttt{is\_adult()} always returns True.
\end{onehalfspace}
\end{itemize}
\begin{onehalfspace}
\noindent The \texttt{Student} class inherits from \texttt{Person},
such that the \texttt{is\_adult()} method always returns False.

\noindent Add your program code to

\noindent \texttt{TASK4\_2\_<your name>\_<centre number>\_<index number>.py}\hfill{}{[}4{]}

\noindent The text file, \texttt{people.txt}, contains data items
for a number of people. Each data item is separated by a comma, with
each person's data on a new line as follows:
\end{onehalfspace}
\begin{itemize}
\begin{onehalfspace}
\item full name
\item date of birth in the form YYYY-MM-DD
\item a string indicating whether the person is ``\texttt{Staff}'', ``\texttt{Student}''
or ``\texttt{Person}''.
\end{onehalfspace}
\end{itemize}
\begin{onehalfspace}
\noindent Write program code to read in the information from the text
file, \texttt{people.txt}, creating instance of the appropriate class
for each person (either \texttt{Staff}, \texttt{Student} or \texttt{Person}).\hfill{}{[}4{]}

\noindent Write program code to insert all information from the file
into the \texttt{school.db} database.

\noindent Run the program.

\noindent Add your program code to

\noindent \texttt{TASK4\_2\_<your name>\_<centre number>\_<index number>.py}\hfill{}{[}8{]}
\end{onehalfspace}
\begin{onehalfspace}

\subsubsection*{Task 4.3}
\end{onehalfspace}

The screen names of the people in the text file, \texttt{people.txt},
are to be displayed in a web browser.

Write a Python program and the necessary files to create a web application
that enables the list of people to be displayed.

For each record the web page should include the:
\begin{itemize}
\item full name
\item screen name
\item identity as student, staff or person.
\end{itemize}
Save your program as

\begin{onehalfspace}
\noindent \texttt{TASK4\_3\_<your name>\_<centre number>\_<index number>.py}
\end{onehalfspace}

with any additional files / sub-folders as needed in a folder named

\begin{onehalfspace}
\noindent \texttt{TASK4\_3\_<your name>\_<centre number>\_<index number>}\hfill{}{[}5{]}
\end{onehalfspace}

Run the web application and save the output of the program as 

\begin{onehalfspace}
\noindent \texttt{TASK4\_2\_<your name>\_<centre number>\_<index number>.html}\hfill{}{[}3{]}
\end{onehalfspace}