\item \textbf{{[}PJC/PRELIM/9597/2015/P1/Q1{]} }

\texttt{Rainfall\_mth.csv} is a file which contains monthly total
rainfall (in millimetres), for years 1984 to 2014. The format of the
record is \textquotedblleft \textbf{\emph{{[}Year{]} M {[}Month{]}
, {[}rainfall in millimetres{]}}}\textquotedblright . Three sample
records are: 
\begin{itemize}
\item \textquotedblleft \texttt{1984M01, 251.2}\textquotedblright , which
means total rainfall for January 1984 is 251.2; 
\item \textquotedblleft \texttt{1999M07, 225.4}\textquotedblright , which
means total rainfall for July 1999 is 225.4; 
\item \textquotedblleft \texttt{2014M11, 250.8}\textquotedblright , which
means total rainfall for November 2014 is 250.8.
\end{itemize}
The total annual rainfall for a particular year can be calculated
by adding the monthly total rainfall for the twelve months. 

\subsection*{Task 1.1 }

Write program code to find total annual rainfall from 1984 to 2014
and display in a table with a heading and borders as follows: 
\noindent \begin{center}
\begin{tabular}{|c|c|}
\hline 
Year & Total Annual Rainfall (millimetres in 1 d.p.)\tabularnewline
\hline 
1984 & 2686.7\tabularnewline
\hline 
1985 & 1483.9\tabularnewline
\hline 
1986 & 2536.1\tabularnewline
\hline 
... & ... ...\tabularnewline
\hline 
... & ... ...\tabularnewline
\hline 
... & ... ...\tabularnewline
\hline 
2014 & 1538.1\tabularnewline
\hline 
\end{tabular}
\par\end{center}

\subsection*{Evidence 1: }

The program code. \hfill{}{[}7{]}

\subsection*{Evidence 2:}

Screenshot to display total annual rainfall. \hfill{}{[}1{]}

\texttt{Rainfall\_day.csv} is another file which contains number of
rainy days in a month, for years 2009 to 2014. The format of the record
is \textquotedblleft {[}Year{]} M {[}Month{]} , {[}number of days{]}\textquotedblright .
Three sample record are: 
\begin{itemize}
\item \texttt{\textquotedblleft 2009M03, 19\textquotedblright }, which means
there are 19 rainy days in March 2009;
\item \texttt{\textquotedblleft 2011M05, 15\textquotedblright }, which means
there are 15 rainy days in May 2011; 
\item \texttt{\textquotedblleft 2014M09, 9\textquotedblright }, which means
there are 9 rainy days in September 2014. 
\end{itemize}
The average rainfall for a rainy day for a particular year can be
calculated by dividing the total annual rainfall by the total number
of rainy days in a year. 

\subsection*{Task 1.2 }

Amend your program code using the following specifications: 
\begin{itemize}
\item Allow user to input a year from 2009 to 2014
\item Output error message if input is outside these years -- 2009 to 2014
\item Calculate the average rainfall for a rainy day for that year
\item Output the result 
\end{itemize}

\subsection*{Evidence 3:}

Your amended program code. \hfill{}{[}5{]}

\subsection*{Evidence 4: }

Screenshots that show 2 test cases from running Task 1.2. \hfill{}{[}2{]}