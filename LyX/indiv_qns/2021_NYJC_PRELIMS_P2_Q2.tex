\item \textbf{{[}NYJC/PRELIM/9569/2021/P2/Q2{]} }

Name your Jupyter Notebook as 

\texttt{TASK2\_<your name>\_<centre number>\_<index number>.ipynb }

The task is to implement a todo list using a linkedlist data structure.
For each of the sub-tasks, add a comment statement, at the beginning
of the code using the hash symbol \textquoteleft \texttt{\#}\textquoteright ,
to indicate the sub-task the program code belongs to, for example: 
\noindent \begin{center}
\begin{tabular}{c|l|}
\cline{2-2} 
\multirow{2}{*}{\texttt{In{[}1{]}:}} & \texttt{\# Task 2.1}\tabularnewline
 & \texttt{Program Code}\tabularnewline
\cline{2-2} 
\multicolumn{1}{c}{} & \multicolumn{1}{l}{\texttt{Output:}}\tabularnewline
\end{tabular}
\par\end{center}

\subsubsection*{Task 2.1 }

The class \texttt{TodoList} represents a LinkedList and has the following
attributes: 
\begin{itemize}
\item \texttt{\_\_head} -- a pointer to the first node of the LinkedList;
if empty, it has a value of \texttt{None} 
\item \texttt{\_\_tail} -- a pointer to the last node of the LinkedList;
if empty, it has a value of \texttt{None} 
\end{itemize}
\texttt{TodoList} has the following methods defined on it: 
\begin{itemize}
\item \texttt{add(item)} -- wraps \texttt{item} in a \texttt{TodoItem}
instance, and adds it to the end of the LinkedList 
\item \texttt{remove(item)} -- removes the first \texttt{TodoItem} containing
\texttt{item} from the LinkedList 
\item \texttt{list()} -- returns a Python list containing each \texttt{item}
in the TodoList 
\end{itemize}
The class \texttt{TodoItem} represents a Node of the LinkedList and
has the following attributes: 
\begin{itemize}
\item \texttt{title} -- a short description of the todo item 
\item \texttt{\_\_next} -- a pointer to the next node in the LinkedList;
if this is the last node, it has a value of None 
\end{itemize}
\texttt{TodoItem} has the following methods defined on it: 
\begin{itemize}
\item l\texttt{ink\_to(todoitem)} -- links this \texttt{TodoItem} instance
to \texttt{todoitem}, another instance of the TodoItem class 
\end{itemize}
Implement the above classes. \hfill{}{[}13{]}

\subsubsection*{Task 2.2 }

Add the following items to a new \texttt{TodoList}: 
\begin{itemize}
\item \textquotedblleft Buy milk\textquotedblright{} 
\item \textquotedblleft Buy flour\textquotedblright{} 
\item \textquotedblleft Buy eggs\textquotedblright{}
\item \textquotedblleft Bake cake\textquotedblright{} 
\end{itemize}
Display the contents of the \texttt{TodoList}. \hfill{}{[}7{]}

\subsubsection*{Task 2.3 }

Remove the following items from the \texttt{TodoList}: 
\begin{itemize}
\item \textquotedblleft Buy milk\textquotedblright{} 
\item \textquotedblleft Buy eggs\textquotedblright{} 
\end{itemize}
Display the contents of the \texttt{TodoList}. \hfill{}{[}3{]}

Save your Jupyter Notebook for Task 2.