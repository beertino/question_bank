\item \textbf{{[}DHS/PRELIM/9569/2020/P2/Q2{]} }

Write a program that sorts a list of student last names, but the sort
only uses the first two letters of the name. Nothing else in the name
is used for sorting. However, if two names have the same first two
letters, they should stay in the same order as in the input (this
is known as a 'stable sort'). Sorting is case sensitive based on ASCII
order (with uppercase letters sorting before lowercase letters i.e.
A<B<\dots <Z<a<b<\dots <z). 

\subsubsection*{Input }

The input file \texttt{LASTNAMES.txt} consists of a sequence of up
to 500 test cases. Each case starts with a line containing an integer
$1\le n\leq200$. After this follow $n$ last names made up of only
letters (a--z, lowercase or uppercase), one name per line. Names
have between 2 and 20 letters. Input ends when $n=0$. 

\subsubsection*{Output }

For each case, print the last names in sort-of-sorted order, one per
line. Print a blank line between cases. 

\subsubsection*{Sample Input }

\noindent %
\noindent\begin{minipage}[t]{1\columnwidth}%
\texttt{3 }

\texttt{Mozart }

\texttt{Beethoven }

\texttt{Bach }

\texttt{5 }

\texttt{Hilbert }

\texttt{Godel}

\texttt{Poincare }

\texttt{Ramanujan }

\texttt{Pochhammmer }

\texttt{0 }%
\end{minipage}

\subsubsection*{Sample Output }

\noindent %
\noindent\begin{minipage}[t]{1\columnwidth}%
\texttt{Bach }

\texttt{Beethoven }

\texttt{Mozart }

\texttt{Godel }

\texttt{Hilbert }

\texttt{Poincare }

\texttt{Pochhammmer}

\texttt{Ramanujan }%
\end{minipage}

You should make use of an appropriate data structure and one or more
suitable sorting algorithms from the syllabus. Indicate as comments
your choice of how you have adapted them or your case for designing
and implementing your own. \hfill{}{[}13{]}