\item \textbf{{[}RVHS/PRELIM/9569/2021/P1/Q8{]} }

You are also to design an Object-Oriented solution for the above-mentioned
project. Both contractor and workers are to create a \texttt{User}
account on the platform, with details such as \texttt{user\_id}, \texttt{password}
and \texttt{gender}. 

The contractors will have to register their company details such as
company \texttt{name} and \texttt{address}, while the workers need
to register their bank \texttt{account} number. 
\begin{enumerate}
\item Draw a class diagram, with base class \textbf{User}, showing: 
\begin{itemize}
\item appropriate sub-classes, 
\item inheritance, 
\item the properties required, 
\item appropriate methods, including but not limited to the \textbf{constructor}
methods, and at least \textbf{one} pair of \textquoteleft \textbf{get}\textquoteright{}
and \textquoteleft \textbf{set}\textquoteright{} methods for each
class, 
\item circle the polymorphed methods. \hfill{} {[}6{]}
\end{itemize}
\item Using the above example, state the definition of inheritance and explain
its purpose/advantage in object-oriented programing. \hfill{} {[}3{]}
\end{enumerate}
The platform hopes to expand its function to allow register of homeowner
accounts. The homeowners can view which are the workers came to their
home address for renovation work on the date/time specified by the
contractors. 
\begin{enumerate}
\item[(c)] State how this would affect the class, properties and methods in
the current example. \hfill{} {[}3{]}
\item[(d)] State how this would affect the tables, attributes and relationships
of the relational database stated in \textbf{7(d)} and \textbf{(e)}.
\hfill{} {[}3{]}
\item[(e)] Explain how NoSQL addresses shortcomings of relational databases.
\hfill{} {[}4{]}
\end{enumerate}
Some homeowners request to have access to the hourly rate and personal
contact of renovation workers. 
\begin{enumerate}
\item[(f)]  From the perspective of the company, explain to the homeowners how
such a feature is against the data protection obligations stated in
the Personal Data Protection Act (PDPA). \hfill{} {[}2{]}
\end{enumerate}