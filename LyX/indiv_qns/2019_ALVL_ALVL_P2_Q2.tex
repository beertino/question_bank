\item \textbf{{[}ALVL/9597/2019/P2/Q2{]} }

A bakery bakes bread and cakes to sell in its own shop and to other
shops throughout a city. 

Its drivers visit every shop each day, delivering that day\textquoteright s
order and collecting the order for the next day. 

Order forms are pre-printed with the name of each shop and every item
that the bakery bakes. The manager of each shop writes onto the form
the quantity of each item required. When the drivers return to the
bakery. the data from the order forms are collated to give the bakers
the total of each item to bake. 

Copies of the order forms are made and used as delivery notes for
the next day\textquoteright s deliveries. The accounts department
use the original order forms to prepare a weekly invoice for each
shop.

The bakery wants the shops to submit their orders online. 

A program is needed to determine the number of each item needed and
produce the weekly invoice for each shop.

The new program will use a relational database with three tables:
Product, Shop and Order. 

Each product has a description. price. and a unique product ID number. 

Each shop has a name. an address, telephone number. manager's name,
and a unique shop iD number. 

Each order has a product lD, a quantity, a shop ID and a date for
delivery. 
\begin{enumerate}
\item Draw an Entity-Relationship (E-R) diagram showing the three tables
and the relationships between them. \hfill{}{[}5{]}
\item A table description can be expressed as: 

\texttt{TableName (}\texttt{\uline{Attribute1}}\texttt{, Attribute2,
Attribute3, ...) }

The primary key is indicated by underlining one or more attributes. 

Write table descriptions for the three tables. \hfill{}{[}4{]}
\end{enumerate}
The bakery can change the price of an item at any time. Validation
ensures that the new price is within specified limits and is more
likely to be correct.
\begin{enumerate}
\item[(c)] {}
\begin{enumerate}
\item Explain why this could still result in incorrect weekly invoices,
assuming that the new price input is correct. \hfill{}{[}2{]}
\item Describe changes to the database and draw a modified E-R diagram to
ensure correct invoices are created. \hfill{}{[}4{]}
\end{enumerate}
\end{enumerate}