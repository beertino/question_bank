\item \textbf{{[}IJC/PRELIM/9597/2018/P1/Q2{]} }

In a school, students are identified by student numbers. These numbers
are stored in a hash table which uses the hashing function 
\noindent \begin{center}
\texttt{Address <- StudentNumber MOD X }
\par\end{center}

The hash table is implemented as a one-dimensional array with elements
indexed \texttt{0} to\texttt{ (X-1)}. 

\subsection*{Task 2.1}

Write program code to: 
\begin{itemize}
\item Read student numbers from a text file and store them in a hash table.
For the purpose of testing the program, \texttt{X} is to be set to
the value of 12. 

Assume different student numbers will hash to different addresses
(no collisions).
\item Print out the contents of the hash table in the order in which the
elements are stored in the array.
\end{itemize}
Use \texttt{KEYS.TXT} to test your program code.

\subsection*{Evidence 2}

Your program code. 

Screenshot of the program output. \hfill{}{[}7{]}

\subsection*{Task 2.2 }

Linear probing is a method to handle collisions. This means that a
collision is resolved by searching sequentially from the hashed address
for an empty location and storing the student number at this empty
location. If the end of the table is reached, the search for an empty
location is continued from the start of the table.

Use \texttt{KEYS2.TXT} to test your amended program code.

\subsection*{Evidence 3}

Your amended program code which performs linear probing. 

Screenshot of the program output. \hfill{}{[}4{]}

\subsection*{Task 2.3 }

Add code to your Task 2.2 program. The program is to: 
\begin{itemize}
\item Take as input a student number 
\item Search the hash table and output the address (index number) of the
hash table where the student number was found.
\end{itemize}
Use KEYS2.TXT to test your program code. Run the program three times.
Use the following inputs: 15, 23, 88. 

\subsection*{Evidence 4}

Your program code.

Screenshot of the program output.\hfill{} {[}7{]}