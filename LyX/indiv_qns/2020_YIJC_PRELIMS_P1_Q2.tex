\item \textbf{{[}YIJC/PRELIM/9569/2020/P1/Q2{]} }

An array \texttt{seq} contains a list of sorted data items except
the last element. \texttt{{[}1,2,5,8,9,6{]}} is an example of such
an array. 

The function \texttt{sortInner} takes two parameters, the array \texttt{seq}
and the index position \texttt{j} of the last element, and returns
the mutated array \texttt{seq}. 

\noindent\begin{minipage}[t]{1\columnwidth}%
\texttt{def sortInner(seq, j): }

\texttt{\qquad{}if j == 0: }

\texttt{\qquad{}\qquad{}return seq }

\texttt{\qquad{}else: }

\texttt{\qquad{}\qquad{}if seq{[}j{]} <= seq{[}j-1{]}: }

\texttt{\qquad{}\qquad{}\qquad{}seq{[}j{]}, seq{[}j-1{]} = seq{[}j-1{]},
seq{[}j{]} }

\texttt{\qquad{}return sortInner(seq, j-1)}%
\end{minipage} 
\begin{enumerate}
\item State what is meant by a recursive function.\hfill{} {[}2{]} 
\item Describe what happens when the function \texttt{sortInner({[}1,2,5,8,9,6{]},5)}
is executed. \hfill{}{[}2{]} 
\item Write a recursive function \texttt{insertionSort} for the Insertion
Sort algorithm by using the given function \texttt{sortInner(seq,j}).\hfill{}
{[}2{]} 
\item Explain whether the Insertion Sort algorithm in \textbf{(c)} is performing
an \textquotedbl in-place\textquotedbl{} or \textquotedbl non in-place\textquotedbl{}
sorting and whether it is stable or unstable.\hfill{} {[}4{]} 
\item State and explain the efficiencies of the Insertion Sort algorithm
in \textbf{(c)} in the worst case scenario, using the Big-O notation
for the time complexity. \hfill{}{[}2{]} 
\end{enumerate}