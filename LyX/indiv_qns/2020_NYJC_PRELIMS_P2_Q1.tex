\item \textbf{{[}NYJC/PRELIM/9569/2020/P2/Q1{]} }

The task is to implement a hash table to retrieve data about waste
disposal in Singapore. 

For each of the sub-tasks, add a comment statement, at the beginning
of the code using the hash symbol '\texttt{\#}', to indicate the sub-task
the program code belongs to, for example:
\noindent \begin{center}
\begin{tabular}{c|l|}
\cline{2-2} 
\multirow{2}{*}{\texttt{In{[}1{]}:}} & \texttt{\# Task 1.1}\tabularnewline
 & \texttt{Program Code}\tabularnewline
\cline{2-2} 
\multirow{2}{*}{\texttt{In{[}2{]}:}} & \texttt{\# Task 1.2}\tabularnewline
 & \texttt{Program Code}\tabularnewline
\cline{2-2} 
\multirow{2}{*}{\texttt{In{[}3{]}:}} & \texttt{\# Task 1.3}\tabularnewline
 & \texttt{Program Code}\tabularnewline
\cline{2-2} 
\multicolumn{1}{c}{} & \multicolumn{1}{l}{\texttt{Output:}}\tabularnewline
\end{tabular}
\par\end{center}

The file \texttt{waste.csv} contains the following fields in each
line: 

\texttt{Year \textendash{} \textquotedblleft YYYY\textquotedblright{} }

\texttt{Waste Disposed Of \textendash{} \textquotedblleft Numeric\textquotedblright{}
(Million Tons) }

\texttt{Waste Recycled \textendash{} \textquotedblleft Numeric\textquotedblright{}
(Million Tons) }

The first line of the file contains the headings. 

\subsection*{Task 1.1 }

Write a program to: 
\begin{itemize}
\item read data from \texttt{waste.csv} 
\item into a hash table of size 20 
\item by creating a function \texttt{GetKeyAddress(Year)} to generate the
hash address 
\item and directly inserting the data into the correct location in the hash
table 
\item taking care of any potential collisions using any suitable methods 
\end{itemize}
Display the contents of the hash table showing the data from the first
slot to the last slot. \hfill{}{[}14{]}

\subsection*{Task 1.2}

Create a menu with the following options: 
\begin{enumerate}
\item[1.] \texttt{ Get Waste Disposed and Recycled by year }
\item[2.] \texttt{ Display year(s) where Recycled waste > Waste disposed }
\item[3.] \texttt{ Return Average waste disposed between two years }
\item[4.] \texttt{ -1 to Exit }
\end{enumerate}
\begin{itemize}
\item implement the functions for each menu choice. 
\item use only direct access to retrieve data for options 1 and 3. 
\item option 1 and 3 requires asking users to input in the year(s). 
\item validate the user input. 
\item test option 1 with the year 2007 and show the output. 
\item test option 3 with the range 2002 to 2008 and show the output. 
\item show the output for option 2. \hfill{}{[}10{]}
\end{itemize}
Save your program code and output for Task 1 as 

\texttt{TASK1\_<your name>.ipynb }