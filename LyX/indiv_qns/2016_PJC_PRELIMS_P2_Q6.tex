\item \textbf{{[}PJC/PRELIM/9597/2016/P2/Q6{]} }

The following shows some data that are stored in a college. 
\noindent \begin{center}
\begin{tabular}{|c|c|c|c|c|c|c|}
\hline 
Student no & Student Name & Programme & Programme Duration (years) & Module No & Module Name & Lecturer\tabularnewline
\hline 
\hline 
13828 & Elvin Gan & P302 & 2 & M165, M121 & Visual Arts, Networking & Fang, Jason\tabularnewline
\hline 
13253 & Goh Seng Lee & P305 & 2 & M121, M110 & Networking, Database & Jason, Kabu\tabularnewline
\hline 
13423 & Yong Kee Le & P502 & 3 & M181, M107, M110 & Music, Accounting, Database & Sunny, Honto, Kabu\tabularnewline
\hline 
13098 & Mahesh Babu & P306 & 4 & M121, M110 & Networking, Database & Jason, Kabu \tabularnewline
\hline 
\end{tabular} 
\par\end{center}

A student is enrolled onto a programme and may take several modules
as part of this programme. A module is only delivered by one lecturer.
\begin{enumerate}
\item These data are in its un-normalised form. Explain the problems associated
with it. \hfill{}{[}2{]}
\item Normalise the data and write them in \textbf{four} tables. \hfill{}{[}6{]}
\item Draw an ER diagram that shows the relationships between these four
tables. \hfill{} {[}2{]}
\end{enumerate}