\item \textbf{{[}ALVL/9597/2016/P1/Q4{]} }

Numbers in Computing are often represented in hexadecimal form. 

A program is required to convert a hexadecimal number into a denary
number and vice versa. 

\subsubsection*{Task 4.1}

Write program code with the following specification: .
\begin{itemize}
\item Input a hexadecimal number as a string 
\item Validate the input 
\item Calculate the denary value of each hexadecimal digit (write this code
as a function)
\item Calculate the denary value of the hexadecimal number input
\item Output the denary value. 
\end{itemize}

\subsubsection*{Evidence 9}

Your program code. \hfill{}{[}10{]}

\subsubsection*{Task 4.2}

Draw up a list of \textbf{three} suitable test cases. Complete a table
with the following headings: 
\begin{center}
\begin{tabular}{|l|l|l|}
\hline 
Hexadecimal Number & Purpose of the test & Expected output\tabularnewline
\hline 
 &  & \tabularnewline
\hline 
 &  & \tabularnewline
\hline 
 &  & \tabularnewline
\hline 
\end{tabular}
\par\end{center}

Provide screenshot evidence for your testing. 

\subsubsection*{Evidence 10}

The completed table. 

Screenshots for each test data run. \hfill{}{[}5{]} 

\subsubsection*{Task 4.3}

Write additional code to convert a denary number into a hexadecimal
number. 

\subsubsection*{Evidence 11}

Your program code. \hfill{}{[}10{]}

\subsubsection*{Task 4.4}

Draw up a list of\textbf{ three} suitable test cases. Complete a table
with the following headings:
\begin{center}
\begin{tabular}{|l|l|l|}
\hline 
Denary Number & Purpose of the test & Expected output\tabularnewline
\hline 
 &  & \tabularnewline
\hline 
 &  & \tabularnewline
\hline 
 &  & \tabularnewline
\hline 
\end{tabular}
\par\end{center}

Provide screenshot evidence for your testing.

\subsubsection*{Evidence 12}

The completed table.

Screenshots for each test data run. \hfill{}{[}5{]}