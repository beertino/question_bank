\item \textbf{{[}JPJC/PRELIM/9569/2020/P2/Q4{]} }

A computer company has several offices throughout Singapore, each
with several salespersons. Each salesperson is assigned to one office
only. A record of the sales made by each salesperson has been set
up using a relational database. 

The following tables hold the data: 

\texttt{CUSTOMER (}\texttt{\uline{CustomerID}}\texttt{, CustomerName,
Email, Telephone) }

\texttt{OFFICE (}\texttt{\uline{OfficeID}}\texttt{, PostalCode,
Telephone) }

\texttt{SALE (}\texttt{\uline{SalesPersonID{*}}}\texttt{, }\texttt{\uline{CustomerID{*}}}\texttt{,
}\texttt{\uline{SaleDate}}\texttt{, Amount) }

\texttt{SALESPERSON (}\texttt{\uline{SalesPersonID}}\texttt{, SalesPersonName,
OfficeID{*}) }

\textbf{Note}: Underlined field indicates primary key. Asterisk ({*})
indicates a foreign key. 

\subsection*{Task 4.1 }

Write the SQL code to create the four tables in the database named
\texttt{computercompany.db}. 

Save the SQL file as \texttt{TASK4\_1\_<your class>\_<your name>.sql}.
\hfill{}{[}4{]}

\subsection*{Task 4.2 }

The files \texttt{CUSTOMER.CSV}, \texttt{OFFICE.CSV}, \texttt{SALE.CSV}
and \texttt{SALESPERSON.CSV} contain information exported by their
spreadsheets files. Write Python code to migrate them to the database. 

Save your Python code as \texttt{TASK4\_2\_<your class>\_<your name>.py}.
\hfill{} {[}6{]}

\subsection*{Task 4.3 }

Write SQL code to show the \texttt{SaleDate}, \texttt{SalesPersonName},
\texttt{CustomerName} and \texttt{Amount} of all sale transactions
performed at the office with ID \texttt{1}. 

Save the SQL file as \texttt{TASK4\_3\_<your class>\_<your name>.sql}.
\hfill{} {[}4{]}

\subsection*{Task 4.4 }

A report is produced to show the top salesperson in each office each
month. 

Write Python code to: 
\begin{enumerate}
\item[i.]  generate a web form that allows a user to enter the month (\texttt{1}
to \texttt{12}) and year, 
\item[ii.]  use the data submitted by the web form to query the database, and 
\item[iii.]  return a HTML page with the report displayed. 
\end{enumerate}
The following is a sample report for March, 2020. 
\noindent \begin{center}
\begin{tabular}{|lll|}
\hline 
\multicolumn{3}{|c|}{Top Performers in March 2020}\tabularnewline
\textbf{Office ID} & \textbf{Salesperson} & \textbf{Total Amount S\$}\tabularnewline
1 & Low Kok Kheong & 7880\tabularnewline
2 & Mindy Tan & 6935\tabularnewline
3 & Monish Chandr & 10700\tabularnewline
\hline 
\end{tabular}
\par\end{center}

Save your Python program as \texttt{TASK4\_4\_<your class>\_<your
name>.py} with any additional files/ sub-folders as needed in a zipped
folder named 

\texttt{Task4\_<your class>\_<your name>.zip}.\hfill{} {[}12{]}

\subsection*{Task 4.5 }

Deploy the web app on your local host and enter the following data: 
\begin{enumerate}
\item month -- \texttt{8}, and 
\item year -- \texttt{2020}. 
\end{enumerate}
Save the screenshot of table generated as 

\texttt{TASK4\_5\_<your class>\_<your name>.jpg}. \hfill{} {[}2{]}