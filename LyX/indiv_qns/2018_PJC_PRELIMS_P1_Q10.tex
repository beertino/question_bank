\item \textbf{{[}PJC/PRELIM/9597/2018/P1/Q10{]} }

A dataset of fruit names is to be stored in a binary search tree. 

The names of the fruits are inserted into the tree in the order shown: 
\noindent \begin{center}
\texttt{Papaya, Mango, Durian, Strawberry, Orange, Rambutan, Watermelon}
\par\end{center}
\begin{enumerate}
\item Draw the binary search tree. \hfill{} {[}3{]}
\end{enumerate}
The binary tree is implemented using these identifiers. 
\noindent \begin{center}
\begin{tabular}{|c|c|c|}
\hline 
\textbf{Variable} & \textbf{Data Type} & \textbf{Description}\tabularnewline
\hline 
\texttt{RootPtr} & \texttt{INTEGER} & Array subscript of the root of tree\tabularnewline
\hline 
\texttt{Fruit} & \texttt{ARRAY {[}1..100{]} of STRING} & Array of fruit names\tabularnewline
\hline 
\texttt{LeftPtr} & \texttt{ARRAY {[}1..100{]} of INTEGER} & Array of left pointer values\tabularnewline
\hline 
\texttt{RightPtr} & \texttt{ARRAY {[}1..100{]} of INTEGER} & Array of right pointer values \tabularnewline
\hline 
\end{tabular}
\par\end{center}
\begin{enumerate}
\item[(b)]  Draw a diagram to show the contents of the binary tree in array
form and the root pointer variable for the fruits inserted in \textbf{(a)}
above. \hfill{}{[}3{]}
\item[(c)]  The pseudocode shows an algorithm to search for a particular fruit
in the binary tree. Additional variables \texttt{SearchFruit}, \texttt{IsFound},
and \texttt{Current} are used. 

\noindent\fbox{\begin{minipage}[t]{1\columnwidth - 2\fboxsep - 2\fboxrule}%
\texttt{\textbf{INPUT}}\texttt{ SearchFruit }

\texttt{IsFound \textleftarrow{} False }

\texttt{Current \textleftarrow{} RootPtr }

\texttt{\textbf{REPEAT}}\texttt{ }

\texttt{\qquad{}}%
\fbox{\begin{minipage}[t]{0.4\columnwidth}%

\subsubsection*{\texttt{... ... ...}}

\subsubsection*{\texttt{... ... ...}}

\subsubsection*{\texttt{... ... ...}}%
\end{minipage}}

\texttt{\textbf{UNTIL}}\texttt{ Current = 0 }\texttt{\textbf{OR}}\texttt{
IsFound = }\texttt{\textbf{TRUE}}\texttt{ }

\texttt{\textbf{IF}}\texttt{ IsFound = False }\texttt{\textbf{THEN}}\texttt{ }

\texttt{\qquad{}}\texttt{\textbf{OUTPUT}}\texttt{ SearchFruit \textquotedbl Not
found\textquotedbl{} }

\texttt{\textbf{ENDIF}}\texttt{ }%
\end{minipage}}

Complete the algorithm in the \texttt{\textbf{REPEAT-UNTIL}} loop
by writing the missing lines. \hfill{}{[}6{]}
\end{enumerate}