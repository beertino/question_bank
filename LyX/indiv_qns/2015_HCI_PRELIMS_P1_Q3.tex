\item \textbf{{[}HCI/PRELIM/9597/2015/P1/Q3{]} }

A message is encrypted and passed between two parties. To decrypt
the message, a \textquotedblleft key\textquotedblright{} is applied.
Both the sending and receiving parties hold the key which enables
them to encrypt and decrypt the message. 

An approach of cryptography is the simple substitution cipher, a method
of encryption by which each letter of a message is substituted with
another letter. The receiving party deciphers the text by performing
an inverse substitution. 

The substitution system is created by first writing out a \texttt{\emph{phrase}}.
The \texttt{\emph{key}} is then derived from the \texttt{\emph{phrase}}
by removing all the repeated letters. The \texttt{\emph{cipher text}}
alphabet is then constructed starting with the letters of the key
and then followed by all the remaining letters in the alphabet. 

Using this system, the phrase \textquotedbl\texttt{apple}\textquotedbl{}
gives us the \texttt{\emph{key}} as \textquotedbl\texttt{APLE}\textquotedbl{}
and the following substitution scheme: 

\begin{tabular}{lccccccccccccccccccccccccccc}
\textbf{Plain text alphabet:} & \texttt{a} & \texttt{b} & \texttt{c} & \texttt{d} & \texttt{e} & \texttt{f} & \texttt{g} & \texttt{h} & \texttt{i} & \texttt{j} & \texttt{k} & \texttt{l} & \texttt{m} & \texttt{n} & \texttt{o} & \texttt{p} & \texttt{q} & \texttt{r} & \texttt{s} & \texttt{t} & \texttt{u} & \texttt{v} & \texttt{w} & \texttt{x} & \texttt{y} & \texttt{z} & \tabularnewline
 & \texttt{$\downarrow$} &  &  & \texttt{$\downarrow$} &  &  &  &  &  &  &  &  &  &  &  &  &  &  &  &  &  &  &  &  &  & \texttt{$\downarrow$} & is substituted by\tabularnewline
\textbf{Cipher text alphabet:} & \texttt{A} & \texttt{P} & \texttt{L} & \texttt{E} & \texttt{B} & \texttt{C} & \texttt{D} & \texttt{F} & \texttt{G} & \texttt{H} & \texttt{I} & \texttt{J} & \texttt{K} & \texttt{M} & \texttt{N} & \texttt{O} & \texttt{Q} & \texttt{R} & \texttt{S} & \texttt{T} & \texttt{U} & \texttt{V} & \texttt{W} & \texttt{X} & \texttt{Y} & \texttt{Z} & \tabularnewline
\end{tabular}

\texttt{'a'} will be substituted by \texttt{'A'}, \texttt{'b'} will
be substituted by \texttt{'P'}, \texttt{'c'} will be substituted by
\texttt{'L'}, \texttt{'d'} will be substituted by \texttt{'E'}, \texttt{'e'}
will be substituted by \texttt{'B'}, and so on. 

\subsection*{Task 3.1 }

Write program code for a function to create cipher text using the
following specification: 
\noindent \begin{center}
\texttt{FUNCTION CreateCipher (phrase: STRING): STRING }
\par\end{center}

The function \texttt{CreateCipher} has a single parameter \texttt{phrase}
and returns the cipher text alphabet as a string. 

\subsection*{Evidence 8: }

Your program code for Task 3.1. \hfill{}{[}8{]}

\subsection*{Task 3.2 }

Write program code for a procedure \texttt{CreateCipherTest} which
does the following:
\begin{itemize}
\item read the phrases from file \texttt{PHRASES.txt} 
\item create cipher text for each of the phrases 
\item display each phrase and cipher text on the screen as follows: 

\noindent\fbox{\begin{minipage}[t]{1\columnwidth - 2\fboxsep - 2\fboxrule}%
\texttt{Phrase: apple}

\texttt{Cipher text: APLEBCDFGHIJKMNOQRSTUVWXYZ }

\texttt{... ... }

\texttt{... ... }%
\end{minipage}}
\end{itemize}

\subsection*{Evidence 9: }

Your program code for Task 3.2. \hfill{}{[}3{]}

\subsection*{Evidence 10: }

Screenshot for running Task 3.2.\hfill{} {[}1{]}

\subsection*{Task 3.3}

Write program code for a function to decrypt a message using the following
specification: 
\noindent \begin{center}
\texttt{FUNCTION Decrypt(enc\_message: STRING, cipher: STRING): STRING }
\par\end{center}

The function \texttt{Decrypt} accepts parameters \texttt{enc\_message}
and \texttt{cipher}, and returns the decrypted message as a string.
Parameter \texttt{enc\_message} is the encrypted message, and parameter
\texttt{cipher} is the cipher text alphabet.

\subsection*{Evidence 11: }

Your program code for Task 3.3.\hfill{} {[}6{]}

\subsection*{Task 3.4 }

Write program code which does the following: 
\begin{itemize}
\item read the phrase and encrypted message from file \texttt{CIPHER.txt}
\item cipher text is generated from \texttt{CreateCipher} function 
\item message is decrypted from \texttt{Decrypt} function
\item display decrypted message on the screen together with the phrase and
encrypted message 

\noindent\fbox{\begin{minipage}[t]{1\columnwidth - 2\fboxsep - 2\fboxrule}%
\texttt{Phrase: ... }

\texttt{Encrypted message: ... }

\texttt{Decrypted message: ...}%
\end{minipage}} 
\end{itemize}

\subsection*{Evidence 12: }

Your program code for Task 3.4. \hfill{}{[}3{]}

\subsection*{Evidence 13:}

Screenshot for running Task 3.4.\hfill{} {[}1{]}

\subsection*{Task 3.5 }

Write program code for a function to encrypt a message using the following
specification:
\noindent \begin{center}
\texttt{FUNCTION Encrypt(message: STRING, cipher: STRING): STRING }
\par\end{center}

The function Encrypt accepts parameters \texttt{message} and \texttt{cipher},
and returns the encrypted message as a string. Parameter \texttt{message}
is the message to be encrypted while parameter \texttt{cipher} is
the cipher text. 

\subsection*{Evidence 14: }

Your program code for Task 3.5. \hfill{}{[}4{]}

\subsection*{Task 3.6 }

Write program code which does the following: 
\begin{itemize}
\item encrypt the message: \textquotedbl\texttt{do not give up!}\textquotedbl{} 
\item use the phrase: \textquotedbl\texttt{skyhigh}\textquotedbl{} 
\item generate cipher text from \texttt{CreateCipher} function
\item message is encrypted using \texttt{Encrypt} function 
\item encrypted message is displayed on screen as follows: 

\noindent\fbox{\begin{minipage}[t]{1\columnwidth - 2\fboxsep - 2\fboxrule}%
\texttt{Phrase: skyhigh}

\texttt{Encrypted Message: ... }%
\end{minipage}}
\end{itemize}

\subsection*{Evidence 15: }

Your program code for Task 3.6. \hfill{} {[}3{]}

\subsection*{Evidence 16:}

Screenshot for running Task 3.6. \hfill{}{[}1{]}