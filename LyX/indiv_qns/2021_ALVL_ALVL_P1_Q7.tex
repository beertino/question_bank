\item \textbf{{[}ALVL/9569/2021/P1/Q7{]} }

The nodes of a binary search tree holding names in alphabetical order
are stored in the elements of an array, \texttt{Names}. 

Each element of the array \texttt{Names} comprises three parts: a
left pointer, the data and a right pointer. 
\noindent \begin{center}
\texttt{}%
\begin{tabular}{|c|c|c|}
\hline 
\texttt{\textbf{LPtr}} & \texttt{\textbf{Data}} & \texttt{\textbf{RPtr}}\tabularnewline
\hline 
\end{tabular} 
\par\end{center}

The pointers contain the array index of a node to either the left
or right of the current node. \textbf{Null} indicates there are no
further nodes in a particular direction. 

An integer variable, \texttt{Root}, holds the index of the root node. 

The contents of the array \texttt{Names} are shown: 
\begin{center}
\textbf{}%
\begin{tabular}{|c|}
\hline 
\texttt{Root}\tabularnewline
\hline 
\textbf{1}\tabularnewline
\hline 
\end{tabular}\textbf{~}%
\begin{tabular}{c|c|c|c|}
\multicolumn{1}{c}{\texttt{Index}} & \multicolumn{1}{c}{\texttt{LPtr}} & \multicolumn{1}{c}{\texttt{Data}} & \multicolumn{1}{c}{\texttt{RPtr}}\tabularnewline
\cline{2-4} \cline{3-4} \cline{4-4} 
0 & \texttt{\textbf{Null}} & \texttt{\textbf{Peter}} & \texttt{\textbf{Null}}\tabularnewline
\cline{2-4} \cline{3-4} \cline{4-4} 
1 & \texttt{\textbf{3}} & \texttt{\textbf{Leona}} & \texttt{\textbf{5}}\tabularnewline
\cline{2-4} \cline{3-4} \cline{4-4} 
2 & \texttt{\textbf{Null}} & \texttt{\textbf{Alice}} & \texttt{\textbf{Null}}\tabularnewline
\cline{2-4} \cline{3-4} \cline{4-4} 
3 & \texttt{\textbf{2}} & \texttt{\textbf{Bobbie}} & \texttt{\textbf{6}}\tabularnewline
\cline{2-4} \cline{3-4} \cline{4-4} 
4 & \texttt{\textbf{Null}} & \texttt{\textbf{Tom}} & \texttt{\textbf{Null}}\tabularnewline
\cline{2-4} \cline{3-4} \cline{4-4} 
5 & \texttt{\textbf{0}} & \texttt{\textbf{Simone}} & \texttt{\textbf{4}}\tabularnewline
\cline{2-4} \cline{3-4} \cline{4-4} 
6 & \texttt{\textbf{Null}} & \texttt{\textbf{David}} & \texttt{\textbf{Null}}\tabularnewline
\cline{2-4} \cline{3-4} \cline{4-4} 
\end{tabular}
\par\end{center}
\begin{enumerate}
\item Draw the binary search tree represented by the data in the array \texttt{Wurdle}
and the value in \texttt{Root}. \hfill{}{[}2{]}
\item A new word, \textbf{Eric}, is to be inserted into the binary search
tree. Show the changes to the array \texttt{Names} after this insertion.
\hfill{}{[}2{]}
\item Using pseudo-code, write a recursive procedure \texttt{P}, together
with line numbers, that takes the value in \texttt{Root} and outputs
the result of an in-order traversal of \texttt{Names}.\hfill{} {[}4{]}
\end{enumerate}