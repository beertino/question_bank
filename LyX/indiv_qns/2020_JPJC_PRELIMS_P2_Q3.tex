\item \textbf{{[}JPJC/PRELIM/9569/2020/P2/Q3{]} }

The library in Jurong Pioneer Primary School uses a hybrid data structure
to keep track of its inventory. Each record in the hash table stores
a simple list \texttt{{[}<Category>, <Binary Search Tree object>{]}}.
Each node in the Binary Search Tree (BST) stores the ISBN number and
title of a book. The nodes in each BST share the same book category
and are sorted, in ascending order, according to their ISBN numbers. 

\subsection*{Task 3.1 }

Write the object-oriented code for the \texttt{BST} and \texttt{BSTNode}
classes described above. \hfill{}{[}10{]}

A checksum is applied to determine the Hash Value for each \texttt{<Category>},
where the ASCII value of each character in the title is multiplied
by its position in the \texttt{<Category>} string (starting from left
to right), and then summed. 

For example, given the category \textquotedblleft classics\textquotedblright ,
the summed value would thus be: $99\times1+108\times2+97\times3+115\times4+115\times5+105\times6+99\times7+115\times8=3884$.

$(3884\mod19)+1=9$

A weighted modulus 19 operation is then applied and 1 is added to
the remainder to determine the final Hash Value. 

\subsection*{Task 3.2 }

Write the code for the function \texttt{CalcHash(my\_string)}, which
takes in a string argument and returns its resultant Hash Value. \hfill{}{[}4{]}

\subsection*{Task 3.3 }

Write the code to declare and initialise \texttt{HashTable}, an empty
hash table array that may store up to 19 records. {[}2{]} 

\subsection*{Task 3.4 }

\texttt{CATEGORIES.TXT} is a text file containing the book categories.
Read the entire contents of \texttt{CATEGORIES.TXT} and update the
records in the hash table. Collisions are handled using \textbf{linear
probing}. \hfill{}{[}4{]}

\texttt{BOOKS.TXT} holds the details of books in the library. The
format of the data in the file is: \texttt{<Category>},\texttt{ <ISBN>},\texttt{
<Title>}. 

\subsection*{Task 3.5 }

Write program code to: 
\begin{itemize}
\item read the lines from the file, 
\item extract the \texttt{<Category>}, \texttt{<ISBN>} and \texttt{<Title>}
values, and 
\item add each book to the BST of its category. \hfill{}{[}6{]}
\end{itemize}

\subsection*{Task 3.6 }

Write the code to display the ISBN and title of each book belonging
to \texttt{\textquotedblleft classics\textquotedblright{}} category.
The output is sorted according to the ISBN numbers. Ensure your output
uses headings to identify the data displayed. 

Download your program code for Task 3 as 

\texttt{TASK3\_<your class>\_<your name>.ipynb}\hfill{} {[}4{]}