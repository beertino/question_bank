\begin{onehalfspace}
\item \textbf{{[}ALVL/9569/2020/P2/Q1{]} }
\end{onehalfspace}

\begin{onehalfspace}
\noindent Name your Jupyter Notebook as 

\noindent \texttt{TASK1\_<your name>\_<centre number>\_<index number>.ipynb}

\noindent The task is to implement a hashing function using the modulus
function and ASCII codes.

\noindent The hash is implemented with the following pseudocode, acting
on a string \texttt{string\_value}, which returns an integer, \texttt{h},
representing the hash.

\noindent \texttt{h $\leftarrow$ 0}

\noindent \texttt{FOR i $\leftarrow$ 0 TO length(string\_value)-1}

\noindent \texttt{\quad{}~~val $\leftarrow$ 33 {*} (ASCII value
of string\_value{[}i{]})}

\noindent \texttt{\quad{}~~h $\leftarrow$ (h+val) \% 1024}

\noindent \texttt{NEXT i}

\noindent \texttt{RETURN h}

\noindent For each of the sub-tasks, add a comment statement at the
beginning of the code, using the hash symbol '\#' to indicate the
sub-task the program code belongs to, for example:

\noindent %
\begin{tabular}{l|>{\raggedright}p{0.77\textwidth}|}
\cline{2-2} 
\texttt{In{[}1{]}:} & \texttt{\emph{\# Task 1.1}}\tabularnewline
 & \texttt{\emph{Program code}}\tabularnewline
\cline{2-2} 
\multicolumn{1}{l}{} & \multicolumn{1}{>{\raggedright}p{0.77\textwidth}}{\texttt{Output :}}\tabularnewline
\end{tabular}
\end{onehalfspace}
\begin{onehalfspace}

\subsubsection*{Task 1.1}
\end{onehalfspace}

\begin{onehalfspace}
\noindent Write a function \texttt{task1\_1(string\_value)} that:
\end{onehalfspace}
\begin{itemize}
\begin{onehalfspace}
\item takes a string value \texttt{string\_value}
\item implements the hash algorithm to produce an integer value
\item returns that integer value. \hfill{}{[}5{]}
\end{onehalfspace}
\end{itemize}
\begin{onehalfspace}
\noindent Test your function using the following \textbf{three} calls:
\end{onehalfspace}
\begin{itemize}
\begin{onehalfspace}
\item \texttt{task1\_1('Hello')}
\item \texttt{task1\_1('Hallo')}
\item \texttt{task1\_1('Hullo')}\hfill{}{[}5{]}
\end{onehalfspace}
\end{itemize}
\begin{onehalfspace}

\subsubsection*{Task 1.2}
\end{onehalfspace}

\begin{onehalfspace}
\noindent Strings are often combined with an original value (known
as the seed) before their hash is calculated. This makes it harder
to use reverse engineering to retrieve the original string.

\noindent Write a function task1\_2(seed, string\_value) that:
\end{onehalfspace}
\begin{itemize}
\begin{onehalfspace}
\item takes two string values seed and string\_value
\item concatenates these two string values together
\item uses the function in Task 1.1 to return the hash generated for the
concatenated value. \hfill{}{[}2{]}
\end{onehalfspace}
\end{itemize}
\begin{onehalfspace}
\noindent Test your function with the following three calls:
\end{onehalfspace}
\begin{itemize}
\begin{onehalfspace}
\item \texttt{task1\_2('seed-one','Hello')}
\item \texttt{task1\_2('seed-two','Hello')}
\item \texttt{task1\_2('seed-three','Hello')}\hfill{}{[}3{]}
\end{onehalfspace}
\end{itemize}
\begin{onehalfspace}
\noindent Save your Jupyter Notebook for Task 1. 
\end{onehalfspace}