\item \textbf{{[}JPJC/PRELIM/9569/2021/P2/Q1{]} }

Your program code and output for Task 1 should be saved in a single
\texttt{.ipynb} file. 

Name your Jupyter Notebook as \texttt{TASK1\_<your name>\_<class>\_<index
number>.ipynb }

The file \texttt{marathon.CSV} contains the full list of athletes
who took part in the 42.195km marathon race. The first line in the
file is the heading for the records. Each subsequent line is a record
of a runner in the form: 
\noindent \begin{center}
\texttt{<name of athlete>,<country code>,<timing in h:mm:ss>} 
\par\end{center}

For example, \texttt{Abdi ABDIRAHMAN,USA,2:18:27} 

Several athletes did not complete the race and their timing is recorded
as 'DNF' to indicate \textquoteleft did not finish\textquoteright . 

For example: \texttt{Alemu BEKELE,BRN,DNF }

\subsubsection*{Task 1.1 }

Write program code to find out the number of athletes who did not
finish the race and output the following three statements: 

\noindent %
\noindent\begin{minipage}[t]{1\columnwidth}%
\texttt{Number of DNF: x }

\texttt{Total number of athletes: y }

\texttt{Percentage of athletes who finished race: z }%
\end{minipage}

\texttt{x} is the number of athletes who did not finish the race, 

\texttt{y} is the total number of athletes who participated in the
marathon race, 

and \texttt{z} is the percentage (rounded to 1 decimal place) of athletes
who finished the race.\hfill{} {[}7{]}

\subsubsection*{Task 1.2 }

Write a function \texttt{insertionSort} that takes an unsorted list
as a parameter, sorts the list using the insertion sort algorithm,
and returns the sorted list. \hfill{}{[}6{]}

\subsubsection*{Task 1.3 }

By making use of the \texttt{insertionSort} function from Task 1.2,
or otherwise, find out the top 20 athletes and list them in order
of rank under the heading (Rank, Country, Name, Timing). \hfill{}{[}5{]} 

Sample Output: 

\noindent\begin{minipage}[t]{1\columnwidth}%
\texttt{Rank Country Name ~~~~~~Timing }

\texttt{1 ~~~ABC ~~~~Harry TAN ~2:08:38 }

\texttt{2 ~~~XYZ ~~~~Andy LEE ~~2:09:58 }

\texttt{3 ~~~... ~~~~... ~~~~~~~... }%
\end{minipage}