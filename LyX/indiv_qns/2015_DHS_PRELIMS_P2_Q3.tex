\item \textbf{{[}DHS/PRELIM/9597/2015/P2/Q3{]} }

Some game tournament such as badminton, bridge, chess, Pokemon, Scabble,
and many eSports employ the Swiss system of play, in which players
are not eliminated after each round and are paired with players with
the same number of wins or as close as possible. The following shows
an example of a 16-player Scrabble tournament. Assume that all games
have a winner and there are no draws. 

First round pairing is by random draw i.e there will be 8 random pairs.
After round 1, there will be a group of 8 players with a score of
1 (win), and a group of 8 players with a score of 0 (loss). 

For the second round, players in each scoring group will be paired
against each other -- 1\textquoteright s versus 1\textquoteright s
and 0\textquoteright s versus 0\textquoteright s.

After round 2, there will be three scoring groups: 
\begin{itemize}
\item 4 players who have won both games and have 2 points 
\item 8 players who have won a game and lost a game and have 1 point
\item 4 players who have lost both games and have no points. 
\end{itemize}
Again, for the third round, players are paired with players in their
scoring group. After round 3, the typical scoring groups will be: 
\begin{itemize}
\item 2 players who have won 3 games (3 points) 
\item 6 players with 2 wins (2 points) 
\item 6 players with 1 win (1 point) 
\item 2 players with no wins (0 points) 
\end{itemize}
For the fourth (and in this case final) round, the process repeats,
and players are matched with others in their scoring group. Note that
there are only 2 players who have won all of their games so far --
they will be matched against each other for the \textquotedbl championship\textquotedbl{}
game. After the final round, we will have something that looks like
this: 
\begin{itemize}
\item 1 player with 4 points -- the winner! 
\item 4 players with 3 points -- tied for second place
\item 6 players with 2 points 
\item 4 players with 1 point 
\item 1 player with 0 points 
\end{itemize}
The Swiss system produces a clear winner in just a few rounds, no
one is eliminated and almost everyone wins at least one game, but
there are many ties to deal with. 
\begin{enumerate}
\item Propose and justify a suitable data structure to store and initialise
the player names, intermediate and final opponents, points and results
of the 16-player tournament. You may denote the players' names with
the alphabets A -- P. \hfill{} {[}3{]}
\item Devise an algorithm to update the data structure in (a) and determine
the final ranks of the players. Players who are tied will have their
names output in alphabetical order. The next rank after 2 is 3. \hfill{}{[}7{]}
\end{enumerate}