\item \textbf{{[}PJC/PRELIM/9597/2015/P1/Q2{]} }

Quicksort is a sorting algorithm that employs a divide-and-conquer
strategy. 

Here is a high-level description of Quicksort applied to an array
A{[}0 : n -- 1{]}: 
\begin{enumerate}
\item[1.]  Select an element from A{[}0 : n -- 1{]} to be the pivot. 
\item[2.]  Rearrange the elements of A to partition A into a left subarray
and a right subarray, such that no element in the left subarray is
larger than the pivot and no element in the right subarray is smaller
than the pivot. 
\item[3.]  Recursively sort the left and the right subarrays.
\end{enumerate}

\subsection*{Task 2.1 }

Study the identifier table and incomplete quicksort algorithm. The
missing parts of the algorithm are labelled A, B and C. 
\noindent \begin{center}
\begin{tabular}{|c|c|c|}
\hline 
Variable & Data Type & Description\tabularnewline
\hline 
\hline 
ThisArray & ARRAY OF INTEGER & Array containing the dataset\tabularnewline
\hline 
First & INTEGER & First index of array\tabularnewline
\hline 
Last & INTEGER & Last index of array\tabularnewline
\hline 
Temp & INTEGER & Temporary variable\tabularnewline
\hline 
Low & INTEGER & Index of array\tabularnewline
\hline 
High & INTEGER & Index of array\tabularnewline
\hline 
Pivot & INTEGER & Reference value in array\tabularnewline
\hline 
\end{tabular}
\par\end{center}

\noindent %
\noindent\begin{minipage}[t]{1\columnwidth}%
\texttt{FUNCTION QuickSort(ThisArray, First, Last) RETURNS NULL }

\texttt{\qquad{}DECLARE Temp:INTEGER, Low:INTEGER, High:INTEGER,
Pivot:INTEGER }

\texttt{\qquad{}Low <- First }

\texttt{\qquad{}High <- Last }

\texttt{\qquad{}.........A......... //Assign reference value }

\texttt{\bigskip{}
}

\texttt{\qquad{}WHILE Low <= High }

\texttt{\qquad{}\qquad{}WHILE(ThisArray{[}Low{]} < Pivot) //Scan
left}

\texttt{\qquad{}\qquad{}\qquad{}.........B......... }

\texttt{\qquad{}\qquad{}ENDWHILE }

\texttt{\qquad{}\qquad{}WHILE(ThisArray{[}High{]} > Pivot) //Scan
right}

\texttt{\qquad{}\qquad{}\qquad{}High <- High \textendash{} 1 }

\texttt{\qquad{}\qquad{}ENDWHILE}

\texttt{\qquad{}\qquad{}IF Low <= High //Swapping }

\texttt{\qquad{}\qquad{}\qquad{}Temp <- ThisArray{[}Low{]} }

\texttt{\qquad{}\qquad{}\qquad{}ThisArray{[}Low{]} <- ThisArray{[}High{]} }

\texttt{\qquad{}\qquad{}\qquad{}ThisArray{[}High{]} <- Temp; }

\texttt{\qquad{}\qquad{}\qquad{}Low <- Low + 1 //Shift right by
1 element}

\texttt{\qquad{}\qquad{}\qquad{}High <- High - 1 //Shift left by
1 element }

\texttt{\qquad{}\qquad{}ENDIF}

\texttt{\qquad{}ENDWHILE}

\texttt{\bigskip{}
}

\texttt{\qquad{}IF First < High }

\texttt{\qquad{}\qquad{}QuickSort(ThisArray, First, High)}

\texttt{\qquad{}ENDIF}

\texttt{\qquad{}IF Low < Last}

\texttt{\qquad{}\qquad{}.........C.........}

\texttt{\qquad{}ENDIF}

\texttt{ENDFUNCTION }%
\end{minipage}

\subsection*{Evidence 5: }

What are the three missing lines of this pseudocode?\hfill{} {[}3{]}

\subsection*{Task 2.2}

Write a program to implement the quicksort. 

The program will:
\begin{itemize}
\item Call procedure \texttt{InitialiseList}. 
\item Use the function \texttt{QuickSort} to sort an array of integer \texttt{{[}435,646,344,54,23,98,789,212,847,201,733{]}}.
Copy and paste this array from the file \texttt{Number.txt} into your
program. 
\item Output the array \uline{before} and \uline{after} the quicksort
algorithm is applied.
\end{itemize}

\subsection*{Evidence 6: }

Program code for Task 2.2.\hfill{} {[}7{]}

\subsection*{Evidence 7: }

Screenshot to show running of program code in Task 2.2.\hfill{} {[}1{]}

\subsection*{Task 2.3 }

Amend the program as follows: 

The program must also output the number of function calls carried
out. 

\subsection*{Evidence 8: }

The amended program code.\hfill{} {[}3{]}

\subsection*{Task 2.4 }

By selecting different reference values (pivot) and input datasets,
and making use of the number of function calls, evaluate the efficiency
of the algorithm. 

\subsection*{Evidence 9: }

Evaluation of efficiency of quicksort algorithm with accompanying
screenshots (showing runs of function) for different reference values
and input datasets.\hfill{} {[}4{]}