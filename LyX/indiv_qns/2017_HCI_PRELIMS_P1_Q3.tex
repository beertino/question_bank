\item \textbf{{[}HCI/PRELIM/9597/2017/P1/Q3{]} }

An application is to be created to store a Football League table data.
The team names are stored in the file \texttt{TEAMS.txt}. The results
of the football matches are provided in file \texttt{RESULTS.txt}.

Each match data takes up one line, for example: \texttt{MadUnited
2 Chelsand 1 }

That is, \texttt{MadUnited} won \texttt{Chelsand}, scoring two goals
and conceding one goal, or Chelsand lost to Mad United, scoring one
goal and conceding two goals. 

The League table that needs to be created has the following information: 
\noindent \begin{center}
\begin{tabular}{llllllllll}
\textbf{Team} &  & \textbf{P} & \textbf{W} & \textbf{D} & \textbf{L} & \textbf{GF} & \textbf{GA} & \textbf{GD} & \textbf{Points}\tabularnewline
\hline 
\texttt{MadUnited} &  & \texttt{4} & \texttt{3} & \texttt{1} & \texttt{0} & \texttt{8} & \texttt{3} & \texttt{5} & \texttt{10}\tabularnewline
\texttt{Chelsand} &  & \texttt{4} & \texttt{2} & \texttt{1} & \texttt{1} & \texttt{10} & \texttt{7} & \texttt{3} & \texttt{7}\tabularnewline
\texttt{Lovepool} &  & \texttt{5} & \texttt{2} & \texttt{1} & \texttt{2} & \texttt{4} & \texttt{6} & \texttt{-2} & \texttt{7}\tabularnewline
$\dots$ &  &  &  &  &  &  &  &  & \tabularnewline
$\dots$ &  &  &  &  &  &  &  &  & \tabularnewline
\end{tabular}
\par\end{center}

\noindent %
\noindent\begin{minipage}[t]{1\columnwidth}%
\textbf{Legend}: 

\textbf{P} -- games played 

\textbf{W} -- games won

\textbf{D} -- games drawn

\textbf{L} -- games lost 

\textbf{GF} -- goals for (scored against opponents)

\textbf{GA} -- goals against (goals conceded by team)

\textbf{GD} -- goal difference, i.e. GD = GF -- GA

\textbf{Points} -- computed based on 3 points per win, 1 point per
draw and zero points per loss%
\end{minipage}

\subsection*{Task 3.1 }

Write program code for a procedure \texttt{CreateUpdateFile} which
does the following: 
\begin{itemize}
\item the program reads the \textbf{first} match results from \texttt{RESULTS.txt}
\item appends to the results of each team to a text file \texttt{NEWFILE.txt}
with the following information: team name, result of match (W/D/L),
goals for (GF), goals against (GA)
\item for e.g. the data \textquotedblleft \texttt{MadUnited 2 Chelsand 1}\textquotedblright{}
will result in the following two records to be appended to \texttt{NEWFILE.txt}.

\noindent\begin{minipage}[t]{1\columnwidth}%
\texttt{MadUnited,W,2,1}

\texttt{Chelsand,L,1,2}%
\end{minipage}
\end{itemize}

\subsection*{Evidence 12 }

Your CreateUpdateFile program code. \hfill{}{[}5{]}

\subsection*{Task 3.2 }

Amend your \texttt{CreateUpdateFile} program code from Task 3.1 so
that all the match results are read from \texttt{RESULTS.txt}, and
the \texttt{NEWFILE.txt} updated accordingly. 

\subsection*{Evidence 13 }

Your program code for the amended procedure \texttt{CreateUpdateFile}.
\hfill{}{[}2{]}

\subsection*{Evidence 14 }

2 screenshots of \texttt{NEWFILE.txt} showing first 10 and last 10
records. \hfill{} {[}2{]}

\subsection*{Task 3.3 }

Write program code for a function \texttt{ComputeTeamStat} which does
the following: 
\begin{itemize}
\item receives a team name as a parameter 
\item the function searches the file \texttt{NEWFILE.txt} for all occurrences
of that team
\item calculates and outputs the team\textquoteright s league table information, 

e.g. for team \textquoteleft \texttt{MadUnited}\textquoteright , it
may output the following: 
\noindent \begin{center}
\begin{tabular}{llllllllll}
\textbf{Team} &  & \textbf{P} & \textbf{W} & \textbf{D} & \textbf{L} & \textbf{GF} & \textbf{GA} & \textbf{GD} & \textbf{Points}\tabularnewline
\hline 
\texttt{MadUnited} &  & \texttt{4} & \texttt{3} & \texttt{1} & \texttt{0} & \texttt{8} & \texttt{3} & \texttt{5} & \texttt{10}\tabularnewline
\end{tabular}
\par\end{center}
\end{itemize}

\subsection*{Evidence 15 }

Your \texttt{ComputeTeamStat} program code.\hfill{} {[}6{]}

\subsection*{Evidence 16 }

A screenshot showing the output for the team \texttt{Everlong}. \hfill{}{[}1{]}

\subsection*{Task 3.4 }

Write program code for a procedure \texttt{GenerateTable} which does
the following: 
\begin{itemize}
\item reads the data from files \texttt{TEAMS.txt} and \texttt{NEWFILE.txt}
\item and makes use of the function \texttt{ComputeTeamStat} from Task 3.3 
\item to output the complete league table information ordered by the team
with the highest points first. If two or more teams having the same
points, team having more goal difference (GD) will be listed first. 
\end{itemize}

\subsection*{Evidence 17 }

Your \texttt{GenerateTable} program code.\hfill{} {[}7{]}

\subsection*{Evidence 18 }

A screenshot showing the output for the complete League table.\hfill{}
{[}2{]}