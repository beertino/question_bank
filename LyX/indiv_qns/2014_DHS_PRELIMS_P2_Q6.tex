\item \textbf{{[}DHS/PRELIM/9597/2014/P2/Q6{]} }

While English is the official working language in Singapore, beneficiaries
of the PGP are elderly and a significant segment of them do not speak
or understand English. As a multi-racial society, the pioneer generation
also consists of Chinese, Malays and Indians. 
\begin{enumerate}
\item What is Unicode and why is it an appropriate representation for PGP
information compared to ASCII? \hfill{}{[}3{]}
\item Give two disadvantages of using Unicode in this context. \hfill{}
{[}2{]}
\end{enumerate}
Mailing information of the pioneers is currently held in a sequential
file in NRIC order: 
\noindent \begin{center}
\texttt{<NRIC><Statutory name><Address line><Postal code> }
\par\end{center}

Postal code is a 6-digit string representing the geographic location
of an address. To mail the PGP information to the pioneers efficiently,
three methods are proposed: 
\begin{itemize}
\item M1 - Sort the contents of the sequential file using quick sort on
the postal code field. 
\item M2 - Reorganise the contents of the sequential file to a linked list
of linked lists in ascending postal code order. Each node of the linked
list points to a linked list of addresses with the same postal code. 
\item M3 - Reorganize the contents of the sequential file to a binary search
tree using postal code as the key field. Each node in the binary search
tree points to a linked list of addresses with the same postal code. 
\end{itemize}
\begin{enumerate}
\item[(c)]  Why is quick sort appropriate or inappropriate for method M1? \hfill{}{[}3{]}
\item[(d)]  Draw diagrams to represent the scenario in 
\begin{enumerate}
\item method M2 using linked list 
\item method M3 using binary tree 
\end{enumerate}
Your diagrams should contain at least 3 nodes for each scenario. \hfill{}
{[}4{]}
\item[(e)]  Write an algorithm to insert a new entry to the linked list in method
M2, assuming a successful appeal. \hfill{} {[}4{]}
\item[(f)]  Write an algorithm to delete an entry from the binary search tree
in method M3, assuming the demise of a pioneer. \hfill{}{[}4{]}
\end{enumerate}