\item \textbf{{[}DHS/PRELIM/9569/2021/P1/Q6{]} }

Long-distance optical fibre lines and submarine cables which are a
vital part of the global internet infrastructure are vulnerable to
solar superstorms which happen once in a century. The last solar superstrom
was in 1921. 
\begin{enumerate}
\item State and explain why websites would or would not be accessible by
web browsers if a solar superstorm shuts down all DNS servers. \hfill{}{[}2{]}
\item Explain packet switching and its importance in ensuring global internet
connectivity when some parts of the earth are hit by a solar superstorm.
\hfill{}{[}4{]}
\end{enumerate}
An international company based in many countries updates its network
structure to ensure Internet connectivity during solar superstorms. 

Employees can now access files on a shared virtual space which is
made up of 3 servers located in the United States, Europe, and Asia.
All servers hold identical information (any changes made on one server
would update the other servers immediately) so even if one server
is affected by the solar superstorm, employees still can access their
files on the other servers. 

Employees must be connected physically to the company\textquoteright s
intranet network at each country\textquoteright s office to access
the virtual space. 
\begin{enumerate}
\item[(c)] Draw a network diagram of the above configuration and label the LAN,
internet, router(s), WAN link(s), intranet, servers, and employee
laptops. \hfill{}{[}5{]}
\item[(d)] Why is version control vital when employees from different countries
work in teams? \hfill{}{[}1{]}
\end{enumerate}