\item \textbf{{[}DHS/PRELIM/9597/2018/P1/Q1{]} }

A programmer is writing a treasure island game to be played on the
computer. 

The island is a rectangular grid, 30 squares by 10 squares. Each square
of the island is represented by an element in a 2D array. The top
left square of the island is represented by the array element {[}0,
0{]}.

There are 30 squares across and 10 squares down. 

The computer will: 
\begin{itemize}
\item generate three random locations where treasure will be buried 
\item prompt the player for the location of one square where the player
choose to dig
\item display the contents of the array by outputting for each square: 
\begin{itemize}
\item '.' for only sand in this square
\item 'T' for treasure still hidden in sand 
\item 'X' for a hole dug where treasure was found
\item 'O' for a hole dug where no treasure was found. 
\end{itemize}
\end{itemize}
Here is an example display after the player has chosen to dig at location
{[}9, 3{]}: 

\noindent %
\noindent\begin{minipage}[t]{1\columnwidth}%
\texttt{.............................. }

\texttt{.............................. }

\texttt{.............................. }

\texttt{.............................. }

\texttt{..................T........... }

\texttt{.............................. }

\texttt{...T.......................... }

\texttt{.............................. }

\texttt{.............................. }

\texttt{...X..........................}%
\end{minipage}

The game is to be implemented using object oriented programming. 

The programmer has designed the class \texttt{IslandClass}. The identifier
table for this class is: 
\begin{itemize}
\item \texttt{Grid : ARRAY{[}0:9, 0:29{]} OF CHAR} - 2D array to represent
the squares of the island 
\item \texttt{Constructor()} - instantiates an object of class \texttt{IslandClass}
and initialises all squares to sand 
\item \texttt{HideTreasure()} - generates a pair of random numbers used
as the grid location of treasure and marks the square with \texttt{'T'} 
\item \texttt{DigHole(Row, Column)} - takes as parameter a valid grid location
and marks the square with \texttt{'X'} or \texttt{'O'} as appropriate 
\item \texttt{GetSquare(Row, Column) : CHAR} - takes as parameter a valid
grid location and returns the grid value for that square from the
\texttt{IslandClass} object 
\item \texttt{DisplayGrid()} - shows the current grid data. \texttt{DisplayGrid}
should make use of the getter method \texttt{GetSquare} of the \texttt{IslandClass}
class
\end{itemize}

\subsection*{Task 1.1 }

Write program code for the class \texttt{IslandClass} including the
\texttt{Constructor}, \texttt{GetSquare} and \texttt{DisplayGrid}
methods. The code should follow the specification given. 

The value to represent sand should be declared as a constant. Do not
attempt to write the methods \texttt{HideTreasure} or \texttt{DigHole}
at this stage.

\subsection*{Evidence 1}

Program code for Task 1.1\hfill{} {[}5{]}

\subsection*{Task 1.2 }

Write program code for the \texttt{HideTreasure} method. Your method
should check that the random location generated does not already contain
treasure. The value to represent treasure should be declared as a
constant. 

\subsection*{Evidence 2 }

Your program code for Task 1.2\hfill{} {[}3{]}

\subsection*{Task 1.3 }

Write a main program to: 
\begin{itemize}
\item create an IslandClass object 
\item generate three random locations where treasures will be buried
\item your program will then call the DisplayGrid method. 
\end{itemize}

\subsection*{Evidence 3 }

The program code for Task 1.3 \hfill{}{[}3{]}

\subsection*{Evidence 4}

Screenshot showing the output from running the program in Task 1.3\hfill{}
{[}1{]}

\subsection*{Task 1.4 }

Write program code for the \texttt{DigHole} method. This method takes
two integers as parameters. These parameters form a valid grid location.
The location is marked with \texttt{'X'} or \texttt{'O'} as appropriate. 

The values to represent treasure, found treasure and hole should be
declared as constants. 

\subsection*{Evidence 5}

Program code for Task 1.4. \hfill{}{[}3{]}

\subsection*{Task 1.5 }

Add code to the main program in Task 1.3. The program is to: 
\begin{itemize}
\item prompt the player for a location to dig 
\item validate the user input
\item call the \texttt{DigHole} method and then the \texttt{DisplayGrid}
method. 
\end{itemize}

\subsection*{Evidence 6 }

The program code.\hfill{} {[}3{]}

\subsection*{Task 1.6}

Run the program by inputting a location where: 
\begin{itemize}
\item the treasure is not found
\item the treasure is found. 
\end{itemize}

\subsection*{Evidence 7}

Screenshot evidence similar to that shown which shows: 
\begin{itemize}
\item The player has chosen to dig at location {[}2, 25{]} where no treasure
was found 

\noindent\begin{minipage}[t]{1\columnwidth}%
\texttt{.............................. }

\texttt{....T......................... }

\texttt{.........................O.... }

\texttt{..............................}

\texttt{..........T................... }

\texttt{.............................. }

\texttt{..................T........... }

\texttt{.............................. }

\texttt{.............................. }

\texttt{..............................}%
\end{minipage}
\item The player has chosen to dig at location {[}5, 3{]} where treasure
was found 

\noindent\begin{minipage}[t]{1\columnwidth}%
\texttt{.............................. }

\texttt{..............................}

\texttt{...............T..............}

\texttt{.............................. }

\texttt{.........................T.... }

\texttt{...X.......................... }

\texttt{.............................. }

\texttt{..............................}

\texttt{.............................. }

\texttt{..............................}%
\end{minipage}

\hfill{} {[}2{]}
\end{itemize}