\item \textbf{{[}PJC/PRELIM/9597/2016/P1/Q2{]} }

Write a program to encrypt and decrypt user passwords. 

\subsection*{Task 2.1}

A set of encryption key is given in the data file \texttt{ENCRYPT\_KEY.txt}.
The encryption key will map each alphabet and number according to
the table below.
\noindent \begin{center}
\texttt{}%
\begin{tabular}{ccc}
\texttt{a \textrightarrow{} m} & \texttt{m \textrightarrow{} q} & \texttt{y \textrightarrow{} c}\tabularnewline
\texttt{b \textrightarrow{} h} & \texttt{n \textrightarrow{} s} & \texttt{z \textrightarrow{} a}\tabularnewline
\texttt{c \textrightarrow{} t} & \texttt{o \textrightarrow{} l} & \texttt{0 \textrightarrow{} 7}\tabularnewline
\texttt{d \textrightarrow{} f} & \texttt{p \textrightarrow{} n} & \texttt{1 \textrightarrow{} 3}\tabularnewline
\texttt{e \textrightarrow{} g} & \texttt{q \textrightarrow{} i} & \texttt{2 \textrightarrow{} 8}\tabularnewline
\texttt{f \textrightarrow{} k} & \texttt{r \textrightarrow{} u} & \texttt{3 \textrightarrow{} 9}\tabularnewline
\texttt{g \textrightarrow{} b} & \texttt{s \textrightarrow{} o} & \texttt{4 \textrightarrow{} 5}\tabularnewline
\texttt{h \textrightarrow{} p} & \texttt{t \textrightarrow{} x} & \texttt{5 \textrightarrow{} 6}\tabularnewline
\texttt{i \textrightarrow{} j} & \texttt{u \textrightarrow{} z} & \texttt{6 \textrightarrow{} 0}\tabularnewline
\texttt{j \textrightarrow{} w} & \texttt{v \textrightarrow{} y} & \texttt{7 \textrightarrow{} 1}\tabularnewline
\texttt{k \textrightarrow{} e} & \texttt{w \textrightarrow{} v} & \texttt{8 \textrightarrow{} 4}\tabularnewline
\texttt{l \textrightarrow{} r} & \texttt{x \textrightarrow{} d} & \texttt{9 \textrightarrow{} 2}\tabularnewline
\end{tabular}
\par\end{center}

For example, \texttt{a} is mapped to \texttt{m}, \texttt{b} mapped
to \texttt{h}, m mapped to \texttt{q}, \texttt{9} mapped to \texttt{2},
etc. The mapping is case-sensitive for alphabets. Therefore a password
\texttt{WhizKid123} will be encrypted to \texttt{VpjaEjf389}.

Conversely, decryption works in the opposite way. Hence the password
\texttt{VpjaEjf389} will be decrypted to \texttt{WhizKid123}. 

There is no mapping for symbols. Some examples of symbols are \texttt{!},\texttt{
@}, \texttt{\#}, \texttt{\$}, \texttt{\%}, \texttt{\textasciicircum}.
Hence, a password \texttt{Extr@123} will be encrypted to \texttt{Gdxu@389},
where the symbol remains the same. 

\textbf{\emph{Copy and paste}} the encryption key (from data file
\texttt{ENCRYPT\_KEY.txt})\textbf{\emph{ into your program code}}
and \textbf{\emph{make use of it}} to write a program that 
\begin{itemize}
\item Allows a user to select option to \textbf{\emph{encrypt}} or \textbf{\emph{decrypt}}
a user password
\item Gets user input of the password,
\item Encrypts (or decrypts) the password according to user\textquoteright s
option, 
\item Displays the encrypted or decrypted password 
\end{itemize}

\subsection*{Evidence 5: }

Your program code for task 2.1.\hfill{} {[}8{]}

\subsection*{Evidence 6:}

Screenshot of running program by encrypting \texttt{WhizKid123} and
decrypting \texttt{XgteV\textasciicircum cg84}.\hfill{} {[}1{]}

\subsection*{Task 2.2 }

Amend your program code into a function that accepts two parameters
-- \textquotedbl\texttt{password}\textquotedbl{} and \textquotedbl\texttt{encrypt}\textquotedbl ,
and returns the encrypted or decrypted password. 

\texttt{FUNCTION Cryptograph(password:STRING, encrypt:BOOLEAN):RETURN
STRING }

\subsection*{Evidence 7: }

Your program code for task 2.2.\hfill{} {[}4{]}

\subsection*{Task 2.3 }

Write program code that makes use of the function \texttt{Cryptograph}
from task 2.2 that reads all the passwords in data file \texttt{PASSWORDS.txt},
encrypts and writes them into another file \texttt{CONVERTED.txt}. 

\begin{tabular}{l|l|l}
\cline{2-2} 
Sample input file:  & \texttt{WhizKid123 } & \tabularnewline
 & \texttt{Extr@123} & \tabularnewline
\cline{2-2} 
\multicolumn{1}{l}{} & \multicolumn{1}{l}{~~~~~~~~~~~$\downarrow$} & Encrypt and write\tabularnewline
\cline{2-2} 
Sample output file: & \texttt{VpjaEjf389} & \tabularnewline
 & \texttt{Gdxu@389} & \tabularnewline
\cline{2-2} 
\end{tabular}

\subsection*{Evidence 8: }

Program code for task 2.3. \hfill{}{[}3{]}

\subsection*{Evidence 9: }

Screenshot of output file \texttt{CONVERTED.txt}. \hfill{}{[}1{]}