\item \textbf{{[}NYJC/PRELIM/9569/2021/P2/Q3{]} }

Name your Jupyter Notebook as 

\texttt{TASK3\_<your name>\_<centre number>\_<index number>.ipynb }

The task is to write a function that takes a sequence of characters
representing a colour, and translates the colour into a different
number base. 

8-bit colours are represented with three numbers, indicating the level
of the colours red (R), green (G), and blue (B) respectively. Each
number is an integer from 0 to 255. 255 represents the fully saturated
colour, while 0 represents zero saturation (black).

In HTML, these colours may be represented using hex code as well.
In hex code, the R, G, and B values are converted to hexadecimal.
Hex codes begin with the symbol \textquoteleft \#\textquoteright{}
followed by the three R, G, and B hexadecimal values. 

For example, the hex code \#0A0B0C represents a colour with RGB values
10, 11, and 12 respectively. 

For each of the sub-tasks, add a comment statement, at the beginning
of the code using the hash symbol \textquoteleft \#\textquoteright ,
to indicate the sub-task the program code belongs to, for example: 
\noindent \begin{center}
\begin{tabular}{c|l|}
\cline{2-2} 
\multirow{2}{*}{\texttt{In{[}1{]}:}} & \texttt{\# Task 3.1}\tabularnewline
 & \texttt{Program Code}\tabularnewline
\cline{2-2} 
\multicolumn{1}{c}{} & \multicolumn{1}{l}{\texttt{Output:}}\tabularnewline
\end{tabular}
\par\end{center}

\subsubsection*{Task 3.1 }

Write a function called \texttt{task3\_1(hex)} that: 
\begin{itemize}
\item takes \texttt{hex}, a string representing a hex code, beginning with
a \textquoteleft \#\textquoteright{} symbol followed by three valid
hexadecimal values between 00 and FF 
\item returns and displays either: 
\begin{itemize}
\item a 3-integer tuple representing RGB values 

or 
\item the error message, \textquotedbl\texttt{invalid data}\textquotedbl{}
\hfill{}{[}5{]}
\end{itemize}
\end{itemize}
Test the function fully with suitable test data. 

For example, 

\texttt{task3\_1(\textquotedbl\#FFFFFF\textquotedbl ) }

should return and display \texttt{(255, 255, 255)}\hfill{} {[}3{]}

\subsubsection*{Task 3.2 }

Some image programs do not represent colours using 8-bit integers.
Instead, they represent them as a normalised float value. In this
representation, a value of 1.0 represents the fully saturated colour
and a value of 0 represents zero saturation (black). 

Write a second function \texttt{task3\_2(rgb)} that: 
\begin{itemize}
\item takes a 3-integer tuple rgb representing RGB values 
\item returns and displays either: 
\begin{itemize}
\item a 3-float tuple representing normalised RGB 

or 
\item the error message, \textquotedbl invalid data\textquotedbl{} \hfill{}{[}5{]}
\end{itemize}
\end{itemize}
Test the function fully with suitable test data. 

For example, 

\texttt{task3\_2((128, 128, 128))} 

should return and display\texttt{ (0.50196, 0.50196, 0.50196) }

\texttt{task3\_2((255, 255, 255)) }

should return and display\texttt{ (1.0, 1.0, 1.0)}\hfill{} {[}3{]}

\subsubsection*{Task 3.3 }

Image filters are functions that take in image data and change the
RGB values of its colours according to an algorithm. The algorithm
for converting an image to grayscale calculates the average of the
RGB values and sets the R, G, and B values to this average. 

Write a third function \texttt{task3\_3(hex)} that:
\begin{itemize}
\item takes hex, a string representing a hex code 
\item returns and displays a 3-float tuple representing normalised RGB of
the colour converted to grayscale \hfill{}{[}4{]}
\end{itemize}
Test the function fully with \textbf{two} suitable values.

For example, 

\texttt{task3\_3(\textquotedbl\#FF8000\textquotedbl ) }

should return and display \texttt{(0.5, 0.5, 0.5)}\hfill{} {[}2{]}

Save your Jupyter Notebook for Task 3.