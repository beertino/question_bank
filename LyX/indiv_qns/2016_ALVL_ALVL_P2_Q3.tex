\item \textbf{{[}ALVL/9597/2016/P2/Q3{]} }

The recursive procedure \texttt{X} was two parameters, \texttt{Value}
and \texttt{Index}. The procedure processes the contents of an array,
\texttt{T}.

\noindent %
\noindent\begin{minipage}[t]{1\columnwidth}%
\texttt{01 PROCEDURE X(Value, Index)}

\texttt{02 \qquad{}IF T{[}Index{]} > 0}

\texttt{03 \qquad{}\qquad{}THEN}

\texttt{04 \qquad{}\qquad{}\qquad{}IF T{[}Index{]} > Value}

\texttt{05 \qquad{}\qquad{}\qquad{}\qquad{}THEN}

\texttt{06 \qquad{}\qquad{}\qquad{}\qquad{}\qquad{}X(Value, Index
{*} 2) }

\texttt{07 \qquad{}\qquad{}\qquad{}ENDIF}

\texttt{08 \qquad{}\qquad{}\qquad{}IF T{[}Index{]} < Value}

\texttt{09 \qquad{}\qquad{}\qquad{}\qquad{}THEN}

\texttt{10 \qquad{}\qquad{}\qquad{}\qquad{}\qquad{}X(Value, Index
{*} 2 + 1) }

\texttt{11 \qquad{}\qquad{}\qquad{}ENDIF}

\texttt{12 \qquad{}\qquad{}\qquad{}IF T{[}Index{]} = Value}

\texttt{13 \qquad{}\qquad{}\qquad{}\qquad{}THEN}

\texttt{14 \qquad{}\qquad{}\qquad{}\qquad{}\qquad{}OUTPUT \textquotedbl True\textquotedbl}

\texttt{15 \qquad{}\qquad{}\qquad{}ENDIF}

\texttt{l6 \qquad{}ENDIF}

\texttt{17 ENDPROCEDURE}%
\end{minipage}
\begin{enumerate}
\item {}
\begin{enumerate}
\item State what is meant by a recursive procedure. \hfill{}{[}1{]}
\item Give the two line numbers which indicate that procedure x is recursive.
\hfill{} {[}1{]}
\end{enumerate}
\item An array T is used to store the data for a binary tree. A program
places items in the array in the order in which they joined the tree
structure. 
\begin{center}
\begin{tabular}{|c|c|c|c|c|c|c|c|c|c|c|c|c|c|c|}
\hline 
1 & 2 & 3 & 4 & 5 & 6 & 7 & 8 & 9 & 10 & 11 & 12 & 13 & 14 & 15\tabularnewline
\hline 
17 & 11 & 19 & 9 & 12 & 18 & 23 & 0 & 4 & 0 & 0 & 0 & 0 & 0 & 0\tabularnewline
\hline 
\end{tabular}
\par\end{center}
\begin{enumerate}
\item Draw the binary tree for the array \texttt{T} dataset. \hfill{} {[}3{]}
\item Copy and then complete the trace table for the procedure call \texttt{X(18,
1)}.
\begin{center}
\begin{tabular}{|c|c|c|c|}
\hline 
Procedure call & \texttt{Value} & \texttt{Index} & Output\tabularnewline
\hline 
\texttt{1} & \texttt{18} & \texttt{1} & \tabularnewline
\hline 
 &  &  & \tabularnewline
\hline 
 &  &  & \tabularnewline
\hline 
 &  &  & \tabularnewline
\hline 
 &  &  & \tabularnewline
\hline 
\end{tabular} 
\par\end{center}

\begin{center}
\hfill{}{[}3{]}
\par\end{center}
\item Describe the purpose of procedure \texttt{X}. \hfill{}{[}2{]}
\end{enumerate}
\end{enumerate}