\item \textbf{{[}RVHS/PRELIM/9597/2018/P1/Q2{]} }

\textbf{Health Workshops }

To promote healthy lifestyle, a local company decided to sponsor its
500 employees up to 3 health workshops of their choice. The company
decides to maintain this piece of information using a binary search
tree (BST) with each node contains a string called \texttt{employeeID}
which is used as the key of the BST, and an array structure called
workshops to store the name(s) and cost(s) of the health workshop(s)
chosen by the employee. 

\subsection*{Task 2.1}

Implement the BST structure and perform insertion using the 8-character
employee ID as key. During the insertion process, the employee\textquoteright s
respective health workshop(s) information must also be stored in the
node. The information of the employees and their chosen workshops
can be found in \textquotedblleft \emph{T2\_healthworkshops.txt}\textquotedblright{}
and its format is as follow: 

\texttt{<employeeID>-{[}<workshopName1>,<cost1> {[}<workshopName2>,<cost2>{]} }

For example: 

\texttt{5175590R-{[}Yoga With Yoyo,60{]}{[}Decoding The Nutrition
Label,55{]} }

This means that the employee with \texttt{employeeID} 5175590R has
chosen \textquotedblleft Yoga With Yoyo\textquotedblright{} and \textquotedblleft Decoding
The Nutrition Label\textquotedblright{} workshops and the cost of
these 2 workshops are \$60 and \$55 respectively. 

The BST structure must be able to handle up to 500 nodes. You need
to implement the following functions. 
\begin{itemize}
\item Functions
\begin{itemize}
\item \texttt{createBST()} This function creates a BST, reads the file \textquotedblleft \emph{T2\_workshops.txt}\textquotedblright ,
perform insertion of data and returns a BST structure. \hfill{}{[}5{]}
\item \texttt{inOrderTraversal()} This procedure outputs all employee IDs
from the BST in-order. \hfill{}{[}2{]}
\item \texttt{findWorkshopsById (employeeID)} This function returns a list
of names of all the workshops chosen by the employee with employee
ID as \texttt{employeeID}. \hfill{}{[}3{]}
\item \texttt{findIdsByWorkshop (workshopName)} This function returns a
list of employee IDs of employees who have chosen the workshop indicated
by the string \texttt{workshopName}. \hfill{}{[}2{]}
\item \texttt{findTotalCost()} This function returns the cost of all the
workshops chosen by all the employees in the company as an integer.
\hfill{}{[}2{]}
\end{itemize}
\end{itemize}

\subsection*{Evidence 10}

Program code for Task 2.1. \hfill{}{[}24{]}

\subsection*{Evidence 11 }

Screenshot the output of the following code. \hfill{}{[}1{]}

\noindent\begin{minipage}[t]{1\columnwidth}%
\texttt{def test1(): }

\texttt{\qquad{}b1 = createBST() }

\texttt{\qquad{}print(b1.findTotalCost()) }

\texttt{\qquad{}print(b1.findWorkshopsById(\textquotedbl 1001278B\textquotedbl ))
\qquad{}print(b1.findWorkshopsById(\textquotedbl 1019563R\textquotedbl )) }

\texttt{\qquad{}print(b1.findWorkshopsById(\textquotedbl 3161202Y\textquotedbl ))}

\texttt{\qquad{}print(b1.findWorkshopsById(\textquotedbl 5095845H\textquotedbl ))}

\texttt{\qquad{}print(b1.findWorkshopsById(\textquotedbl 9965997Y\textquotedbl ))}

\texttt{\qquad{}print(b1.findWorkshopsById(\textquotedbl 9998622F\textquotedbl )) }

\texttt{\qquad{}print(b1.findIdsByWorkshop('Diabetes 101')) }

\texttt{\qquad{}print(b1.findIdsByWorkshop('The Truth About Carbs')) }

\texttt{\qquad{}print(b1.findIdsByWorkshop('Nutrition Nuts And Bolts'))}

\texttt{\qquad{}print(b1.findIdsByWorkshop('Yoga With Yoyo')}

\texttt{test1() }%
\end{minipage}

\subsection*{Task 2.2 }

Write a menu which has the following options.

\noindent\begin{minipage}[t]{1\columnwidth}%
\texttt{1) Read file to generate BST }

\texttt{2) Find workshop(s) by user ID. }

\texttt{3) Find user ID(s) by workshop. }

\texttt{4) Display users in order. }

\texttt{5) Total cost.}

\texttt{6) Quit.}%
\end{minipage}

The validation of the employee ID follows the rules below: 
\begin{itemize}
\item The last of employee ID is the check code. 
\item The algorithm to generate the check code is as follows:
\begin{itemize}
\item Obtain the weighted sum of the 7 digits using the weights <2,7,6,5,4,3,2> 
\item Find the remainder of the weighted sum when divided by 17 
\item Look the check code up in the table below. 
\end{itemize}
\noindent \begin{center}
\begin{tabular}{|c|c|c|c|c|c|c|c|c|c|c|c|c|c|c|c|c|}
\hline 
0 & 1 & 2 & 3 & 4 & 5 & 6 & 7 & 8 & 9 & 10 & 11 & 12 & 13 & 14 & 15 & 16\tabularnewline
\hline 
A & B & C & D & E & F & G & H & J & M & N & Q & R & S & T & Y & Z\tabularnewline
\hline 
\end{tabular}
\par\end{center}

\end{itemize}
Below is an example of how the menu works.

\noindent\begin{minipage}[t]{1\columnwidth}%
\texttt{>\textcompwordmark >\textcompwordmark >menu()}

\texttt{1) Read file to generate BST }

\texttt{2) Find workshop(s) by user ID.}

\texttt{3) Find user ID(s) by workshop.}

\texttt{4) Display users in order. }

\texttt{5) Total cost. }

\texttt{6) Quit. }

\texttt{Choose an action: 1}

\texttt{File read. BST created. }

\texttt{1) Read file to generate BST}

\texttt{2) Find workshop(s) by user ID. }

\texttt{3) Find user ID(s) by workshop. }

\texttt{4) Display users in order. }

\texttt{5) Total cost. }

\texttt{6) Quit. }

\texttt{Choose an action: 2 }

\texttt{Type a userID: 1001278B}

\texttt{{[}'Stress Management For Managers: Helping You and Your Team
Stress Less'{]} }

\texttt{1) Read file to generate BST }

\texttt{2) Find workshop(s) by user ID. }

\texttt{3) Find user ID(s) by workshop.}

\texttt{4) Display users in order.}

\texttt{5) Total cost. }

\texttt{6) Quit. }

\texttt{Choose an action: 3}

\texttt{Type a workshop name: Yoga With Yoyo }

\texttt{{[}'1026960S', '1425017Z', '1518324H', '1602033B', '1712684E',
'1855773Y', '1914889R', '1918475F', '2131933A', '2174550G', '2448074G',
'2544510E', '2570896Z', '2882640B', '3307418R', '4583774N', '4881580J',
'5015391T', '5175590R', '6010700R', '6237846Z', '6285730A', '6382864S',
'6397129T', '6695719R', '6801170S', '6865739T', '7357796T', '7498335T',
'7839365D', '8154005Y', '8958192T', '9023193T', '9038581D', '9540693Y',
'9750174Q', '9964906Y'{]} }

\texttt{1) Read file to generate BST }

\texttt{2) Find workshop(s) by user ID. }

\texttt{3) Find user ID(s) by workshop. }

\texttt{4) Display users in order. }

\texttt{5) Total cost. }

\texttt{6) Quit.}

\texttt{Choose an action: 4}

\texttt{1001278B }

\texttt{1006415N}

\texttt{... }

\texttt{9978729Q }

\texttt{9998622F}

\texttt{1) Read file to generate BST }

\texttt{2) Find workshop(s) by user ID.}

\texttt{3) Find user ID(s) by workshop. }

\texttt{4) Display users in order.}

\texttt{5) Total cost.}

\texttt{6) Quit. }

\texttt{Choose an action: 6}

\texttt{Quit. }%
\end{minipage}

\subsection*{Evidence 12 }

Program code of \texttt{menu} with full input validations. \hfill{}
{[}10{]}