\item \textbf{{[}YIJC/PRELIM/9569/2020/P1/Q3{]} }

A wildlife information application is being developed to store and
display information about birds. The application uses a binary search
tree to store the name of the bird. 
\begin{enumerate}
\item The binary search tree has its data inserted in the following order. 

\texttt{Magpie }

\texttt{Cockatiel }

\texttt{Jay }

\texttt{Pigeon }

\texttt{Mynah }

\texttt{Crow }

\texttt{Albatross }

\texttt{Quail }

Draw the binary search tree. \hfill{}{[}4{]}
\item The binary search tree in part \textbf{(a)} can be implemented using
object-oriented programming that involves the use of two pointers
and an array. 
\begin{enumerate}
\item Describe the purpose of the two pointers in the implementation of
the binary search tree class. \hfill{} {[}2{]}
\item Describe the purpose of the array in the implementation of the binary
search tree class. \hfill{}{[}1{]}
\end{enumerate}
\item {}
\begin{enumerate}
\item List the nodes, in order, that are visited for an in-order traversal.
\hfill{}{[}2{]}
\item State the property exhibited by the list of items produced in part
(c)(i). \hfill{}{[}1{]}
\end{enumerate}
\item Describe an algorithm, using pseudocode, which uses a stack to perform
an in- order traversal for the tree \hfill{} {[}5{]}
\end{enumerate}