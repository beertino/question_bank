\item \textbf{{[}HCI/PRELIM/9597/2016/P2/Q1{]} }

Using the information in the table below, assuming that the project
team will work a standard working week (5 working days in 1 week)
and that all tasks will start as soon as possible: 
\noindent \begin{center}
\begin{tabular}{|c|l|c|c|}
\hline 
Task & Description & Duration (Working Days) & Predecessor/s \tabularnewline
\hline 
A & Requirement Analysis & 5 & -\tabularnewline
\hline 
B & Systems Design & 15 & A\tabularnewline
\hline 
C & Programming & 25 & B\tabularnewline
\hline 
D & Telecoms & 15 & B\tabularnewline
\hline 
E & Hardware Installation & 30 & B\tabularnewline
\hline 
F & Integration & 10 & C, D\tabularnewline
\hline 
G & System Testing & 10 & E, F\tabularnewline
\hline 
H & Training/Support & 5 & G\tabularnewline
\hline 
I & Handover and Go-Liv & 5 & H\tabularnewline
\hline 
\end{tabular}
\par\end{center}
\begin{enumerate}
\item Draw a Program Evaluation and Review Technique (PERT) chart, show
clearly the early start and late start time of each task, showing
dummy tasks, where necessary. \hfill{}{[}4{]}
\item Explain the nature and purpose of a dummy activity. \hfill{}{[}2{]}
\item Explain dependent stages and concurrent stages, giving examples from
your chart. \hfill{}{[}2{]}
\item State the critical path and the minimum time in which the project
can be completed in weeks.\hfill{} {[}2{]}
\item Identify any non-critical tasks and the float (free slack) on each.\hfill{}
{[}2{]}
\item Produce a Gantt chart based on the above information.\hfill{} {[}2{]}
\end{enumerate}