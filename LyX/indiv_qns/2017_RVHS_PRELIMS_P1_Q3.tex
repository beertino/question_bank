\item \textbf{{[}RVHS/PRELIM/9597/2017/P1/Q3{]} }

\textbf{Unsorted Linked List}

The classes Node and LinkedList is partially implemented for you according
to the specification given below. The attribute start of class LinkedList
stores the first node instance of the linked list. If the linked list
has no node, start has the value of None. Functions in bold are not
implemented. 
\begin{center}
\texttt{}%
\begin{tabular}{|l|}
\hline 
\texttt{\hspace{0.25\columnwidth}Node}\tabularnewline
\hline 
\texttt{-data : STRING}\tabularnewline
\texttt{-next : Node}\tabularnewline
\hline 
\texttt{+constructor()}\tabularnewline
\texttt{+set\_data(data)}\tabularnewline
\texttt{+get\_data()}\tabularnewline
\texttt{+set\_next(node)}\tabularnewline
\texttt{+get\_next()}\tabularnewline
\texttt{+\_\_str\_\_()}\tabularnewline
\hline 
\end{tabular}\texttt{}%
\begin{tabular}{|l|}
\hline 
\texttt{\hspace{0.25\columnwidth}LinkedList}\tabularnewline
\hline 
\texttt{-start : Node}\tabularnewline
\tabularnewline
\hline 
\texttt{+constructor()}\tabularnewline
\texttt{+set\_start(node)}\tabularnewline
\texttt{+get\_start()}\tabularnewline
\texttt{\textbf{+insert\_data(data)}}\tabularnewline
\texttt{\textbf{+transfer(linkedlist)}}\tabularnewline
\texttt{\textbf{+delete\_pos(pos)}}\tabularnewline
\texttt{+display() }\tabularnewline
\hline 
\end{tabular}
\par\end{center}

\subsection*{Task 3}

Implement the following \texttt{LinkedList} class functions according
to its specification given: 
\begin{itemize}
\item \texttt{+insert\_data(data) : }This function takes a string \texttt{data}
as input and uses its value to create a new \texttt{Node} instance.
The new Node instance is then inserted as the first node of the linked
list\texttt{.}
\item \texttt{+transfer(llist) :} This function takes a \texttt{LinkedList}
instance \texttt{llist} as input and transfers all its \texttt{Node}
instances in the same order to the \texttt{LinkedList} instance that
calls for this function. 

For example: 

\noindent\begin{minipage}[t]{1\columnwidth}%
\texttt{A:'boy'->'man'->'woman'}

\texttt{B:'girl'->'baby' }

\texttt{>\textcompwordmark >\textcompwordmark > A.transfer(B) }

\texttt{A: 'boy'->'man'->'woman'->'girl'->'baby'}

\texttt{B: None}%
\end{minipage}
\item \texttt{+delete\_pos(pos) : }This function takes in an integer pos
as input and deletes the node at position pos. Take note that the
first node of the linked list is position 1.

For example: 

\noindent\begin{minipage}[t]{1\columnwidth}%
\texttt{A:'boy'->'man'->'woman' }

\texttt{>\textcompwordmark >\textcompwordmark > A. delete\_pos(2) }

\texttt{A:'boy'->'woman'}%
\end{minipage}
\end{itemize}

\subsection*{Evidence 11 }

Program code for: 
\begin{itemize}
\item \texttt{insert\_data(node)} \hfill{}{[}3{]}
\item \texttt{transfer(linkedlist) }\hfill{}{[}3{]}
\item \texttt{delete\_pos()}\hfill{} {[}4{]}
\end{itemize}