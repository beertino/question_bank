\item \textbf{{[}NYJC/PRELIM/9597/2019/P1/Q4{]} }

To encrypt a message, a keyword cipher is used. It is a form of monoalphabetic
substitution where a keyword is used as the key. The key is used to
determine the letter matchings of the cipher alphabet to the plain
alphabet. Repeating letters in the key is removed. For instance, if
the key used is \texttt{'SECRET'}, the cipher alphabet generated will
be as follows: 

\begin{tabular}{lcccccccccccccccccccccccccc}
Plain text: & \texttt{A} & \texttt{B} & \texttt{C} & \texttt{D} & \texttt{E} & \texttt{F} & \texttt{G} & \texttt{H} & \texttt{I} & \texttt{J} & \texttt{K} & \texttt{L} & \texttt{M} & \texttt{N} & \texttt{O} & \texttt{P} & \texttt{Q} & \texttt{R} & \texttt{S} & \texttt{T} & \texttt{U} & \texttt{V} & \texttt{W} & \texttt{X} & \texttt{Y} & \texttt{Z}\tabularnewline
Cipher Alphabet: & \texttt{S} & \texttt{E} & \texttt{C} & \texttt{R} & \texttt{T} & \texttt{A} & \texttt{B} & \texttt{D} & \texttt{F} & \texttt{G} & \texttt{H} & \texttt{I} & \texttt{J} & \texttt{K} & \texttt{L} & \texttt{M} & \texttt{N} & \texttt{O} & \texttt{P} & \texttt{Q} & \texttt{U} & \texttt{V} & \texttt{W} & \texttt{X} & \texttt{Y} & \texttt{Z}\tabularnewline
\end{tabular}

After the key\textquoteright s unique letters is used up, the rest
of the ciphertext letters are used in alphabetical order, excluding
those already used in the key. Thus, to encode the word \textquoteleft \texttt{Attack}\textquoteright ,
\textquoteleft \texttt{A}\textquoteright{} is replaced with \texttt{'S'},
\texttt{'t'} is replaced with \texttt{'Q'}, and so on giving the encrypted
word as \texttt{'SQQSCH'}. 

\subsection*{Task 4.1 }

Write program code for a function to generate an array of the cipher
alphabet given a key. 
\noindent \begin{center}
\texttt{FUNCTION Cipher (NewAlphabet : ARRAY, Key : STRING) }
\par\end{center}

The function has two parameters and returns the \texttt{NewAlphabet}
array with the correct cipher alphabet based on the \texttt{Key} parameter. 

\subsection*{Evidence 14:}

Your program code. \hfill{} {[}6{]}

\subsection*{Task 4.2 }

Write driver code that asks the user to enter a key that contains
only letters, calls function Cipher and displays the cipher alphabet
(all in uppercase) in one line. Do appropriate data validation on
the input key. 

\subsection*{Evidence 15: }

Your program code. \hfill{} {[}3{]}

\subsection*{Task 4.3 }

Design three suitable test cases and provide screenshot evidence for
your testing. 

\subsection*{Evidence 16: }

Annotated screenshots for each test data run. \hfill{} {[}3{]}

\subsection*{Task 4.4 }

Develop your program further to display the following menu: 
\begin{enumerate}
\item[1.]  Encode a message
\item[2.]  Decode a message 
\item[3.]  -1 to Quit
\end{enumerate}
Implement the menu options to allow a line of text to be encoded or
decoded. For each option, ask for the cipher key and the message that
is to be encoded or decoded. Test your program with the key \texttt{'TOPSECRET'}
and the message \texttt{'I will score A for Computing}'. 

\subsection*{Evidence 17: }

Your program code. \hfill{}{[}10{]}

\subsection*{Evidence 18: }

Screenshot(s) showing your output. \hfill{}{[}2{]}

\subsection*{Task 4.5 }

Frequency analysis is a common method used by code breakers to break
monoalphabetic substitution ciphers. The first step is to analyse
the coded message and construct a frequency table of all the letters
appearing in the coded message. 

Write a program that reads the content from the file \texttt{INTERCEPT.txt}.
The text in this file contains a coded message. Construct a frequency
table (ignore punctuation marks) as follows using a suitable data
structure: 

\begin{tabular}{lcccccccccccccccccccccccccc}
Ciphertext Letter: & \texttt{A} & \texttt{B} & \texttt{C} & \texttt{D} & \texttt{E} & \texttt{F} & \texttt{G} & \texttt{H} & \texttt{I} & \texttt{J} & \texttt{K} & \texttt{L} & \texttt{M} & \texttt{N} & \texttt{O} & \texttt{P} & \texttt{Q} & \texttt{R} & \texttt{S} & \texttt{T} & \texttt{U} & \texttt{V} & \texttt{W} & \texttt{X} & \texttt{Y} & \texttt{Z}\tabularnewline
Frequency: & \texttt{5} & \texttt{2} & $\cdots$ &  &  &  &  &  &  &  &  &  &  &  &  &  &  &  &  &  &  &  &  &  &  & \tabularnewline
\end{tabular}

Your program will then display the table sorted by descending order
showing the most used letter and its frequency first followed by the
next highest and so on.

Partial sample output: 

\begin{tabular}{lcccc}
Ciphertext Letter: & \texttt{S} & \texttt{O} & \texttt{G} & \texttt{$\cdots$}\tabularnewline
Frequency: & \texttt{88} & \texttt{85} & \texttt{67} & $\cdots$\tabularnewline
\end{tabular}

\subsection*{Evidence 19: }

Your program code. \hfill{} {[}7{]}

\subsection*{Evidence 20: }

Screenshot of output. \hfill{}{[}1{]}