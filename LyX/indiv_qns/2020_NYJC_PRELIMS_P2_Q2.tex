\item \textbf{{[}NYJC/PRELIM/9569/2020/P2/Q2{]} }

A school library stores the following data in a file named \texttt{story.csv}: 

\begin{tabular}{ll}
\textbf{Field} & \textbf{Format}\tabularnewline
book\_title & text\tabularnewline
text subject & text\tabularnewline
author\_name & text\tabularnewline
published & \textquoteleft YYYY\textquoteright{} (year) \tabularnewline
\end{tabular}

Merge sort is an efficient sorting algorithm which falls under divide
and conquer paradigm and produces a stable sort. It operates by dividing
a large array into two smaller subarrays and then recursively sorting
the subarrays. 

For each of the sub-tasks, add a comment statement at the beginning
of the code using the hash symbol '\texttt{\#}', to indicate the sub-task
the program code belongs to, for example: 
\noindent \begin{center}
\begin{tabular}{c|l|}
\cline{2-2} 
\multirow{2}{*}{\texttt{In{[}1{]}:}} & \texttt{\# Task 2.1}\tabularnewline
 & \texttt{Program Code}\tabularnewline
\cline{2-2} 
\multirow{2}{*}{\texttt{In{[}2{]}:}} & \texttt{\# Task 2.2}\tabularnewline
 & \texttt{Program Code}\tabularnewline
\cline{2-2} 
\end{tabular}
\par\end{center}

\subsection*{Task 2.1 }

Write program code to: 
\begin{itemize}
\item read data from \texttt{story.csv} into an array of records. 
\item ask user to input in which field to sort the records by.
\item validate that the choice must be \texttt{\textquoteleft B\textquoteright },
\texttt{\textquoteleft S\textquoteright }, \texttt{\textquoteleft A\textquoteright },
or \texttt{\textquoteleft P\textquoteright{}} representing \texttt{book\_title},
\texttt{subject}, \texttt{author\_name} and \texttt{published} fields. 
\item implement a \texttt{MergeSort(ArrayData, Sortby)} function that takes
in two parameters, \texttt{ArrayData} (array of records) and \texttt{Sortby},
and sorts the records in ascending order according to the specified
field. \texttt{MergeSort(ArrayData, Sortby)} will return the sorted
\texttt{ArrayData} using a mergesort algorithm to do the sorting. 
\item display \texttt{ArrayData}. 
\item test your program twice and show your output for sorting by \texttt{subject}
and by \texttt{author\_name}. \hfill{} {[}12{]}
\end{itemize}

\subsubsection*{Task 2.2 }

Write program code to: 
\begin{itemize}
\item implement a \texttt{QuickSort(ArrayData)} function that uses the quicksort
algorithm to sort the \texttt{ArrayData} by \texttt{book\_title} in
descending order. \hfill{} {[}8{]}
\end{itemize}
Design 2 test cases to test your QuickSort(ArrayData) function and
explain the purpose of the test data. Show the output of your test
cases. \hfill{}{[}4{]}

Save your program code and output for Task 2 as 

\texttt{TASK2\_<your name>.ipynb}