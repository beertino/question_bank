\item \textbf{{[}YIJC/PRELIM/9569/2021/P1/Q2{]} }

A merge sort algorithm consists of the function \texttt{merge\_sort(seq)}
that takes in an unsorted list \texttt{seq} as an input and returns
a sorted list. The function uses the helper function \texttt{merge(left,right)}
which merges the two sorted lists, \texttt{left} and \texttt{right},
and returns a sorted list. 

The following is the pseudocode for the function \texttt{merge(left,right)}: 

\noindent\begin{minipage}[t]{1\columnwidth}%
\texttt{01 FUNCTION merge(left: LIST, right: LIST) RETURNS LIST }

\texttt{02 \qquad{}IF LENGTH(left) = 0 }

\texttt{03 \qquad{}\qquad{}THEN }

\texttt{04\qquad{}\qquad{}\qquad{} RETURN right }

\texttt{05 \qquad{}\qquad{}ELSE }

\texttt{06\qquad{}\qquad{}\qquad{} IF LENGTH(right) = 0 }

\texttt{07 \qquad{}\qquad{}\qquad{}\qquad{}THEN }

\texttt{08\qquad{}\qquad{}\qquad{}\qquad{}\qquad{} RETURN left }

\texttt{09 \qquad{}\qquad{}\qquad{}ENDIF }

\texttt{10 \qquad{}ENDIF }

\texttt{11 \qquad{}IF left{[}0{]} < right{[}0{]} }

\texttt{12 \qquad{}\qquad{}THEN }

\texttt{13\qquad{}\qquad{}\qquad{} RETURN {[}left{[}0{]}{]} + merge(left{[}1:{]},
right) }

\texttt{14 \qquad{}\qquad{}ELSE }

\texttt{15 \qquad{}\qquad{}\qquad{}RETURN {[}right{[}0{]}{]} +
merge(left, right{[}1:{]}) }

\texttt{16 \qquad{}ENDIF }%
\end{minipage} 
\begin{enumerate}
\item Explain why this is a recursive function. \hfill{}{[}2{]} 
\item State whether this \texttt{merge} implementation is stable or unstable
and explain with an example. \hfill{}{[}2{]} 
\item Complete the merge sort algorithm by writing the pseudocode for the
function \texttt{merge\_sort(seq)}. \hfill{}{[}4{]} 
\item State \textbf{one} advantage and \textbf{one} disadvantage of a merge
sort algorithm over a bubble sort algorithm. \hfill{} {[}2{]} 
\item Explain whether using a merge sort algorithm or a bubble sort algorithm
will be more efficient in arranging the data list \texttt{{[}2,1,3,4,6,5,8,7,9,10,12,11{]}}
in an ascending order. \hfill{}{[}3{]}
\end{enumerate}