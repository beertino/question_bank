\item \textbf{{[}RVHS/PRELIM/9597/2019/P1/Q1{]} }

\textbf{CID3 Team Grouping} 

In this question, you will help the CID3 students in forming CID3
groups for their projects. 

In \textquotedblleft \texttt{student\_cid.txt}\textquotedblright{} 

There are three fields on each line which indicates \texttt{name},
\texttt{role} and \texttt{gender} of 50 cid3 students. The fields
are separated by \textquoteleft ;\textquoteright{} 

\noindent %
\noindent\begin{minipage}[t]{1\columnwidth}%
\texttt{Rufus Schuck;Coder;F }

\texttt{Ione Wolfe;Dealer;F }

\texttt{Hillary Curl;Coder;M }

\texttt{\dots{} }%
\end{minipage}

\subsection*{Task 1.1 }

Implement the function \texttt{read\_data(filename)} which takes \texttt{filename}
as a string and returns a 2-dimension list that follows the format
as shown in the example below.

\noindent %
\noindent\begin{minipage}[t]{1\columnwidth}%
\texttt{>\textcompwordmark >\textcompwordmark > read\_data(\textquotedbl student\_cid.txt\textquotedbl ) }

\texttt{{[}{[}'Rufus Schuck', 'Coder', 'F'{]}, {[}'Ione Wolfe', 'Dealer',
'F'{]}, {[}'Hillary Curl', 'Coder', 'M'{]},\dots {]} }%
\end{minipage}

\subsection*{Evidence 1 }

Program code of function \texttt{read\_data}. \hfill{}{[}2{]}

\subsection*{Task 1.2}

Implement the function \texttt{gender\_count(cid\_student\_lst, is\_female)}
which takes a list \texttt{cid\_student\_lst} and a boolean \texttt{is\_female}
as inputs and returns the number of female students in \texttt{cid\_student\_lst}
if \texttt{is\_female} is \texttt{True}, otherwise return the number
of male students. \texttt{cid\_student\_lst} is the list obtained
in \textbf{Task 1.1}. 

\subsection*{Evidence 2 }

Program code of function \texttt{gender\_count}.\hfill{} {[}2{]}

\subsection*{Evidence 3 }

Screenshot of the output of the following: 

\noindent %
\noindent\begin{minipage}[t]{1\columnwidth}%
\texttt{cid\_student\_lst = read\_data(\textquotedbl student\_cid.txt\textquotedbl ) }

\texttt{print(gender\_count(cid\_student\_lst, True)) }

\texttt{print(gender\_count(cid\_student\_lst, False))}\hfill{}\texttt{
}{[}1{]}%
\end{minipage}

\subsection*{Task 1.3 }

Implement the procedure \texttt{role\_statistics(cid\_student\_lst)}
which takes a list \texttt{cid\_student\_lst} as input and output
the number of students for each role in the following format. (There
are more than 5 types of roles.) 

For example: 

\noindent %
\noindent\begin{minipage}[t]{1\columnwidth}%
\texttt{>\textcompwordmark >\textcompwordmark > cid\_student\_lst
= read\_data(\textquotedbl student\_cid.txt\textquotedbl ) }

\texttt{>\textcompwordmark >\textcompwordmark > role\_statistics(cid\_student\_lst) }

\texttt{Role ~~~~~~~Number }

\texttt{Coder ~~~~~~11 }

\texttt{Dealer ~~~~~13 }

\texttt{Designer ~~~14 }

\texttt{Empathizer ~14 }

\texttt{Maker ~~~~~~12 }%
\end{minipage}

Take note that the roles and numbers shown above is just an example.

\subsection*{Evidence 4 }

Program code of procedure \texttt{role\_statistics}.\hfill{} {[}3{]}

\subsection*{Evidence 5}

Screenshot of the output of the following: 

\noindent %
\noindent\begin{minipage}[t]{1\columnwidth}%
\texttt{cid\_student\_lst = read\_data(\textquotedbl student\_cid.txt\textquotedbl ) }

\texttt{role\_statistics(cid\_student\_lst)}\hfill{} {[}1{]}%
\end{minipage}

\subsection*{Task 1.4 }

Implement the function \texttt{form\_random\_group(cid\_student\_lst)}
which takes a list \texttt{cid\_student\_lst} as input and returns
a list consists of 5 student names. This list of students forms a
group and must consist of one coder, one maker, one dealer, one empathizer
and one designer. The student picked for each role must be random.
If there is not sufficient roles or students to form a group, return
an empty list. 

For example: 

\noindent %
\noindent\begin{minipage}[t]{1\columnwidth}%
\texttt{>\textcompwordmark >\textcompwordmark > cid\_student\_lst
= read\_data(\textquotedbl student\_cid.txt\textquotedbl ) }

\texttt{>\textcompwordmark >\textcompwordmark > form\_random\_group(cid\_student\_lst) }

\texttt{{[}'Fredricka Gormley', 'Jalisa Stoudemire', 'Laverna Halpern',
'Chadwick Griffin', 'Abdul Boland'{]} }%
\end{minipage}

Note: 

\noindent %
\noindent\begin{minipage}[t]{1\columnwidth}%
\texttt{Fredricka Gormley is a coder }

\texttt{Jalisa Stoudemire is a dealer}

\texttt{Laverna Halpern is a designer }

\texttt{Chadwick Griffin is an empathizer }

\texttt{Abdul Boland is a maker}%
\end{minipage}

\subsection*{Evidence 6 }

Program code of procedure \texttt{form\_random\_group}.\hfill{} {[}5{]}

\subsection*{Evidence 7 }

Screenshot of the output of the following: 

\noindent %
\noindent\begin{minipage}[t]{1\columnwidth}%
\texttt{cid\_student\_lst = read\_data(\textquotedbl student\_cid.txt\textquotedbl ) }

\texttt{for i in range(3): }

\texttt{\qquad{}print(form\_random\_group(cid\_student\_lst))} \hfill{}{[}1{]}%
\end{minipage}

\subsection*{Task 1.5 }

Implement the function \texttt{remove\_students} which takes \texttt{cid\_student\_lst}
and \texttt{one\_cid\_group} as inputs where \texttt{cid\_studnet\_lst}
is the list obtained from \textbf{task 1.1} and \texttt{one\_cid\_group}
is the list of 5 student names obtained from task 1.4. The function
removes 5 records in \texttt{cid\_student\_lst} specified by the student
names in \texttt{one\_cid\_group} and returns the removed records
in a list. For example: 

\noindent %
\noindent\begin{minipage}[t]{1\columnwidth}%
\texttt{>\textcompwordmark >\textcompwordmark > cid\_student\_lst
= read\_data(\textquotedbl student\_cid.txt\textquotedbl )}

\texttt{>\textcompwordmark >\textcompwordmark > one\_cid\_group
= form\_random\_group(cid\_student\_lst) }

\texttt{>\textcompwordmark >\textcompwordmark > one\_cid\_group
{[}'Rufus Schuck', 'Kathlene Collar', 'Luanne Lett', 'Phyliss Rolen',
'Tobias Kimmer'{]} }

\texttt{>\textcompwordmark >\textcompwordmark > remove\_students(cid\_student\_lst,
one\_cid\_group) {[}{[}'Rufus Schuck', 'Coder', 'F'{]}, {[}'Kathlene
Collar', 'Empathizer', 'M'{]}, {[}'Luanne Lett', 'Dealer', 'F'{]},
{[}'Phyliss Rolen', 'Maker', 'M'{]}, {[}'Tobias Kimmer', 'Designer',
'F'{]}}

\texttt{>\textcompwordmark >\textcompwordmark > len(cid\_student\_lst) }

\texttt{45 }%
\end{minipage}

After \texttt{remove\_students(cid\_student\_lst, one\_cid\_group)}
is executed \texttt{cid\_student\_lst} should not contain any records
with students name \texttt{Fredricka Gormley}, \texttt{Jalisa Stoudemire},
\texttt{Laverna Halpern}, \texttt{Chadwick Griffin} and \texttt{Abdul
Boland}. Since the 5 names are removed. \texttt{cid\_student\_lst}
should now have 45 records. 

\subsection*{Evidence 8 }

Program code of function \texttt{remove\_students}. \hfill{}{[}4{]}

\subsection*{Evidence 9 }

Screenshot of the output of the following code. 

\noindent %
\noindent\begin{minipage}[t]{1\columnwidth}%
\texttt{def test\_15(): }

\texttt{\qquad{}print(\textquotedbl -{}-{}-{}-{}-{}-Task 1.5-{}-{}-{}-{}-{}-\textquotedbl ) }

\texttt{\qquad{}cid\_student\_lst = read\_data(\textquotedbl student\_cid.txt\textquotedbl ) }

\texttt{\qquad{}one\_cid\_group = form\_random\_group(cid\_student\_lst) }

\texttt{\qquad{}removed\_records = remove\_students(cid\_student\_lst,
one\_cid\_group) }

\texttt{\qquad{}print(\textquotedbl removed records\textquotedbl ) }

\texttt{\qquad{}for removed\_record in removed\_records: }

\texttt{\qquad{}\qquad{}print(removed\_record) }

\texttt{\qquad{}print(\textquotedbl remaining records\textquotedbl ) }

\texttt{\qquad{}for cid\_student in cid\_student\_lst: }

\texttt{\qquad{}\qquad{}print(cid\_student) }

\texttt{test\_15()} \hfill{} {[}1{]}%
\end{minipage}

\subsection*{Task 1.6 }

Using your solutions in task 1.1, 1.4 and 1.5, write a procedure \texttt{form\_max\_cid\_group}
which takes a list \texttt{cid\_student\_lst} as input and write to
a file named \textquotedblleft \texttt{result.txt}\textquotedblright{}
the suggested maximum number of CID3 groups that can be formed from
\texttt{cid\_student\_lst}. The content in \textquotedbl\texttt{result.txt}\textquotedbl{}
should also include the group number and its group members. For example,
the content of \textquotedbl\texttt{result.txt}\textquotedbl{} can
be: 

\noindent %
\noindent\begin{minipage}[t]{1\columnwidth}%
\texttt{Group 0}

\texttt{Rufus Schuck Coder F}

\texttt{Lashawna Meals Dealer M}

\texttt{Phyliss Rolen Maker M }

\texttt{Laverna Halpern Designer F }

\texttt{Apryl Soileau Empathizer F }

\texttt{Group 1 }

\texttt{Claudette Bode Maker F }

\texttt{Angle Linck Coder F }

\texttt{Grazyna Kitzman Designer M }

\texttt{Virgilio Britt Dealer F }

\texttt{Dannette Raasch Empathizer F }

\texttt{Group 2 }

\texttt{Carolann Kintner Designer M }

\texttt{Ola Markell Empathizer F }

\texttt{Jaye Galle Maker F }

\texttt{Lanita Sciortino Coder M}

\texttt{Joella Wessner Dealer F }

\texttt{Group 3 }

\texttt{Hertha Dossantos Dealer F }

\texttt{Chadwick Griffin Empathizer M }

\texttt{Fredricka Gormley Coder F }

\texttt{Marcella Daigneault Designer F}

\texttt{Farah Quon Maker F}

\texttt{Group 4 }

\texttt{Hillary Curl Coder M }

\texttt{Elvia Dubrey Designer F}

\texttt{Terrence Shannon Empathizer M }

\texttt{Luanne Lett Dealer F }

\texttt{See Borne Maker F }

\texttt{Group 5 }

\texttt{Toney Mcnab Coder M }

\texttt{Jalisa Stoudemire Dealer M }

\texttt{Abdul Boland Maker M }

\texttt{Russell Gillison Designer F }

\texttt{Reiko Stack Empathizer F}%
\end{minipage}

\subsection*{Evidence 10 }

Program code of procedure \texttt{form\_max\_cid\_group}. \hfill{}{[}4{]}

\subsection*{Evidence 11 }

Screenshot of the content of \textquotedbl\texttt{result.txt}\textquotedbl .
\hfill{}{[}1{]}