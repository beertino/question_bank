\item \textbf{{[}ALVL/9569/2021/P1/Q2{]} }

A construction company provides employees to work on projects fro
clients. Each employee has a specific skill.
\begin{itemize}
\item Each project has an ID number and a title.
\item Each employee has a name, ID number and a SINGLE skill.
\item Each employee can work on a number of projects.
\item Each skill has an ID number, name and cost per hour.
\item The cost to the client for each employee is the employee's cost per
hour multiplied by the number of hours worked on the project.
\end{itemize}
This table shows typical data from which server charges can be calculated. 
\noindent \begin{center}
\begin{tabular}{|c|c|c|c|c|c|c|c|}
\hline 
Project ID & Project Title & Employee ID & Employee Name & Skill ID & Skill Name & Cost per Hour (\$) & Hours worked\tabularnewline
\hline 
\hline 
1 & New roof  & 1 & Smith & 1 & Carpentry & 40 & 45\tabularnewline
\hline 
1 & New roof  & 2 & Jones & 1 & Carpentry & 40 & 30\tabularnewline
\hline 
1 & New roof  & 3 & Roberts & 2 & Bricklaying & 45 & 12\tabularnewline
\hline 
2 & Refurbish pool & 4 & Harrison & 3 & Electrical & 50 & 15\tabularnewline
\hline 
2 & Refurbish pool & 5 & Harris & 4 & Plastering & 42 & 20\tabularnewline
\hline 
2 & Refurbish pool & 6 & Patel & 5 & Tiling & 30 & 35\tabularnewline
\hline 
2 & Refurbish pool & 7 & Staples & 50 & Tiling & 30 & 42\tabularnewline
\hline 
3 & Replace kitchen & 2 & Jones & 1 & Carpentry & 40 & 20\tabularnewline
\hline 
3 & Replace kitchen & 5 & Harris & 4 & Plastering & 42 & 14\tabularnewline
\hline 
3 & Replace kitchen & 4 & Harrison & 3 & Electrical & 50 & 30\tabularnewline
\hline 
3 & Replace kitchen & 8 & Charles & 5 & Tiling & 30 & 17\tabularnewline
\hline 
\end{tabular}
\par\end{center}
\begin{enumerate}
\item Explain, giving an example, whether the above table is in first normal
form (1NF). \hfill{}{[}2{]}
\end{enumerate}
The company wants to construct a relational database to store the
data shown in the table on page 3. 

The following tables contain the data:

\begin{tabular}{|c|c|}
\hline 
\multicolumn{2}{|c|}{Region}\tabularnewline
\hline 
Project ID & Project Title\tabularnewline
\hline 
1 & New roof\tabularnewline
\hline 
2 & Refurbish pool\tabularnewline
\hline 
3 & Replace kitchen\tabularnewline
\hline 
\end{tabular} %
\begin{tabular}{|c|c|c|c|c|}
\hline 
\multicolumn{5}{|c|}{Employee}\tabularnewline
\hline 
Employee ID & Employee Name & Skill ID & Skill Name & Cost per Hour (\$)\tabularnewline
\hline 
1 & Smith & 1 & Carpentry & 40\tabularnewline
\hline 
2 & Jones & 1 & Carpentry & 40\tabularnewline
\hline 
3 & Roberts & 2 & Bricklaying & 45\tabularnewline
\hline 
4 & Harrison & 3 & Electrical & 50\tabularnewline
\hline 
5 & Harris & 4 & Plastering & 42\tabularnewline
\hline 
6 & Patel & 5 & Tiling & 30\tabularnewline
\hline 
7 & Staples & 50 & Tiling & 30\tabularnewline
\hline 
8 & Charles & 5 & Tiling & 30\tabularnewline
\hline 
\end{tabular}

\begin{tabular}{|c|c|c|}
\hline 
\multicolumn{3}{|c|}{ProjectEmployee}\tabularnewline
\hline 
Project ID & Employee ID & Hours Worked\tabularnewline
\hline 
1 & 1 & 45\tabularnewline
\hline 
1 & 2 & 30\tabularnewline
\hline 
1 & 3 & 12\tabularnewline
\hline 
2 & 4 & 15\tabularnewline
\hline 
2 & 5 & 20\tabularnewline
\hline 
2 & 6 & 35\tabularnewline
\hline 
2 & 7 & 42\tabularnewline
\hline 
3 & 2 & 20\tabularnewline
\hline 
3 & 5 & 14\tabularnewline
\hline 
3 & 4 & 30\tabularnewline
\hline 
3 & 8 & 17\tabularnewline
\hline 
\end{tabular}
\begin{enumerate}
\item[(b)]  Explain why the table \textbf{Employee} is not in third normal form
(3NF). \hfill{}{[}2{]}
\item[(c)]  A table description can be expressed as:

\texttt{TableName (Attributel, Attribute2, Attribute3, ...)}

The primary key is indicated by underlining one or more attributes.

Write table descriptions for two tables to hold the data from the
\textbf{Employee} table each of which are in third normal form (3NF).
\hfill{}{[}3{]}
\item[(d)]  State the primary key for the table \textbf{ProjectEmployee}.\hfill{}
{[}1{]}
\item[(e)]  Draw an entity-relationship (ER) diagram showing the necessary four
tables and the relationships between them.\hfill{}{[}4{]}
\end{enumerate}
For each project one employee is nominated to be the Project Manager
\begin{enumerate}
\item[(f)]  Explain the change that needs to be made to the existing table design
to allow the Project Manager on each project to be identified.\hfill{}
{[}2{]}
\end{enumerate}
The cost of employing an electrician increases to \$52 per hour. The
client receives an invoice, at the end of the project, showing the
hours worked and charge for each skill. An employee with the skill
Electircal can be called an electrician.
\begin{enumerate}
\item[(g)]  Explain a problem that may arise if the \textbf{Cost per Hour (\$)}
field for Electrical in the \textbf{Employee} table is changed from
\$50 to \$52. \hfill{}{[}2{]}
\end{enumerate}
An employee with the skill Tiling can be called a tiler.
\begin{enumerate}
\item[(h)] Write an SQL query to output the names and running worked of the
tilers who worked on the Refurbish pool project, in descending hours
of \textbf{Hours Worked}.\hfill{}{[}7{]}
\end{enumerate}
The video game company maintains a table of employee names and addresses,
so the company can send letters to them.

The company also maintains a table of employee bank account details,
so monthly payments for the game can be transferred automatically
to their bank accounts.
\begin{enumerate}
\item[(i)] ) State \textbf{four} actions the construction company must take
regarding the collection, disclosure and use of this data under the
Personal Data Protection Act. \hfill{}{[}4{]}
\end{enumerate}