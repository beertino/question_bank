\item \textbf{{[}ACJC/PRELIM/9569/2021/P2/Q1{]} }

A programmer is writing a program to implement a role-playing computer
game using Object-Oriented Programming (OOP).

The players have to collect food items. A food item has the following
attributes:
\begin{itemize}
\item \texttt{name : STRING }
\item \texttt{value : INTEGER}
\end{itemize}
and the following methods:
\begin{itemize}
\item \texttt{get\_name() }
\item \texttt{get\_value()}
\end{itemize}
A player takes on the role of a person. A person has the following
attributes:
\begin{itemize}
\item \texttt{name : STRING }
\item \texttt{health : INTEGER} which is initialised at a value of \texttt{100}
\item \texttt{strength : INTEGER} which is initialised at a value of \texttt{100}
\end{itemize}
and the following methods:
\begin{itemize}
\item \texttt{get\_name() }
\item \texttt{get\_health() }
\item \texttt{get\_strength() }
\item \texttt{eat(food)} adds the value of the food to the strength. The
code should display the player\textquoteright s new strength. 
\item \texttt{attack(opponent)}
\end{itemize}
For the \texttt{attack} method, \texttt{opponent} is another person.
\begin{itemize}
\item A random integer \texttt{r} between \texttt{1} and \texttt{10} (inclusive)
is generated. 
\item If the player\textquoteright s \texttt{strength} is less than \texttt{r},
then the player does not have enough strength to attack and there
is no change to \texttt{opponent}\textquoteright s \texttt{health}. 
\item If the player\textquoteright s \texttt{strength} is at least \texttt{r},
then the attack is successful and \texttt{opponent}\textquoteright s
\texttt{health} is decreased by \texttt{r}. 
\begin{itemize}
\item If opponent\textquoteright s health is now negative, then opponent
has been defeated. 
\end{itemize}
\item The player\textquoteright s \texttt{strength} is decreased by r.
\end{itemize}
There are two subclasses of the \texttt{Person} class -- \texttt{Healer}
and the \texttt{Warrior}.

Healer has one additional method:
\begin{itemize}
\item \texttt{heal(patient)}
\end{itemize}
\quad{}

For the \texttt{heal} method, \texttt{patient} is another person.
\begin{itemize}
\item A random integer \texttt{r} between \texttt{1} and \texttt{10} (inclusive)
is generated. 
\item If the player\textquoteright s \texttt{strength} is less than \texttt{r},
then the player does not have enough strength to heal and there is
no change to \texttt{patient}\textquoteright s \texttt{health}. 
\item If the player\textquoteright s \texttt{strength} is at least \texttt{r},
then the healing is successful and \texttt{patient}\textquoteright s
\texttt{health} is increased by \texttt{r}, up to a maximum of \texttt{100}.
\end{itemize}
\texttt{Warrior}\textquoteright s \texttt{attack} method is twice
as effective, meaning that if the player has enough strength to attack,
\texttt{opponent}\textquoteright s \texttt{health} is decreased by
\texttt{2{*}r}, while the player\textquoteright s \texttt{strength}
is decreased by \texttt{r}.

\subsubsection*{Task 1.1}

Write program code to define the class \texttt{Food}. \hfill{}{[}3{]}

\subsubsection*{Task 1.2}

Write program code to define the class \texttt{Person}.

The code should display appropriate messages about the outcome of
\texttt{attack}, including the new value of opponent\textquoteright s
\texttt{health}. \hfill{}{[}10{]}

\subsubsection*{Task 1.3}

Use appropriate inheritance to write program code to define the class
\texttt{Healer}.

The code should display appropriate messages about the outcome of
\texttt{heal}, including the new value of \texttt{patient}\textquoteright s
\texttt{health}. \hfill{}{[}4{]}

\subsubsection*{Task 1.4}

Use appropriate inheritance and polymorphism to write program code
to define the class \texttt{Warrior}. \hfill{}{[}2{]}

Test your code with the following steps in order:
\begin{itemize}
\item Create a \texttt{Food} item with name \texttt{'Cheese'} and value
\texttt{10}. 
\item Create a \texttt{Warrior} with \texttt{name} \texttt{'Sam'}. 
\item Create a \texttt{Healer} with name \texttt{'Alex'}. 
\item Create a \texttt{Person} with name \texttt{'Jan'}. 
\item \texttt{'Jan'} attacks \texttt{'Sam'}. 
\item \texttt{'Sam'} attacks \texttt{'Jan'}. 
\item \texttt{'Alex'} heals \texttt{'Jan'}. 
\item \texttt{'Sam'} eats \texttt{'Cheese'}.
\end{itemize}
Download your program code and output for Task 1 as \texttt{TASK1\_<your
name>\_<centre number>\_<index number>.ipynb} \hfill{}{[}3{]}