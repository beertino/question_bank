\begin{onehalfspace}
\item \textbf{{[}ALVL/9569/2020/P2/Q3{]} }
\end{onehalfspace}

\begin{onehalfspace}
\noindent Name your Jupyter Notebook as

\noindent \texttt{TASK3\_<your name>\_<centre number>\_<index number>.ipynb}

\noindent The task is to write a function that takes a sequence of
characters that represents a quantity of data and unit, and translates
this quantity to a different unit.

\noindent The basic unit of data is the byte (B):
\end{onehalfspace}
\begin{itemize}
\begin{onehalfspace}
\item A kilobyte (KB) is $10^{3}$ bytes
\item A megabyte (MB) is $10^{6}$ bytes
\item A gigabyte (GB) is $10^{9}$ bytes
\item A terabyte (TB) is $10^{12}$ bytes
\end{onehalfspace}
\end{itemize}
\begin{onehalfspace}
\noindent For example, 8KB has 8000 bytes.

\noindent For each of the sub-tasks, add a comment statement at the
beginning of the code using the hash symbol '\#' to indicate the sub-task
the program code belongs to, for example:

\noindent %
\begin{tabular}{l|>{\raggedright}p{0.77\textwidth}|}
\cline{2-2} 
\texttt{In{[}1{]}:} & \texttt{\emph{\# Task 3.1}}\tabularnewline
 & \texttt{\emph{Program code}}\tabularnewline
\cline{2-2} 
\multicolumn{1}{l}{} & \multicolumn{1}{>{\raggedright}p{0.77\textwidth}}{\texttt{Output :}}\tabularnewline
\end{tabular}
\end{onehalfspace}
\begin{onehalfspace}

\subsubsection*{Task 3.1}
\end{onehalfspace}

\begin{onehalfspace}
\noindent Write a function called \texttt{task3\_1 (quantity\_of\_data)}
that:
\end{onehalfspace}
\begin{itemize}
\begin{onehalfspace}
\item takes a string, \texttt{quantity\_of\_data}
\item tests that the given string is a sequence of digits followed by one
of the four approved units shown above (KB, MB, GB, TB).
\item returns and displayes either:
\end{onehalfspace}
\begin{itemize}
\begin{onehalfspace}
\item the actual number of bytes represented by the input string\\
or
\item the error message, \texttt{``invalid data''}.\hfill{}{[}5{]}
\end{onehalfspace}
\end{itemize}
\end{itemize}
\begin{onehalfspace}
\noindent Test the function fully with suitable test data, including
all four approved units.

\noindent For example,

\noindent \texttt{task3\_1(``8KB'')}

\noindent should return and display \texttt{8000}.\hfill{}{[}3{]}
\end{onehalfspace}
\begin{onehalfspace}

\subsubsection*{Task 3.2}
\end{onehalfspace}

\begin{onehalfspace}
\noindent Companion units are also defined in terms of powers of 2.
These have similar abbreviations, as shown:
\end{onehalfspace}
\begin{itemize}
\begin{onehalfspace}
\item A kibibyte (KiB) is $2^{10}$ bytes
\item A mebibyte (MiB) is $2^{20}$ bytes
\item A gibibyte (GiB) is $2^{30}$ bytes
\item A tebibyte (TiB) is $2^{40}$ bytes
\end{onehalfspace}
\end{itemize}
\begin{onehalfspace}
\noindent Write a second function \texttt{task3\_2(quantity\_of\_data)}
that:
\end{onehalfspace}
\begin{itemize}
\begin{onehalfspace}
\item takes a string, \texttt{quantity\_of\_data}
\item tests that the given string is a sequence of digits followed by one
of the eight approved units (KB, KiB, MB, MiB, GB, GiB, TB, TiB)
\item returns and displays either:
\end{onehalfspace}
\begin{itemize}
\begin{onehalfspace}
\item the number of bytes represented by the input string\\
or
\item the error message, ``\texttt{invalid data}''\hfill{}{[}5{]}
\end{onehalfspace}
\end{itemize}
\end{itemize}
\begin{onehalfspace}
\noindent Test the function fully with suitable test data, including
all eight approved units.

\noindent For example,

\noindent \texttt{task3\_2('2MiB')}

\noindent should return and display \texttt{20197152}.\hfill{}{[}3{]}
\end{onehalfspace}
\begin{onehalfspace}

\subsubsection*{Task 3.2}
\end{onehalfspace}

\begin{onehalfspace}
\noindent Write a third function, \texttt{task3\_3(quantity\_of\_data,
target\_unit)} that:
\end{onehalfspace}
\begin{itemize}
\begin{onehalfspace}
\item takes two strings, \texttt{quantity\_of\_data} and \texttt{target\_unit}
\item tests that \texttt{target\_unit} is one of the eight approved units
from task 3.2
\item uses your function \texttt{task3\_2} to generate the actual number
of bytes represented by \texttt{quantity\_of\_data}
\item converts the generated number of bytes in \texttt{target\_unit}
\item returns and displays either:
\end{onehalfspace}
\begin{itemize}
\begin{onehalfspace}
\item the \texttt{quantity\_of\_data} in terms of the \texttt{target\_unit}
\item the error message, ``invalid data''\hfill{}{[}4{]}
\end{onehalfspace}
\end{itemize}
\end{itemize}
\begin{onehalfspace}
\noindent Test the function with three suitable set of values.

\noindent For example,

\noindent \texttt{task3\_3(``512MiB'',''GiB'')}

\noindent should return and display 0.5\hfill{}{[}3{]}
\end{onehalfspace}