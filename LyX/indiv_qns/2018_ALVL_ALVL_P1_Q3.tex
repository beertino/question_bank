\item \textbf{{[}ALVL/9597/2018/P1/Q3{]} }

The file,\texttt{ HASHEDDATA.TXT}, holds details of the names and
telephone numbers of 250 people. 

There are a total of 500 lines in the file, and a number of these
lines are empty of name and telephone number.

An index is stored for each line of the file. 

The format of the data in the file is: 
\begin{center}
<Index>,<PersonName>,<TelephoneNumber> 
\par\end{center}

The first 10 lines from the file are shown as follows: 

\noindent\fbox{\begin{minipage}[t]{1\columnwidth - 2\fboxsep - 2\fboxrule}%
0, ,

1, ,

2, ,

3, Boon Keng V., 07492 546415

4, ,

5, ,

6, Ahmad Yusof, 07439 778665

7, Durno Peter, 07662 863518

8, Batisah Wong, 07362 156265

9, ,%
\end{minipage}}

The values in the file are separated by the comma character. 

A record structure is used to store a name and telephone number. A
data structure of 500 records is needed to store all the names and
telephone numbers. Each line in the file is written to a corresponding
position in the data structure.

The records with index six to eight from the data structure are: 
\begin{center}
\begin{tabular}{r|c|c|}
\cline{2-3} \cline{3-3} 
\textbf{Index} & \textbf{PersonName} & \textbf{TelephoneNumber}\tabularnewline
\cline{2-3} \cline{3-3} 
6 & Ahmad Yusof & 07439 778665\tabularnewline
\cline{2-3} \cline{3-3} 
7 & Durno Peter & 07662 863518\tabularnewline
\cline{2-3} \cline{3-3} 
\texttt{8} & Batisah Wong & 07362 156265\tabularnewline
\cline{2-3} \cline{3-3} 
\end{tabular}
\par\end{center}

\subsubsection*{Task 3.1}

Use program code to create a:
\begin{itemize}
\item record structure to hold the name and telephone number for one person
\item data structure, using this record structure to store 500 records.
\end{itemize}

\subsubsection*{Evidence 7}

Your program code. \hfill{}{[}6{]}

\subsubsection*{Task 3.2}

Write program code to:
\begin{itemize}
\item read the lines from the file
\item extract the <Index>, <PersonName> and <TelephoneNumber> values
\item store these values in the data structure.
\end{itemize}
Create a procedure called \texttt{DisplayValues} that will loop though
the data structure and display the index, name and telephone number
for every record where the name is present.

Ensure your procedure uses headings to identify the data displayed.

\subsubsection*{Evidence 8}

Your program code. \hfill{}{[}13{]}

\subsubsection*{Evidence 9}

A Screenshot showing the output.\hfill{} {[}1{]}

A hashing function was used to create the file. The same hashing function
can be used to search the data structure for a particular name. The
hashing function generates a hash. This is calculated as follows:

\noindent\begin{minipage}[t]{1\columnwidth}%
\texttt{Get SearchName}

\texttt{Set HashTotal to 0}

\texttt{FOR each Character in SearchName}

\texttt{\qquad{}Get the ASCII code for Character}

\texttt{\qquad{}Multiply the ASCII code by the position of Character
in SearchName}

\texttt{\qquad{}Add the result to the HashTotal}

\texttt{Calculate Hash as HashTotal MOD 500}

\texttt{RETURN Hash}%
\end{minipage}

\subsubsection*{Task 3.3}

Add the program code for the hashing function. Use the following specification:
\begin{center}
\texttt{FUNCTION GenerateHash(SearchName : STRING) : INTEGER}
\par\end{center}

The function has a single parameter \texttt{SearchName} and returns
an integer value.

Write additional code for your program to allow you to test the implementation
of this function.

The following test data will assist you.

\qquad{}\textquotedblleft Tait Davinder\textquotedblright{} should
return a hash of 87

\qquad{}\textquotedblleft Anandan Yeo\textquotedbl{} should return
a hash of 156

\subsubsection*{Evidence 10}

Your program code. \hfill{}{[}8{]}

\subsubsection*{Evidence 11}

A screenshot (or screenshots) of your program to show the results
of the hash calculation for both the given test data values.\hfill{}{[}2{]}

The hash calculated from the \texttt{SearchName} can be used to find
a corresponding record in the data structure.

If the \texttt{SearchName} is not found in the record given by the
hash \textbf{and} the record is not empty:
\begin{itemize}
\item compare \texttt{SearchName} with the next record
\item until the \texttt{SearchName} is found or an empty record is found.
\end{itemize}
If an empty record is found then the program will report that the
name is \textquotedblleft NOT FOUND\textquotedblright .

If the record is found, the program will output the index, name and
telephone number.

\subsubsection*{Task 3.4}

Add the program code to implement the search as described.

\subsubsection*{Evidence 12}

Your program code. \hfill{}{[}7{]}

\subsubsection*{Evidence 11}

A screenshot (or screenshots) of your program to show the results
of the following searches:

Search 1: Charlie Love

Search 2: Chin Tan

Search 3: John Barrowman\hfill{}{[}3{]}