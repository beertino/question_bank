\item \textbf{{[}NYJC/PRELIM/9569/2021/P1/Q3{]} }

A printing shop needs to set up a print queue system to serve its
customers. This print queue will manage print tasks, by sending them
one at a time to available printers.

For all print tasks, the data that will be stored include: 

\noindent %
\noindent\begin{minipage}[t]{1\columnwidth}%
\texttt{\qquad{}}User 

\texttt{\qquad{}}Printer address 

\texttt{\qquad{}}Job name 

\texttt{\qquad{}}Status %
\end{minipage}

The print queue itself stores the following data: 

\texttt{\qquad{}}Number of jobs 

When a print task is added to the queue: 
\begin{itemize}
\item The task is stored inside the queue, in FIFO order 
\item The jobs count is incremented by one 
\end{itemize}
When a print task is sent to a printer: 
\begin{itemize}
\item The print task is removed from the queue, in FIFO order 
\item The jobs count is decremented by one
\end{itemize}
\begin{enumerate}
\item {}
\begin{enumerate}
\item Draw a class diagram that shows the following for the situation described
above. 
\begin{itemize}
\item The classes 
\item properties 
\item appropriate methods \hfill{}{[}9{]}
\end{itemize}
\item Explain the meaning of the terms: 
\begin{enumerate}
\item[1.]  inheritance \hfill{}{[}2{]}
\item[2.]  polymorphism \hfill{}{[}2{]}
\end{enumerate}
\end{enumerate}
\end{enumerate}
The printing shop wishes to implement a circular queue to limit the
maximum number of pending jobs and improve the performance of their
system. 
\begin{enumerate}
\item[(b)]  {}
\begin{enumerate}
\item State two differences between a linear queue and a circular queue.
\hfill{}{[}2{]}
\item Suggest whether inheritance or polymorphism is a more suitable principle
to apply in the implementation of both linear queue and circular queue
in the same program. Explain your answer. \hfill{}{[}4{]}
\end{enumerate}
\item[(c)]  Using a suitable diagram, pseudocode, or other method, show how
an item would be added to a circular queue implemented with a static
array. \hfill{}{[}4{]}
\end{enumerate}