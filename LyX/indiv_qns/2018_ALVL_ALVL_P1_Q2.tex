\item \textbf{{[}ALVL/9597/2018/P1/Q2{]} }

The following algorithm is an implementation of a quick sort that
operates on an array \texttt{Scores}. 

This algorithm assumes that the first element of an array is the zeroth
element. This means that \texttt{Scores{[}0{]}} is the first element
in the array.

This pseudocode is available in the file \texttt{QUICKSORT.TXT}

\noindent\begin{minipage}[t]{1\columnwidth}%
\noindent \texttt{FUNCTION QuickSort(Scores)}

\noindent \texttt{\qquad{}QuickSortHelper(Scores, 0, LENGTH(Scores)
- 1)}

\noindent \texttt{\qquad{}RETURN Scores}

\noindent \texttt{ENDFUNCTION}

\bigskip{}

\noindent \texttt{FUNCTION QuickSortHelper(Scores, First, Last)}

\noindent \texttt{\qquad{}IF First < Last}

\noindent \texttt{\qquad{}THEN}

\noindent \texttt{\qquad{}\qquad{}SplitPoint <- PartitioniScores,
First, Last)}

\noindent \texttt{\qquad{}\qquad{}QuickSortHelper(Scores, First,
SplitPoint \textemdash{} 1)}

\noindent \texttt{\qquad{}\qquad{}QuickSoRtHelper(Scores, SplitPoint
+ 1, Last)}

\noindent \texttt{\qquad{}ENDIF}

\noindent \texttt{\qquad{}RETURN Scores}

\noindent \texttt{ENDFUNCTION }\bigskip{}

\noindent \texttt{FUNCTION Partition(Scores, First, Last)}

\noindent \texttt{\qquad{}PivotValue <- ScoresiFirst{]}}

\noindent \texttt{\qquad{}Lefthark <- First + 1}

\noindent \texttt{\qquad{}RightMark <- Last}

\noindent \texttt{\qquad{}Done <- FALSE}

\noindent \texttt{\qquad{}WHILE (Done = FALSE)}

\noindent \texttt{\qquad{}\qquad{}WHILE LeftMark <= RightMark AND
Scores{[}LeftMark{]} <= PivotValue}

\noindent \texttt{\qquad{}\qquad{}\qquad{}LeftMark <- LeftMark
+ 1}

\noindent \texttt{\qquad{}\qquad{}ENDWHILE}

\noindent \texttt{\qquad{}\qquad{}WHILE Scores{[}RightMark{]} >=
PivotValue AND RightMark >= LeftMark}

\noindent \texttt{\qquad{}\qquad{}\qquad{}RightMark <- RightMark
\textemdash{} 1}

\noindent \texttt{\qquad{}\qquad{}ENDWHILE}

\noindent \texttt{\qquad{}\qquad{}IF RightMark < LeftMark}

\noindent \texttt{\qquad{}\qquad{}\qquad{}THEN}

\noindent \texttt{\qquad{}\qquad{}\qquad{}\qquad{}Done <- TRUE}

\noindent \texttt{\qquad{}\qquad{}ELSE}

\noindent \texttt{\qquad{}\qquad{}\qquad{}Temp <- Scores{[}LeftMark{]}}

\noindent \texttt{\qquad{}\qquad{}\qquad{}Scores{[}LeftMark{]}
<- Scores{[}RightMark{]}}

\noindent \texttt{\qquad{}\qquad{}\qquad{}Scores{[}RightMark{]}
<- Temp}

\noindent \texttt{\qquad{}ENDIF}

\noindent \texttt{ENDWHILE }\bigskip{}

\noindent \texttt{\textbf{\emph{<swap Scores{[}First{]} with Scores{[}RightMark{]}>}}}\texttt{
}\bigskip{}

\noindent \texttt{\qquad{}RETURN RightMark}

\noindent \texttt{ENDFUNCTION}%
\end{minipage}

\subsubsection*{Task 2.1}

Write program code to implement this algorithm. Ensure that you add
the missing code to complete the algorithm. The area of missing code
is highlighted as:
\begin{center}
\texttt{\textbf{\emph{<swap Scores {[}First{]} with Scores {[}RightMark{]}>}}}
\par\end{center}

Copy the sample data available in the \texttt{SCORES.TXT} file. Paste
this into your programming code to set up the data to be sorted.

\subsubsection*{Evidence 4}

Your program code. \hfill{}{[}12{]}

\subsubsection*{Task 2.2}

Add a function to your code to output Scores. Call this function before
and after the operation of the quick sort so that the unsorted and
sorted data is displayed.

\subsubsection*{Evidence 5}

Your program code. \hfill{}{[}2{]}

\subsubsection*{Evidence 6}

Screenshot showing the unsorted and sorted \texttt{Scores} data.\hfill{}
{[}1{]}