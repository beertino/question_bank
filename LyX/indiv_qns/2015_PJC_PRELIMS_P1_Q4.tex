\item \textbf{{[}PJC/PRELIM/9597/2015/P1/Q4{]} }

Implement a stack class using array using the following properties
and methods: 
\begin{center}
\begin{tabular}{|l|l|l|}
\hline 
\multicolumn{3}{|c|}{\texttt{Class: Stack}}\tabularnewline
\hline 
\multicolumn{3}{|c|}{Properties}\tabularnewline
\hline 
\texttt{\hspace{0.01\columnwidth}}Identifier & \texttt{\hspace{0.01\columnwidth}}Data Type & \texttt{\hspace{0.05\columnwidth}}Description\tabularnewline
\hline 
\texttt{Data} & \texttt{ARRAY{[}x{]} of STRING} & x is the limit, which must be supplied when the object is called \tabularnewline
\hline 
\texttt{Limit} & \texttt{INTEGER} & The maximum number of elements the stack can hold \tabularnewline
\hline 
\end{tabular}
\par\end{center}

\begin{center}
\begin{tabular}{|l|l|l|}
\hline 
\multicolumn{3}{|c|}{\texttt{Class: Stack}}\tabularnewline
\hline 
\multicolumn{3}{|c|}{Methods}\tabularnewline
\hline 
\texttt{\hspace{0.01\columnwidth}}Identifier & \texttt{\hspace{0.01\columnwidth}}Data Type & \texttt{\hspace{0.05\columnwidth}}Description\tabularnewline
\hline 
\texttt{IsEmpty} & \texttt{FUNCTION RETURNS BOOLEAN} & Indicates whether any elements are stored in stack or not \tabularnewline
\hline 
\texttt{IsFull} & \texttt{FUNCTION RETURNS BOOLEAN} & Indicates whether stack is full or not\tabularnewline
\hline 
\texttt{Push} & \texttt{PROCEDURE} & Inserts data onto stack\tabularnewline
\hline 
\texttt{Pop} & \texttt{PROCEDURE} & Removes and returns the last inserted element from the stack\tabularnewline
\hline 
\texttt{Peek} & \texttt{PROCEDURE} & Returns the last inserted element without removing it\tabularnewline
\hline 
\texttt{Size} & \texttt{PROCEDURE} & Returns the number of elements stored in stack \tabularnewline
\hline 
\texttt{Display} & \texttt{PROCEDURE} & Displays the content of the stack with top of stack clearly indicated\tabularnewline
\hline 
\end{tabular}
\par\end{center}

\subsection*{Task 4.1 }

Write program code for the stack class with all the properties and
methods above.

\subsection*{Evidence 16:}

Your program code. \hfill{}{[}12{]}

A stack can be used to evaluate an arithmetic expression. An arithmetic
expression can first be converted from infix notation to postfix notation,
then the postfix notation can be evaluated to get the value of the
infix notation. 

For example, the infix notation \texttt{5 {*} (6 + 2) - 12 / 4} can
first be converted to postfix notation \texttt{5 6 2 + {*} 12 4 /
-}, and then evaluated to \texttt{37} using a stack. 

The following is an algorithm for converting infix notation to postfix
notation:
\begin{enumerate}
\item[1.]  Create an empty stack called \texttt{opStack} for keeping operators. 
\item[2.]  Scan the token list from left to right. 
\begin{itemize}
\item If the token is an operand, append it to the end of the output list. 
\item If the token is a left parenthesis, push it on the \texttt{opStack}. 
\item If the token is an operator, \texttt{{*}, /, +}, or \texttt{-}, push
it on the \texttt{opStack}. However, first remove any operators already
on the \texttt{opStack} that have higher or equal precedence and append
them to the output list. 
\item If the token is a right parenthesis, pop the \texttt{opStack} until
the corresponding left parenthesis is removed. Append each operator
to the end of the output list. 
\end{itemize}
\end{enumerate}
When the input expression has been completely processed, check the
\texttt{opStack}. Any operators still on the stack can be removed
and appended to the end of the output list. 

\subsection*{Task 4.2 }

Write program code for the algorithm to convert infix notation to
postfix notation using the following specification: 
\noindent \begin{center}
\texttt{FUNCTION infixToPostfix(infixexpression:STRING):STRING}
\par\end{center}

\subsection*{Evidence 17: }

Your program code. \hfill{}{[}7{]}

The following algorithm can be used to evaluate postfix notation: 
\begin{enumerate}
\item 1. Create an empty stack called \texttt{operandStack}. 
\item 2. Scan the token list from left to right. 
\begin{itemize}
\item If the token is an operand, convert it from a string to an integer
and push the value onto the \texttt{operandStack}. 
\item If the token is an operator, \texttt{{*}, /, +, }or \texttt{-}, it
will need two operands. Pop the \texttt{operandStack} twice. The first
pop is the second operand and the second pop is the first operand.
Perform the arithmetic operation. Push the result back on the \texttt{operandStack}.
\end{itemize}
\end{enumerate}
When the input expression has been completely processed, the result
is on the stack. Pop the \texttt{operandStack} and return the value.

\subsection*{Task 4.3 }

Write program code for the algorithm to evaluate postfix notation
using the following specification: 
\noindent \begin{center}
\texttt{FUNCTION postfixEval(postfixexpression:STRING):FLOAT }
\par\end{center}

\subsection*{Evidence 18: }

Your program code.\hfill{} {[}7{]} 

\subsection*{Task 4.4}

Write program code to read the infix expressions from the file Infix.txt
and output its postfix expressions and its evaluation. 

\subsection*{Evidence 19: }

Your program code. \hfill{}{[}2{]}

\subsection*{Evidence 20: }

Screenshot of running your program code in Task 4.4.\hfill{} {[}2{]}