\item \textbf{{[}NYJC/PRELIM/9569/2021/P2/Q4{]} }

A bookstore uses a text file to store data about its inventory of
books. The bookshop carries two kinds of books: printed books and
virtual books. The bookshop wishes to transfer this information into
a database. 

The bookshop also wishes to create an online bookstore that allows
users to add books to a shopping cart for purchase.

\subsubsection*{Task 4.1}

Create an SQL file called \texttt{TASK4\_1\_<centre number>\_<index
number>.sql} to show the SQL code to create database \texttt{bookstore.db}
with three tables: \texttt{Book}, \texttt{Printed}, and \texttt{Virtual}.
The Printed and Virtual tables represent physical and virtual books
respectively, and stores properties unique to each type of book. 

The \texttt{Book} table will have the following fields: 
\begin{itemize}
\item \texttt{BookID} -- the primary key, an integer value 
\item \texttt{Title} -- the title of the book 
\item \texttt{Price} -- the price of the book, in cents 
\item \texttt{Type} -- the type of book: \textquotedbl physical\textquotedbl{}
or \textquotedbl virtual\textquotedbl{} 
\end{itemize}
The \texttt{Printed} table will have the following additional field: 
\begin{itemize}
\item \texttt{Weight} -- the weight of the book 
\end{itemize}
The \texttt{Virtual} table will have the following additional field: 
\begin{itemize}
\item \texttt{DownloadLink} -- the download link for the book 
\end{itemize}
Save your SQL code as 

\texttt{TASK4\_1\_<your name>\_<centre number>\_<index number>.sql}
\hfill{}{[}6{]}

\subsubsection*{Task 4.2 }

Python programming language and object-oriented programming will be
used to implement the online bookstore and shopping cart on a web
page. 

The class \texttt{Book} will store the following data: 
\begin{itemize}
\item \texttt{title} -- stored as a string 
\item \texttt{price} -- stored as an integer 
\end{itemize}
The class \texttt{Cart} will store the following data: 
\begin{itemize}
\item \texttt{items} -- stored as a list of \texttt{Book} objects 
\end{itemize}
The class \texttt{Cart} has a method defined on it: 
\begin{itemize}
\item \texttt{total\_price()} -- returns an integer representing the total
price of books in the cart 
\end{itemize}
Save your program code as 

\texttt{TASK4\_2\_<your name>\_<centre number>\_<index number>.py}
\hfill{}{[}6{]}

The \texttt{PrintedBook} class inherits from \texttt{Book}, and stores
the following additional data: 
\begin{itemize}
\item weight -- stored as an integer 
\end{itemize}
\texttt{Virtu}alBook class inherits from \texttt{Bo}ok, and stores
the following additional data: 
\begin{itemize}
\item download\_link -- stored as a string 
\end{itemize}
Add your program code to 

\texttt{TASK4\_2\_<your name>\_<centre number>\_<index number>.py}
{[}3{]} 

The text file, \texttt{bookstore.txt}, contains data items for books
stocked by the bookstore. Each data item is separated by a comma,
with each book\textquoteright s data on a new line as follows: 
\begin{itemize}
\item book title 
\item price 
\item type 
\item weight 
\item download link 
\end{itemize}
Write program code to read in the information from the text file,
\texttt{bookstore.txt}, creating an instance of the appropriate class
for each book (either \texttt{PrintedBook} or \texttt{VirtualBook}).
\hfill{}{[}4{]}

Write program code to insert all information from the file into the
\texttt{bookstore.db} database.

Run the program. Add your program code to 

\texttt{TASK4\_2\_<your name>\_<centre number>\_<index number>.py}
\hfill{}{[}8{]}

\subsubsection*{Task 4.3 }

The data from the text file, \texttt{bookstore.txt}, is to be used
to implement a shopping cart in a web browser. 

Write a Python program and the necessary files to create a web application
that: 
\begin{itemize}
\item displays a list of books stocked by the bookstore 
\item enables the user to add books to a shopping cart using an ID 
\item displays the contents of the shopping cart
\item shows the total price of items in the shopping cart 
\end{itemize}
For each book displayed the web page should include the: 
\begin{itemize}
\item book ID 
\item book title 
\item price 
\end{itemize}
Save your program as 

\texttt{TASK4\_3\_<your name>\_<centre number>\_<index number>.py }

with any additional files / sub-folders as needed in a folder named 

\texttt{TASK4\_3\_<your name>\_<centre number>\_<index number>}\hfill{}
{[}7{]}

Run the web application and add the following books to the shopping
cart: 
\begin{itemize}
\item Title: \textquotedbl Northanger Abbey\textquotedbl , Price: 13.99,
Type: Physical, Weight: 178g 
\item Title: \textquotedbl War and Peace\textquotedbl , Price: 17.49,
Type: Physical, Weight: 432g 
\item Title: \textquotedbl Computer Programs\textquotedbl , Price: 20.99,
Type: Virtual, Link: https://mybookstore.com/dJHtFy 
\item Title: \textquotedbl Data Science\textquotedbl , Price: 14.99, Type:
Virtual, Link: https://mybookstore.com/fJynJk 
\end{itemize}
Save the output of the program as

\texttt{TASK4\_3\_<your name>\_<centre number>\_<index number>.html}
\hfill{}{[}4{]}