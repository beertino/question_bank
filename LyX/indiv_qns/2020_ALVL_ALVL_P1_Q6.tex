\item \textbf{{[}ALVL/9569/2020/P1/Q6{]} }

A college is designing a database to store data about:
\begin{itemize}
\begin{singlespace}
\item students
\item courses
\item subjects
\item teachers
\item classrooms.
\end{singlespace}
\end{itemize}
The designers are told that:
\begin{itemize}
\begin{singlespace}
\item each student takes four courses
\item each teacher delivers all their lessons in one room 
\item each room may be used by more than one teacher 
\item each teacher may teach more than one course
\item a course can only be taught by one teacher
\item a subject may be taught by more than one teacher. 
\end{singlespace}
\end{itemize}
A first attempt is represented by the following table:

\begin{tabular}{|c|c|c|c|c|c|c|c|}
\hline 
\multirow{2}{*}{\textbf{Student ID}} & \multirow{2}{*}{\textbf{First Name}} & \multirow{2}{*}{\textbf{Last Name}} & \textbf{Course} & \multirow{2}{*}{\textbf{Subject}} & \textbf{Teacher} & \textbf{Teacher} & \textbf{Room}\tabularnewline
 &  &  & \textbf{ID} &  & \textbf{ID} & \textbf{Name} & \textbf{Number}\tabularnewline
\hline 
\hline 
1279 & Joe & Smith & 934 & Geography & 334 & Mansoor & 12\tabularnewline
\hline 
 &  &  & 926 & Maths & 451 & Yang & 16\tabularnewline
\hline 
 &  &  & 882 & Physics & 628 & Lee & 12\tabularnewline
\hline 
 &  &  & 425 & Computing & 329 & James & 14\tabularnewline
\hline 
1395 & Muhammad & Hilmi & 934 & Geography & 334 & Mansoor & 12\tabularnewline
\hline 
 &  &  & 927 & Maths & 723 & Morris & 8\tabularnewline
\hline 
 &  &  & 883 & Physics & 534 & Weston & 10\tabularnewline
\hline 
 &  &  & 586 & French & 271 & Dubois & 16\tabularnewline
\hline 
2883 & Sumiko & Chong & 425 & Computing & 329 & James & 14\tabularnewline
\hline 
 &  &  & 882 & Physics & 628 & Lee & 12\tabularnewline
\hline 
 &  &  & 934 & Geography & 334 & Mansoor & 12\tabularnewline
\hline 
 &  &  & 586 & French & 271 & Dubois & 16\tabularnewline
\hline 
\end{tabular}
\begin{enumerate}
\item Explain why this table is not in first normal form (1NF). \hfill{}{[}2{]}
\end{enumerate}
The following is an attempt to reduce data redundancy. 

Student

\begin{tabular}{|c|c|c|}
\hline 
\textbf{Student ID} & \textbf{First Name} & \textbf{Last Name}\tabularnewline
\hline 
\hline 
1279 & Joe & Smith\tabularnewline
\hline 
1395 & Muhammad & Hilmi\tabularnewline
\hline 
2883 & Sumiko & Chong\tabularnewline
\hline 
\end{tabular}

Course

\begin{tabular}{|c|c|c|c|c|}
\hline 
\textbf{Course ID} & \textbf{Subject} & \textbf{Teacher ID} & \textbf{Teacher Name} & \textbf{Room Number}\tabularnewline
\hline 
\hline 
934 & Geography & 334 & Mansoor & 12\tabularnewline
\hline 
926 & Maths & 451 & Yang & 16\tabularnewline
\hline 
882 & Physics & 628 & Lee & 12\tabularnewline
\hline 
425 & Computing & 329 & James & 14\tabularnewline
\hline 
927 & Maths & 723 & Morris & 8\tabularnewline
\hline 
883 & Physics & 534 & Weston & 10\tabularnewline
\hline 
586 & French & 271 & Dubois & 16\tabularnewline
\hline 
\end{tabular}

IsTaking

\begin{tabular}{|c|c|}
\hline 
\textbf{Student ID} & \textbf{Course ID}\tabularnewline
\hline 
\hline 
1279 & 934\tabularnewline
\hline 
1279 & 926\tabularnewline
\hline 
1279 & 882\tabularnewline
\hline 
1279 & 425\tabularnewline
\hline 
1395 & 934\tabularnewline
\hline 
1395 & 927\tabularnewline
\hline 
1395 & 883\tabularnewline
\hline 
1395 & 586\tabularnewline
\hline 
2883 & 425\tabularnewline
\hline 
2883 & 882\tabularnewline
\hline 
2883 & 934\tabularnewline
\hline 
2883 & 586\tabularnewline
\hline 
\end{tabular}

\newpage
\begin{enumerate}
\item[\textbf{(b)}] Give suitable primary keys for each of the following three tables. 
\begin{enumerate}
\item Student \hfill{}{[}1{]}
\item Course \hfill{} {[}1{]}
\item lsTaking \hfill{}{[}1{]}
\end{enumerate}
\item[\textbf{(c)}] Create an entity-relationship (ER) diagram showing the degree of
all relations. \hfill{} {[}3{]}
\item[\textbf{(d)}] Explain why table Course is not in third normal form (3NF). \hfill{}
{[}2{]}
\item[\textbf{(e)}] A table description can be expressed as: 

\texttt{TableName (Attributel, Attribute2, Attribute3 , ...) }

The primary key is indicated by underlining one or more attributes.
Foreign keys are indicated by using a dashed underline. Write table
descriptions for the required tables in the database so they are in
third nennai form (3NF). \hfill{}{[}7{]}
\item[\textbf{(e)}] Explain the reasons for reducing data redundancy in a relational
database. \hfill{} {[}2{]}
\item[\textbf{(f)}] Write an SQL query to output the subjects, teacher names and room
numbers for the courses taken by the student with Student ID of 1395. 

The output is to be in alphabetical order of subject. \hfill{}{[}5{]}
\end{enumerate}