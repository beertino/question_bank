\item \textbf{{[}NYJC/PRELIM/9597/2019/P1/Q1{]} }

The number of rainy days for each year and month is stored in the
file \texttt{RAINFALL.txt}. The first line of the file contains the
heading description for the data. Each line of data is stored in the
format \texttt{YYYY-MM,99} where \texttt{YYYY} is the year, \texttt{MM}
is the month and \texttt{99} is the number of days. Thus '\texttt{1982-
01,10}' means there were 10 rainy days in the month of January, 1982. 

You are required to write a program to: 
\begin{itemize}
\item Read the data in the file. 
\item Calculate the total number of rainy days for each year by adding all
the months\textquoteright{} rainy days for that year. 
\item Create a new file \texttt{RAINFALLYEAR.txt}. 
\item Write the heading description in the first line as \textquotedbl Year,Rainy
Days\textquotedblright . 
\item For each subsequent line, write to the file in the format \texttt{YYYY,999}
where \texttt{YYYY} is the year and \texttt{999} is the total number
of rainy days (up to 3 digits) for that year. 
\end{itemize}

\subsection*{Task 1.1 }

Write program code for this task. 

\subsection*{Evidence 1: }

Your program code. \hfill{}{[}8{]}

\subsection*{Task 1.2 }

Write program code for a procedure \texttt{ShowMenu} to display the
following menu: 
\begin{enumerate}
\item[1.] \texttt{ Query total rainy days in any year}
\item[2.] \texttt{ Query by year the month of highest rainy days }
\item[3.] \texttt{ -1 to Exit}
\end{enumerate}

\subsection*{Evidence 2: }

Your program code.\hfill{} {[}2{]}

\subsection*{Task 1.3 }

Implement a program that displays \texttt{ShowMenu} and asks the user
for their choice. Create functions \texttt{Query1} and \texttt{Query2}
which corresponds to the menu selection option \texttt{1} and \texttt{2}
respectively. When option \texttt{1} is selected, \texttt{Query1}
should run and ask the user to input the year. \texttt{Query1} will
return the total number of rainy days or a suitable message if data
for that year is not available. 

When option \texttt{2} is selected, \texttt{Query2} should execute
and ask the user for the year. \texttt{Query2} will return the month
with the highest number of rainy days in that year in words (e.g.
January, August, or December) or a suitable message if data for that
year is not available. 

For both \texttt{Query1} and \texttt{Query2}, appropriate validation
of the user input for year should be done. The program will display
\texttt{ShowMenu} after each valid query until option \texttt{3} is
selected.

\subsection*{Evidence 3: }

Your program code. \hfill{}{[}8{]}

\subsection*{Task 1.4 }

Design 3 test data which tests the functionality of your program. 

\subsection*{Evidence 4: }

A screenshot for each test case you considered. Annotate the screenshot
explaining the purpose of each test. \hfill{}{[}3{]}