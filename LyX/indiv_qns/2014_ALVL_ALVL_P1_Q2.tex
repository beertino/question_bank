\item \textbf{{[}ALVL/9597/2019/P1/Q2{]} }

A binary search (binary chop) is a technique to search for a value
in an ordered dataset.

\subsubsection*{Task 2.1}

Study the identifier table and incomplete recursive algorithm. 

The missing parts of the algorithm are labelled A, B and C.
\begin{center}
\begin{tabular}{|l|c|l|}
\hline 
\texttt{\hspace{0.01\columnwidth}}Variable & \texttt{\hspace{0.01\columnwidth}}Data Type & \texttt{\hspace{0.05\columnwidth}}Description\tabularnewline
\hline 
\texttt{ThisArray} & \texttt{ARRAY OF STRING} & Array containing the the dataset\tabularnewline
\hline 
\texttt{FindValue} & \texttt{STRING} & Item to be found\tabularnewline
\hline 
\texttt{Low} & \texttt{INTEGER} & Lowest index of the considered list\tabularnewline
\hline 
\texttt{High} & \texttt{INTEGER} & Highest index of the considered list\tabularnewline
\hline 
\texttt{Middle} & \texttt{INTEGER} & The array index for the middle position of the current list considered\tabularnewline
\hline 
\end{tabular}
\par\end{center}

\noindent %
\noindent\begin{minipage}[t]{1\columnwidth}%
\texttt{FUNCTION BinarySearch(ThisArray, FindValue, Low, High) RETURNS
INTEGER}

\texttt{\qquad{}DECLARE Middle : INTEGER}

\texttt{\qquad{}IF ............... A ...............}

\texttt{\qquad{}\qquad{}THEN}

\texttt{\qquad{}\qquad{}\qquad{}RETURN -1 // not found}

\texttt{\qquad{}ELSE}

\texttt{\qquad{}\qquad{}// calculate new Middle value}

\texttt{\qquad{}\qquad{}Middle \textleftarrow{} ............... B
...............}

\texttt{\qquad{}\qquad{}IF ThisArray{[}Middle{]} > FindValue}

\texttt{\qquad{}\qquad{}\qquad{}THEN}

\texttt{\qquad{}\qquad{}\qquad{}\qquad{}RETURN BinarySearch(ThisArray,
FindValue, Low, Middle - 1)}

\texttt{\qquad{}\qquad{}\qquad{}ELSE}

\texttt{\qquad{}\qquad{}\qquad{}\qquad{}IF ThisArray{[}Middle{]}
< FindValue}

\texttt{\qquad{}\qquad{}\qquad{}\qquad{}\qquad{}THEN}

\texttt{\qquad{}\qquad{}\qquad{}\qquad{}\qquad{}\qquad{}............... C
...............}

\texttt{\qquad{}\qquad{}\qquad{}\qquad{}\qquad{}ELSE}

\texttt{\qquad{}\qquad{}\qquad{}\qquad{}\qquad{}\qquad{}RETURN
Middle // found at position Middle}

\texttt{\qquad{}\qquad{}\qquad{}\qquad{}ENDIF}

\texttt{\qquad{}\qquad{}ENDIF}

\texttt{\qquad{}ENDIF}

\texttt{ENDFUNCTION}%
\end{minipage}

\subsubsection*{Evidence 3}

What are the three missing lines in this pseudocode? \hfill{}{[}3{]}

\subsubsection*{Task 2.2}

Write a program to implement binary search. 

The program will
\begin{itemize}
\item Call procedure InitialiseAnimals 
\item Input an animal name
\item Use the function BinarySearch 
\item Report whether or now this animal name was found. If found, also output
the index position.
\end{itemize}
The array in the program has identifier \texttt{MyAnimal}. 

Use the dataset given in the file \texttt{ANIMALS.TXT}. You should
paste the contents of this file into your program. The statements
will form the basis of the code for the procedure \texttt{InitialiseAnimals}.

\subsubsection*{Evidence 4}

Program code for Task 2.2 \hfill{}{[}7{]}

\subsubsection*{Evidence 5}

Screenshot to confirm that an animal wich is present in the list was
found with its index position displayed. \hfill{}{[}1{]}

\subsubsection*{Task 2.3}

Amend the program as follows: 

The program must also output the number of function calls carried
out.

\subsubsection*{Evidence 6}

The amended program code. \hfill{}{[}4{]}

\subsubsection*{Evidence 7}

Screenshots showing the amended ouput for runs of the program where: 
\begin{itemize}
\item the animal is found
\item the animal is not found. \hfill{}{[}2{]}
\end{itemize}