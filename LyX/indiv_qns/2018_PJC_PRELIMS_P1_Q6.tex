\item \textbf{{[}PJC/PRELIM/9597/2018/P1/Q6{]} }

The following pseudo-code algorithm describes one method of finding
an arbitrary visitor name in an alphabetically ordered array of N
unique names. 

\noindent %
\noindent\begin{minipage}[t]{1\columnwidth}%
set first to 1 

set last to N 

repeat 

\texttt{\qquad{}}set mid to the integer part of (first + last) /2 

\texttt{\qquad{}}If the mid name precedes the wanted name then 

\texttt{\qquad{}\qquad{}}set first to mid + 1 

\texttt{\qquad{}}else 

\texttt{\qquad{}\qquad{}}set last to mid - 1

\texttt{\qquad{}}endif 

until first > last or midth name is the wanted name%
\end{minipage}
\begin{enumerate}
\item If 142 names are stored in the array, and JOSEPH is the 44th name,
state the elements of the array that are examined when searching for
JOSEPH. \hfill{}{[}4{]}
\item If a search is made for a name that is not in the array, what is the
largest number of elements that might need to be examined before one
could say that the name is not present? Explain how you arrive at
your answer. \hfill{}{[}3{]}
\end{enumerate}