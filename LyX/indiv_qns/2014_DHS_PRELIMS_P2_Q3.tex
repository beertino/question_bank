\item \textbf{{[}DHS/PRELIM/9597/2014/P2/Q3{]} }

To enhance the security of online transactions, a one-time randomly
generated 5-digit Personal Identification Number (PIN) will be sent
to the user's registered mobile phone number as part of the authentication
process. The security algorithm, \texttt{Secret()}, makes use of the
sum of squares of the prime factors of the rightmost 5 digits of the
user's mobile number modulo $k$, where $k$ is a randomly generated
prime in the range 100000 -- 999999. 

As an illustration, consider the mobile phone number 87654321 ($n=54321$):

Sum of squares of prime factors of $n=3^{2}+19^{2}+953^{2}=908579$

PIN = 908579 modulo $k$ (let's say $k=102077$) = 91963 

So a one-time 5-digit PIN 91963 will be sent to the mobile number
87654321. 

The algorithm for finding the prime factors of a number n is detailed
recursively as follows: 
\begin{itemize}
\item If $n$ is even, one factor is 2, then find the prime factors of n/2. 
\item If $n$ is divisible by 3, remove this factor and then find the prime
factors of the remainder. 
\item If $n$ is divisible by 5, remove this factor and then find the prime
factors of the remainder. 
\item If $n$ is divisible by 7, remove this factor and then find the prime
factors of the remainder. 
\item ...
\end{itemize}
You may assume the availability of a random function \texttt{Random()}
but should indicate how it is used. 
\begin{enumerate}
\item Identify, giving examples, any pitfall(s) with your \texttt{Secret()}
algorithm, and suggest possible solution(s) to overcome them. \hfill{}{[}4{]}
\item Devise, incorporating your answer to (a), an efficient means of generating
a PIN using \texttt{Secret()}. Your function should include a recursive
algorithm for finding the prime factors of n. State explicitly any
necessary assumption(s) made. \hfill{}{[}6{]}
\end{enumerate}