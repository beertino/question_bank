\item \textbf{{[}DHS/PRELIM/9569/2021/P2/Q3{]} }

\textbf{Task 3} 

In 2021, Singapore's Health Science Authority (HSA) recalled 18 brands
of hand sanitisers due to high levels of acetaldehyde and/or methanol.
The HSA keeps information on current hand sanitisers and uses it to
monitor the types of chemical ingredients used to make the sanitisers. 

\subsubsection*{Task 3.1 }

Create an SQL file to show the SQL code to create database \texttt{sanitisers.db}
with the single table, \texttt{sanitisers}. 

The table will have the following fields: 
\begin{itemize}
\item \texttt{product\_name} which is the primary key 
\item \texttt{active\_ingredient} 
\item \texttt{alcohol-based} 
\end{itemize}
Save your SQL code as 

\texttt{TASK3\_<index\_number>\_<name>.sql} \hfill{}{[}3{]}

\subsubsection*{Task 3.2}

The text file, \texttt{sanitisers.txt}, contains data items for a
number of sanitisers. It contains a header line. Each data item is
separated by a comma, with each item data on a new line, as follows:
\begin{itemize}
\item product name 
\item active ingredient used to make the sanitiser product 
\item \textquotedbl Yes\textquotedbl{} or \textquotedbl No\textquotedbl{}
to indicate if the product is alcohol-based 
\end{itemize}
Write program code to read in the information from the text file,
\texttt{sanitisers.txt}, and insert all the information into the \texttt{sanitisers.db}
database. \hfill{}{[}3{]}

Run the program. 

Save your program as 

\texttt{TASK3\_<index\_number>\_<name>.py} 

\subsubsection*{Task 3.3}

The information is to be displayed in a web browser. 

Write a python program and the necessary files to create a web application
that enables the list of sanitisers to be displayed. 

For each record the web page should include the: 
\begin{itemize}
\item product name 
\item ingredients used to make the sanitiser product
\item \textquotedbl Yes\textquotedbl{} or \textquotedbl No\textquotedbl{}
to indicate if the product is alcohol-based 
\end{itemize}
Save your program as 

\texttt{TASK3\_<index\_number>\_<name>.py} 

with any additional files/subfolders as needed in a folder named 

\texttt{TASK3\_<index\_number>\_<name>} 

Run the web application and save the output of the program as 

\texttt{TASK3\_OUTPUT\_<index\_number>\_<name>.html}\hfill{} {[}6{]}

\subsubsection*{Task 3.4 }

HSA wants a form on the web page that allows users to enter in the
name of an active ingredient and, upon submission, will display all
the information of the products with the matching active ingredient. 

Update your application to include this form feature so that users
will be able to use the form after seeing the list of sanitisers displayed
as required in Task 3.3. 

Run the web application, test your program with the ingredient \textquotedbl\texttt{Triclosan}\textquotedbl{}
and save the output. \hfill{}{[}4{]}

Save, zip up and submit your program code and all related files for
Task 3 as 

\texttt{TASK3\_<index\_number>\_<name>.zip}