\item \textbf{{[}HCI/PRELIM/9569/2020/P2/Q4{]} }

A programmer is writing a program to manipulate different data structures
using Object-Oriented Programming. 

The superclass, \texttt{LinkedStructure}, will store the following
data:
\begin{itemize}
\item A linear linked list of data items held in an \textbf{array} of size
10. Array index starts at 1. 
\item Head pointer, pointing to the first element in the linked list 
\item Tail pointer, pointing to the last element in the linked list 
\item Free pointer, pointing to the first element in the free list
\end{itemize}
The diagram shows the linked structure after four items have been
added and the unused nodes are linked together. 
\noindent \begin{center}
<INSERT\_IMAGE\_HERE>
\par\end{center}

This superclass has the following methods:
\begin{itemize}
\item \texttt{Initialise()} method sets up an empty linked list. Should
link all nodes to form the free list. Initialise values for head pointer,
tail pointer and free pointer 
\item \texttt{Add(item)} appends the parameter into its correct \textbf{alphabetical}
order in the linked list. 
\item \texttt{Remove(item)} removes the parameter from the ordered linked
list 
\item \texttt{Display()} displays the data items in the linked list in alphabetical
order 
\item \texttt{PrintStructure()} displays the current state of all pointers
and the array contents in index order 
\item \texttt{IsEmpty()} tests for empty linked list 
\item \texttt{IsFull()} tests for no unused nodes
\end{itemize}
The superclass is used to implement a linear queue. 

The subclass \texttt{Queue} has the following methods:
\begin{itemize}
\item \texttt{Add(item)} appends the parameter to the queue and overrides
the \texttt{LinkedStructure} add method . 
\item \texttt{Remove()} returns and removes the next item in the queue
\item \texttt{Display()} method should display the queue contents in order
(e.g. the earliest added item first) and should override the \texttt{LinkedStructure}
display method. 
\end{itemize}
Each method updates its appropriate pointers, and produces suitable
errors (or returns different values) to indicate if the actions are
not possible, e.g. if the structure is empty. 

For each of the sub-tasks, add a comment statement, at the beginning
of the code using the hash symbol \textquoteleft \#\textquoteright ,
to indicate the sub-task the program code belongs to, for example:

\subsection*{Task 4.1 }

Write program code for the superclass \texttt{LinkedStructure}. \hfill{}{[}20{]}

\subsection*{Task 4.2}

Write program code to:
\begin{itemize}
\item create a \texttt{LinkedStructure} object 
\item add the following three data items in the order shown to the ordered
linked list 
\noindent \begin{center}
\texttt{Japan, Singapore, China }
\par\end{center}
\item output all pointers and array contents using the \texttt{PrintStructure()}
method after adding the items 
\item output the current contents of the linked list using the \texttt{Display()}
method 
\item remove two data items \texttt{China}, \texttt{Japan} in that order
from the linked list 
\item output all pointers and array contents using the \texttt{PrintStructure()}
method after the removal of the items. \hfill{} {[}5{]}
\end{itemize}

\subsection*{Task 4.3}

Write program code for the subclass \texttt{Queue}.

Use appropriate inheritance and polymorphism in your designs. \hfill{}{[}5{]}

\subsection*{Task 4.4}

The file \texttt{QUEUE.TXT} stores data to test your program.

Write program code to:
\begin{itemize}
\item create a new queue and add the data in the file QUEUE.TXT to the queue 
\item output the current contents of the queue
\item remove and output two items from the queue 
\item output all pointers and the array contents of the queue after the
removal of the items.
\end{itemize}
All outputs should have appropriate messages to indicate what they
are showing. \hfill{}{[}3{]}