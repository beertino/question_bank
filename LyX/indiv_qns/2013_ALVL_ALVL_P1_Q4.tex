\quad{}
\item \textbf{{[}ALVL/9597/2013/P1/Q4{]} }

The task is to input data for a frequency distribution and then output
to the screen a horizontal bar chart.

The data is input as an X value followed by its frequency. Assume
the frequency is always in the range 0 to 60 and there are no more
than six X values.

The input shown below shows the number of sweatshirts sold in a retail
shop over a one week period; for example there were 39 XL items sold

\noindent\fbox{\begin{minipage}[t]{1\columnwidth - 2\fboxsep - 2\fboxrule}%
\texttt{Next X value ... <ZZZ to END> XS }

\texttt{Frequency ... 12 }

\texttt{Next X value ... <ZZZ to END> S }

\texttt{Frequency ... 22 }

\texttt{Next X value ... <ZZZ to END> M }

\texttt{Frequency ... 45 }

\texttt{Next X value ... <ZZZ to END> L }

\texttt{Frequency ... 56 }

\texttt{Next X value ... <ZZZ to END> XL }

\texttt{Frequency ... 39 }

\texttt{Next X value ... <ZZZ to END> XXL}

\texttt{Frequency ... 11 }

\texttt{Next X value ... <ZZZ to END> ZZZ}

\texttt{++++++++++++++++++++++++++++++++++++++++}

\texttt{Frequency distribution }

\texttt{++++++++++++++++++++++++++++++++++++++++}

\texttt{XS ~@@@@@@@@@@@@ }

\texttt{S ~~@@@@@@@@@@@@@@@@@@@@@@ }

\texttt{M ~~@@@@@@@@@@@@@@@@@@@@@@@@@@@@@@@@@@@@@@@@@@@@@ }

\texttt{L ~~@@@@@@@@@@@@@@@@@@@@@@@@@@@@@@@@@@@@@@@@@@@@@@@@@@@@@@@@ }

\texttt{XL ~@@@@@@@@@@@@@@@@@@@@@@@@@@@@@@@@@@@@@@@ }

\texttt{XXL @@@@@@@@@@@}%
\end{minipage}}

\subsubsection*{Task 4.1}

Write a program which inputs a set of X values and frequencies and
produces output in the format shown.

\subsubsection*{Evidence 14}

Your program code for Task 4.1.\hfill{}{[}8{]}

\subsubsection*{Evidence 15}

A screenshot to confirm the dataset used and the output produced.
\hfill{}{[}2{]}

The appearance of the bar chart display is to be improved as follows:
\begin{itemize}
\item Each bar is to be represented by more than one line of the same character
to that its bar width is increased. 
\item Each bar will be shown with the same number of lines. 
\item The complete bar chart, including the heading, is to take up no more
than 40 lines. 
\item The line width for the output is exactly 80 characters. 
\item Its appearance could be improved by changing the @ character.
\end{itemize}

\subsubsection*{Task 4.2}

Write code to produce a new chart for the data used in Task 4.1 showing
the maximum possible bar width and any other refinements you have
introduced.

\subsubsection*{Evidence 16}

Your program code for Task 4.2. \hfill{}{[}4{]}

\subsubsection*{Evidence 17}

A screenshot showing the data entry followed by the bar chart. \hfill{}{[}2{]}

Some datasets will have a frequency which is greater than 60 and so
the frequencies of the dataset can no longer be shown with a corresponding
number of characters in the line. The frequencies will need to be
scaled before the output is attempted.

The bar chart would benefit by the inclusion of a horizontal axis
labelled with a scale showing the frequency values.

\subsubsection*{Task 4.3}

Revise your program code to meet these new requirements.

\subsubsection*{Evidence 18}

Your program code for Task 4.3.\hfill{} {[}8{]}

\subsubsection*{Evidence 19}

Screenshots demonstrating: 
\begin{itemize}
\item Dataset 1 as used in Task 4.1 which needs no scaling 
\item Dataset 2 of your choice to demonstrate frequencies which must be
scaled 
\item Dataset 3 of your choice to demonstrate frequencies which must be
scaled differently to Dataset 2 
\end{itemize}
\hfill{}{[}6{]}