\item \textbf{{[}NYJC/PRELIM/9597/2019/P2/Q5{]} }

When a customer orders goods over the phone, the cashier will record
the order in an order form containing the items ordered and quantity,
customer address, delivery date and time and the amount payable. A
copy of this form will be given to the storeman who will pick the
goods and generate a delivery order (DO). The DO will be given to
the delivery man who will deliver the goods. The customer on collecting
the goods will sign on the DO and return a signed copy to the delivery
man. On his return, the delivery man will give the DO to the accounts
department who will generate an invoice. Invoices are kept in a file
until the next day where they will be mailed to the customers.
\begin{enumerate}
\item Draw a data flow diagram of the above processes.\hfill{} {[}8{]} 
\item Goods in the warehouse are divided into 2 main categories -- Kitchen
appliances (e.g. kettle, toasters and ovens) and Entertainment products
(e.g. LCD television, mp3 players and gaming consoles). Each item
has an item name, description, unit price and quantity on hand. Kitchen
appliances have an item weight, packing volume and colour. Entertainment
products have a serial number, country of manufacture and recommended
retail price.
\begin{enumerate}
\item Draw a class diagram of the above showing inheritance, their private
attributes and public methods.\hfill{} {[}6{]}
\item What is the purpose of a public method?\hfill{} {[}1{]}
\item What is the difference between a class and an object?\hfill{} {[}2{]}
\end{enumerate}
\item In relation to the diagram in part (b), explain the terms: 
\begin{enumerate}
\item Encapsulation; \hfill{}{[}2{]}
\item Inheritance;\hfill{} {[}2{]}
\item Data hiding; \hfill{}{[}2{]}
\item Polymorphism. \hfill{}{[}2{]}
\end{enumerate}
\end{enumerate}