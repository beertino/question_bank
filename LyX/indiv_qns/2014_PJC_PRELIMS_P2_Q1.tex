\item \textbf{{[}PJC/PRELIM/9597/2014/P2/Q1{]} }

A manual system for producing school student reports works in the
following manner: 
\begin{itemize}
\item a subject report is completed for each subject that a student takes
by the single teacher teaching that subject; 
\item to help the subject teacher, initially a blank report form is issued
to the student for the student to add their details: student NRIC,
name, contact phone, teacher and class; 
\item the subject report is completed by the teacher with appropriate comments; 
\item all subject reports for the student are passed to the student's tutor; 
\item the tutor puts all the subject reports together to form the student's
report folder; 
\item the tutor adds a tutor's report including attendance data supplied
by the school administration attendance records; 
\item each student's report folder is copied; 
\item the copy is filed in the report storage facility for the school; 
\item the report folder is sent to the student's parents. 
\end{itemize}
The school has decided to replace this manual system with a computerised
system. 

A system developer is employed to carry out the task. The first task
assigned to the system developer is to write a project proposal.
\begin{enumerate}
\item One section of the project proposal is the Problem Statement which
lists the problems in the current system. Write the Problem Statement.\hfill{}
{[}4{]}
\item The proposal is accepted and the main stages of the project have been
identified and durations assessed as follows: 

\begin{tabular}{|c|l|c|l|}
\hline 
Stage & Description & Weeks & Staffing\tabularnewline
\hline 
\hline 
A & analysis of the solution & 4 & Analyst A1\tabularnewline
\hline 
B & design of the solution & 8 & Analysts A2 and A3 \tabularnewline
\hline 
C & development of the solution & 12 & Programmers P1 and P2\tabularnewline
\hline 
D & documentation of the solution & 8 & Clerks C1 and C2\tabularnewline
\hline 
E & implementation of the solution & 6 & Programmer P1\tabularnewline
\hline 
F & testing of the solution & 4 & Programmers P3 and P4\tabularnewline
\hline 
\end{tabular}

B and D cannot start until A is completed 

F and C cannot start until B is completed 

E cannot start until C is completed 

The project will end when D, E and F are completed.
\begin{enumerate}
\item Draw a Program Evaluation and Review Technique (PERT) chart for these
6 project stages (A to F). \hfill{}{[}4{]}
\item Calculate and display on the diagram, with a node layout key, the
earliest and latest start and finish times of each task. \hfill{}{[}4{]}
\item State the critical path.\hfill{} {[}1{]}
\item State the minimum time in which the project could be completed. \hfill{}{[}1{]}
\item Explain dependent stages and concurrent stages. For each type of stage
give an example from this chart. \hfill{}{[}4{]}
\item A decision is made that the PERT chart should show more detail with
regard to testing. It is proposed that stage F (testing) should be
removed from the chart and three new stages added:

L -- black box testing -- 2 weeks 

M -- white box testing -- 2 weeks

N -- beta testing -- 3 weeks 

Redraw the PERT chart to show the effect of these changes.\hfill{}
{[}2{]}
\item Draw a Gantt chart showing all \textbf{eight} stages and their dependencies,
allowing for the resource allocations as indicated above. \hfill{}{[}4{]}
\item List and explain briefly \textbf{TWO} advantages and \textbf{ONE}
disadvantage of using a Program Evaluation and Review Technique (PERT)
chart for a project plan in comparison with using a Gantt chart. \hfill{}{[}3{]}
\end{enumerate}
\item Identify \textbf{FIVE} key stages with brief description of the software
development life cycle (SDLC). \hfill{}{[}3{]}
\item At which stage of the SDLC is top-down analysis used? Explains why
it helps in the solution of complex problems. \hfill{}{[}2{]}
\item The attendance data enter into the new system needs to be validated
and verified. Explain with examples the difference between data validation
and data verification. \hfill{}{[}4{]}
\item Marek is designing a program for this computerised system. His test
strategy includes beta testing and acceptance testing.
\begin{enumerate}
\item Describe what is meant by beta testing and how it can be used to test
Marek\textquoteright s program. \hfill{} {[}2{]}
\item Describe what is meant by acceptance testing and how it can be used
to test Marek\textquoteright s program. \hfill{}{[}2{]}
\end{enumerate}
\item Teachers spend part of their week working from home. A system analyst
will assist in improving their school communication systems. Explain
why it is important to define problem accurately. {[}2{]} 
\item Subject teachers and tutors are worried because so much information
is being stored about their students on the server of the school.
Describe the fears that the teachers may have and explain what the
school can do to allay those fears. \hfill{}{[}3{]}
\item When data is transmitted between devices on a network it is liable
to corruption. One way of checking data for corruption is to carry
out a check sum. What is check sum? \hfill{}{[}1{]}
\item Explain another method of checking data to ensure that it has been
transmitted without corruption. \hfill{}{[}2{]}
\item When data is transmitted on a network it can use a number of different
transmission modes. State what is meant by each of the following modes
of data transmission.
\begin{enumerate}
\item Simplex \hfill{}{[}1{]}
\item Duplex \hfill{}{[}1{]}
\item Half-duplex \hfill{} {[}1{]}
\end{enumerate}
\end{enumerate}