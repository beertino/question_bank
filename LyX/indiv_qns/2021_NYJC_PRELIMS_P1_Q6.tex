\item \textbf{{[}NYJC/PRELIM/9569/2021/P1/Q6{]} }

Alice, a programmer, is implementing a DNS cache using a hash table. 
\begin{enumerate}
\item Explain the purpose of the hash function in a hash table. \hfill{}
{[}2{]}
\end{enumerate}
The description for a particular hashing algorithm using rolling polynomials
is as follows: 

\noindent %
\noindent\begin{minipage}[t]{1\columnwidth}%
For each character in the data, do the following: 
\begin{enumerate}
\item[1]  Let \texttt{i} represent the position of the character (1st char
= 1, 2nd char = 2, ...) 
\item[2]  Let \texttt{ascii} represent the ASCII value of the character 
\item[3]  Calculate the sum of \texttt{i}\texttimes (31\textasciicircum\texttt{ascii})
for all characters 
\end{enumerate}
%
\end{minipage}
\begin{enumerate}
\item[(b)]  Implement this algorithm in pseudocode. 

You may assume that the function \texttt{Ord()} is available, which
takes in a single character and returns the ASCII value of the character.
\hfill{} {[}4{]}
\item[(c)]  Bob, another programmer, suggests that a Binary Search Tree would
be a more appropriate data structure for the DNS cache. 
\begin{enumerate}
\item Describe one advantage of using a hash table for the DNS cache.\hfill{}
{[}2{]}
\item Describe one advantage of using a Binary Search Tree for the DNS cache.\hfill{}
{[}2{]}
\end{enumerate}
\item[(d)]  {}
\begin{enumerate}
\item State the algorithm used to retrieve the sorted contents of a cache
from a Binary Search Tree.\hfill{} {[}1{]}
\item Using any appropriate diagrams, pseudocode, or other appropriate method,
show how this algorithm might be carried out. \hfill{} {[}5{]}
\item Explain why the Binary Search Tree might need to be periodically recreated.\hfill{}
{[}3{]}
\end{enumerate}
\end{enumerate}