\item \textbf{{[}ACJC/PRELIM/9569/2021/P1/Q8{]} }

A programmer is tasked to write a program to store the examination
scores of students in the entire school. For each student, the database
would need to store the following data: name, form class, subject
class, subject, score and subject teacher.
\begin{enumerate}
\item Suggest a suitable data type for each of the following fields: 
\begin{itemize}
\item Name 
\item Class 
\item Score \hfill{}{[}2{]}
\end{itemize}
\item The programmer is considering storing this data in either a relational
or non-relational database.
\begin{enumerate}
\item State three key differences between the two types of databases. \hfill{}{[}3{]}
\item State and explain which database system the programmer should choose.
\hfill{}{[}3{]}
\end{enumerate}
\end{enumerate}
A healthcare group would like to store patient data in a relational
database. When a patient visits the clinic, the clinic will record
the following information:
\begin{itemize}
\item Date and time of visit 
\item Name of attending doctor and his NRIC 
\item Patient name and NRIC number
\end{itemize}
After the doctor has attended to the patient, the patient would be
given a prescription. The prescription would record the medication
which the patient is supposed to take. Each prescription, which consists
of at least one medicine, would have its own unique identification
number. Each medicine would have a unique identification number, name
and price. 

The data is stored in a relational database with five tables: \texttt{Patient},
\texttt{Doctor}, \texttt{Appointment}, \texttt{Prescription} and \texttt{Medicine}.
\begin{enumerate}
\item[(c)]  Draw the Entity-Relationship (E-R) diagram to show the tables in
third normal form (3NF) and the relationships between them. \hfill{}{[}7{]}
\item[(d)]  A table description can be written as:

\texttt{TableName (Attribute1, Attribute2, Attribute3, \dots )}

The primary key is indicated by underlining one or more attributes.
Foreign keys are indicated by using a dashed underline.

Using the information provided, write table descriptions for the tables
you identified in part (c). \hfill{}{[}8{]}
\end{enumerate}