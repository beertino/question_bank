\item \textbf{{[}ALVL/9597/2013/P2/Q3{]} }

A hash table has an index range of 1 to 900. The following pseudocode
describes an algorithm for searching the table using the hashing function
Hash. It is assumed that the key is present in the table.

\noindent %
\noindent\begin{minipage}[t]{1\columnwidth}%
\texttt{01 Index <- Hash(Key) }

\texttt{02 WHILE Table{[}Index, 1{]} <> Key }

\texttt{03 \qquad{}Index <- Index + 1 }

\texttt{04 ENDWHILE }

\texttt{05 Value <- Table{[}Index, 2{]}}%
\end{minipage}
\begin{enumerate}
\item Explain the purpose of:
\begin{enumerate}
\item line 3
\item line 5\hfill{} {[}4{]}
\end{enumerate}
\item Describe a problem that might occur with a key which, when hashed,
produces an index of 900. \hfill{}{[}2{]}
\item What modification to the algorithm is required to overcome this problem?
\hfill{}{[}3{]}
\item Explain how a new item can be added to this hash table. \hfill{}{[}4{]}
\end{enumerate}