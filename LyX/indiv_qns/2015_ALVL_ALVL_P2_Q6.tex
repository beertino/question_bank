\item \textbf{{[}ALVL/9597/2015/P2/Q6{]} }

Students from several schools are entered for examinations in different
subjects. 

A relational database is to be used by the examination board to store
data about examination entries and results. Four tables present in
the database are STUDENT, SCHOOL, SUBJECT and STUDENT-SUBJECT.

Every time a student registers for a subject examination. a new row
is created in the STUDENT-SUBJECT table. When the result becomes available,
this is added to the appropriate row.

Each student, each school, and each subject has a unique identification
code.
\begin{enumerate}
\item {}
\begin{enumerate}
\item Draw an Entity-Relationship (E-R) diagram to show the relationship
between the STUDENT table and the SUBJECT table. \hfill{}{[}1{]}
\item State the type of relationship that exists between the STUDENT and
SUBJECT tables. \hfill{}{[}1{]}
\end{enumerate}
\item Draw an E-R diagram to show the relationship between the four tables
that provides for a fully normalised database design.\hfill{} {[}3{]}
\end{enumerate}
A table description can be expressed as: 

\texttt{TableName(Attribute1, Attribute2, Attribute3, ...)} 

The primary key is indicated by underlining one or more attributes. 

An incomplete STUDENT table is: 

\texttt{STUDENT(}\texttt{\uline{StudentID}}\texttt{, StudentName,
DateOfBirth) }
\begin{enumerate}
\item[(c)]  Give a table description for the SUBJECT table. Ensure there are
\textbf{two} attributes in addition to the primary key. \hfill{}{[}3{]}
\item[(d)]  Give a table description for the STUDENT-SUBJECT table. Ensure there
is \textbf{one} non-key attribute.\hfill{} {[}3{]}
\item[(e)] {} 
\begin{enumerate}
\item State the type of relationship that exists between the STUDENT and
the SCHOOL tables.\hfill{} {[}1{]}
\item Explain how the relationship between the STUDENT table and the SCHOOL
table is established. \hfill{}{[}3{]}
\end{enumerate}
\end{enumerate}