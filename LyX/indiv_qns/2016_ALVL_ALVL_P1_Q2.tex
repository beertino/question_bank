\item \textbf{{[}ALVL/9597/2016/P1/Q2{]} }

Customers are identified by ID numbers. These ID numbers are to be
stored in a hash table. The hashing function to be used is 
\begin{center}
\texttt{Address <- IDnumber MOD Max }
\par\end{center}

The hash table is implemented as a one-dimensional array with elements
indexed 0 to \texttt{(Max-1)}. 

\subsubsection*{Task 2.1}

Write program code to: 
\begin{itemize}
\item Read ID numbers from a text file and store them in a hash table. For
the purpose of testing the program. Max is to be set to the value
20. 

Assume different IDs will hash to different addresses (no collisions). 
\item Print out the contents of the hash table in the order in which the
elements are stored in the array.
\end{itemize}
Use \texttt{KEYS.TXT} to test your program code.

\subsubsection*{Evidence 3}

Your program code. 

Screenshot of the program output.\hfill{} {[}7{]}

\subsubsection*{Task 2.2}

Amend your program code so that collisions can be managed using open
hashing. This means a collision is resolved by searching sequentially
from the hashed address for an empty location and storing the ID at
this empty location. 

Use \texttt{KEYS2.TXT} to test your program code. 

\subsubsection*{Evidence 4}

Your program code.

Screenshot of the program output. \hfill{}{[}4{]}

\subsubsection*{Task 2.3}

Add code to your Task 2.2 program. 

The program is to: 
\begin{itemize}
\item Take as input an ID number 
\item Search the hash table and output the address (index number) of the
hash table where the ID was found. 
\end{itemize}
Use \texttt{KEYS2.TXT} to test your program code. 

Run the program three times. Use the following inputs: 37, 77 and
97.

\subsubsection*{Evidence 5}

Your program code.

Screenshot of the program output. \hfill{}{[}7{]}