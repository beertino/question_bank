\item \textbf{{[}ALVL/9597/2019/P2/Q4{]} }

A company operates a multi-storey car park. All parking bays are identified
by a letter. indicating the floor. and a number indicating the position
of the bay on that floor (for example. C34 indicates bay 34 on floor
C). 

The entrance to the car park is controlled by a barrier. Before the
barrier lifts to allow a car to enter, the driver must press a button
to indicate if they need a standard bay or a special bay. 

Special bay users must present a card to a card reader at the barrier. 

The car park has an additional third type of bay that has a charging
point for electric vehicles. The hourly rate for these bays is not
the same as standard bays. The cost of using this type of bay additionally
depends on the cost of the electricity used. This is monitored by
the charging device and stored.

A camera captures the vehicle registration number. A ticket is printed
showing:
\begin{itemize}
\item current time
\item vehicle registration number 
\item floor letter 
\item position number of a suitable bay where the car must be parked 
\item the card number for the special bay, if a card had been presented
at the barrier. 
\end{itemize}
When the driver takes the ticket from the printer. the entrance barrier
lifts. Before a car is allowed to leave, the ticket must be presented
and a charge paid. The charge is determined by the length of stay
and type of bay. The hourly rate for a standard bay is not the same
as that for a special bay. As a car approaches the exit barrier a
camera captures the vehicle registration. The barrier only lifts if
the charge for this vehicle has been paid. 

This system is to be implemented using object-oriented programming
(OOP). 

The base class PARKING\_BAY has a property to store whether or not
a bay is occupied. 
\begin{enumerate}
\item Draw a class diagram, showing:
\begin{itemize}
\item any derived classes and inheritance from the base class
\item the properties needed in the base, and any derived classes
\item suitable methods to support the system with at least one getter and
one setter method. \hfill{}{[}8{]}
\end{itemize}
\item Add a class, CAR\_PARK. thathas properties to store:
\begin{itemize}
\item a list of all bays
\item the number of unoccupied bays. \hfill{}{[}3{]}
\end{itemize}
\item Explain why polymorphism is useful in object-oriented programming.
\hfill{}{[}2{]}
\item Explain the purpose of making the attributes of an object private.
\hfill{}{[}2{]}
\end{enumerate}