\item \textbf{{[}ALVL/9597/2015/P1/Q4{]} }

Users of a local area network each have a network account ID. The
IDs have the format 2015\_NNNN. where N is a digit. 

\subsubsection*{Task 4.1}

Complete the test case table with the addition of \textbf{three} more
invalid User IDs. The reasons for their invalidity should be different. 

The return value is a code as follows: 
\begin{itemize}
\item 0 - valid User lD 
\item 1 - the User ID was not 9 characters 
\item you will use other integer numbers for other invalid cases. 
\begin{center}
\begin{tabular}{|>{\centering}p{0.1\columnwidth}|>{\centering}p{0.1\columnwidth}|c|c|}
\hline 
Test Number & User ID & Return value & Explanation of the test case\tabularnewline
\hline 
1 & \texttt{2015\_0987} & \texttt{0} & Valid User ID\tabularnewline
\hline 
2 &  &  & \tabularnewline
\hline 
3 &  &  & \tabularnewline
\hline 
4 &  &  & \tabularnewline
\hline 
\end{tabular}
\par\end{center}

\end{itemize}
\hfill{}{[}10{]}

\subsubsection*{Evidence 13}
\begin{itemize}
\item The completed test case table. \hfill{}{[}6{]}
\end{itemize}

\subsubsection*{Task 4.2}

Write program code for a function to validate a User ID. The function
header has the format: 
\begin{center}
\texttt{FUNCTION ValidateUserID (ThisUserID : STRING) RETURNS INTEGER }
\par\end{center}

Write a program to:
\begin{itemize}
\item Input an ID entered by the user 
\item Validate the input using the function\texttt{ ValidateUserID }
\item Output a message describing the validity of the input. 
\end{itemize}
\hfill{}{[}10{]}

\subsubsection*{Evidence 14}
\begin{itemize}
\item Program code for the function \texttt{ValidateUserID} \hfill{}{[}4{]}
\item \textbf{Three} screenshots showing the testing of Test Numbers 2,
3, and 4. \hfill{}{[}3{]}
\end{itemize}
You are to design an object-oriented program which simulates a print
queue for a printer on a local area network (LAN).The print queue
consists at any time of none, one, or more print jobs. 

Each user can send a print job from any of the terminals on the LAN.
Each terminal on the network is identified by an integer number in
the range 1 to 172. 

The program you are to design will record for each printjob: 
\begin{itemize}
\item the user lD
\item the terminal number from which the print request was sent
\item the file size (integer in Kbytes).
\end{itemize}
In practice, there are several print queues each associated with a
different printer. Each printer is identified by a short name, such
as \texttt{Room16}. 

Task 4.3

Design and write program code to define one or more classes and other
appropriate data structures for this application. 

\subsubsection*{Evidence 15}

Program code for the class(es). \hfill{}{[}6{]} 

A print queue behaves as a queue data structure. 

Assume, for testing purposes: 
\begin{itemize}
\item there is a single printer on the LAN 
\item the maximum print queue size for the printer is five print jobs. 
\end{itemize}
The main program will simulate: 
\begin{itemize}
\item the sending of print jobs to the printer by different users 
\begin{itemize}
\item that is, the addition of a print job to the print queue 
\end{itemize}
\item the output of a job from the print queue 
\begin{itemize}
\item that is, the removal of a print job from the print queue 
\end{itemize}
\end{itemize}
The program design has the following menu: 
\begin{center}
\noindent\ovalbox{\begin{minipage}[t]{1\columnwidth - 2\fboxsep - 0.8pt}%
\begin{enumerate}
\item[1]  New print job added to print queue 
\item[2] Next print job output from printer
\item[3] Current print queue displayed
\item[4] End
\end{enumerate}
%
\end{minipage}}
\par\end{center}

The program simulates the working of the print queue as follows: 
\begin{enumerate}
\item[1.] The empty print queue is initialised. 
\item[2.]  The program user selects menu options 1, 2 and 3 in any order. 
\item[3.]  The program user selects menu opt\textbf{two}ion 4.
\end{enumerate}

\subsubsection*{Task 4.4}

Write program code to: 
\begin{itemize}
\item display the main menu 
\item input the choice by the user 
\item run the appropriate code for the choice made.
\end{itemize}

\subsubsection*{Evidence 16}

The program code. \hfill{}{[}3{]}

\subsubsection*{Task 4.5}

Write program code to initialise the print queue for the \texttt{Room16}
printer. 

Write program code to display the current state of the print queue.

\subsubsection*{Evidence 17}

The program code for:
\begin{itemize}
\item initialising the print queue
\item output of the current print queue. \hfill{}{[}6{]}
\end{itemize}

\subsubsection*{Task 4.6}

Write program code to add a new print job to the print queue.

The requirement will be:
\begin{itemize}
\item program user enters data for the new print job
\item print job is added to the print queue.
\end{itemize}
Test the code by adding one new print job.

\subsubsection*{Evidence 18}
\begin{itemize}
\item Program code to add a new print job.
\item Screenshot following menu option 1 then menu option 3. \hfill{}{[}4{]}
\end{itemize}

\subsubsection*{Task 4.7}

Write program code to output the next print job from the printer.

This code will execute from menu option 2.

Test the code by:
\begin{itemize}
\item adding three print jobs
\item outputting the next print job.
\end{itemize}

\subsubsection*{Evidence 19}
\begin{itemize}
\item Program code to output next print job.
\item Screenshot following menu option 1 three times. then menu option 2,
and menu option 3. \hfill{}{[}6{]}
\end{itemize}