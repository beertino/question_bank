\item \textbf{{[}HCI/PRELIM/9597/2014/P1/Q1{]} }

A program is to process daily high scores recorded from an online
game. The program can be run every day. 

All scores are integers in the range 1 to 500.

The program reads from file \texttt{HIGHEST.txt} the current highest
score and player name from running the program on previous days. 

The program specification is to: 
\begin{itemize}
\item input up to 5 player names and their corresponding scores for today 
\item calculate and display on screen: 
\begin{itemize}
\item the highest score with the player name for today 
\item a message saying whether or not the highest score today beat the current
highest score 
\end{itemize}
\item update the file \texttt{HIGHEST.txt} if a higher score was input today. 
\end{itemize}

\subsection*{Task 1.1 }

Write program code for this task. 

\subsection*{Evidence 1: }

Your program code. \hfill{}{[}7{]}

\subsection*{Task 1.2 }

Draw up a set of test data which tests the functioning of your program.
Consider carefully all cases which could occur for both the scores
input and the two processing requirements. 

\subsection*{Evidence 2: }

A screenshot for each test case you considered. Annotate the screenshot
explaining the purpose of each test. \hfill{}{[}8{]}

A \textbf{\emph{palindrome}} is an integer that reads the same backwards
and forwards -{}- so 6, 11 and 121 are all palindromes, while 10,
12, 223 and 2244 are not (even though 010=10, we don't consider leading
zeroes when determining whether a number is a palindrome). 

\subsection*{Task 1.3 }

Write program code with the following specification: 
\begin{itemize}
\item input two integers -{}- the endpoints of an interval e.g. \texttt{10
120 }
\item output the number of integers that are palindromes in the interval. 
\end{itemize}

\subsection*{Evidence 3: }

Your program code. \hfill{}{[}8{]}

\subsection*{Evidence 4: }

Produce two screenshots showing the output of 1 4 and 10 120 by the
user.\hfill{} {[}2{]}