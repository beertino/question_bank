\item \textbf{{[}PJC/PRELIM/9597/2017/P1/Q3{]} }

A linked list of nodes is used to store data for a college. The data
include name of student and exam mark. 

The linked list Abstract Data Type (ADT) has commands to create a
new linked list, add data items to the list and display the list. 

The program to implement this ADT will use the classes \texttt{Node}
and \texttt{LinkedList} as follows: 
\begin{center}
\begin{tabular}{|l|}
\hline 
\texttt{\hspace{0.25\columnwidth}Node}\tabularnewline
\hline 
\texttt{name : STRING}\tabularnewline
\texttt{mark : INTEGER}\tabularnewline
\texttt{nextPtr : INTEGER}\tabularnewline
\hline 
\texttt{constructor()}\tabularnewline
\texttt{setName(name : STRING)}\tabularnewline
\texttt{setMark(mark : INTEGER) }\tabularnewline
\texttt{setNextPtr(ptr : INTEGER)}\tabularnewline
\texttt{getName() : STRING}\tabularnewline
\texttt{getMark() : INTEGER}\tabularnewline
\texttt{getNextPtr() : INTEGER}\tabularnewline
\hline 
\end{tabular}~~~~~~%
\begin{tabular}{|l|}
\hline 
\texttt{\hspace{0.25\columnwidth}LinkedList}\tabularnewline
\hline 
\texttt{nodes : ARRAY OF Node}\tabularnewline
\texttt{head : INTEGER}\tabularnewline
\hline 
\texttt{constructor()}\tabularnewline
\texttt{addInOrder(name, mark)}\tabularnewline
\texttt{print() }\tabularnewline
\hline 
\end{tabular}
\par\end{center}

In the \texttt{Node} class, \texttt{name} and \texttt{mark} store
a student\textquoteright s name and exam mark respectively, while
\texttt{nextPtr} is a pointer to the next node. 

In the \texttt{LinkedList} class, \texttt{head} is a pointer to the
first node in the linked list. When the linked list has no data, \texttt{head}
will be set to --1.

Data added to the linked list will be stored in alphabetical order
of name. 

The print method will output for each node, in array order, the data
and pointer of each node. Page 6 

\subsection*{Task 3.1}

Write program code to define the classes \texttt{Node} and \texttt{LinkedList}.

\subsection*{Evidence 8: }

Program code for Task 3.1. \hfill{} {[}20{]}

\subsection*{Task 3.2 }

Write code to create a linked list object in the main program, read
from data file \texttt{COLLEGE.txt} and add in all the data items,
and print the array contents. The file contains name and mark of each
student in the following format:
\noindent \begin{center}
\texttt{<name>|<mark>}
\par\end{center}

Sample record: 
\noindent \begin{center}
\texttt{Jenny Tan|49}
\par\end{center}

\subsection*{Evidence 9 }

Program code for Task 3.2. Screenshot of running Task 3.2. \hfill{}
{[}5{]}

\subsection*{Task 3.3 }

Write code for a method \texttt{countNodes} to count the number of
nodes used for the data in the linked list. 

\subsection*{Evidence 10:}

Program code for Task 3.3 countNodes. \hfill{}{[}3{]}

\subsection*{Task 3.4 }

Another method \texttt{sortByMark} is to be added to the \texttt{LinkedList}
class to sort the linked list in descending order of exam mark.

Write program code to implement this method.

Test your program code by sorting the linked list from Task 3.2 in
descending order of mark. Page 7 

\subsection*{Evidence 11 }

Program code for Task 3.4 \texttt{sortByMark}.

Screenshot of running Task 3.4 \texttt{sortByMark}. \hfill{} {[}8{]}

\subsection*{Task 3.5 }

Write another method \texttt{displayByMark} to display the list of
students in descending order of mark by traversing the sorted linked
list from Task 3.4. 

\subsection*{Evidence 12 }

Program code for Task 3.5 \texttt{displayByMark}. \hfill{}{[}4{]}