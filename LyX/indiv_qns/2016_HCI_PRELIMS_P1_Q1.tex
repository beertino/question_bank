\item \textbf{{[}HCI/PRELIM/9597/2016/P1/Q1{]} }

A \textbf{\emph{palindrome}} is an integer that reads the same backwards
and forwards -- so 6, 11 and 121 are all palindromes, while 10, 12,
223 and 2244 are not (even though 010=10, we don't consider leading
zeroes when determining whether a number is a palindrome). 

A \textbf{\emph{fair and square}} number is an integer that is a \emph{palindrome}
\textbf{and} the \emph{square of a palindrome} at the same time. For
instance, 1, 9 and 121 are fair and square (being palindromes and
squares, respectively, of 1, 3 and 11), while 16, 22 and 676 are not
fair and square: 16 is \textbf{not} a palindrome, 22 is not a square,
and while 676 is a palindrome and a square number, it is the square
of 26, which is not a palindrome.

\subsection*{Task 1.1 }

Write program code with the following specification: 
\begin{itemize}
\item Input two integers -{}- the endpoints of an interval e.g. \texttt{100
1000 }
\item Output all fair and square numbers, if any, in the interval (inclusive
of endpoints). 
\item Output a count of the number of fair and square numbers in the interval.
\end{itemize}

\subsection*{Evidence 1: }

Your program code for Task 1.1.\hfill{} {[}8{]}

\subsection*{Evidence 2: }

Produce three screenshots showing the output of \texttt{1 4}, \texttt{10
120} and \texttt{100 1000} by the user.\hfill{} {[}3{]}

\subsection*{Task 1.2:}

Write program code to output the first 10 positive fair and square
numbers. 

\subsection*{Evidence 3: }

Your program code for Task 1.2.\hfill{} {[}3{]}

\subsection*{Evidence 4:}

Screenshot of output.\hfill{} {[}1{]}