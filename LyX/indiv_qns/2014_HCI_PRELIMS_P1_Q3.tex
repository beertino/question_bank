\item \textbf{{[}HCI/PRELIM/9597/2014/P1/Q3{]} }

Morse Code is a type of code in which letters are represented by combinations
by long or short signals, e.g. the letter \textquoteleft A\textquoteright{}
is represented by a short followed by a long signal, i.e. \textquoteleft .-\textquoteleft{} 

Here we use the period (\textquoteleft .\textquoteright ) to represent
the short signal and the dash (\textquoteleft -\textquoteleft ) to
present the long signal. The Morse code equivalent for the letters
\textquoteleft A\textquoteright{} to \textquoteleft Z\textquoteright{}
is provided in the file \texttt{MORSE.txt}. In this implementation,
we use a space (\textquoteleft{} \textquoteleft ) to simulate the
inter-character gap and the slash (\textquoteleft /\textquoteright )
to represent the inter-word gap. 

\subsection*{Task 3.1 }

Write the program code for a function that will convert a given word
into its Morse Code equivalent using the following specification. 
\noindent \begin{center}
\texttt{FUNCTION ConvertWord(SingleWord : STRING): STRING }
\par\end{center}

The function has a single parameter \texttt{SingleWord} and returns
the Morse Code equivalent for that word as a string. 

\subsection*{Evidence 10: }

Your \texttt{ConvertWord} program code. \hfill{} {[}7{]}

\subsection*{Evidence 11: }

A screenshot showing the correct Morse Code equivalent for the word
\textquotedblleft COMPUTING\textquotedblright . \hfill{} {[}1{]}

\subsection*{Task 3.2 }

Write the program code that does the following: 
\begin{itemize}
\item the user inputs a sentence of words not exceeding 5 words. 
\item uses the function \texttt{ConvertWord} in Task 3.1 
\item outputs the Morse Code equivalent of the sentence of words that was
entered. 
\end{itemize}

\subsection*{Evidence 12: }

Your program code. \hfill{} {[}6{]}

\subsection*{Task 3.3}

Draw up \textbf{two} suitable tests to assess that your program is
working properly, explaining the reason for each test and provide
screenshot evidence for your testing. 

\subsection*{Evidence 13: }

Annotated screenshots for each test data run. \hfill{} {[}4{]}

\subsection*{Task 3.4 }

Write the program code that will do the following: the user enters
a word string in Morse Code the program converts the Morse Code word
to its alphabetical equivalent outputs the converted word.

\subsection*{Evidence 14: }

Your program code. \hfill{}{[}6{]}

\subsection*{Evidence 15: }

A screenshot showing the correct word equivalent for the following
Morse Code string, \textquotedblleft \dots{} \texttt{-{}-{}-} \dots \textquotedblright{}
\hfill{}{[}1{]}