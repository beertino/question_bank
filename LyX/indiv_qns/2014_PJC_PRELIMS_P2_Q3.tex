\item \textbf{{[}PJC/PRELIM/9597/2014/P2/Q3{]} }

The words COW, BEEF and FORTY have all their letters written in alphabetical
order. Here is an algorithm for a function which checks whether all
the letters in a word are in alphabetical order. 

\noindent\begin{minipage}[t]{1\columnwidth}%
\texttt{01 FUNCTION IsInOrder(Word) }

\texttt{02 \qquad{}IF LENGTH(Word) = 1 THEN }

\texttt{03 \qquad{}\qquad{}RETURN TRUE }

\texttt{04 \qquad{}ELSE }

\texttt{05 \qquad{}\qquad{}FirstChar = First character in Word }

\texttt{06 \qquad{}\qquad{}RestOfWord = All characters in Word except
the first }

\texttt{07 \qquad{}\qquad{}IF FirstChar > RestOfWord THEN }

\texttt{08 \qquad{}\qquad{}\qquad{}RETURN FALSE }

\texttt{09 \qquad{}\qquad{}ELSE }

\texttt{10 \qquad{}\qquad{}\qquad{}RETURN IsInOrder(RestOfWord) }

\texttt{11 \qquad{}\qquad{}END IF }

\texttt{12 \qquad{}END IF }

\texttt{13 END FUNCTION} %
\end{minipage}
\begin{enumerate}
\item State what is meant by recursion using this algorithm as an example.
\hfill{}{[}2{]}
\item The algorithm is tested with the call \texttt{IsInOrder(\textquotedbl Z\textquotedbl )}.
State the value which will be returned. State the lines of the algorithm
which will be executed. \hfill{}{[}2{]}
\item Explain what happens if the algorithm is tested with a call \texttt{IsInOrder(\textquotedbl{}
\textquotedbl )} where the value of the argument is the empty string.
\hfill{}{[}2{]}
\item Explain what happens when the algorithm is tested with the call \texttt{IsInOrder(\textquotedbl APE\textquotedbl )}.
You should show each call made, the lines of the algorithm executed
and the return value of each call. You may use a diagram.\hfill{}
{[}4{]}
\end{enumerate}