\item \textbf{{[}PJC/PRELIM/9597/2018/P1/Q3{]} }

\noindent A Foreign Government Agency is looking for a Tourist Information
Management System. The system should be able to cater for the following
two requirements: 
\begin{enumerate}
\item[a)]  Capture and provision of information for migration control propose
and other aspects of citizen identification 

This is to facilitate the processing of Disembarkation/ Embarkation
(D/E) cards collected from visitors at the checkpoints. It also captures
visitors\textquoteright{} arrival data e.g., the number of arrivals
by countries of residence, their modes of arrival and demographics
(e.g., age and gender). 
\item[b)]  Data Warehouse for analysis 

The Data Warehouse has to receive data and code information on Disembarkation/Embarkation
cards (D/E cards). Information gathered in this manner is to analyse
visitor arrival trends and serve as input to the computation of key
performance indicators (Tourism Receipts, TourismSector Value, etc.)
\end{enumerate}
The Agency wishes to replace this manual system with a computerised
system.

The design for the new system includes the provision of a network
of computers in the office with a central file server. Each office
staff will have access to a computer to retrieve and update visitors\textquoteright{}
data held on the central file server. Some support staff are allowed
to access the data but not change it. In addition the system has an
Internet link which allows staff to access the system from outside
the office. 

Describe \textbf{three} ways in which the security of this system
can be implemented. \hfill{}{[}3{]}