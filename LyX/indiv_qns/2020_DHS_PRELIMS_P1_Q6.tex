\item \textbf{{[}DHS/PRELIM/9569/2020/P1/Q6{]} }

On 14 September 2020, it was reported that GrabCar was fined S\$10,000
for a 4th user data privacy violation. The Personal Data Protection
Commission (PDPC) said the update risked the personal data of 21,541
drivers and passengers, including profile pictures, names and vehicle
plate numbers. 

GrabCar rolled back the app to the previous version within about 40
minutes and took other remedial action, PDPC said. 

On Aug 30, 2019, GrabCar notified the PDPC that profile data of 5,651
GrabHitch drivers was exposed to the risk of unauthorised access by
other GrabHitch drivers for a \textquotedbl short period of time
on the same day\textquotedbl{} through the Grab app. 

Grab's investigations traced the cause of the breach to a deployment
of an update to the app on the same day. The purpose of the update
was to address a potential vulnerability discovered within the Grab
app. 

In PDPC's findings, the application programming interface URL which
allowed GrabHitch drivers to access their data, had contained a \textquotedbl userID\textquotedbl{}
portion that could potentially be manipulated to allow access to other
drivers' data. According to GrabCar, there was no evidence that this
vulnerability was exploited. 

To fix the vulnerability, the update removed the \textquotedbl userID\textquotedbl{}
from the URL, which shortened it to a hard-coded \textquotedbl users/profile\textquotedbl .
However, it failed to take into account the URL-based caching mechanism
in the app, which was configured to refresh every 10 seconds. 

The mechanism served cached content in response to data requests,
so as to reduce the load of direct access to GrabCar's database. 

With the update, all URLs in the Grab app ended with \textquotedbl users/profile\textquotedbl .
Without the \textquotedbl userID\textquotedbl{} in the URL, which
directed data requests to the correct GrabHitch driver's accounts,
the caching mechanism could no longer differentiate between drivers. 

Thus, the mechanism provided the same data to all GrabHitch drivers
for 10 seconds before new data was retrieved from GrabCar's database
and cached for the next 10 seconds. 

PDPC said GrabCar did not put in place \textquotedbl sufficiently
robust processes\textquotedbl{} to manage changes to its IT system
that may put personal data it was processing at risk. 

\textquotedbl This was a particularly grave error given that this
is the second time the (GrabCar) is making a similar mistake, albeit
with respect to a different system,\textquotedbl{} he said.

In a statement in response to Reuters' query, Grab said: \textquotedbl To
prevent a recurrence, we have since introduced more robust processes,
especially pertaining to our IT environment testing, along with updated
governance procedures and an architecture review of our legacy application
and source codes.\textquotedbl{} 

In 2019, GrabCar was ordered to pay a financial penalty of S\$16,000
after it sent out more than 120,000 marketing emails to customers
containing the name and mobile phone number of another customer. 

The PDPC had found that GrabCar \textquotedblleft failed to make reasonable
security arrangements\textquotedblright{} to detect the errors in
their database when sending out the emails. 

Source: Reuters/CNA/lk 

Adapted from https://www.channelnewsasia.com/news/business/grab-car-hitch-pdpc-personal-data-risk-fin
e-13108144 
\begin{enumerate}
\item For each of the following, suggest two ways in which the data leaks
could have been exploited by a malicious hacker with reference to 
\begin{enumerate}
\item profile pictures, names and vehicle plate numbers of drivers and passengers\hfill{}
{[}2{]}
\item name and mobile phone number of customers \hfill{}{[}2{]}
\end{enumerate}
\item What could have caused 
\begin{enumerate}
\item the URL related data leak? 
\item the marketing emails related data leak? 
\item the repeated data leaks? 
\end{enumerate}
You should provide a balance of technical and human related reasons.
\hfill{}{[}6{]}
\item How can Grab ensure and assure PDPC that sufficiently robust processes
have been put in place? \hfill{}{[}3{]}
\end{enumerate}