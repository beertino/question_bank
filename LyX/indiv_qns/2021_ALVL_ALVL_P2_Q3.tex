\item \textbf{{[}ALVL/9569/2021/P2/Q3{]} }

A programmer is writing a class, \texttt{LinkedList}, to represent
a linked list of unique integers. A linked list is a collection of
data elements, whose order is not given by their physical placement
in memory. Instead, each element points to the next.

For each of the sub-tasks, add a comment statement at the beginning
of the code using the hash symbol \textquoteleft \#' to indicate the
sub-task the program code belongs to, for example:

\begin{singlespace}
\noindent \texttt{}%
\begin{tabular}{c|lcccccccccccccccccccccc|}
\cline{2-24} \cline{3-24} \cline{4-24} \cline{5-24} \cline{6-24} \cline{7-24} \cline{8-24} \cline{9-24} \cline{10-24} \cline{11-24} \cline{12-24} \cline{13-24} \cline{14-24} \cline{15-24} \cline{16-24} \cline{17-24} \cline{18-24} \cline{19-24} \cline{20-24} \cline{21-24} \cline{22-24} \cline{23-24} \cline{24-24} 
\texttt{In {[}1{]} :} & \texttt{\#Task 3.1} &  &  &  &  &  &  &  &  &  &  &  &  &  &  &  &  &  &  &  &  &  & \tabularnewline
 & \texttt{Program Code} &  &  &  &  &  &  &  &  &  &  &  &  &  &  &  &  &  &  &  &  &  & \tabularnewline
\cline{2-24} \cline{3-24} \cline{4-24} \cline{5-24} \cline{6-24} \cline{7-24} \cline{8-24} \cline{9-24} \cline{10-24} \cline{11-24} \cline{12-24} \cline{13-24} \cline{14-24} \cline{15-24} \cline{16-24} \cline{17-24} \cline{18-24} \cline{19-24} \cline{20-24} \cline{21-24} \cline{22-24} \cline{23-24} \cline{24-24} 
\multicolumn{1}{c}{} & \texttt{Output:} &  &  &  &  &  &  &  &  &  &  &  &  &  &  &  &  &  &  &  &  &  & \multicolumn{1}{c}{}\tabularnewline
\end{tabular}
\end{singlespace}

\subsection*{Task 3.1 }

Write the \texttt{LinkedList} class in Python. Use of a simple Python
list is not sufficient. Include the following methods: 
\begin{itemize}
\item \texttt{insert(integer\_value)} inserts the \texttt{integer\_value}
at the beginning (head) of the list 
\item \texttt{delete(integer\_value)} attempts to delete \texttt{integer\_value}
from the list; if the item was not present, return \texttt{None} 
\item \texttt{search(integer\_value)} returns a Boolean value: \texttt{True}
if \texttt{integer\_value} is in the list, \texttt{False} if not in
the list 
\item \texttt{count()} should return the number of elements in the list,
or zero if empty 
\item \texttt{to\_String()} should return a string containing a suitably
formatted list with the elements separated by a comma and a space,
with square brackets at either end, eg. in the form: 

\texttt{\qquad{}{[}11, 2, 7, 4{]}} \hfill{}{[}8{]}
\end{itemize}
Test \texttt{LinkedList} by using the data in the file \texttt{Task3data.txt}.
Use the \texttt{to\_String()} method to print the resulting contents
of the list. \hfill{}{[}3{]}

\subsection*{Task 3.2 }

Write a Python subclass \texttt{SortedLinkedList} using \texttt{LinkedList}
as its superclass. 

The insert method in the \texttt{SortedLinkedList} subclass should
ensure that the elements are stored in ascending order. \hfill{}{[}5{]}

Test \texttt{SortedLinkedList} by using the data in the file \texttt{Task3data.txt}.
Use the \texttt{to\_String()} method to print the resulting contents
of the list. 

Print the result of searching the SortedLinkedList for the value 94.
\hfill{}{[}2{]}

\subsection*{Task 3.3 }

Write a Python subclass \texttt{Queue} using \texttt{LinkedList} as
its superclass.

Additional \texttt{enqueue} and \texttt{dequeue} methods are to be
defined on the \texttt{Queue} class;
\begin{itemize}
\item \texttt{enqueue(integer\_value)} will insert \texttt{integer\_value}
to the end of the queue
\item \texttt{dequeue()} will return the first element in the queue. If
the queue is empty, return \texttt{None.} \\
\textcolor{white}{.}\hfill{}{[}6{]}
\end{itemize}
Test \texttt{Queue} by using the data in the file \texttt{Task3data.txt}.
Print the first five elements to be dequeued from the list. \hfill{}{[}1{]}

Save your Jupyter notebook for Task 3.