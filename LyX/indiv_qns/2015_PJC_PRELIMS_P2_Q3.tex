\item \textbf{{[}PJC/PRELIM/9597/2015/P2/Q3{]} }

A recursive algorithm for finding a value, SearchItem, in an ordered
array, X, is as follows: 

\noindent %
\noindent\begin{minipage}[t]{1\columnwidth}%
\texttt{Search(Low, High) }

\texttt{\qquad{}Mid =(Low + High) div 2 }

\texttt{\qquad{}If X(Mid) = SearchItem then output \textquotedbl found\textquotedbl{}
: exit }

\texttt{\qquad{}If X(Mid) > Searchltem then Search(Low, Mid-1) }

\texttt{\qquad{}\qquad{}Else Search(Mid+1, High) }

\texttt{End Search }

\texttt{\qquad{}}Note: the \texttt{div} operation returns an integer
value after division e.g.\texttt{ 7 div 2 = 3} %
\end{minipage}

Using the above algorithm: 
\begin{enumerate}
\item Explain what is meant by a recursive algorithm. \hfill{}{[}1{]}
\item Describe what might occur during execution with an incorrectly written
recursive routine. 

Array \texttt{X} has 15 elements and the subscripts start at 1. \hfill{}
{[}2{]}
\item If the algorithm was used to search the array \texttt{X} for the value
stored at \texttt{X(3)} state the calls to Search as the recursion
executes. \hfill{} {[}2{]}
\item The algorithm does not handle the case where SearchItem is not present
in \texttt{X}. Indicate what changes need to be made to Search to
rectify this problem. \hfill{}{[}3{]}
\item For this method of searching state the maximum number of comparisons
and the minimum number of comparisons for array \texttt{X}, justifying
your answers. \hfill{}{[}2{]}
\end{enumerate}