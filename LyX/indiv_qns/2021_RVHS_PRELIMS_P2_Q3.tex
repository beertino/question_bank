\item \textbf{{[}RVHS/PRELIM/9569/2021/P2/Q3{]} }

\textbf{Task 3.1}

Write program code to read the csv file \textquotedbl\texttt{health\_facilities.csv}\textquotedbl{}
and insert all information in the file as documents into a NoSQL MongoDB
database called \textquotedbl\texttt{Health}\textquotedbl{} with
one collection called \textquotedbl\texttt{facilities}\textquotedbl .
The \textquotedbl\texttt{\_id}\textquotedbl{} of the documents in
the database should start from \texttt{1}, \texttt{2}, \texttt{3}
and \texttt{4} etc. The correct data type of each field is expected
to be inserted into the database. \hfill{}{[}10{]}

\subsubsection*{Task 3.2 }
\begin{enumerate}
\item Write a MongoDB Pymongo query to retrieve all public acute hospital
documents with their corresponding number of beds more than 7200.
\hfill{}{[}4{]}
\item Write program code to bubble sort the results retrieved in \textbf{Task
3.2 a)} according to the average number of beds per facility. Then,
display the top 3 years which has the highest average number of beds
per facility using the format below. 

\texttt{The three years that have the highest average number of beds
per facility are: \_\_\_\_\_, \_\_\_\_\_ and \_\_\_\_\_. }

\hfill{}{[}7{]}
\end{enumerate}

\subsubsection*{Task 3.3}
\begin{enumerate}
\item Write a MongoDB Pymongo query and program code to display all \textquotedbl\texttt{\_id}\textquotedbl s
of Not-for-Profit health facilities documents that have no facility.
\hfill{}{[}3{]} 
\item Write MongoDB Pymongo code to update the fields \textquotedbl\texttt{no\_of\_facilities}\textquotedbl{}
and \textquotedbl\texttt{no\_beds}\textquotedbl{} of only 3 documents
retrieved in \textbf{Task 3.3 a)} to \texttt{1} and a random number
from \texttt{10} to \texttt{20} respectively. \hfill{}{[}6{]} 
\end{enumerate}