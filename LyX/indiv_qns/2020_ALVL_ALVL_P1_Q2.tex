\item \textbf{{[}ALVL/9569/2020/P1/Q2{]} }

Quicksort is an algorithm to arrange data items into ascending or
descending order. The algorithm selects a pivot from the data set.
The data set is divided into two subsets around the pivot.
\begin{enumerate}
\item {}
\begin{enumerate}
\item State the ideal pivot for the quicksort algorithm to execute most
efficiently. \hfill{}{[}1{]} 
\item State the difficulty in locating this ideal pivot. \hfill{}{[}1{]}
\end{enumerate}
\end{enumerate}
Sometimes the item in the first or last position in the data set is
used as the pivot. An alternative is to pick the pivot at random.

A given data set is largely sorted.
\begin{enumerate}
\item[\textbf{(b)}]  Explain what advantage random selection has over selecting the item
in the first or last position. \hfill{}{[}2{]}
\item[\textbf{(c)}]  Explain why a programmer might choose to use an insertion sort rather
than quicksort in this situation. \hfill{}{[}4{]}
\end{enumerate}