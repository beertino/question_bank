\item \textbf{{[}RVHS/PRELIM/9569/2021/P1/Q3{]} }

The following recursive procedure is created to encode a character
\texttt{char} based on a \texttt{shift} value. For example, if letter
\textquotedbl\texttt{a}\textquotedbl{} is shifted by \texttt{3},
it will become letter \textquotedbl\texttt{d}\textquotedbl ; and
if shifted by \texttt{-3}, it will become letter \textquotedbl\texttt{x}\textquotedbl . 

The \texttt{ORD} and \texttt{CHR} function will help to convert the
character to its ASCII value and vice versa.

\noindent\begin{minipage}[t]{1\columnwidth}%
\texttt{01 PROCEDURE ENCODE\_CHAR(char: STRING, shift: INT): }

\texttt{02 \qquad{}IF shift == 0: }

\texttt{03 \qquad{}\qquad{}RETURN char }

\texttt{04 \qquad{}ELSE IF shift > 0: }

\texttt{05 \qquad{}\qquad{}DECLARE new\_char: STRING }

\texttt{06 \qquad{}\qquad{}new\_char = CHR((ORD(char) + 1) \% 26) }

\texttt{07 \qquad{}\qquad{}shift -= 1 }

\texttt{08 \qquad{}\qquad{}RETURN ENCODE\_CHAR(new\_char, shift) }

\texttt{09 \qquad{}ELSE: }

\texttt{10 \qquad{}\qquad{}Shift += 26 }

\texttt{11 \qquad{}\qquad{}RETURN ENCODE\_CHAR(char, shift) }

\texttt{12 \qquad{}END IF }

\texttt{13 END PROCEDURE}%
\end{minipage} 
\begin{enumerate}
\item Identify one error from the above code, state the type of the error,
including its definition and explain how the errors can be fixed.
\hfill{}{[}2{]}
\item Assume the above error has been fixed. Copy the following trace table
to your answer booklet. State the line number each time one of the
return statements is called and complete it based on the following
function call. 

\texttt{ENCODE\_CHAR(\textquotedbl y\textquotedbl , -24) }\hfill{}{[}3{]}
\noindent \begin{center}
\begin{tabular}{|c|c|c|c|}
\hline 
\texttt{Line No. } & \texttt{char } & \texttt{shift } & \texttt{new\_char }\tabularnewline
\hline 
 &  &  & \tabularnewline
\hline 
 &  &  & \tabularnewline
\hline 
 &  &  & \tabularnewline
\hline 
\end{tabular}
\par\end{center}

\end{enumerate}