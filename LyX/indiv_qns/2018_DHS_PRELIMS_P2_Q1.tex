\item \textbf{{[}DHS/PRELIM/9597/2018/P2/Q1{]} }

The Singapore Quick Response Code (SGQR) is a single QR code that
combines multiple e-payment solutions into one. It is intended to
simplify QR e-payments in Singapore for both consumers and merchants.

Currently, consumers see multiple QR codes at merchant stores promoting
various e-payment solutions. This can be confusing for consumers who
have to manually find if their preferred e-payment option is accepted.
Merchants are also impacted by the aesthetic and logistics constraints
of supporting multiple QR codes on their limited display and retail
space. With SGQR, consumers will see a single SGQR label that shows
all QR payment options that the merchant accepts. For merchants, SGQR
will be an infrastructure-light and cheaper way to accept multiple
types of e-payments.
\begin{center}
\includegraphics[width=0.25\paperwidth]{C:/Users/Admin/Desktop/Github/question_bank/LyX/static/img/9597-DHS-2018-P2-Q1}
\par\end{center}

Merchants that currently offer QR code payments will have their existing
QR codes replaced with a single SGQR label over the next six months.
The first phase of SGQR label replacement, starting with merchants
in the Central Business District, will be commencing in late September
2018.
\begin{enumerate}
\item You have been engaged as a project manager to oversee the implementation
of SGQR in Dunman High School (DHS) canteen. Produce a project proposal
outlining the key activities to make DHS canteen cashless by March
2019. Your proposal should include the essential elements such as
problem statement, project management processes and tools (e.g. PERT
chart and Gantt chart), roles of team members, etc. \hfill{}{[}19{]}
\item Beyond SGQR and in line with the Smart Nation drive, you have also
engaged a systems analyst to come up with an online food ordering
application to allow students and staff to avoid long queues and streamline
the food preparation process using their mobile devices. The school
management also wishes to keep track of the situation to provide feedback
to the canteen vendors. 

Outline the deliverables in the various phases of the software development
life cycle (specification, design, development, documentation, implementation,
testing/modification and maintenance). Be sure to adapt your answer
to the question context. \hfill{}{[}12{]}
\item Networking is critical in such a project/system. Give an example of
where each of the following networking concept is applicable in your
project/system.
\begin{enumerate}
\item synchronous and asynchronous data transmission
\item simplex, half duplex and full duplex mode of data transmission
\item packet switching and circuit switching for data transmission \hfill{}{[}6{]}
\end{enumerate}
\item You are also mindful about and is determined to prevent cybersecurity
attacks like the recent SingHealth data breach of the personal information
of 1.5 million patients. Outline a comprehensive organisational security
plan which goes beyond technical controls (user authentication, access
levels, antivirus, firewalls) to ensure that both the hardware infrastructure
and software applications are well secured against cybersecurity hacks.\hfill{}
{[}8{]}
\end{enumerate}