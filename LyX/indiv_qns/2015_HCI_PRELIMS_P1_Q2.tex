\item \textbf{{[}HCI/PRELIM/9597/2015/P1/Q2{]} }

The Russian peasant algorithm is an alternative method to perform
multiplication of whole numbers by consecutive application of doubling
numbers, halving numbers and addition. 

Consider the multiplication of \texttt{57} by \texttt{86} (= \texttt{4902}): 

Write each number side by side: 
\noindent \begin{center}
\texttt{\sout{57 ~~86}}
\par\end{center}

Double the first number and halve the second number (by performing
integer division by 2 i.e. drop the remainder).

If the second number or halving result is even, cross out this entire
row. Keep doubling, halving, and crossing out until the halving result
is 1.
\noindent \begin{center}
\noindent\begin{minipage}[t]{1\columnwidth}%
\texttt{\sout{57 ~~86}}

\texttt{\sout{114 ~43}}\texttt{ }

\texttt{\sout{228 ~21}}

\texttt{\sout{456 ~10}}

\texttt{\sout{912 ~5 }}

\texttt{\sout{1824 2 }}

\texttt{\sout{3648 1}}\texttt{ }%
\end{minipage}
\par\end{center}

Add up the remaining numbers in the first column. The total is the
product of the original numbers. 
\noindent \begin{center}
\noindent\begin{minipage}[t]{1\columnwidth}%
\texttt{~~~114}

\texttt{~~~228}

\texttt{~~~912}

\texttt{\uline{+ 3648 }}

\texttt{~~4902}%
\end{minipage}
\par\end{center}


\subsection*{Task 2.1 }

Write a function to implement the Russian peasant multiplication algorithm.
Test your function with the numbers \texttt{50} and \texttt{22}. 

\subsection*{Evidence 3: }

Your program code for Task 2.1. \hfill{} {[}7{]}

\subsection*{Evidence 4: }

Screenshot showing the output from running the program. \hfill{}
{[}1{]}

The Russian peasant algorithm is actually related to binary numbers.
Doubling a decimal number is to shift its binary equivalent left,
while halving a decimal number is to shift its binary equivalent right. 

Consider the multiplication of \texttt{13} by \texttt{12} (\texttt{=
156}) 

\noindent\begin{minipage}[t]{1\columnwidth}%
\texttt{~~~~}In Binary\texttt{ ~~~~~~~~~~~~~}In Decimal\texttt{ }

\texttt{~~~}\texttt{\sout{1101 1100}}\texttt{ ~~~~~~~~~~~}\texttt{\sout{13
12}}\texttt{ }

\texttt{~ }\texttt{\sout{11010 0110}}\texttt{ ~~~~~~~~~~~}\texttt{\sout{26
6}}\texttt{ }

\texttt{~110100 0011 ~~~~~~~~~~~52 3}

\texttt{1101000 0001 ~~~~~~~~~~104 1 }

\bigskip{}

\texttt{110100 + 1101000 = 10011100 = 156 (in decimal) }%
\end{minipage}

\subsection*{Task 2.2 }

Write a function \texttt{DecToBin()} to convert a decimal number to
its binary equivalent.

\subsection*{Evidence 5: }

Your program code for Task 2.2. \hfill{}{[}3{]}

\subsection*{Task 2.3}

Implement the Russian peasant multiplication algorithm using binary
numbers. Your program should accept 2 decimal numbers as input, convert
them to binary, and then perform the necessary operations to output
a binary string as the result. You should make use of your \texttt{DecToBin()}
function in Task 2.2. 

\subsection*{Evidence 6: }

Your program code for Task 2.3. \hfill{} {[}8{]}

\subsection*{Evidence 7: }

One screenshot showing the output from running the program code. \hfill{}
{[}1{]}