\item \textbf{{[}ALVL/9597/2019/P1/Q4{]} }

Design and code a computer program to simulate the following: 

A garden has a rectangular fish pond measuring 15 metres by 8 metres. 

The pond is to be represented on the screen by a rectangular grid.
Each square metre of the pond is represented by an x-coordinate and
a y-coordinate. The top left square metre of the pond display has
x = 1 and y = 1. 

A boy throws a stone into the pond. The user will input the x-coordinate
and y-coordlnate of the stone impact position. 

A grid representing the pond is then displayed with the stone's impact
position: 

\texttt{X coordinate <1 to 15>? 9 }

\texttt{Y coordinate <1 to 8>? 3 }

\texttt{. . . . . . . . . . . . . . . }

\texttt{. . . . . . . . . . . . . . .}

\texttt{. . . . . . . . S . . . . . .}

\texttt{. . . . . . . . . . . . . . .}

\texttt{. . . . . . . . . . . . . . .}

\texttt{. . . . . . . . . . . . . . .}

\texttt{. . . . . . . . . . . . . . .}

\texttt{. . . . . . . . . . . . . . .}

\subsubsection*{Task 4.1}

The following are the suggested characters to use for the visual representation
of the pond: 
\begin{center}
\begin{tabular}{|c|c|l|}
\hline 
Character & ASCII code (decimal) & \texttt{\hspace{0.01\columnwidth}}Represents\tabularnewline
\hline 
. & 46 & One square metre of water\tabularnewline
\hline 
S & 83 & Stone impact position\tabularnewline
\hline 
\end{tabular}
\par\end{center}

Decide on the design to be used for: 
\begin{itemize}
\item The data structure to represent the grid
\item The contents of each square metre of the pond 
\item Procedure(s) and/or function(s)s to be used 
\end{itemize}

\subsubsection*{Evidence 17}

Show your program design (Task 4.1). \hfill{} {[}6{]}

\subsubsection*{Task 4.2}

Write program code to display the pond contents after a single stone
has been thrown. 

\subsubsection*{Evidence 18}

The program code. \hfill{} {[}7{]}

\subsubsection*{Evidence 19}

Screenshot for a single run of the program. \hfill{} {[}1{]}

\subsubsection*{Task 4.3}

The boy has been told to stop throwing stones into the pond because
the pond now has three fish. The fish randomly swim around. Each fish
will occupy a unique grid position.

Using a random number generator, simulate the positioning of the three
fish. 

Use the following character for a fish: 
\begin{center}
\begin{tabular}{|c|c|l|}
\hline 
Character & ASCII code (decimal) & \texttt{\hspace{0.01\columnwidth}}Represents\tabularnewline
\hline 
F & 70 & Fish\tabularnewline
\hline 
\end{tabular}
\par\end{center}

Write program code to show the pond containing the three fish at a
particular instance of time. The program will now only display the
pond and fish. 

\subsubsection*{Evidence 20}

The program code for Task 4.3. \hfill{} {[}6{]}

\subsubsection*{Evidence 21}

Screenshot for a single run of the program. \hfill{} {[}1{]}

\subsubsection*{Task 4.4}

The boy has been asked to feed the fish. He cannot see the fish in
the pond. He throws a food pellet into the pond which lands inside
one of the square metres. If one of the fish is in this square. it
eats the food and becomes a happy fish. 

Use character symbols for the pond\textquoteleft s grid display as
follows: 
\begin{center}
\begin{tabular}{|c|c|l|}
\hline 
Character & ASCII code (decimal) & \texttt{\hspace{0.01\columnwidth}}Represents\tabularnewline
\hline 
. & 46 & One square metre of water\tabularnewline
\hline 
P & 80 & Pellet (if not eaten by one of the fish)\tabularnewline
\hline 
H & 72 & Happy (fed) fish\tabularnewline
\hline 
F & 70 & Fish\tabularnewline
\hline 
\end{tabular}
\par\end{center}

Write program code to simulate the boy throwing one food pellet into
the pond. The user will input an x-coordinate and y-coordinate for
the food pellet position. You should consider the possible reuse of
any code from Tasks 4.2 and 4.3.

\subsubsection*{Evidence 22}

The program code.\hfill{} {[}6{]}

\subsubsection*{Evidence 23}

Screenshotevldence similar to that shown which shows: 
\begin{itemize}
\item one throw which did not feed a fish 

\texttt{X coordinate <1 to 15>? 2 }

\texttt{Y coordinate <1 to 8>? 5 }

\texttt{. . . . . . . . . . . . . . . }

\texttt{. . . . . . . . . . . . . . .}

\texttt{. . . . . . . . F . . . . . .}

\texttt{. . . . . . . . . . . . . . .}

\texttt{. P . . . . . . . . F . . . .}

\texttt{. . . . . . . . . . . . . . .}

\texttt{. . . . F . . . . . . . . . .}

\texttt{. . . . . . . . . . . . . . .}
\item a second throw where a fish was fed 

\texttt{X coordinate <1 to 15>? 1 }

\texttt{Y coordinate <1 to 8>? 5 }

\texttt{. . . . . . . . . . . . . . . }

\texttt{. . . . . . . . . . . . . . .}

\texttt{. . . . . . . . . . . . . . F}

\texttt{. . . . . . . . . . . . . . .}

\texttt{H . . . . . . . . . . . . . .}

\texttt{. . . . . . . . . . . . . . .}

\texttt{. . . . . . . F . . . . . . .}

\texttt{. . . . . . . . . . . . . . .}\hfill{} {[}3{]}
\end{itemize}