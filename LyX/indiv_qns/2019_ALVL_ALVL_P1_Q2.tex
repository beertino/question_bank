\item \textbf{{[}ALVL/9597/2019/P1/Q2{]} }

Characters are numerically encoded using ASCII codes.
\begin{itemize}
\item 'A' has the denary value 65; \textquoteleft B' has the denary value
66 and so on.
\item 'a' has the denary value 97; 'b' has the denary value 98 and so on.
\end{itemize}
The ROT-13 encoding function replaces a letter with the letter that
is 13 positions after it in the alphabet. Characters that are not
letters remain unchanged. .

The function wraps around from the end of the alphabet back to the
beginning. The case of the coded letter should match the case of the
original letter.

For example:
\begin{itemize}
\item 'A' is replaced with 'N'; 'a' is replaced with 'n'
\item 'B' is replaced with '0\textquoteleft ; 'b' is replaced with 'o' 
\item 'Z' is replaced with 'M'; 'z\textquoteleft{} is replaced with 'm'
\end{itemize}

\subsubsection*{Task 2.1}

Write program code that:
\begin{itemize}
\item reads a string of characters as input
\item encodes the string in ROT-13 form
\item outputs the encoded string.
\end{itemize}
Run the program \textbf{three} times with the inputs: 

\noindent\begin{minipage}[t]{1\columnwidth}%
\texttt{This is a word.}

\texttt{ALL \&\&\&\& CAPITALS}

\texttt{UpperCamelCase12()}%
\end{minipage}

\subsubsection*{Evidence 3}

Your program code.

Screenshots of your outputs. \hfill{}{[}9{]}

\subsubsection*{Task 2.2}

A string is encoded using ROT-13. The resulting string is then encoded
using ROT-13. The output of the second encoding should be identical
to the original string.

Amend your program code to apply HOT-13 twice, in the method described.
Show that the resulting string is identical to the original string.

\subsubsection*{Evidence 4}

Your program code.

Screenshot of the output from one of the given inputs. \hfill{}{[}3{]}