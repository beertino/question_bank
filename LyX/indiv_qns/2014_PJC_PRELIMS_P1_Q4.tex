\item \textbf{{[}PJC/PRELIM/9597/2014/P1/Q4{]} }

A message is encrypted and passed between two parties. To decrypt
the message, a \textquotedblleft key\textquotedblright{} is applied.
Both the sending and receiving parties hold the key which enables
them to encrypt and decrypt the message. 

An approach of cryptography is the simple substitution cipher, a method
of encryption by which each letter of a message is substituted with
another letter. The receiving party deciphers the text by performing
an inverse substitution. 

The substitution system is created by first writing out a \emph{phrase}.
The key is then derived from the phrase by removing all the repeated
letters. The \emph{cipher text} alphabet is then constructed starting
with the letters of the \emph{key} and then followed by all the remaining
letters in the alphabet. 

Using this system, the phrase \textquotedbl\texttt{apple}\textquotedbl{}
gives us the key as \textquotedbl\texttt{APLE}\textquotedbl{} and
the following substitution scheme: 

\begin{tabular}{cccccccccccccccccccccccccccc}
\textbf{Plain text alphabet :} & a & b & c & d & e & f & g & h & i & j & k & l & m & n & o & p & q & r & s & t & u & v & w & x & y & z & \tabularnewline
 & $\downarrow$ &  &  & $\downarrow$ & \multicolumn{21}{c}{$\cdots$$\cdots$$\cdots$$\cdots$$\cdots$$\cdots$$\cdots$$\cdots$} & $\downarrow$ & is substituted by\tabularnewline
\textbf{Cipher text alphabet :} & A & P & L & E & B & C & D & F & G & H & I & J & K & M & N & O & Q & R & S & T & U & V & W & X & Y & Z & \tabularnewline
\end{tabular}

\texttt{'a'} will be substituted by \texttt{'A'}, \texttt{'b'} will
be substituted by \texttt{'P'}, \texttt{'c'} will be substituted by
\texttt{'L'}, \texttt{'d'} will be substituted by\texttt{ 'E'},\texttt{
'e' }will be substituted by \texttt{'B'}, and so on. 

\subsection*{Task 4.1 }

Write program code for a function to create cipher text using the
following specification:
\noindent \begin{center}
\texttt{FUNCTION CreateCipher (phrase : STRING) : STRING }
\par\end{center}

The function \texttt{CreateCipher} has a single parameter \texttt{phrase}
and returns the cipher text alphabet as a string. 

\subsection*{Evidence 14: }

Your program code for task 4.1.\hfill{} {[}8{]}

\subsection*{Task 4.2 }

Write program code for a procedure \texttt{CreateCipherTest} which
does the following: 
\begin{itemize}
\item read the phrases from file phrases.txt 
\item create cipher text for each of the phrases 
\item display each phrase and cipher text on the screen as follows: 

\noindent\fbox{\begin{minipage}[t]{1\columnwidth - 2\fboxsep - 2\fboxrule}%
\texttt{Phrase: apple }

\texttt{Cipher text: APLEBCDFGHIJKMNOQRSTUVWXYZ }

\texttt{... ...}

\texttt{... ... }%
\end{minipage}}
\end{itemize}

\subsection*{Evidence 15: }

Your program code for task 4.2.\hfill{} {[}3{]}

\subsection*{Evidence 16: }

Screenshot for running task 4.2. \hfill{}{[}1{]}

\subsection*{Task 4.3}

Write program code for a function to decrypt a message using the following
specification: 
\noindent \begin{center}
\texttt{FUNCTION Decrypt (enc\_message:STRING, cipher:STRING) : STRING }
\par\end{center}

The function \texttt{Decrypt} accepts parameters \texttt{enc\_message}
and \texttt{cipher}, and returns the decrypted message as a string.
Parameter \texttt{enc\_message} is the encrypted message, and parameter
\texttt{cipher} is the cipher text alphabet. 

\subsection*{Evidence 17: }

Your program code for task 4.3. \hfill{}{[}6{]}

\subsection*{Task 4.4 }

Write program code which does the following: 
\begin{itemize}
\item read the phrase and encrypted message from file \texttt{cipher.txt }
\item cipher text is generated from \texttt{CreateCipher} function 
\item message is decrypted from \texttt{Decrypt} function 
\item display decrypted message on the screen together with the phrase and
encrypted message 

\noindent\fbox{\begin{minipage}[t]{1\columnwidth - 2\fboxsep - 2\fboxrule}%
\texttt{Phrase: ...}

\texttt{Encrypted message: ...}

\texttt{Decrypted message: ...}%
\end{minipage}}
\end{itemize}

\subsection*{Evidence 18: }

Your program code for task 4.4.\hfill{} {[}3{]}

\subsection*{Evidence 19: }

Screenshot for running task 4.4. \hfill{}{[}1{]}

\subsection*{Task 4.5 }

Write program code for a function to encrypt a message using the following
specification:

\texttt{FUNCTION Encrypt (message:STRING, cipher:STRING) : STRING}

The function \texttt{Encrypt} accepts parameters \texttt{message}
and \texttt{phrase}, and returns the encrypted message as a string.
Parameter \texttt{message} is the message to be encrypted while parameter
\texttt{cipher} is the cipher text. 

\subsection*{Evidence 20: }

Your program code for task 4.5.\hfill{} {[}4{]}

\subsection*{Task 4.6 }

Write program code which does the following: 
\begin{itemize}
\item encrypt the message: \textquotedblleft \texttt{do not give up!}\textquotedblright{}
\item use the phrase: \textquotedblleft \texttt{skyhigh}\textquotedblright{} 
\item generate cipher text from \texttt{CreateCipher} function
\item message is encrypted using \texttt{Encrypt} function
\item encrypted message is displayed on screen as follows: 

\noindent\fbox{\begin{minipage}[t]{1\columnwidth - 2\fboxsep - 2\fboxrule}%
\texttt{Phrase: skyhigh}

\texttt{Encrypted message: ...}%
\end{minipage}}
\end{itemize}

\subsection*{Evidence 21: }

Your program code for task 4.6. \hfill{}{[}3{]}

\subsection*{Evidence 22: }

Screenshot for running task 4.6. \hfill{}{[}1{]}