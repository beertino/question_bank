\item \textbf{{[}ALVL/9597/2019/P1/Q3{]} }

A program is to be written to implement a to-do list using object-oriented
programming (OOP). The list shows tasks that need to be done.

Each task is given a category and a description.

The base class will be called \texttt{ToDo} and is designed as follows: 
\begin{center}
\begin{tabular}{|l|}
\hline 
\texttt{\hspace{0.25\columnwidth}ToDo}\tabularnewline
\hline 
\texttt{category : STRING }\tabularnewline
\texttt{description : STRING}\tabularnewline
\hline 
\texttt{constructor(c : STRING, d : STRING) }\tabularnewline
\texttt{set\_category(s : STRING)}\tabularnewline
\texttt{set\_description(s : STRING) }\tabularnewline
\texttt{get\_category() : STRING }\tabularnewline
\texttt{get:\_description() : STRING }\tabularnewline
\texttt{summary() : STRING}\tabularnewline
\hline 
\end{tabular}
\par\end{center}

The \texttt{summary()} method returns the category and description
as a single string.

\subsubsection*{Task 3.1}

Write program code to define the class \texttt{ToDo}.

\subsubsection*{Evidence 5}

Your program code. \hfill{}{[}5{]}

Tasks should be sorted alphabetically by category. Within each category.
tasks should be sorted alphabetically by description. 

A task to be added to the list is compared to the tasks already in
the list to determine its correct position in the list. If the list
is empty. it is added to the beginning of the list. 

This comparison will use an additional member method,
\begin{center}
\texttt{compare\_with(td : ToDo) : INTEGER }
\par\end{center}

This function compares the instance (the item in the list) and the
\texttt{ToDo} object passed to it, returning one of three values:

\texttt{\qquad{}}-1 if the instance is before the given \texttt{ToDo}

\texttt{\qquad{}}0 if the two are equal

\texttt{\qquad{}}+1 if the instance is after the given \texttt{ToDo}

\subsubsection*{Task 3.2}

There are four objects defined in the text file \texttt{TASK3\_2.TXT}.

Write program code to:
\begin{itemize}
\item implement the \texttt{compare\_with()} method
\item create an empty list of \texttt{ToDo} objects
\item add each of the four objects in the text file \texttt{TASK3\_2.TXT}
to its appropriate place in the list
\item printout the list contents using the \texttt{summary()} method.
\end{itemize}

\subsubsection*{Evidence 6}

Your program code.

Screenshot of test run. \hfill{}{[}13{]}

The to-do list can have items with extra information. One such item
has a date by which the task should be completed. 

The \texttt{DatedToDo} class inherits from the \texttt{ToDo} class,
extending it to have a \texttt{due\_date}. designed as follows: 
\begin{center}
\begin{tabular}{|l|}
\hline 
\hspace{0.25\columnwidth}DatedToDo : ToDo\tabularnewline
\hline 
\texttt{due\_date : DATE}\tabularnewline
\hline 
\texttt{constructor(dt : DATE, 0 : STRING, d : STRING)}\tabularnewline
\texttt{set\_due\_date(d : DATE)}\tabularnewline
\texttt{get\_due\_date() : DATE }\tabularnewline
\hline 
\end{tabular}
\par\end{center}

The \texttt{DatedToDo} class should extend the \texttt{compare\_with()}
method to ensure that tasks are ordered by ascending \texttt{due\_date},
and then by the ordering used by the base \texttt{compare\_with()}
method. The \texttt{summary()} method should also be extended to return
the \texttt{due\_date} and the return values of the base \texttt{summary()}
method.

\subsubsection*{Task 3.3}

There are seven objects defined in the text file \texttt{TASK3\_3.TXT}.

Amend your program code to:
\begin{itemize}
\item implement the \texttt{DatedToDo} class, with \texttt{constructor},
\texttt{get\_due\_date} and \texttt{set\_due\_date}
\item implement the extended \texttt{compare\_with()} method
\item implement the extended \texttt{summary()} method
\item ensure all seven objects in the text file \texttt{TASK3\_3.TXT} are
added to the list
\item print out the list contents using the \texttt{summary()} method.
\end{itemize}

\subsubsection*{Evidence 7}

Your program code. 

Screenshot of the test run.\hfill{}{[}5{]}

\noindent When a task in the to-do list has been completed, it should
be removed.

\subsubsection*{Task 3.4}

There are four completed tasks defined in the text file \texttt{TASK3\_4.TXT}.

If any of the four tasks exists in the list, it should be removed.

Amend your program to: 
\begin{itemize}
\item recreate the list of seven tasks from Task 3.3
\item check if each of the four completed tasks in the text file \texttt{TASK3\_4.TXT}
exists in the list and: 
\begin{itemize}
\item remove it from the list if it does or 
\item print a warning message it the completed task does not exist
\end{itemize}
\item print out the list after all four objects have been processed. 
\end{itemize}

\subsubsection*{Evidence 8}

Your amended code.

Screenshot of the test run.\hfill{}{[}10{]}