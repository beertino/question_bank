\item \textbf{{[}ACJC/PRELIM/9569/2021/P2/Q4{]} }

A company keeps records of the employees working for it. The following
are the information stored in the company\textquoteright s database:
\begin{itemize}
\item \texttt{Employee\_name} -- name of the employee 
\item \texttt{Employee\_ID} -- unique ID number allocated to each employee 
\item \texttt{Job\_type} -- type of job the employee is employed for (\texttt{'Sales'}
or \texttt{'Tech\_support}') 
\item \texttt{Date\_of\_employment} -- date the employee joined the company 
\item \texttt{Service\_status} -- whether the employee is still in service
(\texttt{'True'} means the employee is still in the company, \texttt{'False'}
means the employee has left the company) 
\end{itemize}
For sales employees, the following extra information is recorded:
\begin{itemize}
\item \texttt{Total\_sales} -- the amount of sales in dollars made by the
employee 
\end{itemize}
For tech support employees, the following extra information is recorded:
\begin{itemize}
\item \texttt{Bugs\_resolved} -- the total number of bugs the employee
has resolved 
\end{itemize}
The database is expected to be normalised and stored in three different
tables:

\texttt{Employee }

\texttt{Sales }

\texttt{Tech\_support}

\subsection*{Task 4.1}

Create an SQL file called \texttt{TASK4\_1\_<your name>\_<centre number>\_<index
number>.sql} to show the SQL code to create the database \texttt{records.db}
with the three tables. Primary keys and foreign keys should be defined
where appropriate.

Save your SQL code as 

\texttt{TASK4\_1\_<your name>\_<centre number>\_<index number>.sql}
\hfill{}{[}5{]}

\subsection*{Task 4.2 }

The files \texttt{SALES.txt} and \texttt{TECH\_SUPPORT.txt} contain
information regarding the sales and tech support employees respectively.
The information should be inserted into the database.

For \texttt{SALES.txt}, information is given in the following order: 

\texttt{Employee\_ID}, \texttt{Employee\_name}, \texttt{Date\_of\_Employment},
\texttt{Service\_status}, \texttt{Total\_Sales}

For \texttt{TECH\_SUPPORT.txt}, information is given in the following
order: 

\texttt{Employee\_ID}, \texttt{Employee\_name}, \texttt{Date\_of\_Employment},
\texttt{Service\_status}, \texttt{Bugs\_resolved}

Write a python program to insert all information from the two files
into the \texttt{records} database, \texttt{records.db}. Run the program.

Save your program code as 

\texttt{TASK4\_2\_<your name>\_<centre number>\_<index number>.py}\hfill{}
{[}5{]}

\subsection*{Task 4.3}

The company wants to filter the employees by \texttt{Service\_status}
and display the results in a web browser.

Write a Python program and the necessary files to create a web application
that:
\begin{itemize}
\item receives a \texttt{Service\_status} string from a HTML form, then
\item creates and returns a HTML document that enables the web browser to
display either
\begin{itemize}
\item an ordered list of employees that are still in service, or 
\item an ordered list of employees that are no longer in service,
\end{itemize}
depending on the \texttt{Service\_status} string entered by the user.
\end{itemize}
The list should be sorted in alphabetical order.

Save your Python program as 

\texttt{TASK4\_3\_<your name>\_<centre number>\_<index number>.py }

with any additional files / sub-folders as needed in a folder named 

\texttt{TASK4\_3\_<your name>\_<centre number>\_<index number>}

Run the web application. Save the output of the program when \texttt{'TRUE'}
is entered as the \texttt{Service\_status} as \texttt{TASK4\_3\_<your
name>\_<centre number>\_<index number>.html}. \hfill{}{[}12{]}