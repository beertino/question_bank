\item \textbf{{[}JPJC/PRELIM/9569/2020/P1/Q8{]} }

A company currently uses a computerised flat file to keep track of
the monthly claims submitted by its employee, and has decided to use
a relational database to store and manage the claims submitted by
the employees instead. The following table shows the details of the
computerised flat file.

\begin{tabular}{|c|c|c|c|c|c|}
\hline 
Claims ID  & Item Description  & Staff ID  & Staff Name  & Department  & Amount\tabularnewline
\hline 
\hline 
2818 & Phone charger & P212 & John Lee & Production & \$53.23\tabularnewline
\hline 
3291 & Car Transport & S281 & Chan, Molly & Sales & \$31.40\tabularnewline
\hline 
3998 & Meal, Lunch & O323  & Omar Hairi  & Operations & \$7.20\tabularnewline
\hline 
4820  & AAA Batteries  & E493  & Muthu K.  & Engineering  & \$10.17\tabularnewline
\hline 
6322  & Hard Drive 3TB  & A550  & Jervois F.  & Accounts  & \$27.99\tabularnewline
\hline 
7384  & Medical  & M438  & Zudin B Ali  & Marketing  & \$48.00\tabularnewline
\hline 
\dots .  & \dots .  & \dots .  & \dots .  & \dots . & \dots .\tabularnewline
\hline 
\end{tabular}
\begin{enumerate}
\item State and justify \textbf{one} reason made by the company to migrate
its claims information from the existing flat file system to a relational
database management system. \hfill{}{[}2{]}
\item State \textbf{two} other fields which would be useful for the company
to capture. \hfill{}{[}2{]}
\item Given that the every claim is associated with one item, write the
table descriptions of the relational database in \textbf{first normal
form} and \textbf{second normal form}. You are to include the fields
in \textbf{(b)}. \hfill{}{[}4{]}
\end{enumerate}