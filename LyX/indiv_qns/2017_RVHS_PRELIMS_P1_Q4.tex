\item \textbf{{[}RVHS/PRELIM/9597/2017/P1/Q4{]} }

\textbf{Sorting Algorithms }

The pseudocode procedure below is given a list of integers in random
order, the procedure then outputs a sorted list in ascending order. 

\noindent\fbox{\begin{minipage}[t]{1\columnwidth - 2\fboxsep - 2\fboxrule}%
\texttt{01. FUNCTION Merge\_Sort (ARRAY: arr) }

\texttt{02. \qquad{}n = SIZE(arr) }

\texttt{03. \qquad{}IF n < 2: }

\texttt{04. \qquad{}\qquad{}RETURN arr }

\texttt{05. \qquad{}END IF }

\texttt{06. \qquad{}left = Merge\_Sort(PARTITION(arr, 0, n DIV 2)) }

\texttt{07. \qquad{}right = Merge\_Sort(PARTITION(arr, n DIV 2, n)) }

\texttt{08. \qquad{}RETURN Merge(left, right) }

\texttt{09. END FUNCTION }

\texttt{10.}

\texttt{11. FUNCTION Merge (ARRAY: left, LIST: right) }

\texttt{12. \qquad{}DECLARE results: ARRAY{[}SIZE(left) + SIZE(right){]} }

\texttt{13. \qquad{}DECLARE i, j, index : INTEGER }

\texttt{14. \qquad{}i, j, index = 0, 0, 0 }

\texttt{15. \qquad{}WHILE i < SIZE(left) and j < SIZE(right)}

\texttt{16. \qquad{}\qquad{}IF left{[}i{]} < right{[}j{]}:}

\texttt{17. \qquad{}\qquad{}\qquad{}results{[}index{]} = left{[}i{]} }

\texttt{18. \qquad{}\qquad{}\qquad{}i = i + 1}

\texttt{19. \qquad{}\qquad{}ELSE: }

\texttt{20. \qquad{}\qquad{}\qquad{}results{[}index{]} = right{[}j{]}}

\texttt{21. \qquad{}\qquad{}\qquad{}j = j + 1}

\texttt{22. \qquad{}\qquad{}END IF }

\texttt{23. \qquad{}\qquad{}index = index + 1 }

\texttt{24. \qquad{}END WHILE}

\texttt{25. \qquad{}WHILE i < SIZE(left)}

\texttt{26. \qquad{}\qquad{}results{[}index{]} = left{[}i{]} }

\texttt{27. \qquad{}\qquad{}i = i + 1}

\texttt{28. \qquad{}\qquad{}index = index + 1 }

\texttt{29. \qquad{}END WHILE}

\texttt{30. \qquad{}WHILE j < SIZE(right) }

\texttt{31. \qquad{}\qquad{}results{[}index{]} = right{[}j{]} }

\texttt{32. \qquad{}\qquad{}j = j + 1 }

\texttt{33. \qquad{}\qquad{}index = index + 1 }

\texttt{34. \qquad{}END WHILE}

\texttt{35. \qquad{}RETURN results}

\texttt{36. END FUNCTION }%
\end{minipage}}

\subsection*{Task 4.1 }

Write program code to implement the given procedure. Execute the function
using the following list as the parameter. 

\texttt{{[}1, 15, 2, 4, 3, 9, 7, 10{]}} 

\subsection*{Evidence 12 }

Program Code. Screenshot showing the output. \hfill{}{[}6{]}

\subsection*{Evidence 13}

Analyze and state the time complexity of the algorithm in big $O$
notation.\hfill{} {[}1{]}

\subsection*{Task 4.2 }

Bubble sort is a common sorting algorithm, below is an implementation
of in-place bubble sort using recursion. There are three missing lines
in the pseudocode, indicated as \texttt{A}, \texttt{B} and \texttt{C}. 

\noindent\fbox{\begin{minipage}[t]{1\columnwidth - 2\fboxsep - 2\fboxrule}%
\texttt{01. PROCEDURE Bubble\_Sort (ARRAY: arr, INTEGER: n) }

\texttt{02. \qquad{}IF \_\_\_\_\_\_\_A\_\_\_\_\_\_\_:}

\texttt{03. \qquad{}\qquad{}RETURN }

\texttt{04. \qquad{}END IF }

\texttt{05. }

\texttt{06. \qquad{}FOR i FROM 0 TO (n-1): }

\texttt{07. \qquad{}\qquad{}IF \_\_\_\_\_\_\_B\_\_\_\_\_\_\_: }

\texttt{08. \qquad{}\qquad{}\qquad{}temp = arr{[}i{]} }

\texttt{09. \qquad{}\qquad{}\qquad{}arr{[}i{]} = arr{[}i+1{]} }

\texttt{10. \qquad{}\qquad{}\qquad{}arr{[}i+1{]} = temp }

\texttt{11. \qquad{}\qquad{}END IF }

\texttt{12. \qquad{}END FOR }

\texttt{13.}

\texttt{14. \qquad{}\_\_\_\_\_\_\_C\_\_\_\_\_\_\_ }

\texttt{15. END PROCEDURE }%
\end{minipage}}

Execute the procedure to sort the following list. \texttt{{[}1, 15,
2, 4, 3, 9, 7, 10{]} }

\subsection*{Evidence 14}

Complete the missing lines \texttt{A}, \texttt{B} and \texttt{C}.
\hfill{}{[}3{]}

Insert a counting mechanism to count the number of comparisons made,
and return the value as an integer. Implement the new recursive bubble
sort. 

\subsection*{Evidence 15}

Program Code. 

Screenshot showing the output of the sorted list and count value.
\hfill{}{[}4{]}

\subsection*{Evidence 16 }

Analyze and state the time complexity of the algorithm in big $O$
notation. \hfill{} {[}1{]}

\subsection*{Task 4.3}

Due to the nature of bubble sort, if no swaps are observed in a given
iteration, then there is no need for the next iteration. Implement
the improved bubble sort. Execute the function to sort the following
list.

\texttt{{[}1, 15, 2, 4, 3, 9, 7, 10{]}}

\subsection*{Evidence 17}

Program Code. Screenshot showing the output of the sorted list and
count value. \hfill{}{[}3{]}