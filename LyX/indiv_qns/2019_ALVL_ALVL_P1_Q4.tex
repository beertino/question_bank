\item \textbf{{[}ALVL/9597/2019/P1/Q4{]} }

A stack is used to store characters.

\subsubsection*{Task 4.1}

Write program code to implement the stack and the operations specified.

Your code should allow operations to: 
\begin{itemize}
\item push an item on to the stack
\item pop an item off the stack
\item determine the size of the stack. A size of zero indicates that the
stack is empty.
\end{itemize}

\subsubsection*{Evidence 9}

Your program code for the stack.\hfill{}{[}10{]}

The stack is to be used to identify it an arithmetic expression is
balanced. 

An expression is balanced if each opening bracket has a corresponding
closing bracket. 

Different pairs of brackets can be used. These are: {[}{]}, () or
\{\}. 

This is an example of an expression that is balanced.

\texttt{\qquad{}({[}8-1{]}/(5{*}7)) }

This is an example of an expression that is not balanced. 

\texttt{\qquad{}{[}(8-1{]}/(5{*}7)) }

Note the change in the order of the first two open bracket symbols.
The first closing bracket should be a closing bracket ')' to match
the previous opening bracket \textquoteleft ('. 

Note that an expression is not balanced if the order of the brackets
is incorrect, even if there are the same number of opening and closing
brackets of each bracket type. 

An expression is checked by iterating over it: 
\begin{itemize}
\item if a non-bracket symbol is found, continue to the next character. 
\item If an opening symbol is found, push it on to the stack and continue
to the next character. 
\item If a closing bracket is encountered: 
\begin{itemize}
\item If the stack is empty, return an error (because there is no corresponding
opening bracket) 
\item else pop the symbol from the top of the stack and compare it to the
current closing symbol to see if they make a matching pair 
\item If they do match continue to the next character 
\item else return an error (pairs of brackets must match). 
\end{itemize}
\item When the last symbol is encountered: 
\begin{itemize}
\item return an error if the stack is not empty (too many opening symbols) 
\item else return a success message.
\end{itemize}
\end{itemize}

\subsubsection*{Task 4.2}

Add \textbf{five} other suitable test cases and a reason for choosing
each test case.
\begin{center}
\begin{tabular}{|l|l|l|}
\hline 
\hspace{0.05\columnwidth}\textbf{Test case} & \textbf{Reason for choice} & \textbf{Expected value}\tabularnewline
\hline 
\texttt{({[}8-1{]}/(5{*}7))} & Provided & Succeeds\tabularnewline
\hline 
\texttt{{[}(8-1{]}/(5{*}7))} & Provided & Fails\tabularnewline
\hline 
 &  & Succeeds\tabularnewline
\hline 
 &  & Succeeds\tabularnewline
\hline 
 &  & Fails\tabularnewline
\hline 
 &  & Fails\tabularnewline
\hline 
 &  & Fails\tabularnewline
\hline 
\end{tabular}
\par\end{center}

\subsubsection*{Evidence 10}

The completed table with all seven test cases and a reason for choosing
each test case.\hfill{}{[}6{]}

\subsubsection*{Task 4.3}

Write program code that checks expressions using the given algorithm. 

Use all \textbf{seven} test cases to verify It.

\subsubsection*{Evidence 11}

Your program code for the stack.

Screenshots for each test data run. \hfill{}{[}19{]}