\item \textbf{{[}DHS/PRELIM/9569/2020/P1/Q2{]} }

You have been tasked to use a suitable data structure to manage the
preliminary exam results of students in Dunman High School. Each student
is identified by its centre and index numbers, each of which is 4-digit.
For example, Lim Ah Seng's identification number is 30420188. A typical
range of students' identification numbers are from 30420001 to 30420450,
since each graduating cohort will have about 450 students. 

The preliminary examination details to be stored are as follows: 
\begin{itemize}
\item Subject code (4-digit) eg 9569 
\item Subject name eg H2 Computing 
\item Subject grade (1-character eg 'A') You may assume that each student
will have a valid subject grade in the range of {[}'A', 'B', 'C',
'D', 'E', 'S', 'U', 'T', '0'{]}, where 'T' stands for terminated and
'0' stands for absent. 
\end{itemize}
Using suitable examples, evaluate the pros and cons using each of
the following data structure to store the required information: 
\begin{enumerate}
\item dictionary 
\item hash table \hfill{}{[}12{]}
\end{enumerate}