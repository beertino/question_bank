\item \textbf{{[}ALVL/9569/2020/P1/Q5{]} }

Validation and verification are used in data entry.
\begin{enumerate}
\item {}
\begin{enumerate}
\item State the purpose of verification. \hfill{} {[}1{]}
\item State \textbf{one} method of verification. \hfill{} {[}1{]}
\end{enumerate}
\end{enumerate}
The use of check digits is one validation technique.
\begin{enumerate}
\item[\textbf{(b)}] {}
\begin{enumerate}
\item State the purpose of validation. \hfill{}{[}1{]}
\item State \textbf{three} other validation techniques. \hfill{}{[}3{]}
\item Name \textbf{two} types of error that check digits usually detect.
\hfill{}{[}2{]}
\end{enumerate}
\end{enumerate}
A check digit is to be added to the end of 02757 using Modulus 11.
The weight of each digit, starting with the first digit (0) is 6,
5, 4, 3, 2.
\begin{enumerate}
\item[\textbf{(c)}] Showing your working. determine the check digit for 02757. \hfill{}{[}3{]}
\item[\textbf{(d)}] Give \textbf{two} reasons why the data type of a field storing 02757
with a Modulus 11 check digit should be a string rather than an integer.
\hfill{} {[}2{]}
\end{enumerate}