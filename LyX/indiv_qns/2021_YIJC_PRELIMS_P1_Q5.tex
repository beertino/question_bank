\item \textbf{{[}YIJC/PRELIM/9569/2021/P1/Q5{]} }

The Food Services Industry Digital Plan (IDP) was recently launched
by the Minister of State for Trade and Industry to help F\&B businesses
adopt more digital technologies in their daily operations. The manager
of a Iocal restaurant engaged a consultant to propose a digital solution
for his restaurant operations. 

After conducting a comprehensive study, the consultant proposed a
web-based solution using a client-server model. The solution requires
the following hardware to access the web server wirelessly in the
local area network (LAN): 
\begin{itemize}
\item A tablet device on each table for customers to browse the menu and
order their food items. 
\item Multiple large monitors for the chefs in the kitchen to read the ordered
food items. 
\item A computer station for the service staff to check the table number
before serving the food to the customers. 
\item A computer in the manager\textquoteright s office to update the menu
in the web server and print the daily sales report. 
\end{itemize}
When a customer decides to pay the bill, a QR code will be generated
on the tablet device for him to scan and make online payment using
his personal mobile device.
\begin{enumerate}
\item Explain the meaning of the term client-server model. \hfill{}{[}1{]}
\item Describe the main software components to be developed for the web
server to host this service for the restaurant. \hfill{}{[}2{]}
\item Describe how the customer would use the client tablet device to browse
and order the food items. \hfill{}{[}3{]}
\item Suggest \textbf{one} feature on the digital form that will provide
a positive experience for the customers when using the tablet device
to order their food. Describe the usability principle applied in this
feature. \hfill{}{[}2{]}
\item The manager recommends the proposed solution to the shareholders,
but a number of social issues associated with the solution have been
raised. Describe \textbf{two} possible issues that could have been
raised.\hfill{} {[}2{]}
\end{enumerate}
An alternative to this web-based solution would be to develop a native
application programme for the customers to download and install on
their mobile devices. 
\begin{enumerate}
\item[(f)]  Describe \textbf{one} feature that is only available in the native
application solution and how it is relevant to the solution proposed
for the restaurant. \hfill{}{[}2{]}
\end{enumerate}
The restaurant\textquoteright s manager is also keen to expand his
business to accept online ordering for takeaways. 
\begin{enumerate}
\item[(g)]  Explain \textbf{one} benefit of the web-based solution in this situation.\hfill{}
{[}1{]}
\item[(h)]  Draw the network diagram for the proposed web-based solution and
include all the required hardware for the restaurant to accept online
ordering.\hfill{} {[}5{]}
\item[(i)]  Describe \textbf{two} benefits for the restaurant in implementing
this solution. \hfill{}{[}2{]}
\item[(j)]  For online ordering, the restaurant needs to collect the customer's
name, address and contact number. State and describe \textbf{two}
data protection obligations that the manager needs to comply under
the Personal Data Protection Act. \hfill{}{[}4{]}
\end{enumerate}