\item \textbf{{[}PJC/PRELIM/9597/2018/P1/Q7{]} }

Visitors can claim the GST (Goods and Services Tax) before they leave
the country. A programmer is going to write part of the computer system
using an object-oriented programming (OOP) language, which will store
details of claims by visitor either pay by cash or hand phone transfer.
The claim receive by hand phone transfer will have a rebate of 0.2\%.
while claim receive by cash will have to pay 0.5\% of service charge
and recording the currency exchange rate. 

Properties identified the claims included: 
\begin{itemize}
\item Passport number
\item Receipt number 
\end{itemize}
Type of claims (cash or hand phone transfer)
\begin{enumerate}
\item Draw a diagram that shows how the properties could be distributed
amongst a number of classes. Include in your diagram any inheritance
between classes. Also indicate appropriate methods (including one
pair of 'get' and 'set' methods for one of the properties) that would
be required. One method should demonstrate polymorphism. {[}6{]}
\item In the context of object-oriented programming explain what is meant
by: 
\begin{enumerate}
\item Encapsulation and how classes support information hiding and implementation
independence. \hfill{} {[}3{]}
\item Inheritance and how it promotes software reuse. \hfill{}{[}2{]}
\item Polymorphism and how it enables code generalisation. \hfill{}{[}2{]}
\item Computational thinking and why it is important? \hfill{} {[}5{]}
\end{enumerate}
\item Give \textbf{two} advantages of object-oriented programming. \hfill{}{[}2{]}
\end{enumerate}