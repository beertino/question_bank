\item \textbf{{[}JJC/PRELIM/9597/2018/P2/Q4{]} }

An object-oriented program is being written to store details about
clients at a real estate agency. 

Clients can be either sellers or prospective buyers. A class \texttt{Client}
has been created and two subclasses, \texttt{Seller} and \texttt{Buyer}
are to be developed. A \texttt{Location} class has been created to
store details about an address (e.g. postal code, street name and
district). 

The \texttt{Client} class has data fields \texttt{Name}, \texttt{Address}
and \texttt{DOB}. Part of the class definition for \texttt{Client}
class is: 

\noindent\begin{minipage}[t]{1\columnwidth}%
\texttt{Client = Class}

\texttt{\qquad{}// Private }

\texttt{\qquad{}Name: String}

\texttt{\qquad{}Address: Location }

\texttt{\qquad{}DOB: Date}

\bigskip{}

\texttt{\qquad{}// Public }

\texttt{\qquad{}Constructor() }

\texttt{\qquad{}Function GetName(): String }

\texttt{\qquad{}Function GetDOB(): Date Function}

\texttt{\qquad{}GetAddress(): Location }

\texttt{\qquad{}Procedure SetDetails(String, Location, Date) }%
\end{minipage}

A \texttt{Buyer} has the following additional data fields: 
\begin{itemize}
\item \texttt{NoOfBedroomsRequired}: stores the minimum number of bedrooms
that the buyer requires in the property they purchase.
\item \texttt{ParkingSpace}: stores a value indicating if the buyer requires
parking or not. 
\item \texttt{AreaDesired}: the district the buyer is looking to purchase
a property in. For example, Jurong is in district 22 while Orchard
is in district 9. 
\end{itemize}
\begin{enumerate}
\item Write the class definition for \texttt{Buyer}. \hfill{}{[}4{]}
\item Hence, illustrate the diagram which exhibits the relationships between
all classes as well as inheritance and encapsulation. \hfill{}{[}6{]}
\item Explain why encapsulation is an important feature of object-oriented
programming. \hfill{}{[}2{]}
\end{enumerate}