\item \textbf{{[}DHS/PRELIM/9597/2018/P1/Q2{]} }

The International Bank Account Number (IBAN) is an international bank
account identification standard used in many countries. It uses modulo
arithmetic to perform validation. The IBAN consists of up to 34 alphanumeric
characters, as follows: country code -- two letters check digits
-- two digits, and basic bank account number -- up to 30 alphanumeric
characters that are countryspecific An example of an IBAN in the United
Kingdom Great Britain which is 22 characters is \texttt{GB82WEST12345698765432}
(\texttt{GB} - country code, \texttt{82} - check digits) 

The following is the IBAN check digits generation algorithm: 
\begin{enumerate}
\item[1.]  Initialize the two check digits by \texttt{00} (e.g. \texttt{GB}\texttt{\uline{00}}\texttt{WEST12345698765432}).
\item[2.]  Move the four initial characters to the end of the string (e.g.
\texttt{WEST12345698765432GB}\texttt{\uline{00}}). 
\item[3.]  Replace each alphabet in the string with two digits, using the mapping
A = 10, B = 11, C = 12, . . . . . , Z = 35 (i.e. ASCII value of uppercase
letters - 55) (e.g. \texttt{\uline{32142829}}\texttt{12345698765432}\texttt{\uline{1611}}\texttt{00}). 
\item[4.]  Convert the string to an integer. 
\item[5.]  Calculate the remainder of dividing this number by 97 (e.g. \texttt{3214282912345698765432161100
mod 97 = 16}).
\item[6.]  Subtract the remainder from 98 to give the two check digits (e.g.
check digits \texttt{= 98 - 16 = 82}). If the result is a single digit
number, pad it with a leading 0 to make a two-digit number. 
\end{enumerate}

\subsection*{Task 2.1 }

Write program code for the \texttt{CheckDigits} function using the
following specification. 
\noindent \begin{center}
\texttt{FUNCTION CheckDigits (IBAN : STRING) : STRING }
\par\end{center}

The function has a single string parameter IBAN and returns a two-digit
string result. Use the sample data provided in the text file \texttt{IBANS.txt}
and paste this into your program code.

\subsection*{Evidence 8 }

Your program code. \hfill{}{[}6{]}

\subsection*{Evidence 9 }

One screenshot verifying that your program generates the correct check
digits for the data in \texttt{IBANS.txt}. \hfill{}{[}1{]}

An IBAN is validated by converting it into an integer and performing
a basic mod-97 operation on it. If the IBAN is valid, the remainder
equals 1. The algorithm of IBAN validation is as follows:
\begin{enumerate}
\item[1.]  Move the four initial characters to the end of the string (For the
IBAN \texttt{GB82WEST12345698765432} e.g. \texttt{WEST12345698765432}\texttt{\uline{GB82}}).
\item[2.]  Replace each letter in the string with two digits, thereby expanding
the string, where A = 10, B = 11, C = 12, . . . . . , Z = 35 

(e.g. \texttt{\uline{32142829}}\texttt{12345698765432}\texttt{\uline{1611}}\texttt{82}). 
\item[3.]  Interpret the string as a decimal integer and compute the remainder
of that number on division by 97 

(e.g. \texttt{3214282912345698765432161182 mod 97 = 1}). 
\end{enumerate}
If the remainder is 1, the check digit test is passed and the IBAN
might be valid.

\subsection*{Task 2.2 }

Write a Boolean function \texttt{ValidateIBAN} to determine if a given
IBAN is valid. This function should have a parameter which allows
it to be used for any IBAN. 

\subsection*{Evidence 10 }

Your \texttt{ValidateIBAN} program code. \hfill{}{[}2{]}

\subsection*{Task 2.3 }

A \texttt{TRANSACTIONS.txt} file contains transaction details of customers
of a bank. Each transaction record takes up one line and has three
data fields: customer IBAN, transaction mode (\texttt{W} - withdrawal
or \texttt{D} - deposit) and transaction amount. 

Write a procedure \texttt{CheckIBAN} to read in the IBANs in \texttt{TRANSACTIONS.txt}
and display on the screen: 
\begin{itemize}
\item If an IBAN is valid, the valid IBAN followed by \textquotedblleft \texttt{OK}\textquotedblright . 
\item If an IBAN is invalid, the invalid IBAN followed by \texttt{\textquotedblleft Invalid. Expected
check digits: ??\textquotedblright }, where ?? represents the computed
check digits. 
\item For an invalid IBAN, update the record in \texttt{TRANSACTIONS.txt}
with the expected computed check digits.
\end{itemize}

\subsection*{Evidence 11 }

Your \texttt{CheckIBAN} program code.\hfill{} {[}7{]}

\subsection*{Evidence 12 }

One screenshot showing the output and contents of \texttt{TRANSACTIONS.txt}
from running the program. {[}2{]} 

\subsection*{Task 2.4 }

A master file \texttt{ACCOUNTS.txt} contains the customer IBANs, names
and current balances. 

Write a procedure \texttt{UpdateBalance} to update the current balances
of the customers in \texttt{ACCOUNTS.txt} from \texttt{TRANSACTIONS.txt}.
At the end of the process, your program will output the message: 

\texttt{x records updated. }

\subsection*{Evidence 13}

Your program code for the procedure \texttt{UpdateBalance}.\hfill{}
{[}8{]}

\subsection*{Evidence 14}

One screenshot showing the program output and contents of \texttt{ACCOUNTS.txt}
from running the program.\hfill{} {[}2{]}