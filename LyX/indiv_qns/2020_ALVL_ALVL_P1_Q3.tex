\item \textbf{{[}ALVL/9569/2020/P1/Q3{]} }

A binary search tree is implemented using an array, \texttt{b\_tree}.
Each element of the array comprises three parts: \texttt{l\_ptr} and
\texttt{r\_ptr} are of data type integer and \texttt{data\_item} is
of data type char. 
\begin{center}
\begin{tabular}{|c|c|c|}
\hline 
\texttt{l\_ptr} & \texttt{data\_item} & \texttt{r\_ptr}\tabularnewline
\hline 
\end{tabular}
\par\end{center}

The root of the binary search tree is stored in an integer variable,
\texttt{root}. The contents of \texttt{b\_tree} is shown below. $-1$
represents the null pointer. 
\begin{center}
\begin{tabular}{c|c|c|c|}
\cline{2-4} \cline{3-4} \cline{4-4} 
Index & \texttt{l\_ptr} & \texttt{data\_item} & \texttt{r\_ptr}\tabularnewline
\cline{2-4} \cline{3-4} \cline{4-4} 
$0$ & $-1$ & A & $-1$\tabularnewline
\cline{2-4} \cline{3-4} \cline{4-4} 
$1$ & $0$ & + & $2$\tabularnewline
\cline{2-4} \cline{3-4} \cline{4-4} 
$2$ & $-1$ & B & $-1$\tabularnewline
\cline{2-4} \cline{3-4} \cline{4-4} 
$3$ & $1$ & $\star$ & $5$\tabularnewline
\cline{2-4} \cline{3-4} \cline{4-4} 
$4$ & $-1$ & C & $-1$\tabularnewline
\cline{2-4} \cline{3-4} \cline{4-4} 
$5$ & $4$ & $-$ & $6$\tabularnewline
\cline{2-4} \cline{3-4} \cline{4-4} 
$6$ & $-1$ & D & $-1$\tabularnewline
\cline{2-4} \cline{3-4} \cline{4-4} 
\end{tabular}\qquad{}\qquad{} %
\begin{tabular}{|c|}
\hline 
\texttt{root}\tabularnewline
\hline 
\hline 
$3$\tabularnewline
\hline 
\end{tabular}
\par\end{center}
\begin{enumerate}
\item Draw the binary search tree represented by the array \texttt{b\_tree}.\hfill{}
{[}2{]}
\item State the index of a leaf node in \texttt{b\_tree}.\hfill{} {[}1{]}
\end{enumerate}
A procedure, \texttt{P}, uses recursion.

\begin{algorithm}[H]
\texttt{01 PROCEDURE P (Index: INTEGER)}

\texttt{02 ~~IF b\_tree{[}Index{]}.l\_ptr <> -1 THEN}

\texttt{03 ~~~~~P(b\_tree{[}lndex{]}.1\_pt.r)}

\texttt{04 ~~ENDIF}

\texttt{05 ~~IF b\_tree{[}Index{]}.r\_ptr <> \textemdash 1 THEN}

\texttt{O6 ~~~~~P(b\_tree{[}lndex{]}.r\_ptr)}

\texttt{07 ~~ENDIF}

\texttt{08 ~~OUTPUT b\_tree{[}Index{]}.data\_item}

\texttt{09 ENDPROCEDURE}
\end{algorithm}

\begin{enumerate}
\item[\textbf{(c)}] {}
\begin{enumerate}
\item State what is meant by \textbf{recursion}. \hfill{}{[}1{]} 
\item State the line numbers that indicate procedure \texttt{P} is recursive.
\hfill{}{[}1{]} 
\item State the significance of lines \texttt{02} and \texttt{05}. \hfill{}{[}1{]} 
\end{enumerate}
\end{enumerate}
Procedure \texttt{P} is called with the parameter value of 1. 
\begin{enumerate}
\item[\textbf{(d)}] Copy and then complete the trace table for procedure \texttt{P} showing
the values of \texttt{Index} and the output. 
\begin{center}
\begin{tabular}{|c|c|}
\hline 
\texttt{Index} & Output\tabularnewline
\hline 
\hline 
\texttt{1} & \tabularnewline
\hline 
 & \tabularnewline
\end{tabular}
\par\end{center}

\hfill{}{[}5{]}
\item[\textbf{(e)}] Explain the use of a stack when the recursive procedure \texttt{P}
executes. \hfill{}{[}3{]}
\item[\textbf{(f)}] Identify the type of tree traversal that procedure \texttt{P} performs.
\hfill{}{[}1{]}
\end{enumerate}
A 1-dimensional array is used to hold a queue. 
\begin{enumerate}
\item[\textbf{(g)}] Explain the concept of a queue. \hfill{}{[}2{]}
\end{enumerate}
Queues can be either linear or circular. 
\begin{enumerate}
\item[\textbf{(h)}] State \textbf{two} differences between a circular queue and a linear
queue.\hfill{} {[}2{]}
\end{enumerate}