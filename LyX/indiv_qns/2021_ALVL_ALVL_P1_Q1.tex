\item \textbf{{[}ALVL/9569/2021/P1/Q1{]} }

A programmer is writing software for a baker who makes celebration
(wedding and birthday) cakes to order. All cakes are circular.

The programmer wishes to use object-oriented programming to model
the order book and calculate the total price for a cake. 

For every cake. the following items of data are recorded: 
\begin{itemize}
\item an order number 
\item the customer\textquoteright s name \textquoteleft{} 
\item the customer's telephone number 
\item diameter (in cm) 
\item any special requirements (eg. must not contain nuts) 
\item standard price (in S). 
\end{itemize}
For wedding cakes the number of layers, a maximum of four. is also
recorded. The total price is the number of layers multiplied by the
standard price. 

For birthday cakes the number of candles and price of a candle is
recorded. The total price is the standard price plus the number of
candles multiplied by the price of a candle. 
\begin{enumerate}
\item Draw a class diagram that shows the following for the situation described
above:
\begin{itemize}
\item the superclass
\item any subclasses
\item inheritance
\item properties
\item appropriate methods. \hfill{}{[}12{]}
\end{itemize}
\item Name two suitable validation techniques that might be applied to the
number of layers on a wedding cake.\hfill{} {[}2{]}
\item Explain inheritance using examples from this situation. \hfill{}{[}4{]}
\end{enumerate}