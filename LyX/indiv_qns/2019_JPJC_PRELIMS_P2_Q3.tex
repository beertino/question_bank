\item \textbf{{[}JPJC/PRELIM/9597/2019/P2/Q3{]} }
\begin{enumerate}
\item \textbf{Copy} and \textbf{complete} the algorithm for a binary search
written in pseudocode shown below. It is given that the data being
searched is stored in the array \texttt{SearchData{[}63{]}}, and the
item of data being searched is stored in the variable \texttt{SearchItem}. 

\noindent %
\noindent\begin{minipage}[t]{1\columnwidth}%
\texttt{X <- 0 }

\texttt{Low <- 1 }

\texttt{High <- \dots \dots \dots \dots \dots \dots \dots \dots \dots \dots \dots \dots \dots \dots \dots{} }

\texttt{WHILE (High >= Low) AND (\dots \dots \dots \dots \dots \dots \dots \dots \dots \dots \dots \dots \dots \dots \dots \dots \dots \dots \dots )}

\texttt{\qquad{}Middle INT((High + Low)/2) }

\texttt{\qquad{}IF SearchData{[}Middle{]} = SearchItem }

\texttt{\qquad{}\qquad{}THEN }

\texttt{\qquad{}\qquad{}\qquad{}X <- Middle }

\texttt{\qquad{}\qquad{}ELSE}

\texttt{\qquad{}\qquad{}\qquad{}IF SearchData{[}Middle{]} < SearchItem }

\texttt{\qquad{}\qquad{}\qquad{}\qquad{}THEN}

\texttt{\qquad{}\qquad{}\qquad{}\qquad{}\qquad{}Low <- Middle
+ 1}

\texttt{\qquad{}\qquad{}\qquad{}\qquad{}ELSE }

\texttt{\qquad{}\qquad{}\qquad{}\qquad{}\qquad{}IF SearchData{[}Middle{]}
> SearchItem }

\texttt{\qquad{}\qquad{}\qquad{}\qquad{}\qquad{}\qquad{}THEN}

\texttt{\qquad{}\qquad{}\qquad{}\qquad{}\qquad{}\qquad{}\qquad{}\dots \dots \dots \dots \dots \dots \dots \dots \dots \dots \dots \dots \dots \dots \dots \dots{} }

\texttt{\qquad{}\qquad{}\qquad{}\qquad{}\qquad{}ENDIF }

\texttt{\qquad{}\qquad{}\qquad{}ENDIF }

\texttt{\qquad{}ENDIF }

\texttt{ENDWHILE }%
\end{minipage}

\hfill{}{[}3{]}
\item State the maximum number of comparisons that are required to find
an item which is present in \texttt{SearchData}. \hfill{} {[}1{]}
\item You will change the binary search algorithm to a recursive algorithm
and write the equivalent program code in the form of a procedure.
Name the recursive procedure \texttt{BinarySearch}. Use the following
variables: 
\noindent \begin{center}
\begin{tabular}{|l|l|l|}
\hline 
\textbf{Variable} & \textbf{Data Type} & \textbf{Description}\tabularnewline
\hline 
\texttt{SearchData} & \texttt{ARRAY{[}63{]} : INTEGER} & global array\tabularnewline
\hline 
\texttt{SearchItem} & \texttt{INTEGER} & global variable\tabularnewline
\hline 
\texttt{X} & \texttt{INTEGER} & global variable\tabularnewline
\hline 
\texttt{Low} & \texttt{INTEGER} & parameter\tabularnewline
\hline 
\texttt{High} & \texttt{INTEGER} & global variable\tabularnewline
\hline 
\texttt{Middle} & \texttt{INTEGER} & local variable\tabularnewline
\hline 
\end{tabular} 
\par\end{center}

Write pseudocode or program code for the recursive procedure \texttt{BinarySearch}.
\hfill{}{[}4{]}
\end{enumerate}