\item \textbf{{[}ALVL/9597/2019/P2/Q6{]} }

A function is to be written that returns the sum of all values held
in an array that are greater than a minimum value. The function will
be used with arrays of varying size, but never more than a maximum
of 50 000 elements.

A first attempt at writing the program code for the function is given
below:

\noindent %
\noindent\begin{minipage}[t]{1\columnwidth}%
\texttt{01 FUNCTION TotalSum(Results : ARRAY{[}50000{]} OF REAL, ArraySize
: INTEGER, MinValue : REAL) RETURNS REAL }

\texttt{02 \qquad{}DECLARE Sum, Counter : INTEGER }

\texttt{03 \qquad{}DECLARE Temp : Real }

\texttt{04 \qquad{}Sum = 0.0 }

\texttt{05 \qquad{}\qquad{}FOR Counter = 1 TOO ArraySize }

\texttt{06 \qquad{}\qquad{}\qquad{}Temp = Results{[}Counter{]} }

\texttt{07 \qquad{}\qquad{}\qquad{}IF Temp > MinValue THEN Sum
= Sum {*} Temp }

\texttt{08 \qquad{}\qquad{}ENDEOR }

\texttt{09 \qquad{}RETURN Sum }

\texttt{10 ENDFUNCTION}%
\end{minipage}

The function is included in a program specifically written to test
the function. The main program outputs the value returned by the function.
A compiler was used to compile the source program.
\begin{enumerate}
\item The compiler reported an error at line 5 in the function. Identify
the error and explain why it was flagged as a syntax error. \hfill{}{[}2{]}
\item The compiler also reported an error at line 8. State the type of error
reported by the compiler justifying your answer. \hfill{}{[}2{]}
\end{enumerate}
The errors indicated in \textbf{parts (a)} and \textbf{(b)} were corrected.
A successful compilation produces executable code. When the code was
executed, the program failed to complete and reports an error at line
7.
\begin{enumerate}
\item[(c)]  {} 
\begin{enumerate}
\item State the type of error that occurred. Justify your answer. \hfill{}{[}2{]}
\item The error described in \textbf{part (c) (i)} depends on the detection
of another type of error. Name this other type of error. How should
the code be changed to correct this error?\hfill{}{[}2{]}
\end{enumerate}
\end{enumerate}
When the program finally runs without error, the test plan needs to
be completed. The test plan uses data that tests different sizes of
array, different array values and different minimum values. 

The array \texttt{TempArray} is used in the main program as the array
to be processed.\quad{} 
\begin{enumerate}
\item[(d)]  Each element of \texttt{TempArray} stores a random value between
1.0 and 10.0.
\begin{enumerate}
\item Explain why the function call:

\texttt{TotalSum(TempArray, 1000, 5.0)}

is not an appropriate black box test. \hfill{}{[}2{]}
\item Explain why the function call:

\texttt{TotalSum(TempArray, 10, 10.5)}

is not an appropriate white box test. \hfill{}{[}2{]}
\end{enumerate}
\item[(e)] if each element of \texttt{TempArray} stores the value 1.0, state
a function call that will be an appropriate black box test. Justify
your answer. \hfill{}{[}3{]}
\end{enumerate}