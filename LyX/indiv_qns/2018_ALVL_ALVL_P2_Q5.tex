\item \textbf{{[}ALVL/9597/2018/P2/Q5{]} }

The organisers of a diving championship have created software to calculate
and show the total score for each diver. 

There are nine judges scoring each dive. The two best scores and the
two worst scores are ignored. The other five scores are added together
to give the diver\textquoteright s total score. 
\begin{enumerate}
\item Write an algorithm to take in the nine scores, delete the best two
and the two worst scores, and total the five remaining scores. \hfill{}{[}4{]}
\end{enumerate}
There are 10 divers in the final. The scoreboard shows the order of
diving. 
\begin{center}
\begin{tabular}{|c|c|l|}
\hline 
\textbf{Order} & \textbf{Diver name} & \textbf{Total score}\tabularnewline
\hline 
1 & Daniel Tan & \tabularnewline
\hline 
2 & Parker Lam & \tabularnewline
\hline 
3 & Mohamed Noor & \tabularnewline
\hline 
4 & Hariz Yazid & \tabularnewline
\hline 
5 & Sheryl Xuan & \tabularnewline
\hline 
6 & Karl Lim & \tabularnewline
\hline 
7 & Elaine Ning & \tabularnewline
\hline 
8 & Nadyia Esmadi & \tabularnewline
\hline 
9 & Cai Ng & \tabularnewline
\hline 
10 & Hamid Mahmood & \tabularnewline
\hline 
\end{tabular}
\par\end{center}
\begin{enumerate}
\item The programmers decided to use a 1D array for the scores. They will
also use a bubble sort to sort the scores into descending order.
\begin{enumerate}
\item Explain how a bubble sort can be used to arrange the scores into a
descending or ascending order. \hfill{}{[}2{]}
\end{enumerate}
This is the bubble sort algorithm for sorting into descending order:

\noindent\begin{minipage}[t]{1\columnwidth}%
\texttt{01 WHILE NoSwaps = FALSE}

\texttt{02 \qquad{}NoSwaps <- TRUE}

\texttt{03 \qquad{}UpperBound <- ListLength}

\texttt{04 \qquad{}FOR Posn <- 0 TO ......}\texttt{\textbf{A}}\texttt{......}

\texttt{05 \qquad{}\qquad{}IF List{[}Posn{]} < ......}\texttt{\textbf{B}}\texttt{......}

\texttt{06 \qquad{}\qquad{}\qquad{}THEN}

\texttt{O7 \qquad{}\qquad{}\qquad{}\qquad{}// Swap}

\texttt{O8 \qquad{}\qquad{}\qquad{}\qquad{}NoSwaps <- ......}\texttt{\textbf{C}}\texttt{......}

\texttt{09 \qquad{}\qquad{}\qquad{}\qquad{}Temp <- List{[}Posn{]}}

\texttt{10 \qquad{}\qquad{}\qquad{}\qquad{}List{[}Posn{]} <- ListiPosn
+ 1{]}}

\texttt{11 \qquad{}\qquad{}\qquad{}\qquad{}List{[}Posn + 1{]}
<- ......}\texttt{\textbf{D}}\texttt{......}

\texttt{12 \qquad{}\qquad{}ENDIF}

\texttt{l3 \qquad{}ENDFOR}

\texttt{14 ENDWHILE}%
\end{minipage}
\begin{enumerate}
\item Write the pseudocode for \texttt{\textbf{A}}, \texttt{\textbf{B}},
\texttt{\textbf{C}}\textbf{ a}nd \texttt{\textbf{D}} in the algorithm.
\hfill{} {[}4{]}
\end{enumerate}
\item During the first round of dives, the sorted scores for five divers
are: 

\texttt{48 }

\texttt{45 }

\texttt{40 }

\texttt{37 }

\texttt{36 }

The sixth diver scores 42 and the software appends the score to the
list as follows: 

\texttt{48 }

\texttt{45 }

\texttt{40 }

\texttt{37 }

\texttt{36 }

\texttt{42}
\begin{enumerate}
\item State the number of passes needed through the list to return the list
to its sorted order. \hfill{}{[}1{]}
\item Explain why the bubble sort is efficient in this example. \hfill{}{[}2{]}
\item Name another sort method that could have been used in this situation. 

Compare the speed of sorting the divers\textquoteright{} scores in
your named method with using the bubble sort. \hfill{}{[}2{]}
\end{enumerate}
\end{enumerate}