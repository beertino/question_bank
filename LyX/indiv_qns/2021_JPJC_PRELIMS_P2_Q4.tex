\item \textbf{{[}JPJC/PRELIM/9569/2021/P2/Q4{]}}

JP Mobile sells mobile phones and manages its inventory using a NoSQL
database. Information about the mobile phones is stored in the JSON
file \texttt{items.JSON}. 

The following fields are recorded: 
\begin{itemize}
\item brand of mobile phone, 
\item model, 
\item colour(s) available, 
\item price in dollar, 
\item quantity in stock. 
\end{itemize}

\subsubsection*{Task 4.1 }

Write program code to import the information from the JSON file into
a MongoDB database. Save the information under the \texttt{phone}
collection in the \texttt{jp\_mobile} database. Ensure that the collection
only stores the information from the JSON file. 

Save your program code as \texttt{TASK4\_<your name>\_<class>\_<index
number>.py }\hfill{}{[}4{]}

\subsubsection*{Task 4.2 }

The shop decides to include one or more free gifts for new batches
of mobile phones it sells. 

Write program code for a user to insert information of a mobile phone
by getting user input of the following: brand, model, colour, price,
quantity, free gift(s). 

Your code should allow user to input one or more free gifts. 

If a phone\textquoteright s brand, model and colour already exists,
add the new quantity to the existing quantity in the database, and
replace the existing price with the new price. 

Run your program and insert the following 2 documents: 
\noindent \begin{center}
\begin{tabular}{|c|c|c|c|c|c|c|}
\hline 
No. & Brand  & Model  & Colour  & Price  & Quantity  & Free gift(s)\tabularnewline
\hline 
1  & orange  & 22  & black  & 900  & 11  & power bank\tabularnewline
\hline 
2  & solo  & A33  & red  & 1300  & 7  & power bank, earbuds\tabularnewline
\hline 
\end{tabular}
\par\end{center}

Add your program code to \texttt{TASK4\_<your name>\_<class>\_<index
number>.py} \hfill{}{[}7{]}

\subsubsection*{Task 4.3 }

Write a function \texttt{display\_all} that will display all the information
in the \texttt{phone} collection under these fields: \texttt{brand},
\texttt{model}, \texttt{colour}, \texttt{price}, \texttt{quantity},
\texttt{free gift(s)}. 

If no free gift comes with the phone, print a \texttt{None} statement.

Include a final statement that shows the total number of documents
in the collection. 

Run the \texttt{display\_all} function to show your output. 

Add your program code to \texttt{TASK4\_<your name>\_<class>\_<index
number>.py} \hfill{}{[}6{]}

\subsubsection*{Task 4.4}

The shop uses a web browser to display the database content. The manager
wants to filter the mobile phones by \texttt{brand} and display the
results in a web browser. 

Write additional Python code and the necessary files to create a web
application that 
\begin{itemize}
\item receives a \texttt{brand} string from a HTML form, then 
\item creates and returns a HTML document that enables the web browser to
display an ordered list of mobile phones sorted by \texttt{price}. 
\end{itemize}
For each document, the web page should include the: 
\begin{itemize}
\item brand as the heading 
\item model, 
\item colour, 
\item price 
\end{itemize}
Save your program as 

\texttt{TASK4\_4\_<your name>\_<class>\_<index number>.py} 

with any additional files / sub-folders as needed in a folder named 

\texttt{TASK4\_4\_<your name>\_<class>\_<index number>} 

Run the web application and input brand as \textquoteleft \texttt{solo}\textquoteright{}
in the webpage.

Save the output of the program as 

TASK4\_4\_<your name>\_<class>\_<index number>.html\hfill{} {[}12{]}