\item \textbf{{[}DHS/PRELIM/9569/2021/P2/Q2{]} }

\textbf{Task 2} 

Name your Jupyter Notebook and save all parts for this task as 

\texttt{TASK2\_<index\_number>\_<name>.ipynb }

You will be writing a Math game program for Form Teachers to play
with students in school. The game auto-generates math expressions
and tracks scoring. 

\subsubsection*{Task 2.1 }

Using a stack data structure, write a function \texttt{solver(expr)}
that takes in a string of a mathematical expression expr such as \texttt{\textquotedbl ((1{*}7)+6)\textquotedbl}
and returns \texttt{13}. You may assume that the entire expression
would never have spaces, and would always be enclosed in an opening
and closing parenthesis \textquotedbl ( )\textquotedbl . Do not
use the built-in function \texttt{eval()}. Test your function with
\texttt{solver(\textquotedbl ((1{*}7)+6)\textquotedbl )} and show
your output. \hfill{}{[}6{]}

\subsubsection*{Task 2.2 }

Write a function \texttt{generate\_expression} that takes in an integer
\texttt{operator\_count} and returns a string of a mathematical expression
which has the specified number of operators (i.e. \texttt{+ - {*}
/} ) in \texttt{operator\_count}. The function should use \textbf{recursion}
to form up the operators and operands. The operands, operators and
positions of operands and operators should be random. 

For example, \texttt{generate\_expression(5)} would output \textquotedbl\texttt{(4{*}(6-(2+((1{*}7)+6))))}\textquotedbl{}
and \texttt{generate\_expression(5}) would output \textquotedbl\texttt{((8+(6+((2{*}4)-3)))+6)}\textquotedbl . 

Test your function with \texttt{generate\_expression(5)} and show
your output. \hfill{}{[}7{]}

\subsubsection*{Task 2.3 }

Implement the following using object-oriented programming: 
\begin{itemize}
\item \texttt{Person}, a class, which 
\begin{itemize}
\item initialises with these attributes 
\begin{itemize}
\item \texttt{name: string }
\item \texttt{gender: string} where male is \textquotedbl\texttt{M}\textquotedbl{}
and female is \textquotedbl\texttt{F}\textquotedbl{} 
\item \texttt{score: integer }
\end{itemize}
\item has the following methods 
\begin{itemize}
\item \texttt{display\_info()} which displays the \texttt{Person}\textquoteright s
\texttt{name}, \texttt{gende}r and \texttt{score} 
\begin{itemize}
\item 1. eg \textquotedbl\texttt{Nelson(M)\textquoteright s score is 3.}\textquotedbl{} 
\end{itemize}
\item \texttt{attempts()} which
\begin{itemize}
\item 1. uses the function \texttt{generate\_expression} from Task 2.2 to
generate and display a random math expression of 2 operators 
\item 2. queries the user to give an answer rounded up to the nearest integer 
\item 3. displays \textquotedbl\texttt{Good job!}\textquotedbl{} if the
input is correct or \textquotedbl\texttt{Wrong answer. (Correct
answer: <answer>)}\textquotedbl{} where \texttt{<answer>} is the correct
answer. 
\item 4. increases the score of the student by \texttt{1} if the answer
is correct 
\item 5. displays the user\textquoteright s latest \texttt{sc}ore 
\end{itemize}
\end{itemize}
\end{itemize}
\item \texttt{Student}, a subclass of \texttt{Person}, which 
\begin{itemize}
\item also has the following attribute 
\begin{itemize}
\item \texttt{role: string} which is \textquotedbl\texttt{no role}\textquotedbl{}
by default unless the \texttt{Student} has a class committee role
such as \textquotedbl\texttt{chairperson}\textquotedbl{} 
\end{itemize}
\item also has the following methods 
\begin{itemize}
\item \texttt{student\_role()} which returns a string describing the \texttt{role}
of the \texttt{Student}
\end{itemize}
\end{itemize}
\item \texttt{FormTeacher}, a subclass of \texttt{Person}, which 
\begin{itemize}
\item also has the following methods
\begin{itemize}
\item \texttt{display\_info()} which uses polymorphism to display the\texttt{
FormTeacher}\textquoteright s information with salutation to the FormTeacher\textquoteright s
name 
\begin{itemize}
\item 1. eg: \textquotedbl\texttt{Ms. Norah's score is 0}.\textquotedbl{}
where \textquotedbl\texttt{Norah}\textquotedbl{} is her name, and
\textquotedbl\texttt{Ms}.\textquotedbl{} corresponds to her gender.
\hfill{} {[}14{]}
\end{itemize}
\end{itemize}
\end{itemize}
\end{itemize}

\subsubsection*{Task 2.4}

Write driver code to test the earlier class you created. Also, create
\texttt{groups} which is a list that uses a 2-dimensional array to
store and associate each instance created below with his/her civics
group. Use this 2-dimensional array to display the scores of all persons
in each civics group indicating the student chairperson\textquoteright s
name (if any). Test your code with the following steps in order: 
\begin{itemize}
\item Create an instance of \texttt{Student} with \texttt{name} \textquotedbl\texttt{Melvin}\textquotedbl{}
in civics group \texttt{5C35} 
\item Create an instance of \texttt{Student} with \texttt{name} \textquotedbl\texttt{Susan}\textquotedbl{}
in civics group \texttt{5C35} whose role is \textquotedbl\texttt{chairperson}\textquotedbl{} 
\item Create an instance information of \texttt{FormTeacher} with \texttt{name}
\textquotedbl\texttt{Norah}\textquotedbl{} in civics group \texttt{5C35} 
\item Create an instance of \texttt{Student} with name \textquotedbl\texttt{Ben}\textquotedbl{}
in civics group \texttt{6C35} 
\item Create an instance of \texttt{FormTeacher} with name \textquotedbl\texttt{Jimmy}\textquotedbl{}
in civics group \texttt{6C35} 
\item Display the information of Melvin 
\item Display the information of Susan 
\item Display the information of Norah 
\item Melvin attempts a math question 
\item Susan attempts a math question 
\item Jimmy attempts a math question 
\item Display the scores of all persons in each civics group with a header
for each class 
\end{itemize}
Here is a sample of an expected output: 

\noindent %
\noindent\begin{minipage}[t]{1\columnwidth}%
\texttt{Melvin(M)'s score is 0. }

\texttt{Susan(F)'s score is 0. }

\texttt{Ms. Norah's score is 0. \bigskip{}
}

\texttt{To Melvin : ((5/7)+8) ? }

\texttt{Answer:1 }

\texttt{Wrong answer. (Correct answer: 9 ) }

\texttt{Total score is still 0. \bigskip{}
}

\texttt{To Susan : (3{*}(5{*}6)) ? }

\texttt{Answer:90 }

\texttt{Good job! }

\texttt{New total score for Susan is 1 \bigskip{}
}

\texttt{To Jimmy : ((6-2)-6) ? }

\texttt{Answer:3 }

\texttt{Wrong answer. }

\texttt{(Correct answer: -2 ) }

\texttt{Total score is still 0. \bigskip{}
}

\texttt{5C35\textquoteright s scores:}

\texttt{Melvin(M)'s score is 0. }

\texttt{Susan(F)'s score is 1. (Chairperson) }

\texttt{Ms. Norah's score is 0.\bigskip{}
}

\texttt{6C35\textquoteright s scores:}

\texttt{Ben(M)'s score is 0. }

\texttt{Mr. Jimmy's score is 0. }%
\end{minipage}

Run your program and save your output. \hfill{}{[}7{]}