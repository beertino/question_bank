\item \textbf{{[}PJC/PRELIM/9597/2018/P1/Q4{]} }

\noindent A Foreign Government Agency is looking for a Tourist Information
Management System. The system should be able to cater for the following
two requirements: 
\begin{enumerate}
\item[a)]  Capture and provision of information for migration control propose
and other aspects of citizen identification 

This is to facilitate the processing of Disembarkation/ Embarkation
(D/E) cards collected from visitors at the checkpoints. It also captures
visitors\textquoteright{} arrival data e.g., the number of arrivals
by countries of residence, their modes of arrival and demographics
(e.g., age and gender). 
\item[b)]  Data Warehouse for analysis 

The Data Warehouse has to receive data and code information on Disembarkation/Embarkation
cards (D/E cards). Information gathered in this manner is to analyse
visitor arrival trends and serve as input to the computation of key
performance indicators (Tourism Receipts, TourismSector Value, etc.)
\end{enumerate}
The Agency wishes to replace this manual system with a computerised
system.

The office staff enters information provided by visitor into the computer
system using a graphical user interface. Some of the information required
includes: 
\begin{itemize}
\item Passport number 
\item visitor\textquoteright s salutation (e.g. Dr., Mr., Ms, Mrs, Mdm\dots ) 
\item visitor name and address 
\item visitor gender (e.g. F or M)
\item mobile number 
\end{itemize}
For this application design a simple screen layout which makes use
of appropriate graphical user interface controls. \hfill{}{[}3{]}