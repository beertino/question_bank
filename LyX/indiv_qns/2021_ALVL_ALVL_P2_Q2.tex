\item \textbf{{[}ALVL/9569/2021/P2/Q2{]} }

For this question you are provided with three text files, each contains
a valid list of positive integers, one per line:
\begin{itemize}
\item \texttt{TEN.txt} has 10 lines
\item \texttt{HUNDRED.txt} has 100 lines
\item \texttt{THOUSAND.txt} has 1000 lines.
\end{itemize}
For each of the sub-tasks, add a comment statement at the beginning
of the code using the hash symbol \textquoteleft \#' to indicate the
sub-task the program code belongs to, for example:

\begin{singlespace}
\noindent \texttt{}%
\begin{tabular}{c|lcccccccccccccccccccccc|}
\cline{2-24} \cline{3-24} \cline{4-24} \cline{5-24} \cline{6-24} \cline{7-24} \cline{8-24} \cline{9-24} \cline{10-24} \cline{11-24} \cline{12-24} \cline{13-24} \cline{14-24} \cline{15-24} \cline{16-24} \cline{17-24} \cline{18-24} \cline{19-24} \cline{20-24} \cline{21-24} \cline{22-24} \cline{23-24} \cline{24-24} 
\texttt{In {[}1{]} :} & \texttt{\#Task 2.1} &  &  &  &  &  &  &  &  &  &  &  &  &  &  &  &  &  &  &  &  &  & \tabularnewline
 & \texttt{Program Code} &  &  &  &  &  &  &  &  &  &  &  &  &  &  &  &  &  &  &  &  &  & \tabularnewline
\cline{2-24} \cline{3-24} \cline{4-24} \cline{5-24} \cline{6-24} \cline{7-24} \cline{8-24} \cline{9-24} \cline{10-24} \cline{11-24} \cline{12-24} \cline{13-24} \cline{14-24} \cline{15-24} \cline{16-24} \cline{17-24} \cline{18-24} \cline{19-24} \cline{20-24} \cline{21-24} \cline{22-24} \cline{23-24} \cline{24-24} 
\multicolumn{1}{c}{} & \texttt{Output:} &  &  &  &  &  &  &  &  &  &  &  &  &  &  &  &  &  &  &  &  &  & \multicolumn{1}{c}{}\tabularnewline
\end{tabular}
\end{singlespace}

\subsection*{Task 2.1 }

Write a function \texttt{task2\_1(filename)} that: 
\begin{itemize}
\item takes a string filename which represents the name of a text file 
\item reads in the contents of the text file 
\item returns the content as a list of integers. \hfill{}{[}3{]} 
\end{itemize}
Call your function task2\_l with the file TEN.txt, printing the returned
list and its length, using the following statements:

\texttt{\qquad{}result = task2\_l('TEN.txt')}

\texttt{\qquad{}print(result)}

\texttt{\qquad{}print(len(result)) }\hfill{}{[}2{]} 

\subsection*{Task 2.2 }

One method of sorting is the insertion sort.

Write a function \texttt{task2\_2(list\_of\_integers)} that: 
\begin{itemize}
\item takes a list of integers 
\item implements an insertion sort algorithm 
\item returns the sorted list of integers. \hfill{}{[}5{]}
\end{itemize}
Call your function \texttt{task2\_2} with the contents of the file
\texttt{TEN.txt}, printing the returned list, for example, using the
following statement:

\texttt{\qquad{}print(task2\_2(task2\_1('TEN.txt')))}\hfill{}{[}1{]}

\subsection*{Task 2.3 }

Another method of sorting is the quicksort. 

Write a function t\texttt{ask2\_3(list\_of\_integers)} that: 
\begin{itemize}
\item takes a list of integers 
\item implements a quicksort algorithm 
\item returns the sorted list of integers.\hfill{} {[}7{]}
\end{itemize}
Call your function \texttt{task2\_3} with the contents of the file
\texttt{TEN.txt}, printing the returned list. for example, using the
following statement: 

\texttt{\qquad{}print(task2\_3(task2\_1('TEN.txt')))}\hfill{}{[}1{]} 

\subsection*{Task 2.4 }

The \texttt{timeit} library is built into Python and can be used to
time simple function calls. Example code is shown in \texttt{Task2\_timing.py}.
(The sample code assumes that it has access \texttt{task2\_2} function.) 

Using the \texttt{timeit} module, or other evidence, and the three
text files provided with this question, compare and contrast, including
mention of orders of growth, the time complexity of sort and quicksort
algorithms. 

Save your Jupyter notebook for Task 2.\hfill{}{[}5{]}