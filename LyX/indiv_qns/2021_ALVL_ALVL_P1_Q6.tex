\item \textbf{{[}ALVL/9569/2021/P1/Q6{]} }

ASCII allows characters to be stored in memory by associating a number
with each character. In memory. that number is stored as a pattern
of bits. 

The decimal value representing \textquoteleft A\textquoteright{} in
ASCII is 65. 
\begin{enumerate}
\item Represent this value in:
\begin{enumerate}
\item Binary \hfill{}{[}1{]}
\item Hexadecimal. \hfill{}{[}1{]}
\end{enumerate}
\end{enumerate}
The decimal value representing \textquoteleft a\textquoteright{} in
ASCII is 97. 
\begin{enumerate}
\item[(b)]  State the hexadecimal value that must be added to the ASCII code
for \textquoteleft A\textquoteleft{} to convert it to the ASCII code
for 'a\textquoteleft . \hfill{}{[}1{]}
\end{enumerate}
Unicode is another method of encoding characters.
\begin{enumerate}
\item[(c)]  {}
\begin{enumerate}
\item State the values that are common to both ASCII and Unicode.\hfill{}
{[}1{]}
\item Explain what advantage Unicode has over ASCII. \hfill{}{[}2{]}
\end{enumerate}
\end{enumerate}