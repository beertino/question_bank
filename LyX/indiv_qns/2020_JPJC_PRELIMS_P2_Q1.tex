\item \textbf{{[}JPJC/PRELIM/9569/2020/P2/Q1{]} }

Jurong Pioneer Primary School uses buses to transport students to
school. There are six bus routes labelled \texttt{A} to \texttt{F}.
A survey was conducted to analyse the punctuality statistics of these
buses over a four-week period.

The data from the survey is stored in the file \texttt{SURVEY.TXT}.
The format of the data in the file is: 
\noindent \begin{center}
\texttt{<Day>,<A>,<B>,<C>,<D>,<E>,<F>} 
\par\end{center}

Positive numbers represent minutes early, negative numbers represent
minutes late and 0 represents the bus having been on time. 

\subsection*{Task 1.1 }

Write the program code that: 
\begin{itemize}
\item reads the entire contents of \texttt{SURVEY.TXT} into an appropriate
data structure called \texttt{Records}, and 
\item displays the contents of \texttt{Records} in neat columns. \hfill{}{[}4{]}
\end{itemize}

\subsection*{Task 1.2 }

Extend your program so that the following statistics for the four-week
period may be calculated and output: 
\begin{itemize}
\item the number of late arrivals for each bus route,
\item the average number of minutes late for each bus route, using only
data from days on which it was late, and 
\item the bus route(s) with the highest number of days late. 
\end{itemize}
All the results should be displayed with appropriate annotation. The
following is an example run of the program: 
\noindent \begin{center}
<INSERT\_IMAGE\_HERE> \hfill{}{[}8{]}
\par\end{center}

\subsection*{Task 1.3}

Additional code is to be written for the user to input a specific
day, for example: \texttt{Fri3}, to be used for the analysis of data.
Find and display how many buses were late on this day and for each
late bus, display the route label and how late the bus was on this
day. 

Test your code using the following test data: 

\texttt{Tue1} 

\texttt{Thu2} 

Download your program code for Task 1 as 

\texttt{TASK1\_<your class>\_<your name>.ipynb} \hfill{}{[}4{]}