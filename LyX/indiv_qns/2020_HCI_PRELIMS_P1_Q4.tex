\item \textbf{{[}HCI/PRELIM/9569/2020/P1/Q4{]} }
\begin{enumerate}
\item The ASCII code in denary for the character \textquoteleft \texttt{1}\textquoteright{}
is \texttt{49}.
\begin{enumerate}
\item Using 7 bits, express the ASCII code for the character \textquoteleft \texttt{4}\textquoteright{}
in binary.\hfill{}{[}1{]}
\item Express the character \textquoteleft \texttt{4}\textquoteright{} as
a hexadecimal number. \hfill{}{[}1{]}
\end{enumerate}
\item Convert \texttt{4B1} hexadecimal number to a binary number stored
as two bytes. \hfill{}{[}2{]}
\item In a restaurant, every membership account number is made up of five
digits followed by a letter e.g. \texttt{36514C} where the letter
is a modulus-eleven check digit for the account number. Each digit
is weighted, with the first digit having a weight of \texttt{7} and
each subsequent digit decreases its weight by \texttt{1}. Valid check
digits are in the range of letter C to letter M, with C corresponding
to the value of 1, D corresponds to 2 and so on.
\begin{enumerate}
\item What is the purpose of including a check digit at the end of each
membership account number? \hfill{}{[}1{]}
\item Write, in \textbf{pseudocode}, an algorithm which checks whether a
membership account number is valid. \hfill{}{[}4{]}
\item Using your algorithm, determine whether a person with the account
number \texttt{47938K} is a member of the restaurant. Explain your
answer. \hfill{}{[}1{]}
\end{enumerate}
\end{enumerate}