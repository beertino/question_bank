\item \textbf{{[}RVHS/PRELIM/9597/2018/P1/Q1{]} }

\textbf{Accumulative Season Scores }

For the first task, you are required to read the text file \textquotedblleft \emph{T1\_gamescore.txt}\textquotedblright{}
which contains the accumulative scores of a group of players from
an online game over 12 seasons. 

The format of a line of the text file is as follow: \texttt{<player1\_name>,<class\_1>:<ss1>,<ss2>,<ss3>,<ss4>,<ss5>,<ss6>,<
ss7>,<ss8>,<ss9>,<ss10>,<ss11>,<ss12>\textbackslash n }

E.g.

\texttt{Rufus,priest:4255,17366,22616,31889,49634,58847,78112,86631,94
924,106955,125372,130725 }

The above line represents the player \texttt{Rufus} plays the class
\texttt{priest} and his 12 accumulative seasonal scores are \texttt{4255,
17366, 22616, 31889, 49634, 58847, 78112, 86631, 94924, 106955, 125372}
and \texttt{130725} respectively. 

\subsection*{Task 1.1 -- Read the file }

Implement the procedure readfile. It reads the file \textquotedblleft T1\_gamescore.txt\textquotedblright{}
and returns a list of lists as below. 

\noindent %
\noindent\begin{minipage}[t]{1\columnwidth}%
\texttt{>\textcompwordmark >\textcompwordmark > results = readfile() }

\texttt{>\textcompwordmark >\textcompwordmark > results {[}{[}'Rufus',
'priest', 4255, 13111, 5250, 9273, 17745, 9213, 19265, 8519, 8293,
12031, 18417, 5353{]}, {[}'Ione Wolfe', 'warrior', 2827, 17757, 3612,
6818, 11772, 9161, 4393, 10469, 10567, 15424, 7307, 10014{]},\dots \dots ,{[}'Carola
Tegeler', 'rogue', 6407, 14795, 0, 5004, 16084, 14960, 11879, 16545,
10247, 5617, 7345, 0{]}{]} }%
\end{minipage}

Take note that the returned scores in the lists are not accumulative
scores but the actual season scores. It can be computed by: 

$n$th season score = $n$th -- $(n-1)$th accumulative season score 

\subsection*{Evidence 1 }

Program code of procedure \texttt{readfile}. \hfill{}{[}6{]} 

\subsection*{Task 1.2 -- Count Class }

Using the data from \textquotedblleft \emph{T1\_gamescore.txt}\textquotedblright ,
implement the function \texttt{count\_class}. It takes in the string
\texttt{clss} as parameter and returns an integer which contains the
number of players who play the class indicated by \texttt{clss}. For
example:

\noindent %
\noindent\begin{minipage}[t]{1\columnwidth}%
\texttt{>\textcompwordmark >\textcompwordmark > count\_class(\textquotedbl warrior\textquotedbl ) }

\texttt{57}%
\end{minipage}

\subsection*{This means that the number of players who play the class warrior
is 57. Evidence 2 }

Program code of function \texttt{count\_class}.\hfill{} {[}2{]}

\subsection*{Evidence 3 }

Screenshot of the output of the following code. \hfill{}{[}1{]}

\noindent\begin{minipage}[t]{1\columnwidth}%
\texttt{for clss in {[}\textquotedbl warrior\textquotedbl , \textquotedbl mage\textquotedbl ,
\textquotedbl priest\textquotedbl{]}: }

\texttt{\qquad{}print(\textquotedbl Number of \textquotedbl{} +
clss + \textquotedbl :\textquotedbl{} + str(count\_class(clss))) }%
\end{minipage}

\subsection*{Task 1.3 -- Top Class by Season }

Using the data from \textquotedblleft \emph{T1\_gamescore.txt}\textquotedblright ,
implement the function \texttt{top\_class\_by\_season}. It takes in
the integer \texttt{season} as parameter and returns a string which
contains the class which holds the highest average score of the season.
(Code with the assumption that you do not know how many and what the
classes are available in this game.) 

The average score of the season of a class is calculated by summing
up all the scores of the players who play the class divided by the
total number of players who play the same class.

\noindent\begin{minipage}[t]{1\columnwidth}%
\texttt{>\textcompwordmark >\textcompwordmark > top\_class\_by\_season
(6) }

\texttt{\textquotedbl priest\textquotedbl{} }%
\end{minipage}

\subsection*{Evidence 4}

Program code of function \texttt{top\_class\_by\_season}. \hfill{}{[}6{]}

\subsection*{Evidence 5 }

Screenshot of the output of the following code. \hfill{}{[}1{]}

\noindent\begin{minipage}[t]{1\columnwidth}%
\texttt{for i in range(1, 13): }

\texttt{\qquad{}print(\textquotedbl Top class in season \textquotedbl{}
+ str(i) + \textquotedbl :\textquotedbl ) }

\texttt{\qquad{}print(top\_class\_by\_season(i)) }%
\end{minipage}

\subsection*{Task 1.4 -- Top n Players by Season }

Using the data from \textquotedblleft \emph{T1\_gamescore.txt}\textquotedblright ,
implement the function \texttt{top\_n\_players\_by\_season}. It takes
in the integers \texttt{n} and season as parameters and returns a
list of strings which contains names of the top \texttt{n} players. 

\noindent\begin{minipage}[t]{1\columnwidth}%
\texttt{>\textcompwordmark >\textcompwordmark > top\_n\_players\_by\_season(3,
2) }

\texttt{{[}'bumpkintiger', 'diamondagile', 'milkshakessulky'{]} }

\texttt{>\textcompwordmark >\textcompwordmark > top\_n\_players\_by\_season(10,
4) }

\texttt{{[}'Carolann Kintner', 'lewdsterpub', 'fabindustry', 'eggfun',
'fellradial', 'Grazyna Kitzman', 'chapteridentical', 'chubbysourdough',
'palmears', 'leardfluttering'{]} }%
\end{minipage}

\subsection*{Evidence 6 }

Program code of function \texttt{top\_n\_players\_by\_season}.\hfill{}
{[}5{]}

\subsection*{Evidence 7 }

Screenshot of the list of names of the top 20 players in season 9.\hfill{}
{[}1{]}

\subsection*{Task 1.5 -- Stagnant Players }

Using the data from \textquotedblleft \emph{T1\_gamescore.txt}\textquotedblright ,
implement the function \texttt{find\_stagnant\_players} which returns
a list of strings which contains names of the players who did not
play in at least 1 season. 

\subsection*{Evidence 8 }

Program code of function \texttt{find\_stagnant\_players}. \hfill{}{[}2{]}

\subsection*{Evidence 9 }

Screenshot of the output of program. \hfill{}{[}1{]}