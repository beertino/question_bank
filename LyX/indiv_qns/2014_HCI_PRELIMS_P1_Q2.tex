\item \textbf{{[}HCI/PRELIM/9597/2014/P1/Q2{]} }

The following is a pseudocode algorithm for the quick sort function. 

The function takes in an array and returns a sorted array in \textbf{ascending}
order. The algorithm works for most input but is missing out on a
few edge cases. 

\noindent %
\noindent\begin{minipage}[t]{1\columnwidth}%
\texttt{FUNCTION quicksort(array) }

\texttt{\qquad{}if array length = 1 }

\texttt{\qquad{}\qquad{}return array }

\texttt{\qquad{}select and remove a pivot value from array }

\texttt{\qquad{}create empty subarrays less and greater }

\texttt{\qquad{}for each item in array}

\texttt{\qquad{}\qquad{}if item < pivot }

\texttt{\qquad{}\qquad{}\qquad{}append item to less }

\texttt{\qquad{}\qquad{}else }

\texttt{\qquad{}\qquad{}\qquad{}append item to greater}

\texttt{\qquad{}return quicksort(less) + pivot + quicksort(greater) }%
\end{minipage}

\subsection*{Task 2.1 }

Write program code for this algorithm including all the amendments
you would make to: 
\begin{itemize}
\item make the function work for all cases 
\item adhere to good programming style 
\end{itemize}
Use the \texttt{Team} array sample data available from text file \texttt{SPORT+NAME+TEAM.txt}
and paste this into your program code.

\subsection*{Evidence 5: }

Your program code.\hfill{} {[}6{]}

\subsection*{Evidence 6: }

One screenshot showing the output from running the program code for
array Team.\hfill{} {[}1{]}

The following is a pseudocode algorithm of a binary search written
by a student to search for an item in an array Sport. This array stores
string data arranged in ascending order and has a final subscript
MAX. 

The algorithm is poorly designed and also contains errors. 

\noindent %
\noindent\begin{minipage}[t]{1\columnwidth}%
\texttt{Enter k }

\texttt{A = 1 }

\texttt{B = MAX }

\texttt{While not found }

\texttt{\qquad{}C = (A+B) DIV 2 }

\texttt{\qquad{}If Sport(C) = k }

\texttt{\qquad{}\qquad{}Print \textquotedblleft Found\textquotedblright{} }

\texttt{\qquad{}Else If Sport(C) < k }

\texttt{\qquad{}\qquad{}B = C-1 }

\texttt{\qquad{}Else }

\texttt{\qquad{}\qquad{}A = C+1 }

\texttt{\qquad{}End-If-Else }

\texttt{End-While }%
\end{minipage}

\subsection*{Task 2.2 }

Write program code for this algorithm including all the changes you
would make to: 
\begin{itemize}
\item follow good programming practice 
\item make the algorithm work properly 
\end{itemize}
Use the sample array data available from text file \texttt{SPORT+NAME+TEAM.txt}
and paste this into your programming code. 

\subsection*{Evidence 7: }

Your program code. \hfill{}{[}7{]}

\subsection*{Task 2.3 }

The binary search code could be useful for many programs where a search
routine is required. 

Re-design the program code to have a procedure \texttt{BinarySearch}.
This procedure should have parameters which allow it to be used for
\uline{any} sorted array of string data. 

Use the data provided in the array \texttt{Name} from text file \texttt{SPORT+NAME+TEAM.txt}
and test the procedure with appropriate test cases to ensure it is
working properly. 

\subsection*{Evidence 8: }

Your amended program code. \hfill{}{[}3{]}

\subsection*{Evidence 9: }

A screenshot for each appropriate test case. \hfill{}{[}3{]}