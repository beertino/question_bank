\item \textbf{{[}RVHS/PRELIM/9569/2021/P1/Q5{]} }

The implementation of a Binary Search Tree (BST) using three 1D arrays
is shown below. 

Each unused node that are not in the logical BST is initially connected
in a singly linked list manner using the \texttt{leftPtr} array. The
first position of this linked list is indicated by the variable \texttt{nextFree}. 

When a piece of data is inserted into the BST, a node will be disconnected
from the linked list and added to the logical BST. The root of this
logical BST is indicated by the variable \texttt{root}. The logical
structure of the BST is managed by \texttt{leftPtr} and \texttt{rightPtr}
which are the positions of the left and right child of the node respectively. 

Below is an illustration for such BST with a 0-based index array. 

\begin{tabular}{|c|c|}
\hline 
\texttt{root } & 7 \tabularnewline
\hline 
\texttt{nextFree} & 0\tabularnewline
\hline 
\end{tabular} 

\begin{tabular}{|c|c|c|c|c|c|c|c|c|}
\hline 
Array Index  & 0 & 1 & 2 & 3 & 4 & 5 & 6 & 7\tabularnewline
\hline 
\texttt{Data } & - & 7 & 10 & 6 & 1 & 4 & 2 & 9\tabularnewline
\hline 
\texttt{leftPtr } & -1 & -1 & -1 & 5 & -1 & 6 & 4 & 3\tabularnewline
\hline 
\texttt{rightPtr } & -1 & -1 & -1 & 1 & -1 & -1 & -1 & 2\tabularnewline
\hline 
\end{tabular}
\begin{enumerate}
\item Draw the logical BST at this point of time. \hfill{}{[}1{]}
\item State the post order traversal of the BST \hfill{}{[}1{]}
\item State the values of \texttt{root}, \texttt{nextFree} and the values
in the arrays \texttt{data}, \texttt{leftPtr} and \texttt{rightPtr}
after the \textbf{each} of the following BST operations are executed
sequentially. \hfill{}{[}3{]} 
\begin{itemize}
\item \texttt{Add 8}
\item \texttt{Recursive Delete 6 }
\end{itemize}
\item State an advantage of BST over Hash table. \hfill{}{[}1{]}
\item Explain what can make a Hash Table Search inefficient besides a bad
hash function and how to overcome it. \hfill{}{[}2{]}
\item State 2 characteristics of a good hash function. \hfill{}{[}2{]}
\end{enumerate}