\item \textbf{{[}ALVL/9597/2015/P2/Q3{]} }

A simple queue data structure is implemented using a one-dimensional
array and two pointers, Head and Tail, as shown:
\begin{center}
\begin{tabular}{r|l|l}
\multicolumn{1}{r}{} & \multicolumn{1}{l}{\texttt{Queue}} & \tabularnewline
\cline{2-2} 
\texttt{1} & \texttt{Mac} & \tabularnewline
\cline{2-2} 
\texttt{2} & \texttt{Ben} & $\impliedby$ \texttt{Head}\tabularnewline
\cline{2-2} 
\texttt{3} & \texttt{Dog} & \tabularnewline
\cline{2-2} 
\texttt{4} & \texttt{Can} & \tabularnewline
\cline{2-2} 
\texttt{5} & \texttt{Yog} & \tabularnewline
\cline{2-2} 
\texttt{6} & \texttt{Hur} & \tabularnewline
\cline{2-2} 
\texttt{7} &  & $\impliedby$ \texttt{Tail}\tabularnewline
\cline{2-2} 
\texttt{8} &  & \tabularnewline
\cline{2-2} 
\texttt{9} &  & \tabularnewline
\cline{2-2} 
\texttt{10} &  & \tabularnewline
\cline{2-2} 
\end{tabular}
\par\end{center}
\begin{enumerate}
\item Show the state of the above queue after: 
\begin{itemize}
\item two items, Dap and Eck, are added (in that order) 
\item one item is removed. \hfill{}{[}3{]}
\end{itemize}
When ten items have been added, this simple queue cannot accept any
further items. 
\item A first attempt at an algorithm for adding an item to this queue is:

\noindent\begin{minipage}[t]{1\columnwidth}%
\texttt{01 IF ..............................}

\texttt{02 \qquad{}THEN }

\texttt{03 \qquad{}\qquad{}OUTPUT \textquotedbl No more room to
add items\textquotedblright{} }

\texttt{04 \qquad{}ELSE }

\texttt{05 \qquad{}\qquad{}INPUT \textquotedbl New item to be added\textquotedbl ,
NewItem }

\texttt{06 \qquad{}\qquad{}Queue{[}..............................{]}
<- NewItem }

\texttt{O7 \qquad{}\qquad{}.............................. }

\texttt{O8 ENDIF }%
\end{minipage}

Write the pseudocode to show the completed lines 01, 06, and 07. \hfill{}{[}3{]}
\item Give the initial value for \texttt{Tail} when the queue is created
and justify your answer. \hfill{}{[}2{]}
\end{enumerate}
The programmer can reuse the space released after removing an item.
This maximises the available space.
\begin{enumerate}
\item[(d)]  Describe how the algorithm for adding an item should be amended
so that the released space is made available. \hfill{}{[}2{]}
\end{enumerate}