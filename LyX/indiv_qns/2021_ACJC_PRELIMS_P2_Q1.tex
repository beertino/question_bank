\item \textbf{{[}ACJC/PRELIM/9569/2021/P2/Q1{]} }

\noindent The Universal Product Code (UPC) system is used for tracking
trade items in shipping, inventory, and sales. Each item is given
a 12-digit identification number. The validity of this identification
number can be checked using a checksum. If $x_{i}$ represents the
$i$th digit (starting with $i=1$ as the leftmost digit), then a
valid identification number satisfies the condition that

\noindent 
\[
3x_{1}+x_{2}+3x_{3}+x_{4}+3x_{5}+x_{6}+3x_{7}+x_{8}+3x_{9}+x_{10}+3x_{11}+x_{12}
\]

\noindent has a remainder of 0 when divided by 10.

\noindent The identification number can be encoded into a barcode.
For this Task, the barcode will be represented as a string of '\texttt{0}'s
and '\texttt{1}'s, where '\texttt{0}' represents a white stripe and
'\texttt{1}' represents a black stripe.

\noindent The barcode is divided into seven sections. From left to
right, they are
\begin{itemize}
\item A \textquoteleft quiet zone\textquoteright{} consisting of nine '\texttt{0}'s; 
\item A start pattern which is always '\texttt{101}'; 
\item The first six digits of the identification number are encoded using
the table below; 
\item A middle pattern which is always '\texttt{01010}'; 
\item The last six digits of the identification number are encoded using
the table below; 
\item An end pattern which is always '\texttt{101}'; 
\item A \textquoteleft quiet zone\textquoteright{} consisting of nine '\texttt{0}'s.
\end{itemize}
\noindent The table below shows the encoding system for the digits.
Note that depending on whether the digit occurs in the first six digits
or the last six digits, it would be encoded differently. However,
the two encodings are optical inverses of each other -- a '\texttt{0}'
is changed into a '\texttt{1}', and vice versa.
\noindent \begin{center}
\begin{tabular}{|c|c|c|}
\hline 
Digit  & First six digits & Last six digits\tabularnewline
\hline 
0  & \texttt{'0001101' } & \texttt{'1110010'}\tabularnewline
\hline 
1  & \texttt{'0011001' } & \texttt{'1100110'}\tabularnewline
\hline 
2  & \texttt{'0010011'} & \texttt{'1101100'}\tabularnewline
\hline 
3  & \texttt{'0111101' } & \texttt{'1000010' }\tabularnewline
\hline 
4  & \texttt{'0100011' } & \texttt{'1011100' }\tabularnewline
\hline 
5  & \texttt{'0110001' } & \texttt{'1001110'}\tabularnewline
\hline 
6  & \texttt{'0101111' } & \texttt{'1010000'}\tabularnewline
\hline 
7  & \texttt{'0111011'} & \texttt{'1000100' }\tabularnewline
\hline 
8  & \texttt{'0110111' } & \texttt{'1001000'}\tabularnewline
\hline 
9  & \texttt{'0001011' } & \texttt{'1110100'}\tabularnewline
\hline 
\end{tabular}
\par\end{center}

\noindent The reason for encoding the first and last six digits differently
is that the barcode may inadvertently be scanned upside down. Notice
that in the first six digits, the encoding for each digit contains
an odd number of '\texttt{1}'s, while in the last six digits, the
encoding for each digit contains an even number of '\texttt{1}'s.
This allows the scanning software to detect if the barcode has been
placed upside down and correct it. \quad{} 

\noindent For example, the UPC identification number 036000 291452
would be encoded as:

\begin{tabular}{|c|c|c|c|c|c|}
\hline 
000000000 & 101 & 0001101 & 0111101 & 0101111 & 0001101\tabularnewline
\hline 
Quiet & Start & 0 & 3  & 6  & 0\tabularnewline
\hline 
\end{tabular}

\begin{tabular}{|c|c|c|c|c|c|}
\hline 
0001101  & 0001101  & 01010 & 1101100  & 1110100  & 1100110\tabularnewline
\hline 
0  & 0  & Middle & 2  & 9  & 1\tabularnewline
\hline 
\end{tabular}

\begin{tabular}{|c|c|c|c|c|}
\hline 
1011100 & 1001110  & 1101100  & 101 & 000000000 \tabularnewline
\hline 
4  & 5  & 2  & End & Quiet\tabularnewline
\hline 
\end{tabular}
\noindent \begin{center}
\includegraphics[scale=0.8]{C:/Users/Admin/Desktop/Github/question_bank/LyX/static/img/9597-ACJC-2021-P2-Q1}
\par\end{center}

\noindent (Notice that in an actual barcode, the stripes for the start,
middle and end pattern are usually slightly longer than the surrounding
stripes. This is to help humans to read it.)

\subsection*{Task 1.1}

\noindent Write a function to determine the validity of any input
string as an identification number. \hfill{}{[}5{]}

\subsection*{Task 1.2}

\noindent Write a function to convert a valid identification number,
given as a string, into a barcode (a string of '\texttt{0}'s and '\texttt{1}'s).
\hfill{}{[}5{]}

\subsection*{Task 1.3}

\noindent Write a function that takes in a string, check whether it
represents a valid barcode, and converts it to an identification number
if it does. Note that the barcode may be upside down. \hfill{}{[}11{]}

\noindent Download your program code and output for Task 1 as \texttt{TASK1\_<your
name>\_<centre number>\_<index number>.ipynb}