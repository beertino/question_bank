\item \textbf{{[}RVHS/PRELIM/9597/2019/P2/Q7{]} }

Run length encoding (RLE) is a very simple form of lossless data compression
which runs on sequences having same value occurring many consecutive
times and it encodes the sequence to store only a single value and
its count. For example, \textquotedblleft 111111000\textquotedblright{}
can be converted to \textquotedblleft 6130\textquotedblright .

Below is a draft attempt of implementing this algorithm. 

\noindent %
\noindent\begin{minipage}[t]{1\columnwidth}%
\texttt{01 PROCEDURE RLECompress(s: STRING): }

\texttt{02 \qquad{}DECLARE comp\_str: STRING }

\texttt{03 \qquad{}DECLARE count, i: INT }

\texttt{04 \qquad{}comp\_str = \textquotedbl\textquotedbl{} }

\texttt{05 \qquad{}count, i = 0, 0 }

\texttt{06 }

\texttt{07 \qquad{}WHILE i < LENGTH(s): }

\texttt{08 \qquad{}\qquad{}count = 1 }

\texttt{09 }

\texttt{10 \qquad{}\qquad{}WHILE i < LENGTH(s) - 1 and s{[}i{]}
== s{[}i+1{]}:}

\texttt{11 \qquad{}\qquad{}\qquad{}count += 1 }

\texttt{12 \qquad{}\qquad{}\qquad{}i += 1 }

\texttt{13 \qquad{}\qquad{}END WHILE }

\texttt{14 }

\texttt{15 \qquad{}\qquad{}comp\_str += count + s{[}i{]} }

\texttt{16 \qquad{}END WHILE }

\texttt{17 }

\texttt{18 \qquad{}RETURN comp\_str }

\texttt{19 END PROCEDURE}%
\end{minipage}
\begin{enumerate}
\item The compiler reported an error at line 15. Identity the type of error,
state its definition and suggest how it can be fixed.\hfill{} {[}3{]}
\item After the above error has been corrected, a successful compilation
produces executable code. However, when the code was executed, it
failed to complete. Identity another type of error, state its definition
and suggest how it can be fixed. \hfill{}{[}3{]}
\end{enumerate}
Assuming all errors have been identified and corrected.

The following steps are taken to encode English alphabets:
\begin{enumerate}
\item[1.]  Convert alphabet to hexadecimal ASCII value 
\item[2.]  Convert the hexadecimal number to binary number 
\item[3.]  Encode the binary number using RLE algorithm
\end{enumerate}
For example, the alphabet \textquotedblleft A\textquotedblright{}
has a hexadecimal ASCII value of \textquotedblleft 41\textquotedblright .
\begin{enumerate}
\item[(c)]  State the definition of ASCII.\hfill{} {[}2{]}
\item[(d)]  State the encoded RLE string for letter \textquotedblleft A\textquotedblright .\hfill{}
{[}1{]}
\item[(e)]  State and explain with example one potential limitation of RLE encoding
when compressing long binary strings. \hfill{}{[}2{]}
\end{enumerate}