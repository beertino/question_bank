\item \textbf{{[}RVHS/PRELIM/9597/2017/P1/Q1{]} }

\textbf{Results management }

For this task, you are required to read the text file \textquotedblleft \textbf{overall\_grades.txt}\textquotedblright{}
which contains the subject grades of the first and second semester
examination of 50 students.

The format of the text file is as follow: \texttt{<name>;<subject1>:<grade1>,<grade2>;<subject2>:<grade1>,<grade
2>;<subject3>:<grade1>,<grade2>,\dots etc;\textbackslash n }

For example: 

\texttt{Rufus Schuck;EL:C,B;CL:C,C;Math:C,E;Bio:B,F; }

Student Rufus Schuck takes 4 subjects. They are English (EL), Chinese
(CL), Math (Math) and Biology (Bio). His 1st and 2nd semester grades
for English is C and B respectively. 

\subsection*{Task 1.1 }

Write the program code for the function process\_data() which reads
the text file \textquotedblleft \textbf{overall\_grades.txt}\textquotedblright{}
and returns a dictionary which has the following format. 

\texttt{\{'Rufus Schuck': \{'EL': {[}'C', 'B'{]}, 'CL': {[}'C', 'C'{]},
'Math': {[}'C', 'E'{]}, 'Bio': {[}'B', 'F'{]}\}, ...\} }

The returned dictionary uses the name of student as key and another
dictionary as its return value. This \textquotedblleft another\textquotedblright{}
dictionary then uses the subjects and a list containing the grades
of the 1st and 2nd semesters as its key and value respectively. 

\subsection*{Evidence 1 }

The program code for the function \texttt{process\_data()}. \hfill{}{[}4{]}

\subsection*{Task 1.2}

Write a function \texttt{have\_n\_improvement(n)} which takes in an
integer \texttt{n} and returns a list of student names who have improvement
in exactly \texttt{n} subject(s). Using the example in Task 1.1, '\texttt{Rufus
Schuck}' shows improvement in only English. So, if \texttt{n} is 1,
'\texttt{Rufus Schuck}' must be included in the student name list. 

\subsection*{Evidence 2}

The program code for the function \texttt{have\_n\_improvement(n)}.\hfill{}
{[}3{]}

\subsection*{Task 1.3 }

Write a function \texttt{have\_improvement\_in\_all\_subjs()} which
returns a list of student names who have improvement in all his/her
subjects. 

\subsection*{Evidence 3}

The program code for the function have\_improvement\_in\_all\_subjs(n).\hfill{}
{[}3{]}

\subsection*{Task 1.4 }

Write a function \texttt{count\_combi(combi)} which takes a list of
subjects \texttt{combi} as input and returns the total number of students
who take the subject combination in \texttt{combi}. Take note that
students who take more subjects than what is indicated in \texttt{combi}
are included. 

For example:

\noindent\begin{minipage}[t]{1\columnwidth}%
\texttt{>\textcompwordmark >\textcompwordmark > count\_combi({[}'EL',
'MATH'{]}) }

\texttt{>\textcompwordmark >\textcompwordmark > 50 }%
\end{minipage}

\subsection*{Evidence 4 }

The program code for the function \texttt{count\_combi(n)}. \hfill{}{[}4{]}

\subsection*{Task 1.5}

Using your solution in Task 1.4, find the total number of students
who take \textquotedbl Chem\textquotedbl{} and \textquotedbl Bio\textquotedbl{}
but not \textquotedbl Physics\textquotedbl . 

\subsection*{Evidence 5 }

The program code to find what is required in Task 1.5. \hfill{}{[}2{]}

\subsection*{Task 1.6 }

Write a function \texttt{calculate\_GPA(name)} which takes a string
\texttt{name} as input and calculates the GPA of the student indicated
by \texttt{name}. 

The points for each grade are as follow: 

\texttt{\{'A':5, 'B':4, 'C':3, 'D':2, 'E':1, 'F':0\} }

The 1 st and 2nd semester grades in the data has the same weightages. 

To calculate GPA,
\begin{itemize}
\item Find the average points achieved of each subject.
\begin{itemize}
\item e.g. if a student gets C and D in \textquotedbl CL\textquotedbl ,
his average points for \textquotedbl CL\textquotedbl{} is 3 + 2 =
2.5 
\end{itemize}
\item Add up the average points of each subject 
\item Divide the total by the number subjects the student takes 
\begin{itemize}
\item e.g. \texttt{'Claudette Bode': \{'EL': {[}'C', 'D'{]}, 'CL': {[}'D',
'A'{]}, 'Math': {[}'A', 'D'{]}, 'Phy': {[}'A', 'D'{]}, 'Bio': {[}'D',
'C'{]}\}}, 
\item \texttt{{[}(3 + 2)/2 + (2 + 5)/2 + (5 + 2)/2 + (2 + 5)/2 + (2 + 3)/2{]}/5
= 3.1}
\end{itemize}
\item GPA for '\texttt{Claudette Bode}' is 3.1 
\end{itemize}

\subsection*{Evidence 6 }

The program code for the function \texttt{calculate\_GPA(name)}.\hfill{}
{[}4{]}