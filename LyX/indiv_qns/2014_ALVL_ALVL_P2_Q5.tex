\item \textbf{{[}ALVL/9597/2019/P2/Q5{]} }

A software developer is given the task of producing software for a
college. The software will help to manage information about what students
do after finishing at the college.

The destination of each student after college is classified in one
of three possible ways:
\begin{itemize}
\item University
\item Employment 
\item Other
\end{itemize}
The college wishes to store:
\begin{itemize}
\item name 
\item number of A Level passes 
\item destination (U / E / O) 
\item university attended 
\item main subject studied at university 
\item type of employment 
\item what students do when their destination is classified as 'O'
\end{itemize}
The software developer will use an objectoriented approach to developing
a solution.
\begin{enumerate}
\item Draw a class diagram which exhibits the following:
\begin{itemize}
\item suitable classes with appropriate properties and methods 
\item inheritance 
\item polymorphism \hfill{}{[}6{]}
\end{itemize}
\item Explain how your solution to (a) demonstrates software reuse.\hfill{}
{[}2{]}
\end{enumerate}
The data on the students is to be stored in a serial text file called
STUDENT.DAT. Each line of the file has the same structure:

\texttt{<Name><NoOfPasses><Destination><University><MainSubject><EmpType><Other>}

with the string NULL stored where appropriate.
\begin{enumerate}
\item[(c)]  Write an algorithm, in pseudocode, to read data from \texttt{STUDENT.DAT}
and to output the following:
\begin{itemize}
\item total number of students going to university 
\item average number of passes for the students going to university e total
number of students\hfill{} {[}7{]}
\end{itemize}
\end{enumerate}