\item \textbf{{[}JPJC/PRELIM/9569/2021/P2/Q3{]} }

Your program code and output for Task 3 should be saved in a single
\texttt{.ipynb} file. 

Name your Jupyter Notebook as \texttt{TASK3\_<your name>\_<class>\_<index
number>.ipynb }

JP Fitness Club is a gym that keeps details of its members. You are
tasked to help the club manage the members\textquoteright{} details
and store them in a SQL database. 

There are two types of membership: normal and annual. Each member
has a unique membership number, first name, surname, contact number
and last visit date recorded. 

A normal member deposits a selected amount into their account. Each
time the member visits the gym, the entrance fee is deducted from
the amount held in his account. The member may top up the account
any time. 

An annual member pays a fixed fee per year, starting from the date
of registration. He may then visit the gym any number of times for
the whole year without paying the entrance fee. 

Three classes have been identified: \texttt{Member}, \texttt{NormalMember},
\texttt{AnnualMember}. 

The class \texttt{Member} has these attributes and methods defined
on it. 

\begin{tabular}{|c|c|c|}
\hline 
\textbf{Attribute} & \textbf{Data type} & \textbf{Description}\tabularnewline
\hline 
\texttt{memberID} & String  & 8 digit membership number. First four digits represent the year of
joining gym and last four digits are used to make the \texttt{memberID}
unique, e.g. \texttt{20210357}. \tabularnewline
\hline 
\texttt{first\_name } & String  & First name of member, at most 15 characters.\tabularnewline
\hline 
\texttt{surname } & String  & Surname of member, at most 15 characters.\tabularnewline
\hline 
\texttt{contact\_number } & String  & 8 digit contact number.\tabularnewline
\hline 
\texttt{last\_visit } & String  & The date when member last visited the gym, in the format\texttt{ YYYY-MM-DD}.
Initialises to today\textquoteright s date. \tabularnewline
\hline 
\texttt{memberType} & String  & Indicates type of membership, either \textquotedblleft normal\textquotedblright{}
or \textquotedblleft annual\textquotedblright . Initialises to None. \tabularnewline
\hline 
\textbf{Method} & \textbf{Return type} & \textbf{Description}\tabularnewline
\hline 
\texttt{showMember() } & None  & Outputs member\textquoteright s membership number, first name, surname,
contact number and last visit date.\tabularnewline
\hline 
\texttt{isActive() } & Boolean  & Indicates whether a member is active or not. Returns True if the last
visit date is within 30 days, otherwise returns False.\tabularnewline
\hline 
\end{tabular}

The class \texttt{NormalMember} inherits from \texttt{Member} and
has these additional attributes and methods defined on it. 

\begin{tabular}{|c|c|c|}
\hline 
\textbf{Attribute} & \textbf{Data type} & \textbf{Description}\tabularnewline
\hline 
stored\_value & Float & stored in member\textquoteright s account. Display in 2 decimal places.
Initialise to \$0.00.\tabularnewline
\hline 
\textbf{Method} & \textbf{Return type} & \textbf{Description}\tabularnewline
\hline 
\texttt{showMember()} & None & Output \texttt{memberType} in addition to member\textquoteright s
membership number, first name, surname, contact number and last visit
date. \tabularnewline
\hline 
\end{tabular}

The class \texttt{AnnualMember} also inherits from \texttt{Member}
and has these additional attributes and methods defined on it. 

\begin{tabular}{|c|c|c|}
\hline 
\textbf{Attribute} & \textbf{Data type} & \textbf{Description}\tabularnewline
\hline 
\texttt{annual\_fee } & Integer  & Annual fee paid by member. Initialises to \$500\tabularnewline
\hline 
\texttt{date\_register} & String & Date when member joins annual membership, in format \texttt{YYYY-MM-DD}.
Initialises to today\textquoteright s date.\tabularnewline
\hline 
\textbf{Method} & \textbf{Return type} & \textbf{Description}\tabularnewline
\hline 
\texttt{showMember()} & None & Outputs \texttt{memberType} in addition to member\textquoteright s
membership number, first name, surname, contact number and last visit
date. \tabularnewline
\hline 
\end{tabular}

\subsubsection*{Task 3.1 }

Write program code using object-oriented programming for the classes
\texttt{Member}, \texttt{NormalMember}, and \texttt{AnnualMember}.
Include all the identifiers stated and other appropriate methods to
access and modify the attributes. \hfill{}{[}15{]}

\subsubsection*{Task 3.2 }

The text file, \texttt{members.TXT}, contains data items for a number
of members. Each data item is separated by a comma, with each member\textquoteright s
data on a new line as follows: 
\begin{itemize}
\item membership number 
\item first name 
\item surname
\item contact number 
\item member type 
\end{itemize}
Write program code to read in the information from the text file,
\texttt{member.txt}, creating an instance of the appropriate class
for each member, storing each instance in the same list.

Run \texttt{showMember} method to display each of the members\textquoteright{}
details. \hfill{}{[}5{]}

\subsubsection*{Task 3.3 }

The members\textquoteright{} details are to be stored in a SQL database. 

The file \texttt{JPgym.SQL} contains the SQL code to create database
\texttt{JPgym.db} with the single table, \texttt{Member}. The table
will have the following fields: 
\begin{itemize}
\item MemberID -- primary key, text 
\item FirstName -- first name of member, text 
\item Surname -- surname of member, text 
\item ContactNo -- contact number of member, text 
\item LastVisit -- date that member last visited gym, text 
\item MemberType -- indicates \textquoteleft normal\textquoteright{} or
\textquoteleft annual\textquoteright{} membership, text 
\end{itemize}
Copy and paste this SQL code into your Python program to create the
database and table. 

Also, write program code in Python to insert all the information from
the file into the \texttt{JPgym.db} database. 

Run your program and check that all information has been inserted
using SQLite database software. \hfill{}{[}6{]}

\subsubsection*{Task 3.4 }

Write a SQL query code in Python to display all members with \textquotedblleft normal\textquotedblright{}
membership in ascending order of \texttt{FirstName}. 

Display only the following fields from the query: \texttt{FirstName},
\texttt{Surname}, \texttt{ContactNo}. \hfill{} {[}3{]}