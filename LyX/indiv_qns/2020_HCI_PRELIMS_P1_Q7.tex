\item \textbf{{[}HCI/PRELIM/9569/2020/P1/Q7{]} }

A linked list ADT has the following operations defined: 
\begin{itemize}
\item \texttt{Create(x) }-{}- creates an empty linked list \texttt{x}; 
\item \texttt{Insert(x,item,p)} -{}- inserts new value, \texttt{item}, into
linked list \texttt{x} so that it is at position \texttt{p} in the
linked list; 
\item \texttt{Delete(x,p)} -{}- deletes the item at position \texttt{p}
in the linked list \texttt{x}; 
\item \texttt{Read(x,p)} -{}- returns the item at position \texttt{p} in
the linked list \texttt{x}; 
\item \texttt{Length(x)} -{}- returns the number of items in the linked
list \texttt{x}; 
\item \texttt{IsEmptyList(x)} -{}- returns \texttt{True} if linked list
\texttt{x} is empty, otherwise returns \texttt{False}; 
\item \texttt{Clear(x)} -- empties the linked list \texttt{x};
\end{itemize}
The linked list is implemented by the use of a collection of nodes
that have two parts: the item data and a pointer to the next item
in the list. In addition, there is a \texttt{Start} pointer which
points to the first item in the list.
\begin{enumerate}
\item Write algorithms that could be used to implement the \textquoteleft \texttt{Delete}\textquoteright{}
and \textquoteleft \texttt{Insert}\textquoteright{} operation. \hfill{}{[}8{]}
\end{enumerate}
A stack ADT has the following operations:
\begin{itemize}
\item \texttt{Create()} - creates a new stack; 
\item \texttt{Push(item)} - adds \texttt{item} onto the stack; 
\item \texttt{Pop()} - deletes and returns item from the stack; 
\item \texttt{IsEmpty()} -- if the stack is empty returns True, otherwise
False; 
\item \texttt{Clear()} -- removes all items in the stack;
\end{itemize}
\begin{enumerate}
\item[(b)]  Show how to implement \textquoteleft \texttt{Create}\textquoteright ,
\textquoteleft \texttt{Push}\textquoteright{} and \textquoteleft \texttt{Pop}\textquoteright{}
operation using the list ADT operations.\hfill{} {[}4{]}
\end{enumerate}
The stack implementation above is used to implement the undo/redo
mechanism of a text editor. 

An Undo stack is used to keep the edit history of the editor and the
Redo stack is used to keep the history of the undo operations. The
content of the text editor is stored as a string in the Undo stack
and Redo stack. 

When an undo is invoked, the Undo stack is popped and the content
is pushed into the Redo Stack. When a redo is invoked, the Redo stack
is popped and the content is pushed into the Undo Stack.
\begin{enumerate}
\item[(c)]  Using the stack ADT operations, show how to implement the following
functions which return the contents. Assume that \texttt{undoStack}
and \texttt{redoStack} are created. 
\begin{enumerate}
\item \texttt{FUNCTION Undo() RETURNS STRING} 
\item \texttt{FUNCTION Redo() RETURNS STRING}\hfill{} {[}3{]}
\end{enumerate}
\end{enumerate}