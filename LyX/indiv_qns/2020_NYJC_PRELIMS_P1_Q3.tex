\item \textbf{{[}NYJC/PRELIM/9569/2020/P1/Q3{]} }

The algorithm represented using pseudo-code in \textbf{Figure 3} describes
a method to convert two hexadecimal numbers into decimal. The subroutine
\texttt{ToDecimal} used in\textbf{ Figure 3} is shown in \textbf{Figure
4} and the built-in subroutine \texttt{ASCII} is explained in \textbf{Table
1}. 
\noindent \begin{center}
\textbf{Figure 3} 
\par\end{center}

\noindent %
\noindent\begin{minipage}[t]{1\columnwidth}%
\texttt{FOR Count <- 1 TO 2 }

\texttt{\qquad{}INPUT HexString }

\texttt{\qquad{}Number <- 0 }

\texttt{\qquad{}FOR EACH HexDigit IN HexString }

\texttt{\qquad{}\qquad{}Value <- ToDecimal(HexDigit) }

\texttt{\qquad{}\qquad{}Number <- Number {*} 16 + Value }

\texttt{\qquad{}ENDFOR }

\texttt{\qquad{}OUTPUT Number }

\texttt{ENDFOR}%
\end{minipage}

The \texttt{FOR EACH} command steps through each character in a string
working from left to right. 
\noindent \begin{center}
\textbf{Figure 4} 
\par\end{center}

\noindent %
\noindent\begin{minipage}[t]{1\columnwidth}%
\texttt{SUBROUTINE ToDecimal(HexDigit)}

\texttt{\qquad{}IF HexDigit = \textquotedbl A\textquotedbl{} THEN }

\texttt{\qquad{}\qquad{}Value <- 10}

\texttt{\qquad{}ELSEIF HexDigit = \textquotedbl B\textquotedbl{}
THEN }

\texttt{\qquad{}\qquad{}Value <- 11}

\texttt{\qquad{}ELSEIF HexDigit = \textquotedbl C\textquotedbl{}
THEN}

\texttt{\qquad{}\qquad{}Value <- 12}

\texttt{\qquad{}ELSEIF HexDigit = \textquotedbl D\textquotedbl{}
THEN }

\texttt{\qquad{}\qquad{}Value <- 13}

\texttt{\qquad{}ELSEIF HexDigit {*} \textquotedbl E\textquotedbl{}
THEN }

\texttt{\qquad{}\qquad{}Value <- 14}

\texttt{\qquad{}ELSEIF HexDlgit = \textquotedbl F\textquotedbl{}
THEN }

\texttt{\qquad{}\qquad{}Value <- 15}

\texttt{\qquad{}ELSEIF HexDigit IN (\textquotedbl 0\textquotedbl ,
\textquotedbl l\textquotedbl , ..., \textquotedbl 9\textquotedbl{]}
THEN }

\texttt{\qquad{}\qquad{}Value <- ASCII(HexDigit) - 48}

\texttt{\qquad{}ELSE }

\texttt{\qquad{}\qquad{}Value <- -1}

\texttt{\qquad{}ENDIF}

\texttt{\qquad{}RETURN Value}

\texttt{ENDSUBROUTINE}%
\end{minipage}
\noindent \begin{center}
\textbf{Table 1} 
\par\end{center}

\noindent \begin{center}
\begin{tabular}{|c|c|}
\hline 
Subroutine used in Figure 4 & Description\tabularnewline
\hline 
\hline 
ASCII(Char) & Returns the ASCII code or the chal passed as a parameter. Example:
\texttt{ASCII (\textquotedbl l\textquotedbl )} returns \texttt{49}\tabularnewline
\hline 
\end{tabular}
\par\end{center}
\begin{enumerate}
\item Copy and complete the following table by hand-tracing the algorithm
in Figure 3. Use \textquotedbl\texttt{A2}\textquotedbl{} and \textquotedbl\texttt{1G}\textquotedbl{}
as input strings. You may not need to use all the rows. 
\noindent \begin{center}
\begin{tabular}{|c|c|c|c|c|c|}
\hline 
\texttt{\textbf{Count}} & \texttt{\textbf{HexString}} & \texttt{\textbf{Number}} & \texttt{\textbf{HexDigit}} & \texttt{\textbf{Value}} & \texttt{\textbf{Output}}\tabularnewline
\hline 
 &  &  &  &  & \tabularnewline
\hline 
 &  &  &  &  & \tabularnewline
\hline 
\multicolumn{6}{c}{$\vdots$}\tabularnewline
\end{tabular}
\par\end{center}

\hfill{}{[}6{]}
\item Explain how the algorithm in Figure 3 has attempted to deal with the
conversion of \textquotedbl\texttt{1G}\textquotedbl{} into decimal
and why this method is not fully effective. \hfill{} {[}2{]}
\item Other than a trace table, describe two other debugging methods a programmer
can use to find bugs in his code. \hfill{} {[}4{]}
\end{enumerate}