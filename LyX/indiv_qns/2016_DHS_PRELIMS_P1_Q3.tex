\item \textbf{{[}DHS/PRELIM/9597/2016/P1/Q3{]} }

A Pokemon Go fan who nicknames himself Pokeboy wishes to manage his
Pokemon collection using a binary search tree. Each binary search
tree node stores the name together with the numbers of that particular
Pokemon and candies collected, as well as a reference to a linked
list. Each linked list node stores the combat power (CP) of a Pokemon.
The binary search tree is organised in ascending Pokemon name order.
Each linked list is organised in descending order of CP. For ease
of reference, we shall name this composite data structure Poketree. 

\subsection*{Task 3.1 }

Using object-oriented programming, construct appropriate classes to
initialise, insert and display the contents of the Poketree. In your
main driver program, write code to initialise a Poketree with the
first Pokemon collected which is randomly generated from the contents
in the file \texttt{POKEMONS.TXT} as well as a randomly generated
CP in the range 10 to 200 inclusive. When a Pokemon is caught, 3 candies
of that kind are also added to the candies count of the Poketree binary
search tree node.

\subsection*{Evidence 9: }

Program code for class definition, initialisation and display methods.\hfill{}
{[}7{]}

\subsection*{Evidence 10:}

Screenshot output to show contents of Poketree with first Pokemon
inserted.\hfill{} {[}1{]}

\subsection*{Task 3.2 }

Write a class method \texttt{insert()} and the necessary main driver
program code to insert another 23 randomly generated Pokemons and
their corresponding CPs into the Poketree. 

\subsection*{Evidence 11: }

Program code for Task 3.2.\hfill{} {[}4{]}

\subsection*{Evidence 12:}

Screenshot to show updated contents of Poketree.\hfill{} {[}1{]}

Apart from collecting Pokemons, one can also evolve Pokemons. When
a new Pokemon is evolved, its previous incarnation is deleted from
the Poketree and its new incarnation either inserted (if its kind
is not previously collected) or updated (if its kind is previously
collected) to the Poketree. Evolving also requires a set amount and
type of Pokemon candy. If there is insufficient Pokemon candies, Pokeboy's
preferred action is to exchange (i.e. delete) existing Pokemons of
the same species for candies, starting from the Pokemon with the lowest
CP. Each Pokemon can be exchanged for one candy. Pokeboy also wishes
to keep at least 2 Pokemons of the same species in his collection
(well he is a Pokemon fan). The text file \texttt{CANDIES.TXT} contains
the candies requirement for evolvement. 

\subsection*{Task 3.3 }

Write a Boolean class method \texttt{can\_evolve(pokemon)} to determine
if a particular Pokemon can evolve by Pokeboy's preference. A Pokemon
can evolve if there are sufficient candies or it is possible to exchange
candies, leaving a minimum of 2 Pokemons of that species.

Write a class method evolve(pokemon) which will evolve a Pokemon using
\texttt{can\_evolve(pokemon)}. 

Add necessary class method(s) and main driver program code to evolve
all evolvable Pokemons. 

\subsection*{Evidence 13: }

Program code for Task 3.3.\hfill{} {[}7{]}

\subsection*{Evidence 14:}

Screenshot(s) to show evolution.\hfill{} {[}1{]}

\subsection*{Task 3.4 }

Write a class method \texttt{most\_poke()} to output the frequency
and most collected Pokemon(s) collected by Pokeboy. 

Add necessary program code to exercise \texttt{most\_poke()}.

\subsection*{Evidence 15: }

Program code for Task 3.4.\hfill{} {[}4{]}

\subsection*{Evidence 16:}

Screenshot for Task 3.4. \hfill{}{[}1{]}

\subsection*{Task 3.5}

Pokeboy aspires to join the Catch Them All club. The Catch Them All
club is a group of elites who have managed to catch every species
of Pokemon at least once. Write a class method \texttt{catch\_them\_all()}
to either output the message \textquotedbl Welcome to the club! You
have caught them all!\textquotedbl{} or output the number and remaining
Pokemon names yet to be caught.

Add necessary program code to exercise \texttt{catch\_them\_all()}. 

\subsection*{Evidence 17:}

Program code for Task 3.5.\hfill{} {[}6{]}

\subsection*{Evidence 18: }

Screenshot for annotated test cases. \hfill{}{[}2{]}

\subsection*{Task 3.6 }

After some time, it becomes necessary to reorganise the binary search
tree to ensure optimal search performance. Write program code to rebalance
the binary search tree and include as comments your strategy to do
this. You should output the roots and heights of the old and new binary
search trees.

\subsection*{Evidence 19:}

Program code for Task 3.6.\hfill{} {[}5{]}

\subsection*{Evidence 20:}

Screenshot.\hfill{} {[}1{]}