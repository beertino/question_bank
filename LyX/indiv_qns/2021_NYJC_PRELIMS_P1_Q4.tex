\item \textbf{{[}NYJC/PRELIM/9569/2021/P1/Q4{]} }

An algorithm for sorting an array of elements is shown. 

\noindent %
\noindent\begin{minipage}[t]{1\columnwidth}%
\texttt{01 FOR i = 1 to Array.LENGTH - 1 }

\texttt{02 \qquad{}FOR j = 1 to Array.LENGTH - 1 }

\texttt{03 \qquad{}\qquad{}IF Array{[}j{]} > Array{[}j+1{]} }

\texttt{04 \qquad{}\qquad{}\qquad{}THEN }

\texttt{05 \qquad{}\qquad{}\qquad{}\qquad{}t = Array{[}j{]} }

\texttt{06 \qquad{}\qquad{}\qquad{}\qquad{}Array{[}j{]} = Array{[}j+1{]} }

\texttt{07 \qquad{}\qquad{}\qquad{}\qquad{}Array{[}j+1{]} = t }

\texttt{08 \qquad{}\qquad{}ENDIF }

\texttt{09 \qquad{}ENDFOR }

\texttt{10 ENDFOR }%
\end{minipage}
\begin{enumerate}
\item {} 
\begin{enumerate}
\item State the algorithm represented. \hfill{}{[}1{]}
\item tate the time complexity of this algorithm. \hfill{}{[}1{]}
\item Copy and complete the trace table below with the value of \texttt{Array}
at the end of each iteration of \texttt{i} in the algorithm. \hfill{}{[}5{]}
\end{enumerate}
\noindent \begin{center}
\begin{tabular}{|c|c|}
\hline 
\texttt{i} & \texttt{Array}\tabularnewline
\hline 
Initial  & {[}2, 3, 4, 5, 6, 1{]}\tabularnewline
\hline 
1 & \tabularnewline
\hline 
2 & \tabularnewline
\hline 
3 & \tabularnewline
\hline 
4 & \tabularnewline
\hline 
5 & \tabularnewline
\hline 
6 & \tabularnewline
\hline 
\end{tabular}
\par\end{center}
\item Describe \textbf{two} improvements that could be made to the above
algorithm to improve its efficiency. \hfill{}{[}4{]}
\item Explain why insertion sort is usually used instead of bubble sort,
although both have the same time efficiency.\hfill{} {[}2{]}
\end{enumerate}