\item \textbf{{[}RVHS/PRELIM/9597/2018/P1/Q3{]} }

\textbf{Cards Party}

A deck of French playing cards is the most common deck of playing
cards used today. It includes thirteen ranks of each of the four suits:
clubs ($\clubsuit$), diamonds ($\diamondsuit$), hearts ($\heartsuit$)
and spades ($\spadesuit$).
\noindent \begin{center}
\begin{tabular}{|c|c|c|c|c|c|c|c|c|c|c|c|c|c|}
\hline 
\texttt{Club} & \texttt{A$\clubsuit$} & \texttt{2$\clubsuit$} & \texttt{3$\clubsuit$} & \texttt{4$\clubsuit$} & \texttt{5$\clubsuit$} & \texttt{6$\clubsuit$} & \texttt{7$\clubsuit$} & \texttt{8$\clubsuit$} & \texttt{9$\clubsuit$} & \texttt{10$\clubsuit$} & \texttt{J$\clubsuit$} & \texttt{Q$\clubsuit$} & \texttt{K$\clubsuit$}\tabularnewline
\hline 
\texttt{Diamond} & \texttt{A$\diamondsuit$} & \texttt{2$\diamondsuit$} & \texttt{3$\diamondsuit$} & \texttt{4$\diamondsuit$} & \texttt{5$\diamondsuit$} & \texttt{6$\diamondsuit$} & \texttt{7$\diamondsuit$} & \texttt{8$\diamondsuit$} & \texttt{9$\diamondsuit$} & \texttt{10$\diamondsuit$} & \texttt{J$\diamondsuit$} & \texttt{Q$\diamondsuit$} & \texttt{K$\diamondsuit$}\tabularnewline
\hline 
\texttt{Heart} & \texttt{A$\heartsuit$} & \texttt{2$\heartsuit$} & \texttt{3$\heartsuit$} & \texttt{4$\heartsuit$} & \texttt{5$\heartsuit$} & \texttt{6$\heartsuit$} & \texttt{7$\heartsuit$} & \texttt{8$\heartsuit$} & \texttt{9$\heartsuit$} & \texttt{10$\heartsuit$} & \texttt{J$\heartsuit$} & \texttt{Q$\heartsuit$} & \texttt{K$\heartsuit$}\tabularnewline
\hline 
\texttt{Spade} & \texttt{A$\spadesuit$} & \texttt{2$\spadesuit$} & \texttt{3$\spadesuit$} & \texttt{4$\spadesuit$} & \texttt{5$\spadesuit$} & \texttt{6$\spadesuit$} & \texttt{7$\spadesuit$} & \texttt{8$\spadesuit$} & \texttt{9$\spadesuit$} & \texttt{10$\spadesuit$} & \texttt{J$\spadesuit$} & \texttt{Q$\spadesuit$} & \texttt{K$\spadesuit$}\tabularnewline
\hline 
\end{tabular} 
\par\end{center}

In this task, the \texttt{rank} and \texttt{suits} will follow the
following order when a comparison is needed. Ranks from smallest to
largest (left to right)
\begin{center}
\begin{tabular}{|c|c|c|c|c|c|c|c|c|c|c|c|c|c|}
\hline 
 & \multicolumn{13}{c|}{\texttt{\textbf{Ranks from smallest to largest ( left to right )}}}\tabularnewline
\hline 
\texttt{Rank} & \texttt{A} & \texttt{2} & \texttt{3} & \texttt{4} & \texttt{5} & \texttt{6} & \texttt{7} & \texttt{8} & \texttt{9} & \texttt{10} & \texttt{J} & \texttt{Q} & \texttt{K}\tabularnewline
\hline 
\texttt{Values Represented} & \texttt{1} & \texttt{2} & \texttt{3} & \texttt{4} & \texttt{5} & \texttt{6} & \texttt{7} & \texttt{8} & \texttt{9} & \texttt{10} & \texttt{11} & \texttt{12} & \texttt{13}\tabularnewline
\hline 
\end{tabular}
\par\end{center}

\noindent \begin{center}
\begin{tabular}{|c|c|c|c|c|}
\hline 
 & \multicolumn{4}{c|}{\texttt{\textbf{Suits from smallest to largest ( left to right )}}}\tabularnewline
\hline 
\texttt{Suits} & \texttt{Club} & \texttt{Diamond} & \texttt{Heart} & \texttt{Spade}\tabularnewline
\hline 
\texttt{Python Expression} & \texttt{u\textquotedbl\textbackslash u2663\textquotedbl} & \texttt{u\textquotedbl\textbackslash u2662\textquotedbl} & \texttt{u\textquotedbl\textbackslash u2661\textquotedbl} & \texttt{u\textquotedbl\textbackslash u2660\textquotedbl{} }\tabularnewline
\hline 
\end{tabular}
\par\end{center}

\subsection*{Task 3.1 -- Card Class}

Implement the \texttt{Card} class based on the following UML class
diagram. The descriptions for some class methods can be found below. 
\begin{center}
\begin{tabular}{|l|}
\hline 
\texttt{\hspace{0.25\columnwidth}Card}\tabularnewline
\hline 
\texttt{- suit : string}\tabularnewline
\texttt{- rank\_value : int}\tabularnewline
\hline 
\texttt{+ Card(suit:string, rank\_value: int)}\tabularnewline
\texttt{+ get\_suit():string}\tabularnewline
\texttt{+ get\_rank\_value(): int}\tabularnewline
\texttt{+ get\_suit\_symbol(): string}\tabularnewline
\texttt{+ get\_rank():string}\tabularnewline
\texttt{+ \_\_str\_\_(): string}\tabularnewline
\hline 
\end{tabular}
\par\end{center}

\textbf{Functions and their descriptions:}
\begin{itemize}
\item \texttt{get\_suit\_symbol(): string} Return the Unicode symbol corresponding
to the \texttt{card}\textquoteright s suit. 
\item \texttt{get\_rank():string} Return a string from \textquotedbl\texttt{A}\textquotedbl ,
\textquotedbl\texttt{2}\textquotedbl , \textquotedbl\texttt{3}\textquotedbl ,
\dots{} \textquotedbl\texttt{10}\textquotedbl , \textquotedbl\texttt{J}\textquotedbl ,
\textquotedbl\texttt{Q}\textquotedbl , \textquotedbl\texttt{K}\textquotedbl{}
corresponding to the \texttt{card}\textquoteright s \texttt{rank\_value}. 
\item \texttt{\_\_str\_\_(): string} Return a string with length of 3-character,
stating the \texttt{card}\textquoteright s \texttt{rank} followed
by its \texttt{suit}. For example: \textquotedbl{} \texttt{K$\spadesuit$}
\textquotedbl , \textquotedbl\texttt{10}$\heartsuit$\textquotedbl ,
\textquotedbl{} \texttt{8}$\clubsuit$\textquotedbl{}
\end{itemize}

\subsection*{Evidence 13 }

Screenshot of the program. \hfill{}{[}4{]}

\subsection*{Task 3.2 -- CardList Class }

Implement the \texttt{CardList} class based on the following UML class
diagram. The descriptions for some class methods can be found below.
\begin{center}
\begin{tabular}{|l|}
\hline 
\hspace{0.25\columnwidth}CardList\tabularnewline
\hline 
\texttt{- cards: list = list()}\tabularnewline
\texttt{- rank\_value : int}\tabularnewline
\hline 
\texttt{+ CardList()}\tabularnewline
\texttt{+ add\_card(new\_card)}\tabularnewline
\texttt{+ get\_cards(): list}\tabularnewline
\texttt{+ get\_size(): int}\tabularnewline
\texttt{+ shuffle()}\tabularnewline
\texttt{+ \_\_str\_\_(): string}\tabularnewline
\hline 
\end{tabular}
\par\end{center}

\textbf{Functions and their descriptions:}
\begin{itemize}
\item \texttt{add\_card(new\_card)} Add a new \texttt{card} into the current
list of \texttt{cards}. 
\item \texttt{get\_cards(): list} Return the current list of \texttt{cards}. 
\item \texttt{get\_size():int} Get the size of the current list of \texttt{cards}. 
\item \texttt{shuffle()} Shuffle the current list of \texttt{cards}. 
\item \texttt{\_\_str\_\_(): string} Return a \texttt{string} consisting
of each \texttt{card} in the list of \texttt{cards}, separated by
commas. For example: \textquotedbl{} \texttt{K$\spadesuit$, 9$\spadesuit$,
5$\spadesuit$, A$\spadesuit$, 10$\heartsuit$, 6$\heartsuit$, 2$\heartsuit$,
J$\diamondsuit$, 7$\diamondsuit$, 3$\diamondsuit$, Q$\clubsuit$,
8$\clubsuit$, 4$\clubsuit$}\textquotedbl{} 
\end{itemize}

\subsection*{Evidence 14}

Screenshot of the program. \hfill{}{[}4{]}

\subsection*{Task 3.3 -- Sort your hand }

Implement two additional methods in the \texttt{CardList} class based
on the descriptions below. 

State the best-case and worst-case time complexity of the two methods. 

Functions and their Descriptions
\begin{itemize}
\item \texttt{sort\_by\_suit():} Sort the current list of \texttt{cards}
by suit first, then by rank, in \textbf{ascending} order. Implement
the method by adopting the algorithm of \textbf{insertion sort}.
\item \texttt{sort\_by\_rank():} Sort the current list of \texttt{cards}
by rank first, then by suit, in \textbf{ascending} order. Implement
the method by adopting the algorithm of \textbf{improved bubble sort}.
\end{itemize}

\subsection*{Evidence 15}

Screenshot of the program. 

Best-case and worst-case time complexity of the two methods.\hfill{}
{[}8{]}

\subsection*{Task 3.4 -- Test Sorting}

Algorithms Implement a function test\_sorts()and design 2 test cases.
Run each test case against both methods and test if they are implemented
correctly. 

Write down the justification of each test case using comments. 

\subsection*{Evidence 16}

Screenshot of the program and output. \hfill{}{[}4{]}

\subsection*{Task 3.5 -- Hand Ranking }

A hand of five cards can be ranked according to the following categories
from low to high. 

\textbf{Names and their Examples }
\begin{itemize}
\item Normal - A normal hand containing five cards not of sequential rank
nor of the same suit, such as \texttt{K$\heartsuit$ J$\heartsuit$
8$\clubsuit$ 7$\diamondsuit$ 4}$\spadesuit$. 
\item Straight - A straight hand containing five cards of sequential rank,
not all of the same suit, such as 7\texttt{$\clubsuit$} 6$\spadesuit$
5$\spadesuit$ 4\texttt{$\heartsuit$} 3\texttt{$\heartsuit$}. (AKQJ10
is included.) 
\item Flush - A flush hand containing five cards all of the same suit, not
all of sequential rank, such as K\texttt{$\clubsuit$} 10\texttt{$\clubsuit$}
7\texttt{$\clubsuit$} 6\texttt{$\clubsuit$} 4\texttt{$\clubsuit$}. 
\item Straight Flush - A straight flush is a poker hand containing five
cards of sequential rank, all of the same suit, such as Q\texttt{$\heartsuit$}
J\texttt{$\heartsuit$} 10\texttt{$\heartsuit$} 9\texttt{$\heartsuit$}
8\texttt{$\heartsuit$}. 
\end{itemize}
Implement an function \texttt{compare\_five\_cards()}. The function
should take in 2 \texttt{CardList} objects, \texttt{hand1} and \texttt{han}d2.
If both hands have exactly five cards, compare the hands and return
a string indicating if the first hand wins, loses, or there is a draw. 

For this task, if both hands fall into the same category, it is not
necessary to further compare the value. The comparison result will
be deemed as a draw. 

\subsection*{Evidence 17}

Screenshot of the program. \hfill{} {[}5{]}