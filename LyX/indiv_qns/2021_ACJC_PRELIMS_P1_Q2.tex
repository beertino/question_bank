\item \textbf{{[}ACJC/PRELIM/9569/2021/P1/Q2{]} }

A hash table has 8 spaces to store strings, indexed from 1 to 8 inclusive.

The hash function finds the ASCII number of the first letter of the
string, then counts the number of 1s in its binary representation.
This is the index in which the string will be inserted into the hash
table.

For example, the string \texttt{'Arlington'} will have index 2 because
the ASCII number of '\texttt{A}' is 65, which is \texttt{1000001}
in binary, and there are two \texttt{1}s.

The following strings are to be inserted into the hash table in the
order given.

\texttt{'Grover', }

\texttt{Horsburgh', }

\texttt{'Island',}

\texttt{'Jordan', }

\texttt{'Kalman'}
\begin{enumerate}
\item Find the output of the hash function for each of the strings. \hfill{}{[}5{]}
\item {}
\begin{enumerate}
\item Suppose collisions in the hash table are to be resolved using open
hashing.

Draw the hash table after all five strings are inserted. \hfill{}{[}5{]}
\item Suppose instead that collisions in the hash table are to be resolved
using closed hashing, where spaces 6 to 8 (inclusive) are used as
the overflow storage.

Draw the hash table after all five strings are inserted. \hfill{}
{[}2{]}
\end{enumerate}
\item Explain why the space with index 1 in the hash table will never be
occupied unless there is a collision. \hfill{} {[}2{]}
\end{enumerate}