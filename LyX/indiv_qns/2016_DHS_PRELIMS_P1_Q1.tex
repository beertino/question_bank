\item \textbf{{[}DHS/PRELIM/9597/2016/P1/Q1{]} }

In Olympic diving, scoring is done by a panel comprising a minimum
of 3 judges. The highest and lowest scores are dropped to eliminate
the extremes. The raw score is computed by summing the middle scores.
The raw score is then multiplied by the level of difficulty to give
the total score. 

Sample scoring for a 5-judges panel: Scores: 

6.5, 6, 6.5, 6, 5.5 

Lowest (5.5) and highest (6.5) scores dropped 

Raw Score = 18.5 (6.5 + 6 + 6) 

Total Score = Score (18.5) x Difficulty Level (1.5) = 27.75 

\texttt{DIVE1.TXT} contains the countries, difficulty levels and 5-judges
scores for a diving competition. 

\subsection*{Task 1.1 }

Write program code to determine the podium winners of the competition.
Your program should output the medal (Gold, Silver, Bronze), country
name and the total score (to 2 decimal places). 

Sample output:

\noindent %
\noindent\begin{minipage}[t]{1\columnwidth}%
\texttt{Gold: China 45.90 }

\texttt{Silver: Malaysia 39.20}

\texttt{Bronze: United States 36.00 }%
\end{minipage}

\subsection*{Evidence 1: }

Program code. \hfill{}{[}5{]}

\subsection*{Evidence 2: }

Screenshot of output. \hfill{}{[}1{]}

In most international competitions with more than five judges, the
3/5 method is used. The middle 5 numbers are added and then multiplied
by the difficulty of the dive and then multiplied again by 0.6. \texttt{DIVE2.TXT}
contains the countries, difficulty levels and 9-judges scores for
a diving competition. 

\subsection*{Task 1.2 }

Write program code to determine the podium winners of the competition.
Your program should output the medal (Gold, Silver, Bronze), country
name as well as the total score (to 2 decimal places). 

\subsection*{Evidence 3: }

Program code. \hfill{}{[}8{]}

\subsection*{Evidence 4: }

Screenshot of output. \hfill{}{[}1{]}