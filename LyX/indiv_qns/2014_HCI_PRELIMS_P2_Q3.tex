\item \textbf{{[}HCI/PRELIM/9597/2014/P2/Q3{]} }

All\textquoteright s Well Hospital is in the process of computerizing
its clinic processes. At the clinic, there are various doctors specializing
in different areas e.g. neuroscience specialists, gynaecologists,
etc. All patients of the clinic must be registered in the patient
database and make prior appointments to see the relevant doctors.
No walk-in patients will be entertained. Appointments are made based
on the doctor\textquoteright s weekly schedule i.e. different doctors
have clinic sessions at different day/time of the week. 

On the day of appointment, patients are to register at the self-registration
counter before proceeding to the doctor\textquoteright s room. Upon
seeing the doctor, the doctor will input the diagnosis into the system.
After the doctor\textquoteright s consultation, the patient is to
proceed to make payment. Upon payment, a prescription slip, reference
letter, medical certificate, will be printed as necessary for the
patient. In addition, the next appointment may be scheduled, if necessary. 
\begin{enumerate}
\item Use a diagram to show the data flows, processes, data stores and external
links in the system. \hfill{}{[}10{]}
\item At the self-registration counter, patients can scan the barcode of
their appointment card or NRIC. Alternatively, they can also enter
their NRIC number using the keyboard. When the NRIC is entered using
the keyboard, it must be validated and verified. 
\begin{enumerate}
\item Explain the difference between validation and verification of data.
\hfill{}{[}2{]}
\item Describe two validation tests that can be performed on the NRIC number
entered. \hfill{}{[}2{]}
\item Describe how the NRIC number entered by the patient can be verified
by the system.\hfill{}{[}1{]}
\end{enumerate}
\item As part of the hospital welfare program, there is a fitness center
for staff. To enter the fitness center, staff needs to use a swipe
card together with a 4-digit PIN to gain access. Access is only allowed
if there are fewer than 50 people already in the fitness center, in
order to avoid overcrowding. Access is also restricted to one person
per card. If maintenance is being carried out then access is denied
and a message is output to a screen asking the staff to return in
1 hour. In order to exit the fitness center, the swipe card is used
again. Using appropriate variable names, write in pseudocode, an algorithm
to control the entry system to the fitness center.\hfill{}{[}8{]}
\end{enumerate}