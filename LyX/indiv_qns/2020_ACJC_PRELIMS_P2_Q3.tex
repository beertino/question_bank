\item \textbf{{[}ACJC/PRELIM/9569/2021/P2/Q3{]} }

A text file \texttt{INVENTORY.txt} contains the inventory data for
a certain electronics store. Each line in the file contains tab-delimited
data that shows the product name, product type, purchase price, selling
price and quantity available.

Each line is in the format

\texttt{Name\textbackslash tType\textbackslash tPurchase\_Price\textbackslash tSelling\_Price\textbackslash tQuantity}

where \texttt{\textbackslash t} represents the tab character.

\subsection*{Task 3.1 }

Write program code to:
\begin{itemize}
\item Read the inventory data from the text file; 
\item Find the average selling price of products belonging to the \texttt{Laptop}
product type and display this value; 
\item Count the number of products in each product type and store it in
appropriate data structure called \texttt{TypeCount}; 
\item Display the product type with the greatest number of products. If
there is a tie, display all of the product types with the greatest
number of products.
\end{itemize}
Download your program code and output for Task 3.1 as 

\texttt{TASK3\_1\_<your name>\_<centre number>\_<index number>.ipynb}
\hfill{}{[}6{]}

\subsection*{Task 3.2 }

The profit margin of each product can be calculated by the following
equation:
\noindent \begin{center}
Profit margin = selling price -- purchase price.
\par\end{center}

Write program code to:
\begin{itemize}
\item Calculate and display the total profit the store could make if all
products are sold; 
\item Sort the inventory data using a Merge sort algorithm in descending
order of profit margin; 
\item Display the sorted inventory data in the format given below.
\noindent \begin{center}
\texttt{}%
\begin{tabular}{lll}
\texttt{Product } & \texttt{Product Type } & \texttt{Profit Margin }\tabularnewline
\texttt{ThinkingPad 14} & \texttt{Computer } & \texttt{300 }\tabularnewline
\texttt{Bapple 8 } & \texttt{Phone } & \texttt{450}\tabularnewline
\end{tabular}
\par\end{center}

\end{itemize}
Download your program code and output for Task 3.2 as 

\texttt{TASK3\_2\_<your name>\_<centre number>\_<index number>.ipynb}\hfill{}
{[}9{]}

\subsection*{Task 3.3}

A store manager decided to make some changes to \texttt{INVENTORY.txt}
and saved it as \texttt{INVENTORY\_SERIAL.txt}. Each line in the updated
file contains tab-delimited data that shows the serial number, product
name, product type, purchase price, selling price and quantity available.

Each line is in the format:

\texttt{Serial\_No\textbackslash tName\textbackslash tType\textbackslash tPurchase\_Price\textbackslash tSelling\_Price\textbackslash tQuantity}

where \texttt{\textbackslash t} represents the tab character.

Write program code to insert the data from \texttt{INVENTORY\_SERIAL.txt}
into a NoSQL database \texttt{OUTLETS} under the collection \texttt{GEM}.

Download your program code for Task 3.3 as 

\texttt{TASK3\_3\_<your name>\_<centre number>\_<index number>.py}
\hfill{}{[}5{]}

\subsection*{Task 3.4}

The database administrator wants to validate that the store manager
did not make any errors when he edited the text file. Write program
code to check that the database conforms to the below specifications:
\begin{itemize}
\item \texttt{Serial\_No} consists of one digit followed by two letters,
followed by one digit (e.g. \texttt{1AB7}); 
\item \texttt{Name} consists of only letters, digits and spaces; 
\item \texttt{Quantity} is a positive integer. 
\end{itemize}
Any document that has an error should be removed from the database.
You may assume data fields not specified above are error free. Display
the documents that were removed.

Download your program code for Task 3.4 as 

\texttt{TASK3\_4\_<your name>\_<centre number>\_<index number>.py}
\hfill{}{[}5{]}