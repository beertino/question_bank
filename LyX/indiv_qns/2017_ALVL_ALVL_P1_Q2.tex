\item \textbf{{[}ALVL/9597/2017/P1/Q2{]} }

Every published book has an international Standard Book Number (ISBN).
This ISBN is a 9-digit number plus a check digit which is calculated
and added to the end of the number. A weighted-modulus method is used
to calculate the check digit. 

This method uses a weighted modulus 11. If the check digit is calculated
as 10. it is replaced with the character 'X'. Where the check digit
is calculated as 11, it will be replaced with 0. 

\noindent %
\begin{tabular}{l}
184146208 will be calculated as\tabularnewline
$1\times10=10$\tabularnewline
$8\times9=72$\tabularnewline
$4\times8=32$\tabularnewline
$1\times7=7$\tabularnewline
$4\times6=24$\tabularnewline
$6\times5=30$\tabularnewline
$2\times4=8$\tabularnewline
$0\times3=0$\tabularnewline
$8\times2=16$\tabularnewline
Total = 199\tabularnewline
$199/11=18$ remainder 1\tabularnewline
$11-1=10$\tabularnewline
Therefore, 10 is replaced with X:\tabularnewline
ISBN is 184146208\textbf{X}\tabularnewline
\end{tabular}%
\begin{tabular}{l}
034085045 will be calculated as\tabularnewline
$0\times10=0$\tabularnewline
$7\times9=63$\tabularnewline
$5\times8=40$\tabularnewline
$1\times7=7$\tabularnewline
$5\times6=30$\tabularnewline
$4\times5=20$\tabularnewline
$9\times4=36$\tabularnewline
$2\times3=6$\tabularnewline
$6\times2=12$\tabularnewline
Total = 154\tabularnewline
$154/11=14$ remainder 0\tabularnewline
$11-0=11$\tabularnewline
Therefore, 11 is replaced with 0:\tabularnewline
ISBN is 034085045\textbf{0}\tabularnewline
\end{tabular}%
\begin{tabular}{l}
075154926 will be calculated as\tabularnewline
$0\times10=0$\tabularnewline
$3\times9=27$\tabularnewline
$4\times8=32$\tabularnewline
$0\times7=0$\tabularnewline
$8\times6=48$\tabularnewline
$5\times5=25$\tabularnewline
$0\times4=0$\tabularnewline
$4\times3=12$\tabularnewline
$5\times2=10$\tabularnewline
Total = 214\tabularnewline
$214/11=19$ remainder 5\tabularnewline
$11-5=6$\tabularnewline
Therefore, 6 is added to the\tabularnewline
end of the ISBN: 075154926\textbf{6} \tabularnewline
\end{tabular}

\subsubsection*{Task 2.1}

Study the identifier table and the incomplete recursive algorithm
on the opposite page. 

The missing lines in the algorithm are labelled \textbf{A}, \textbf{B}
and \textbf{C}. 

Write the\textbf{ three} missing lines of code. Label each as \textbf{A},
\textbf{B} or \textbf{C}. \hfill{} {[}3{]}

\subsubsection*{Evidence 3}

The three missing lines of code. \hfill{}{[}3{]}
\begin{center}
\begin{tabular}{|l|c|l|}
\hline 
\textbf{Identifier} & \textbf{Data Type} & \textbf{Description}\tabularnewline
\hline 
\texttt{Number} & \texttt{STRING} & The ISBN to be processed\tabularnewline
\hline 
\texttt{Digit} & \texttt{lNTEGER} & A digit from the iSBN to be processed\tabularnewline
\hline 
\texttt{Total} & \texttt{INTEGER} & Running total for modulus calculation\tabularnewline
\hline 
\texttt{NewNumber} & \texttt{STRING} & A version of the list string shortened by removing the first character\tabularnewline
\hline 
\texttt{CheckDigit} & \texttt{STRING} & The calculated check digit value\tabularnewline
\hline 
\texttt{CalcModulus} & \texttt{INTEGER} & Used to store the result of \texttt{(Total MOD 11)}\tabularnewline
\hline 
\texttt{CheckValue} & \texttt{INTEGER} & Used to store the result of \texttt{(11 - CalcModulus)}\tabularnewline
\hline 
\end{tabular}
\par\end{center}

\noindent %
\noindent\begin{minipage}[t]{1\columnwidth}%
\texttt{FUNCTION CalCheckDigit(Number AS STRING, Total AS INTEGER)
RETURNS STRING }

\texttt{\qquad{}IF LENGTH(Number) > 1 THEN }

\texttt{\qquad{}\qquad{}Digit <- INTEGER(LEFT(Number,1)) }

\texttt{\qquad{}\qquad{}Total <- Total + (Digit {*} (LENGTH(Number)+1)) }

\texttt{\qquad{}\qquad{}NewNumber <- RIGHT(Number, LENGTH(Number)-1) }

\texttt{\qquad{}\qquad{}CheckDigit <- .................. A ................. }

\texttt{\qquad{}ELSE }

\texttt{\qquad{}\qquad{}Digit <- INTEGER(LEFT(Number,1)) }

\texttt{\qquad{}\qquad{}Total <- Total + (Digit 1 (LENGTH(Number)+1)) }

\texttt{\qquad{}\qquad{}CalcModulus <- Total MOD 11 }

\texttt{\qquad{}\qquad{}CheckValue <- 11 - CalcModulus }

\texttt{\qquad{}\qquad{}IF CheckValue = 11 THEN }

\texttt{\qquad{}\qquad{}\qquad{}RETURN STRING(O) }

\texttt{\qquad{}\qquad{}ELSE }

\texttt{\qquad{}\qquad{}\qquad{}IF CheckValue = 10 THEN }

\texttt{\qquad{}\qquad{}\qquad{}....................... B .......................... }

\texttt{\qquad{}\qquad{}\qquad{}ELSE }

\texttt{\qquad{}\qquad{}\qquad{}RETURN STRING(CheckValue) }

\texttt{\qquad{}\qquad{}\qquad{}ENDIF }

\texttt{\qquad{}\qquad{}ENDIF }

\texttt{\qquad{}ENDIF }

\texttt{\qquad{}IF LENGTH(Number) = 9 THEN }

\texttt{\qquad{}\qquad{}RETURN .................... C .................... }

\texttt{\qquad{}ELSE }

\texttt{\qquad{}\qquad{}RETURN CheckDigit; }

\texttt{\qquad{}ENDIF }

\texttt{END FUNCTION}

\bigskip{}

\texttt{// Calculate ISBN, an example of how the function is called. }

\texttt{// Second parameter is always 0. }

\texttt{ISBN = CalCheckDigit(\textquotedbl 184146208\textquotedbl ,0)}%
\end{minipage}

\subsubsection*{Task 2.2}

Write a program to implement the ISBN program using the \texttt{CalCheckDigit}
function.

The program will:
\begin{itemize}
\item read the entire contents of the file \texttt{ISBNPRE.TXT} (seven 9-digit
lSBNs without check digits) into an appropriate data structure
\item use the function \texttt{CalCheckDigit} to calculate the result (ISBN
with check digit) for each entry in the file
\item write each result (ISBN with check digit) to the screen.
\end{itemize}

\subsubsection*{Evidence 4}

Your program code for Task 2.2. \hfill{}{[}11{]}

\subsubsection*{Evidence 5}

Screenshot of the results of processing the \texttt{ISBNPRE.TXT} file.\hfill{}{[}1{]}