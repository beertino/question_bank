\item \textbf{{[}ALVL/9597/2019/P2/Q5{]} }

The function \texttt{z} takes three integer parameters, \texttt{low},
\texttt{high}, \texttt{seek} and returns an integer value. It operates
on the values in the elements of the array \texttt{A}. 

\noindent %
\noindent\begin{minipage}[t]{1\columnwidth}%
\texttt{01 FUNCTION Z(low, high, seek, A) RETURNS INTEGER }

\texttt{02 \qquad{}IF low > high THEN }

\texttt{03 \qquad{}\qquad{}RETURN \textemdash 1 }

\texttt{04 \qquad{}ENDIF }

\texttt{05 \qquad{}mid <- low + INT( (high \textemdash{} low) /2)}

\texttt{06 \qquad{}IF seek = A{[}mid{]} THEN}

\texttt{07 \qquad{}\qquad{}RETURN mid }

\texttt{08 \qquad{}ELSE }

\texttt{09 \qquad{}\qquad{}IF seek < A{[}mid{]} THEN }

\texttt{10 \qquad{}\qquad{}\qquad{}RETURN Z(low, mid - 1, seek,
A) }

\texttt{11 \qquad{}\qquad{}ELSE }

\texttt{12 \qquad{}\qquad{}\qquad{}RETURN Z(mid + 1, high, seek,
A) . }

\texttt{13 \qquad{}\qquad{}ENDIF }

\texttt{14 \qquad{}ENDIF }

\texttt{15 ENDFUNCTION}%
\end{minipage}
\begin{enumerate}
\item {}
\begin{enumerate}
\item State what lines \texttt{10} and \texttt{12} tell you about the function.
\hfill{}{[}1{]}
\item State the purpose for the \texttt{RETURN} statements in lines \texttt{03}
and \texttt{07} of function \texttt{z}. \hfill{} {[}1{]}
\end{enumerate}
\end{enumerate}
The values in each of the eight elements of the array \texttt{A} are:
\begin{center}
\begin{tabular}{|l|c|c|c|c|c|c|c|c|}
\multicolumn{1}{l}{\textbf{Element}} & \multicolumn{1}{c}{0} & \multicolumn{1}{c}{1} & \multicolumn{1}{c}{2} & \multicolumn{1}{c}{3} & \multicolumn{1}{c}{4} & \multicolumn{1}{c}{5} & \multicolumn{1}{c}{6} & \multicolumn{1}{c}{7}\tabularnewline
\hline 
\textbf{Value} & -3 & 8 & 14 & 15 & 96 & 101 & 412 & 500\tabularnewline
\hline 
\end{tabular}
\par\end{center}
\begin{enumerate}
\item Copy and then complete the trace table for the instruction: 

\texttt{OUTPUT z(0, 7, 103, A)}
\begin{center}
\begin{tabular}{|c|c|c|c|c|c|c|}
\hline 
Function call & \texttt{\textbf{low}} & \texttt{\textbf{high}} & \texttt{\textbf{seek}} & \texttt{\textbf{mid}} & \texttt{\textbf{A{[}mid{]}}} & \texttt{\textbf{OUTPUT}}\tabularnewline
\hline 
\texttt{1} & \texttt{0} & \texttt{7} & \texttt{103} &  &  & \tabularnewline
\hline 
 &  &  &  &  &  & \tabularnewline
\hline 
 &  &  &  &  &  & \tabularnewline
\hline 
 &  &  &  &  &  & \tabularnewline
\hline 
 &  &  &  &  &  & \tabularnewline
\hline 
 &  &  &  &  &  & \tabularnewline
\hline 
\end{tabular}
\par\end{center}

\hfill{}{[}4{]}
\item Function \texttt{z} can return two different types of value.

Explain what these represent. \hfill{}{[}2{]}
\item The number of elements in array \texttt{A} may be very large. 

Explain why a programmer might prefer to use an iterative approach
rather than the one used in function \texttt{z}. \hfill{} {[}2{]}
\end{enumerate}