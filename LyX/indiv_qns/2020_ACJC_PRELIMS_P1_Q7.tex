\quad{} 
\item \textbf{{[}ACJC/PRELIM/9569/2021/P1/Q7{]} }
\begin{enumerate}
\item The following is the pseudocode for an in-place quicksort algorithm
for sorting in ascending order.

\noindent %
\noindent\begin{minipage}[t]{1\columnwidth}%
\texttt{01\qquad{}FUNCTION Partition(L, R : INTEGERS, MyList : LIST)
RETURNS INTEGER }

\texttt{02\qquad{}\qquad{}Pivot \textleftarrow{} MyList{[}R{]} }

\texttt{03\qquad{}\qquad{}i \textleftarrow{} L }

\texttt{04\qquad{}\qquad{}j \textleftarrow{} L }

\texttt{05\qquad{}\qquad{}REPEAT }

\texttt{06\qquad{}\qquad{}\qquad{}IF MyList{[}j{]} > Pivot }

\texttt{07\qquad{}\qquad{}\qquad{}\qquad{}THEN }

\texttt{08\qquad{}\qquad{}\qquad{}\qquad{}\qquad{}}\texttt{\textbf{A}}\texttt{ }

\texttt{09\qquad{}\qquad{}\qquad{}\qquad{}ELSE }

\texttt{10\qquad{}\qquad{}\qquad{}\qquad{}Temp \textleftarrow{}
MyList{[}j{]} 11 MyList{[}j{]} \textleftarrow{} MyList{[}i{]} }

\texttt{12\qquad{}\qquad{}\qquad{}\qquad{}}\texttt{\textbf{B}}\texttt{ }

\texttt{13\qquad{}\qquad{}\qquad{}ENDIF }

\texttt{14\qquad{}\qquad{}UNTIL j = R }

\texttt{15\qquad{}\qquad{}MyList{[}R{]} \textleftarrow{} MyList{[}i{]}
// swap elements with index i and R }

\texttt{16\qquad{}\qquad{}MyList{[}i{]} \textleftarrow{} Pivot }

\texttt{17\qquad{}\qquad{}}\texttt{\textbf{C}}\texttt{ }

\texttt{18\qquad{}ENDFUNCTION }

\texttt{19\qquad{}PROCEDURE Quicksort(L, R : INTEGERS, MyList : LIST) }

\texttt{20\qquad{}\qquad{}IF }\texttt{\textbf{D}}\texttt{ }

\texttt{21\qquad{}\qquad{}\qquad{}THEN }

\texttt{22\qquad{}\qquad{}\qquad{}\qquad{}PivotPos = Partition(R,
L, MyList) }

\texttt{23\qquad{}\qquad{}\qquad{}\qquad{}CALL Quicksort(L, PivotPos
- 1, MyList) }

\texttt{24\qquad{}\qquad{}\qquad{}\qquad{}}\texttt{\textbf{E}}\texttt{ }

\texttt{25\qquad{}\qquad{}ENDIF }

\texttt{26\qquad{}ENDPROCEDURE}%
\end{minipage}
\begin{enumerate}
\item Write pseudo-code to replace \texttt{\textbf{A}}, \texttt{\textbf{B}},
\texttt{\textbf{C}}, \texttt{\textbf{D}} and \texttt{\textbf{E}} in
the above algorithm. \hfill{}{[}5{]}
\item State the time complexity of the algorithm in the above pseudo-code.
\hfill{} {[}1{]}
\item State and explain when the worst case scenario (for running time)
for quicksort arises in the above algorithm. \hfill{}{[}3{]}
\item Another programmer suggested insertion sort would be more efficient
during the worst case scenario in (iii).

State and explain if insertion sort is indeed more efficient in this
instance. \hfill{}{[}2{]}
\end{enumerate}
\item A program needs to store an array of names and scores in a two dimensional
array and perform the following:
\begin{itemize}
\item Output the names and scores in alphabetical order. 
\item Check for the presence of a particular name
\end{itemize}
The data to be stored in the array is as follows:

Peter 68 

Mary 70 

Kelvin 48 

Casper 44 

Luther 76
\begin{enumerate}
\item Draw a flowchart to represent a linear search algorithm that returns
the score of a particular name. \hfill{} {[}4{]}
\item Instead of storing the data in an array, it is suggested that the
names could be stored in a hash table instead. 

With reference to the requirements of the program, suggest one advantage
and one disadvantage of storing the names in a hash table instead
of an array. \hfill{} {[}2{]}
\item (iii) An array can be used to create the hash table data structure.
Describe the process of inserting the above data into a hash table.
You may assume there will be no collisions. \hfill{} {[}3{]}
\end{enumerate}
\end{enumerate}