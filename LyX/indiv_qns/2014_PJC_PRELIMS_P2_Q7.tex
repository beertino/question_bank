\item \textbf{{[}PJC/PRELIM/9597/2014/P2/Q7{]} }

Pioneer Dental Group runs a number of clinics and requires its dentists
to use forms, such as the ones shown (on the next page), to keep a
record of treatments given to patients. Each patient has a number,
name, and a category (for example, adult, child, student, senior citizen,
etc.). A patient can received many treatments on the same day, but
the same treatment is not administered twice on the same day. 
\noindent \begin{center}
\begin{tabular}{|c|c|c|}
\hline 
\multicolumn{3}{|c|}{\texttt{\textbf{DENTIST RECORD FORM}}}\tabularnewline
\hline 
\multicolumn{3}{|l|}{\texttt{\textbf{Patient Number}}\texttt{: P102 }}\tabularnewline
\multicolumn{3}{|l|}{\texttt{\textbf{Patient Name}}\texttt{: Yap Kim Meng }}\tabularnewline
\multicolumn{3}{|l|}{\texttt{\textbf{Patient Category Number}}\texttt{: 1 }}\tabularnewline
\multicolumn{3}{|l|}{\texttt{\textbf{Patient Category Description}}\texttt{: Adult}}\tabularnewline
\hline 
\texttt{\textbf{Appointment Date}} & \texttt{\textbf{Treatment ID}} & \texttt{\textbf{Treatment Description}}\tabularnewline
\hline 
\texttt{13-Aug-2013 } & \texttt{T05 } & \texttt{Root canal}\tabularnewline
\hline 
\texttt{13-Aug-2013} & \texttt{T03 } & \texttt{Extraction }\tabularnewline
\hline 
\texttt{21-Oct-2013 } & \texttt{T03 } & \texttt{Extraction }\tabularnewline
\hline 
\end{tabular}
\par\end{center}

\noindent \begin{center}
\begin{tabular}{|c|c|c|}
\hline 
\multicolumn{3}{|c|}{\texttt{\textbf{DENTIST RECORD FORM}}}\tabularnewline
\hline 
\multicolumn{3}{|l|}{\texttt{\textbf{Patient Number}}\texttt{: P104}}\tabularnewline
\multicolumn{3}{|l|}{\texttt{\textbf{Patient Name}}\texttt{: Christopher Thomas}}\tabularnewline
\multicolumn{3}{|l|}{\texttt{\textbf{Patient Category Number}}\texttt{: 2}}\tabularnewline
\multicolumn{3}{|l|}{\texttt{\textbf{Patient Category Description}}\texttt{: Child }}\tabularnewline
\hline 
\texttt{\textbf{Appointment Date}} & \texttt{\textbf{Treatment ID}} & \texttt{\textbf{Treatment Description}}\tabularnewline
\hline 
\texttt{14-Aug-2013} & \texttt{T01} & \texttt{Scale and polish}\tabularnewline
\hline 
\texttt{14-Aug-2013} & \texttt{T02} & \texttt{Fillings}\tabularnewline
\hline 
\texttt{02-Sep-2013} & \texttt{T03} & \texttt{Extraction}\tabularnewline
\hline 
\end{tabular}
\par\end{center}
\begin{enumerate}
\item Explain why inconsistencies may occur as a result of operations to
update and delete records. \hfill{}{[}4{]}
\item Pioneer Dental Group would like to use a relational database to hold
these data and needs to normalise the entities. Explain (i) what entities
are, {[}1{]} (ii) what it means by normalisation. \hfill{}{[}2{]}
\item Show using standard notation, the entities in the database after normalisation.
For each of the entities, identify the primary key(s). \hfill{} {[}6{]}
\item Before using a relational database, the dental group used a series
of application programs that perform services for the end-users, such
as to produce reports for dentists and for the individual clinics.

Discuss \textbf{three} disadvantages of this approach. \hfill{} {[}3{]}
\end{enumerate}