\item \textbf{{[}ALVL/9597/2017/P1/Q1{]} }

A role-playing computer game includes a list of items called the inventory.
This inventory can be represented using a one-dimensional (1 -D) array
or a list structure. 

\texttt{INVENTORY.TXT} is a text file containing the items from the
computer game inventory. Each item type can have many occurrences.
For example: 

\begin{tabular}{lll}
\textbf{Inventory} &  & \textbf{ItemType}\tabularnewline
Iron Ore &  & Iron Ore\tabularnewline
Stone &  & Stone\tabularnewline
Sticky Piston &  & Sticky Piston\tabularnewline
Glass &  & Glass \tabularnewline
Stone &  & Sand \tabularnewline
Stone &  & \tabularnewline
Sand &  & \tabularnewline
Sticky Piston &  & \tabularnewline
Iron Ore &  & \tabularnewline
\end{tabular}

\subsubsection*{Task 1.1}

Design and write program code to: 
\begin{itemize}
\item read the entire contents of \texttt{INVENTORY.TXT} to an appropriate
data structure called Inventory 
\item find each item type in this inventory and write these into a second
similar data structure called \texttt{ItemTypes} 
\item count the number of each item type in the inventory and store this
in a third similar data structure called \texttt{ItemCounts }
\item display the contents of the ItemTypes and ItemCounts data structures
using the format given below. 
\end{itemize}
Example run of the program: 

\begin{tabular}{lllll}
\textbf{Inventory} &  & \multicolumn{3}{l}{The output generated from this input would be:}\tabularnewline
Iron Ore &  &  &  & \tabularnewline
Stone &  & \textbf{ItemType} &  & \textbf{Count}\tabularnewline
Sticky Piston &  &  &  & \tabularnewline
Glass &  & Iron Ore &  & 2\tabularnewline
Stone &  & Stone &  & 3\tabularnewline
Stone &  & Sticky Piston &  & 2\tabularnewline
Sand &  & Glass &  & 1\tabularnewline
Sticky Piston &  & Sand &  & 1\tabularnewline
Iron Ore &  &  &  & \tabularnewline
\end{tabular}

\subsubsection*{Evidence 1}

Your program code.\hfill{}{[}14{]}

\subsubsection*{Evidence 2}

Screenshot of your output. \hfill{}{[}1{]}