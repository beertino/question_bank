\item \textbf{{[}ACJC/PRELIM/9569/2021/P1/Q6{]} }

In a computer game, players\textquoteright{} names and scores are
stored in a binary search tree, in increasing order of score.

The binary search tree has its data inserted in the following order:

Ryan 18 

Bella 25 

Joshua 27 

Shane 20 

Jasmine 17 

Alexis 21 

Leslie 15
\begin{enumerate}
\item Draw the binary search tree. \hfill{}{[}4{]}
\item The binary search tree is implemented using the two dimensional array
shown below. Copy and fill in the entries in the array.
\noindent \begin{center}
\begin{tabular}{|c|c|c|c|c|}
\hline 
Index & Name & Score & Left Pointer & Right Pointer\tabularnewline
\hline 
0 &  &  &  & \tabularnewline
\hline 
1 &  &  &  & \tabularnewline
\hline 
2 &  &  &  & \tabularnewline
\hline 
3 &  &  &  & \tabularnewline
\hline 
4 &  &  &  & \tabularnewline
\hline 
5 &  &  &  & \tabularnewline
\hline 
6 &  &  &  & \tabularnewline
\hline 
\end{tabular}
\par\end{center}

\hfill{}{[}5{]}
\item To delete a node from a binary tree, the following cases are considered:
\noindent \begin{center}
\begin{tabular}{|l|l|}
\hline 
Case & Action\tabularnewline
\hline 
Node has no children & - Node is removed from tree\tabularnewline
\hline 
Node has one child & - Node is replaced with its child\tabularnewline
\hline 
\multirow{4}{*}{Node has two children} & - Call the node to be deleted $D$. Do not delete$D$\tabularnewline
\cline{2-2} 
 & - Look for the node $E$ that comes after $D$ in an in-order traversal\tabularnewline
\cline{2-2} 
 & - Copy the data $E$ into $D$.\tabularnewline
\cline{2-2} 
 & - Delete $E$ using one of the previous two cases.\tabularnewline
\hline 
\end{tabular}
\par\end{center}

Draw the tree at each step after the following players are deleted,
one after another:
\begin{enumerate}
\item Joshua \hfill{}{[}1{]}
\item Jasmine\hfill{} {[}1{]}
\item Ryan \hfill{}{[}2{]}
\end{enumerate}
\item The program has a feature which allows the user to enter an integer.
The program then returns a list of players whose score is greater
than that integer. Describe how the program can create this list using
the binary search tree. \hfill{}{[}4{]}
\end{enumerate}