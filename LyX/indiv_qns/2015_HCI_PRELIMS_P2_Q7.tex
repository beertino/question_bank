\item \textbf{{[}HCI/PRELIM/9597/2015/P2/Q7{]} }

Hash table has an index range of 1 to 400. The following pseudocode
describes an algorithm for searching the table using a hashing method.
It is assumed that the key is present in the table. 

\noindent\begin{minipage}[t]{1\columnwidth}%
\texttt{1. index = hash(key)}

\texttt{2. while table(index, 1) <> key }

\texttt{3. \qquad{}index = index + 1}

\texttt{4. endwhile }

\texttt{5. value = table(index, 2) }%
\end{minipage}
\begin{enumerate}
\item Explain the purpose of: 
\begin{enumerate}
\item line 1 
\item line 2 
\item line 3 
\item line 5 
\end{enumerate}
in this algorithm. \hfill{} {[}8{]}
\item The algorithm fails to handle the upper limit on the range of the
index. What modification to the algorithm is required to overcome
this problem? \hfill{} {[}2{]}
\end{enumerate}