\item \textbf{{[}DHS/PRELIM/9597/2014/P2/Q5{]} }

Pioneers who are eligible for the PGP must meet the following conditions: 
\begin{itemize}
\item still alive 
\item Aged 16 and above in 1965 
\item obtained citizenship on or before 31 December 1986 
\end{itemize}
Eligible pioneers enjoy the following benefits: 
\begin{itemize}
\item additional outpatient care subsidies 
\item Medisave account top-ups 
\item Medishield Life insurance subsidies and top-ups 
\end{itemize}
A panel will assess appeals for individuals who may have marginally
missed out on the PGP on a case-by-case basis. Citizens aged 55 and
above this year who do not qualify for the PGP will receive Medisave
account quantum top-ups for five years. 
\begin{enumerate}
\item Create a decision table showing all the possible conditions and actions.
\hfill{}{[}3{]}
\item Simplify your decision table by removing redundancies. \hfill{}{[}2{]}
\item Draw a program flowchart to determine if an individual is eligible
for the PGP and if not, if they qualify for an appeal or will receive
five-year quantum top-ups. \hfill{}{[}4{]}
\item Give 3 examples of test cases to test the age criteria for your algorithm
in (c). \hfill{}{[}3{]}
\item Given the date of birth in DD/MM/YYYY, write pseudocode to determine
if an individual is aged 16 and above in 1965. \hfill{}{[}3{]}
\end{enumerate}