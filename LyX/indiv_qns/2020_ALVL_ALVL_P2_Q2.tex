\begin{onehalfspace}
\item \textbf{{[}ALVL/9569/2020/P2/Q2{]} }
\end{onehalfspace}

\begin{onehalfspace}
\noindent Name your Jupyter Notebook as 

\noindent \texttt{TASK2\_<your name>\_<centre number>\_<index number>.ipynb}

\noindent The task is to:
\end{onehalfspace}
\begin{itemize}
\begin{onehalfspace}
\item generate a list of random integers
\item write this list to a file
\item read the list from the file
\item sort the list using a merge sort
\item write the sorted list to a second file.
\end{onehalfspace}
\end{itemize}
\begin{onehalfspace}
\noindent For each of the sub-tasks, add a comment statement, at the
beginning of the code using the hash symbol '\#' to indicate the sub-task
the program code belongs to, for example:

\noindent %
\begin{tabular}{l|>{\raggedright}p{0.77\textwidth}|}
\cline{2-2} 
\texttt{In{[}1{]}:} & \texttt{\emph{\# Task 2.1}}\tabularnewline
 & \texttt{\emph{Program code}}\tabularnewline
\cline{2-2} 
\multicolumn{1}{l}{} & \multicolumn{1}{>{\raggedright}p{0.77\textwidth}}{\texttt{Output :}}\tabularnewline
\end{tabular}
\end{onehalfspace}
\begin{onehalfspace}

\subsubsection*{Task 2.1}
\end{onehalfspace}

\begin{onehalfspace}
\noindent Write a function \texttt{task2\_1 (filename, quantity, maximum)}
that:
\end{onehalfspace}
\begin{itemize}
\begin{onehalfspace}
\item accepts three parameters:
\end{onehalfspace}
\begin{itemize}
\begin{onehalfspace}
\item \texttt{filename}, a string representing the name of a file
\item \texttt{quantity,} an integer representing the number of random integers
ot generate
\item \texttt{maximum}, representing the largest value that a random integer
can take
\end{onehalfspace}
\end{itemize}
\begin{onehalfspace}
\item generates \texttt{quantity} random numbers between 0 and \texttt{maximum}
(inclusive)
\item writes those values, on per line, to a file named \texttt{filename}.\hfill{}{[}4{]}
\end{onehalfspace}
\end{itemize}
\begin{onehalfspace}
\noindent Generate 1000 random numbers between 0 and 5000 (inclusive)
and save them to a file called 

\noindent \texttt{randomnumbers\_<your name>\_<centre number>\_<index
number>.txt}\hfill{}{[}2{]}
\end{onehalfspace}
\begin{onehalfspace}

\subsubsection*{Task 2.2}
\end{onehalfspace}

\begin{onehalfspace}
\noindent Write a function \texttt{task2\_2(list\_of\_integers)} that:
\end{onehalfspace}
\begin{itemize}
\begin{onehalfspace}
\item takes a list of integers, \texttt{list\_of\_integers}
\item sorts them into ascending order using merge sort
\item returns to the sorted list\hfill{}{[}7{]}
\end{onehalfspace}
\end{itemize}
\begin{onehalfspace}
\noindent Use the list\texttt{ {[}56,25,4,98,0,18,4,5,7,0{]}} to test
your function.

\noindent For example, the condition

\noindent \texttt{task2\_2({[}56,25,4,98,0,18,4,5,7,0{]})=={[}0,0,4,4,5,7,18,25,56,98{]}}

\noindent should return \texttt{True}.\hfill{}{[}2{]}
\end{onehalfspace}
\begin{onehalfspace}

\subsubsection*{Task 2.3}
\end{onehalfspace}

\begin{onehalfspace}
\noindent Write a function \texttt{task2\_3(filename\_in, filename\_out)}
that:
\end{onehalfspace}
\begin{itemize}
\begin{onehalfspace}
\item accepts two parameters:
\end{onehalfspace}
\begin{itemize}
\begin{onehalfspace}
\item \texttt{filename\_in} represents the input file name
\item \texttt{filename\_out} represents the output file name
\end{onehalfspace}
\end{itemize}
\begin{onehalfspace}
\item reads the integers from the input file
\item uses your \texttt{task2\_2} function to sort the integers
\item writes the integers to the output file.\hfill{}{[}5{]}
\end{onehalfspace}
\end{itemize}
\begin{onehalfspace}
\noindent The function should read the random numbers from

\noindent \texttt{randomnumbers\_<your name>\_<centre number>\_<index
number>.txt}

\noindent and then write the sorted integers to

\noindent \texttt{sortednumbers\_<your name>\_<centre number>\_<index
number>.txt}\hfill{}{[}3{]}

\noindent Save your Jupyter Notebook for Task 2.
\end{onehalfspace}