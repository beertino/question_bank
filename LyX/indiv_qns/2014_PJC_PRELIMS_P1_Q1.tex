\item \textbf{{[}PJC/PRELIM/9597/2014/P1/Q1{]} }

At the Commonwealth Games, the timings for the heats of 100m race
are recorded in a file \texttt{RACE.csv}. 

Each record has the following format: 
\noindent \begin{center}
\texttt{<runnerID>,<country code>,<name of runner>,<race time> }
\par\end{center}

A sample record is: 
\noindent \begin{center}
\texttt{2225,SIN,Kang,10.77} 
\par\end{center}

\subsection*{Task 1.1 }

Write program code to find and output the \textbf{\emph{number of
runners}} who recorded a timing of more than 11 seconds, and \textbf{\emph{list
these runners}} on the screen along with their full records under
this heading: 
\noindent \begin{center}
\texttt{}%
\begin{tabular}{|cccc|}
\hline 
\texttt{Runner ID } & \texttt{Country} & \texttt{Name } & \texttt{Race Time}\tabularnewline
\hline 
\end{tabular}
\par\end{center}

\subsection*{Evidence 1: }

Your program code for task 1.1. \hfill{}{[}6{]}

\subsection*{Evidence 2: }

Screenshot of output. \hfill{}{[}1{]}

\subsection*{Task 1.2 }

Write program code to display the top 10 runners in order of race
time. Runners with the same race time will have the same rank. The
fastest runner will be displayed first, under this heading: Runner
ID 
\noindent \begin{center}
\texttt{}%
\begin{tabular}{|cccc|}
\hline 
\texttt{Runner ID} & \texttt{Country} & \texttt{Name} & \texttt{Race Time}\tabularnewline
\hline 
\end{tabular}
\par\end{center}

\subsection*{Evidence 3: }

Your program code for task 1.2. \hfill{}{[}7{]}

\subsection*{Evidence 4: }

Screenshot of output. \hfill{}{[}1{]}