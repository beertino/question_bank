\item \textbf{{[}DHS/PRELIM/9569/2020/P1/Q1{]} }

Given an array of numbers \texttt{A}, count the minimum number of
'bubble sort' swaps (swap between pair of consecutive items) needed
to sort the array in ascending order. 

For example, if \texttt{A = {[}3, 2, 1, 4{]}}, we need 3 'bubble sort'
swaps to sort A in ascending order i.e. 

\texttt{swap(3, 2)} to get \texttt{{[}}\texttt{\uline{2, 3}}\texttt{,
1, 4{]} }

\texttt{swap(3, 1)} to get \texttt{{[}2, }\texttt{\uline{1, 3}}\texttt{,
4{]}} 

\texttt{swap(1, 2)} to get \texttt{{[}}\texttt{\uline{1, 2}}\texttt{,
3, 4{]}} 
\begin{enumerate}
\item Devise an $O(n^{2})$ algorithm using bubble sort to count the number
of 'bubble sort' swaps. 
\item Devise an $O(n\log n)$ algorithm to count the number of 'bubble sort'
swaps. 
\end{enumerate}
You should also explain why each algorithm has its efficiency. \hfill{}{[}8{]}
\begin{enumerate}
\item[(c)] Would you expect insertion sort to perform better or worse than bubble
sort in (a)? Explain your answer with respect to the number of comparisons
needed using the above example.\hfill{} {[}5{]}
\item[(d)] Why is quick sort not a suitable algorithm in part (b)? Illustrate
your answer with a suitable example. \hfill{}{[}5{]}
\end{enumerate}