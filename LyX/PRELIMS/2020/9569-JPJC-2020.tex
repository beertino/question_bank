%% LyX 2.3.6.1 created this file.  For more info, see http://www.lyx.org/.
%% Do not edit unless you really know what you are doing.
\documentclass[english]{article}
\usepackage[T1]{fontenc}
\usepackage[latin9]{inputenc}
\usepackage{geometry}
\geometry{verbose,tmargin=2.5cm,bmargin=2.5cm,lmargin=2.5cm,rmargin=2.5cm}
\usepackage{amsmath}
\PassOptionsToPackage{normalem}{ulem}
\usepackage{ulem}

\makeatletter

%%%%%%%%%%%%%%%%%%%%%%%%%%%%%% LyX specific LaTeX commands.
%% Because html converters don't know tabularnewline
\providecommand{\tabularnewline}{\\}

\makeatother

\usepackage{babel}
\begin{document}
{[}SPLIT\_HERE{]}
\begin{enumerate}
\item \textbf{{[}JPJC/PRELIM/9569/2020/P1/Q1{]} }

A one-dimensional array \texttt{X} will be used to record the timings
of the participating teams racing in a 200 metres dragon boat race
event. Five dragon boat teams will compete in the event, and the timing
(in seconds) of each team will be captured as records in \texttt{X}. 
\noindent \begin{center}
\begin{tabular}{|c|c|c|c|c|c|}
\hline 
Index & 1  & 2 & 3  & 4  & 5\tabularnewline
\hline 
\hline 
Timing, seconds & 58.61  & 49.01  & 48.54  & 59.32  & 49.78\tabularnewline
\hline 
\end{tabular} 
\par\end{center}

A segment of the pseudocode to perform bubble sort is given below. 

\noindent\begin{minipage}[t]{1\columnwidth}%
\texttt{Line 1: \qquad{}flag <- TRUE }

\texttt{Line 2: \qquad{}WHILE flag = TRUE DO }

\texttt{Line 3: \qquad{}\qquad{}flag <- FALSE }

\texttt{Line 4: \qquad{}\qquad{}FOR i = 1 to N //N is the size of
array X }

\texttt{Line 5: \qquad{}\qquad{}\qquad{}IF X{[}i{]} > X{[}i+1{]} }

\texttt{Line 6: \qquad{}\qquad{}\qquad{}\qquad{}THEN }

\texttt{Line 7: \qquad{}\qquad{}\qquad{}\qquad{}\qquad{}SWAP(X{[}i{]},
X{[}i+1{]})//swaps value of items }

\texttt{Line 8: \qquad{}\qquad{}\qquad{}\qquad{}\qquad{}flag
<- TRUE }

\texttt{Line 9: \qquad{}\qquad{}\qquad{}ENDIF }

\texttt{Line 10: \qquad{}\qquad{}NEXT i }

\texttt{Line 11: \qquad{}ENDWHILE}%
\end{minipage} An error is detected in the pseudocode above. 
\begin{enumerate}
\item Identify the error by stating the line number, and the type of error.\hfill{}
{[}2{]}
\item \textbf{Without} changing the order and the types of constructs used,
rectify the error in \textbf{(a)}. \hfill{}{[}1{]}
\item Using the race timings of the dragon boat event given above, use a
trace table to illustrate that the amended algorithm works. \hfill{}{[}3{]}
\item Describe the worst case scenario, and state the worst case time complexity
for performing the bubble sort using the algorithm given above. \hfill{}{[}2{]}
\item In the worst case scenario, state the total number of comparisons
made by the bubble sort algorithm if 10 lanes are used. \hfill{}{[}1{]}
\end{enumerate}
{[}SPLIT\_HERE{]}
\item \textbf{{[}JPJC/PRELIM/9569/2020/P1/Q2{]} }

Every sales transaction made in JPJC supermarket is stored as a record
in a serial file for auditing purposes. At the end of each day, a
copy of the daily serial file, sorted by transaction amount in descending
order will be archived into the main database. 
\begin{enumerate}
\item State what is meant by a serial file? \hfill{}{[}1{]}
\item Explain how an archive file is different from a backup file, and describe
how a backup file for sales transactions can be created for JPJC supermarket.\hfill{}
{[}2{]}
\end{enumerate}
Merge sort algorithm is used to arrange the sales transaction records
by ordering them in descending order of transaction amounts. The algorithm
will first read all the daily unsorted sales transactions into \texttt{A},
a fixed size array of records with index starting from 1. Then, \texttt{mergesort}
will be applied to sort \texttt{A}. The pseudocode for \texttt{mergesort}
is given below, 

\noindent\begin{minipage}[t]{1\columnwidth}%
\texttt{PROCEDURE mergesort(A: ARRAY of RECORDS, x, y: INTEGERS)}

\texttt{\qquad{}IF x < y }

\texttt{\qquad{}\qquad{}THEN }

\texttt{\qquad{}\qquad{}\qquad{}m x + ((y - x) DIV 2) //DIV performs
integer division }

\texttt{\qquad{}\qquad{}\qquad{}mergesort(A, x, m) }

\texttt{\qquad{}\qquad{}\qquad{}mergesort(A, m + 1, y) }

\texttt{\qquad{}\qquad{}\qquad{}merge(A, x, m, y) }

\texttt{\qquad{}ENDIF }

\texttt{ENDPROCEDURE }

\texttt{//-{}-{}-main program-{}-{}- }

\texttt{mergesort(A, x, y) }%
\end{minipage}

Given that array \texttt{A = {[}39.10, 17.50, 35.40, 42.68, 8.90,
35.40{]}}, and \texttt{merge(A, x, m, y)} will sort and combine elements
of \texttt{A{[}x:m{]}}, and \texttt{A{[}m+1:y{]}} into \texttt{A{[}x:y{]}}
in descending order. 
\begin{enumerate}
\item[(c)]  State the values of \texttt{x}, and \texttt{y} when \texttt{mergesort}
is called in the main program.\hfill{} {[}2{]}
\item[(d)]  State the total number of times \texttt{mergesort} and \texttt{merge}
are called in the entire program.\hfill{} {[}2{]}
\end{enumerate}
The following diagram shows an incomplete trace tree diagram of the
array of sales transaction records represented by its sales amount. 
\begin{enumerate}
\item[(e)]  Draw and complete the trace tree diagram above by applying merge
sort to the unsorted array of records A. \hfill{}{[}4{]}
\item[(f)]  The time complexity for merge sort is $O(N\log_{2}N)$. Explain
why this time complexity is applicable to the best, average and worst
case scenarios.\hfill{} {[}1{]}
\end{enumerate}
{[}SPLIT\_HERE{]}
\item \textbf{{[}JPJC/PRELIM/9569/2020/P1/Q3{]} }
\begin{enumerate}
\item State what is meant by a recursive algorithm? \hfill{}{[}2{]}
\item Explain the difference between an iterative algorithm and a recursive
algorithm. \hfill{}{[}2{]}
\item Design a recursive algorithm \texttt{SumOfCubes(n)} using pseudocode,
that returns the integer value of series 
\[
1^{3}+2^{3}+3^{3}+\ldots+(n-1)^{3}+n^{3},
\]
where $n=1,2,3,\ldots$ \hfill{}{[}3{]}
\item Explain what will happen when the value of \texttt{n }gets too large.
\hfill{}{[}1{]}
\end{enumerate}
{[}SPLIT\_HERE{]}
\item \textbf{{[}JPJC/PRELIM/9569/2020/P1/Q4{]} }

The elections department of a town wishes to store the records of
its voters in a linked list. The stored records are first ordered
by the voter\textquoteright s age (in years), followed by voter\textquoteright s
name in alphabetical order. The voters list is maintained with the
record of the youngest voter at the start of the list. 
\begin{enumerate}
\item Explain why the sequence of nodes in a linked list does not always
reflect how the data is stored in the memory of the computer. {[}2{]}
\quad{} 

\begin{tabular}{c|c|c|c|}
\multicolumn{1}{c}{} & \multicolumn{1}{c}{\textbf{Age}} & \multicolumn{1}{c}{\textbf{Name}} & \multicolumn{1}{c}{\textbf{Link}}\tabularnewline
\cline{2-4} \cline{3-4} \cline{4-4} 
\textbf{1} & \texttt{35} & \texttt{Tim Tan} & \texttt{3}\tabularnewline
\cline{2-4} \cline{3-4} \cline{4-4} 
\textbf{2} & \texttt{23} & \texttt{Annie Hao} & \texttt{a}\tabularnewline
\cline{2-4} \cline{3-4} \cline{4-4} 
\textbf{3} & \texttt{45} & \texttt{Bob Boon} & \texttt{6}\tabularnewline
\cline{2-4} \cline{3-4} \cline{4-4} 
\textbf{4} & \texttt{24} & \texttt{Lester Moh} & \texttt{b}\tabularnewline
\cline{2-4} \cline{3-4} \cline{4-4} 
\textbf{5} & \texttt{18} & \texttt{Ari Bello} & \texttt{c}\tabularnewline
\cline{2-4} \cline{3-4} \cline{4-4} 
\textbf{6} & \texttt{52} & \texttt{Helen How} & \texttt{0}\tabularnewline
\cline{2-4} \cline{3-4} \cline{4-4} 
\textbf{7} & \texttt{23} & \texttt{Cindy Ku} & \texttt{d}\tabularnewline
\cline{2-4} \cline{3-4} \cline{4-4} 
\textbf{8} & \texttt{55} & \texttt{Charles Chu} & \texttt{1}\tabularnewline
\cline{2-4} \cline{3-4} \cline{4-4} 
\textbf{9} & \texttt{53} & \texttt{Mimi Lee} & \texttt{e}\tabularnewline
\cline{2-4} \cline{3-4} \cline{4-4} 
\textbf{10} & \texttt{40} & \texttt{Jenny Tsai} & \texttt{f}\tabularnewline
\cline{2-4} \cline{3-4} \cline{4-4} 
\end{tabular}%
\begin{tabular}{|c|c|}
\hline 
\textbf{Head} & \texttt{g}\tabularnewline
\hline 
\textbf{Free} & \texttt{8}\tabularnewline
\hline 
\end{tabular}
\end{enumerate}
Two linked lists are kept to manage the actual data, and the free
spaces. When a new item is added, a node is taken from the head of
the free space list, and when a node is deleted, the deleted node
will be returned to the tail of the free space list. 
\begin{enumerate}
\item[(b)]  Given that \texttt{Ari Bello} is the youngest voter, state the values
of \texttt{a}, \texttt{b}, \texttt{c}, \texttt{d}, \texttt{e}, \texttt{f},
and \texttt{g}. \hfill{}{[}4{]}
\item[(c)]  Draw the \textbf{linked list diagram} to show its state right after
each of the following successive operations: 
\begin{enumerate}
\item Insert \texttt{18} years old, \texttt{Ahmad Ali}. 
\item Delete \texttt{23} years old, \texttt{Cindy Ku}. 
\item Insert \texttt{37} years old, \texttt{Tania Tan}. \hfill{}{[}6{]}
\end{enumerate}
\item[(d)]  Describe \textbf{one} advantage and \textbf{one} disadvantage of
using a linked list over a static array. \hfill{}{[}2{]}
\end{enumerate}
{[}SPLIT\_HERE{]}
\item \textbf{{[}JPJC/PRELIM/9569/2020/P1/Q5{]} }

The stack is a first in last out data structure where the items are
inserted to and deleted from the top of the stack. The items of the
stack are globally stored in a fixed length array \texttt{S} of size
20. A stack pointer \texttt{sp} points to the top item in the stack,
and is initialised to 0. The three basic methods of the stack are:
\begin{itemize}
\item \texttt{PUSH(X) //inserts X as new item on the top of STACK S }
\item \texttt{POP() ~~//removes and returns item at the top of STACK S. }
\item \texttt{PEEK() ~//returns value of the item on top of STACK S without
removing it. }
\end{itemize}
\begin{enumerate}
\item Write the pseudocode for the algorithms \texttt{PUSH(X)}, \texttt{POP()},
and \texttt{PEEK()}.\hfill{}{[}5{]}
\end{enumerate}
The precedence order of the operators from highest to lowest is as
follows: 
\begin{enumerate}
\item[1.]  Parenthesis 
\item[2.] \texttt{ '\textasciicircum ' }
\item[3.]  \texttt{'{*}'} or \texttt{'/'} with equivalent level of priority 
\item[4.] \texttt{ '+'} or \texttt{'-'} with equivalent level of priority
\end{enumerate}
The pseudocode below shows a stack-based function \texttt{InfixToPostfix}
that converts and returns an input expression represented in infix
notation to its postfix form.

\noindent %
\noindent\begin{minipage}[t]{1\columnwidth}%
\texttt{FUNCTION InfixToPostfix(infix: STRING) RETURNS postfix }

\texttt{\qquad{}Scan through infix expression one token at a time
from leftmost. }

\texttt{\qquad{}Initialise empty STACK S }

\texttt{\qquad{}Initialise empty STRING postfix }

\texttt{FOR token read from infix item by item }

\texttt{\qquad{}\qquad{}CASE of token: }

\texttt{\qquad{}\qquad{}\qquad{}operand : postfix <- postfix +
token}

\texttt{\qquad{}\qquad{}\qquad{}'(' : PUSH(token) }

\texttt{\qquad{}\qquad{}\qquad{}')' : REPEAT postfix <- postfix
+ POP() UNTIL POP() = '(' operator:}

\texttt{\qquad{}\qquad{}\qquad{}\qquad{}WHILE S not empty }

\texttt{\qquad{}\qquad{}\qquad{}\qquad{}\qquad{}IF token = '('
THEN }

\texttt{\qquad{}\qquad{}\qquad{}\qquad{}\qquad{}\qquad{}BREAK }

\texttt{\qquad{}\qquad{}\qquad{}\qquad{}\qquad{}ENDIF }

\texttt{\qquad{}\qquad{}\qquad{}\qquad{}\qquad{}IF PEEK() is
higher or equal precedence than token THEN }

\texttt{\qquad{}\qquad{}\qquad{}\qquad{}\qquad{}\qquad{}postfix
<- postfix + POP() }

\texttt{\qquad{}\qquad{}\qquad{}\qquad{}\qquad{}ENDIF }

\texttt{\qquad{}\qquad{}\qquad{}\qquad{}ENDWHILE }

\texttt{\qquad{}\qquad{}\qquad{}\qquad{}PUSH(token) }

\texttt{\qquad{}\qquad{}END CASE }

\texttt{\qquad{}NEXT token }

\texttt{\qquad{}REPEAT }

\texttt{\qquad{}\qquad{}postfix <- postfix + POP() }

\texttt{\qquad{}UNTIL S is empty }

\texttt{\qquad{}RETURN postfix }

\texttt{ENDFUNCTION }%
\end{minipage}
\begin{enumerate}
\item[(b)]  Complete the trace table given below for \texttt{InfixToPostfix(\textquotedbl A/(B-C){*}D\textasciicircum E\textquotedbl )}. 
\begin{center}
\begin{tabular}{|c|c|c|c|}
\hline 
\texttt{token } & \texttt{Description } & \texttt{STRING postfix } & \texttt{Stack, S}\tabularnewline
\hline 
\hline 
\texttt{A} & \texttt{Appends to postfix } & \texttt{\textquotedbl A\textquotedbl{} } & \texttt{empty}\tabularnewline
\hline 
\texttt{/ } & \texttt{Push to S } & \texttt{\textquotedbl A\textquotedbl{} } & \texttt{/}\tabularnewline
\hline 
\texttt{( } & \texttt{Push to S } & \texttt{\textquotedbl A\textquotedbl{} } & \texttt{/,(}\tabularnewline
\hline 
\texttt{\dots{} } & \texttt{\dots \dots \dots \dots \dots{} } & \texttt{\dots \dots \dots \dots \dots{} } & \texttt{\dots \dots \dots \dots \dots{}}\tabularnewline
\hline 
\end{tabular}
\par\end{center}

\end{enumerate}
\hfill{}{[}4{]}
\begin{enumerate}
\item[(c)]  Show, with the aid of diagrams, how the computer uses a stack to
directly evaluate the value of the postfix expression \texttt{895-/12+{*}4-}.
\hfill{}{[}3{]}
\end{enumerate}
{[}SPLIT\_HERE{]}
\item \textbf{{[}JPJC/PRELIM/9569/2020/P1/Q6{]} }

Traversal was performed on the binary tree given below. 
\begin{enumerate}
\item List the nodes, in the order, that are visited for, 
\begin{enumerate}
\item in-order traversal, \hfill{}{[}1{]}
\item pre-order traversal, and \hfill{}{[}1{]}
\item post-order traversal.\hfill{}{[}1{]}
\end{enumerate}
\item A binary search tree is considered as an ordered binary tree where
the key values of nodes in the left sub-tree are less than the value
of its parent (root) node's key, and key values of nodes in the right
sub-tree are greater than the value of its parent (root) node\textquoteright s
key. 
\begin{enumerate}
\item Explain how a recursive algorithm can be used to locate a node with
key value \texttt{search\_key} by returning \texttt{TRUE} when \texttt{search\_key}
is found, and \texttt{FALSE} otherwise. \hfill{}{[}4{]}
\item State \textbf{one} advantage of using binary search tree as data structure
over linked list, and describe a situation that would negate this
advantage.\hfill{} {[}2{]}
\end{enumerate}
\end{enumerate}
{[}SPLIT\_HERE{]}
\item \textbf{{[}JPJC/PRELIM/9569/2020/P1/Q7{]} }

Car owners who wish to purchase or renew their insurance policy with
XYZ Motor Insurance are required to accumulate not more than 6 demerit
points in their driving records. Under this demerit points system,
a driver who clocks up more than 20 demerit points will have his/her
driving license revoked, thus denying the person from driving and
from purchasing any vehicle insurance. Drivers who have not made any
insurance claims for the past 3 years can get 2 demerit points off,
and current employees of XYZ Motor Insurance can get 1 demerit point
deducted. In addition, drivers awarded with certificate of merit can
get 1 demerit point off as well. XYZ Motor Insurance only allows drivers
to receive a maximum deduction of 3 demerit points per year. Draw
a decision table to reflect the eligibility of car owners who wish
to purchase or renew a car insurance policy with XYZ Motor Insurance.
\hfill{}{[}5{]}

{[}SPLIT\_HERE{]}
\item \textbf{{[}JPJC/PRELIM/9569/2020/P1/Q8{]} }

A company currently uses a computerised flat file to keep track of
the monthly claims submitted by its employee, and has decided to use
a relational database to store and manage the claims submitted by
the employees instead. The following table shows the details of the
computerised flat file.

\begin{tabular}{|c|c|c|c|c|c|}
\hline 
Claims ID  & Item Description  & Staff ID  & Staff Name  & Department  & Amount\tabularnewline
\hline 
\hline 
2818 & Phone charger & P212 & John Lee & Production & \$53.23\tabularnewline
\hline 
3291 & Car Transport & S281 & Chan, Molly & Sales & \$31.40\tabularnewline
\hline 
3998 & Meal, Lunch & O323  & Omar Hairi  & Operations & \$7.20\tabularnewline
\hline 
4820  & AAA Batteries  & E493  & Muthu K.  & Engineering  & \$10.17\tabularnewline
\hline 
6322  & Hard Drive 3TB  & A550  & Jervois F.  & Accounts  & \$27.99\tabularnewline
\hline 
7384  & Medical  & M438  & Zudin B Ali  & Marketing  & \$48.00\tabularnewline
\hline 
\dots .  & \dots .  & \dots .  & \dots .  & \dots . & \dots .\tabularnewline
\hline 
\end{tabular}
\begin{enumerate}
\item State and justify \textbf{one} reason made by the company to migrate
its claims information from the existing flat file system to a relational
database management system. \hfill{}{[}2{]}
\item State \textbf{two} other fields which would be useful for the company
to capture. \hfill{}{[}2{]}
\item Given that the every claim is associated with one item, write the
table descriptions of the relational database in \textbf{first normal
form} and \textbf{second normal form}. You are to include the fields
in \textbf{(b)}. \hfill{}{[}4{]}
\end{enumerate}
{[}SPLIT\_HERE{]}
\item \textbf{{[}JPJC/PRELIM/9569/2020/P1/Q9{]} }
\begin{enumerate}
\item What is the denary value of hexadecimal ABCD? \hfill{} {[}2{]}
\item An integer variable of size 4 bytes is used to keep track of the number
of commuters who travel to work from Jurong bus interchange. State
the maximum number of commuters this variable can keep track. \hfill{}{[}3{]}
\end{enumerate}
{[}SPLIT\_HERE{]}
\item \textbf{{[}JPJC/PRELIM/9569/2020/P1/Q10{]} }

A program written using object-oriented programming has \texttt{point},
\texttt{circle}, and \texttt{cone} as its defined classes. The following
diagram below shows the attributes and methods of the class \texttt{point}. 
\noindent \begin{center}
\texttt{}%
\begin{tabular}{|l|}
\hline 
\texttt{point }\tabularnewline
\hline 
\hline 
\texttt{Properties: }\tabularnewline
\hline 
\texttt{PROTECTED: }\tabularnewline
\texttt{x-value: REAL }\tabularnewline
\texttt{y-value: REAL }\tabularnewline
\hline 
\texttt{Methods: }\tabularnewline
\hline 
\texttt{PUBLIC: }\tabularnewline
\texttt{constructor() }\tabularnewline
\texttt{getCoordinates(): TUPLE }\tabularnewline
\texttt{setCoordinates(x, y: REAL)}\tabularnewline
\hline 
\end{tabular}
\par\end{center}
\begin{enumerate}
\item Draw an inheritance diagram for \textbf{all} the \textbf{three} classes
defined in the program. \hfill{}{[}4{]}
\item Explain the differences between 
\begin{enumerate}
\item private and protected attributes/ methods of a class, \hfill{}{[}2{]}
\item an object and a class. \hfill{}{[}2{]}
\end{enumerate}
\end{enumerate}
{[}SPLIT\_HERE{]}
\item \textbf{{[}JPJC/PRELIM/9569/2020/P1/Q11{]} }

A clinic manages patients\textquoteright{} medical and financial records
through an Internet-based information management portal. Due to several
security incidents related to unintended disclosure of patients\textquoteright{}
information, the clinic\textquoteright s management has decided to
migrate the portal to a local area network (LAN) that consists only
\textbf{four} computers and \textbf{one} printer. Information of patients\textquoteright{}
medical and financial records can only be accessed by authorised staff
on one of the four computers. 
\begin{enumerate}
\item Describe the meaning of the term local area network (LAN). \hfill{}{[}2{]}
\item Explain why ring networks today rarely use physical layout of a ring?
\hfill{}{[}2{]}
\item Describe the functions of a multi-station access unit used in a Token
Ring network. \hfill{}{[}3{]}
\item Describe how token passing enables a computer to send data to the
printer in this Token Ring network. \hfill{}{[}3{]}
\end{enumerate}
{[}SPLIT\_HERE{]}
\item \textbf{{[}JPJC/PRELIM/9569/2020/P2/Q1{]} }

Jurong Pioneer Primary School uses buses to transport students to
school. There are six bus routes labelled \texttt{A} to \texttt{F}.
A survey was conducted to analyse the punctuality statistics of these
buses over a four-week period.

The data from the survey is stored in the file \texttt{SURVEY.TXT}.
The format of the data in the file is: 
\noindent \begin{center}
\texttt{<Day>,<A>,<B>,<C>,<D>,<E>,<F>} 
\par\end{center}

Positive numbers represent minutes early, negative numbers represent
minutes late and 0 represents the bus having been on time. 

\subsection*{Task 1.1 }

Write the program code that: 
\begin{itemize}
\item reads the entire contents of \texttt{SURVEY.TXT} into an appropriate
data structure called \texttt{Records}, and 
\item displays the contents of \texttt{Records} in neat columns. \hfill{}{[}4{]}
\end{itemize}

\subsection*{Task 1.2 }

Extend your program so that the following statistics for the four-week
period may be calculated and output: 
\begin{itemize}
\item the number of late arrivals for each bus route,
\item the average number of minutes late for each bus route, using only
data from days on which it was late, and 
\item the bus route(s) with the highest number of days late. 
\end{itemize}
All the results should be displayed with appropriate annotation. The
following is an example run of the program: 
\noindent \begin{center}
<INSERT\_IMAGE\_HERE> \hfill{}{[}8{]}
\par\end{center}

\subsection*{Task 1.3}

Additional code is to be written for the user to input a specific
day, for example: \texttt{Fri3}, to be used for the analysis of data.
Find and display how many buses were late on this day and for each
late bus, display the route label and how late the bus was on this
day. 

Test your code using the following test data: 

\texttt{Tue1} 

\texttt{Thu2} 

Download your program code for Task 1 as 

\texttt{TASK1\_<your class>\_<your name>.ipynb} \hfill{}{[}4{]}

{[}SPLIT\_HERE{]}
\item \textbf{{[}JPJC/PRELIM/9569/2020/P2/Q2{]} }

In linear queue data structure, elements are inserted until the queue
becomes full. However, after the queue becomes full, new elements
cannot be inserted until all the existing elements are removed from
the queue. Although there are empty spaces in the queue, they remain
unused. This is a disadvantage of a linear queue. 

After inserting all the elements into a linear queue: 
\noindent \begin{center}
\begin{tabular}{|c|c|c|c|c|}
\hline 
\textquotedblleft John\textquotedblright{} & \textquotedblleft Amy\textquotedblright{}  & \textquotedblleft Chetan\textquotedblright{}  & \textquotedblleft Xin Xin\textquotedblright{}  & \textquotedblleft Evan\textquotedblright{}\tabularnewline
\hline 
\multicolumn{1}{c}{$\overset{\uparrow}{\text{Front}}$} & \multicolumn{1}{c}{} & \multicolumn{1}{c}{} & \multicolumn{1}{c}{} & \multicolumn{1}{c}{$\overset{\uparrow}{\text{Rear}}$}\tabularnewline
\end{tabular}
\par\end{center}

Linear queue is still considered full after elements have been dequeued: 
\noindent \begin{center}
\begin{tabular}{|c|c|c|c|c|}
\hline 
\textquotedblleft John\textquotedblright{} & \textquotedblleft Amy\textquotedblright{} & \textquotedblleft Chetan\textquotedblright{} & \textquotedblleft Xin Xin\textquotedblright{} & \textquotedblleft Evan\textquotedblright{}\tabularnewline
\hline 
\multicolumn{1}{c}{} & \multicolumn{1}{c}{} & \multicolumn{1}{c}{} & \multicolumn{1}{c}{$\overset{\uparrow}{\text{Front}}$} & \multicolumn{1}{c}{$\overset{\uparrow}{\text{Rear}}$}\tabularnewline
\end{tabular}
\par\end{center}

To overcome this disadvantage, a circular queue data structure may
be implemented. The next element added to the queue will be stored
at index 0 
\noindent \begin{center}
<INSERT\_IMAGE\_HERE> 
\par\end{center}

\begin{center}
\begin{tabular}{|l|l|l|}
\hline 
\multicolumn{3}{|c|}{\texttt{Queue}}\tabularnewline
\hline 
\multicolumn{3}{|c|}{Attributes}\tabularnewline
\hline 
\texttt{\hspace{0.01\columnwidth}}Identifier & \texttt{\hspace{0.01\columnwidth}}Data Type & \texttt{\hspace{0.05\columnwidth}}Description\tabularnewline
\hline 
\texttt{Items} & \texttt{ARRAY{[}0:4{]} OF STRING} & Stores the elements of queue\tabularnewline
\hline 
\texttt{Front} & \texttt{INTEGER} & Index of the first item added to the queue.\tabularnewline
\hline 
\texttt{Rear} & \texttt{INTEGER} & Index of the last item added to the queue.\tabularnewline
\hline 
\multicolumn{3}{|c|}{\texttt{Methods}}\tabularnewline
\hline 
Identifier & \multicolumn{2}{l|}{Description}\tabularnewline
\hline 
\texttt{Constructor()} & \multicolumn{2}{l|}{Instantiates a \texttt{Queue} object.}\tabularnewline
\hline 
\texttt{IsEmpty()} & \multicolumn{2}{l|}{Returns \texttt{TRUE} if the queue is empty and \texttt{FALSE} otherwise.}\tabularnewline
\hline 
\texttt{IsFull()} & \multicolumn{2}{l|}{Returns \texttt{TRUE} if the queue is full and \texttt{FALSE} otherwise.}\tabularnewline
\hline 
\texttt{Enqueue(STRING)} & \multicolumn{2}{l|}{Inserts a new item to the queue. Displays a suitable message if the
queue is full.}\tabularnewline
\hline 
\texttt{Dequeue():STRING} & \multicolumn{2}{l|}{Returns the item removed from the queue or \textquotedblleft \texttt{NONE}\textquotedblright{}
if the queue is empty.}\tabularnewline
\hline 
\texttt{Display()} & \multicolumn{2}{l|}{Outputs items from the front to the rear of the queue.}\tabularnewline
\hline 
\end{tabular}
\par\end{center}

\begin{center}
\begin{tabular}{|l|l||l|}
\hline 
\multicolumn{3}{|c|}{\texttt{CircularQueue}}\tabularnewline
\hline 
\multicolumn{3}{|c|}{\texttt{Methods}}\tabularnewline
\hline 
Identifier & \multicolumn{2}{l|}{Description}\tabularnewline
\hline 
\texttt{Constructor()} & \multicolumn{2}{l|}{Instantiates a \texttt{CircularQueue} object.}\tabularnewline
\hline 
\texttt{IsFull()} & \multicolumn{2}{l|}{Returns \texttt{TRUE} if the queue is full and \texttt{FALSE} otherwise.Overrides
the method in parent class.}\tabularnewline
\hline 
\texttt{Enqueue(STRING)} & \multicolumn{2}{l|}{Inserts a new item to the queue. Displays a suitable message if the
queue is full.Overrides the method in parent class.}\tabularnewline
\hline 
\texttt{Dequeue():STRING} & \multicolumn{2}{l|}{Returns the item removed from the queue or \textquotedblleft \texttt{NONE}\textquotedblright{}
if the queue is empty.Overrides the method in parent class.}\tabularnewline
\hline 
\texttt{Display()} & \multicolumn{2}{l|}{Outputs items from the front to the rear of the queue.Overrides the
method in parent class.}\tabularnewline
\hline 
\end{tabular}
\par\end{center}

\subsection*{Task 2.1}

Implement the classes \texttt{Queue} and \texttt{CircularQueue} with
object-oriented programming. The first item added to an empty queue
is stored at index 0. The attributes of each object is reinitialised
when the queue becomes empty.\hfill{} {[}20{]} 

\subsection*{Task 2.2 }

There are two printers in the General Office. One of the printers
implements a linear queue while the other implements a circular queue. 

Write the code to instantiate a \texttt{Queue} object and a \texttt{CircularQueue}
object. Test your code, on both queues, using the following steps: 
\begin{enumerate}
\item[i.]  Enqueue five users in the order given in the diagram. 
\item[ii.]  Dequeue twice.
\item[iii.]  Enqueue \textquotedblleft Mohan\textquotedblright .
\item[iv.]  Display the queue. 
\end{enumerate}
Download your program code for Task 2 as 

\texttt{TASK2\_<your class>\_<your name>.ipynb} \hfill{}{[}6{]}

{[}SPLIT\_HERE{]}
\item \textbf{{[}JPJC/PRELIM/9569/2020/P2/Q3{]} }

The library in Jurong Pioneer Primary School uses a hybrid data structure
to keep track of its inventory. Each record in the hash table stores
a simple list \texttt{{[}<Category>, <Binary Search Tree object>{]}}.
Each node in the Binary Search Tree (BST) stores the ISBN number and
title of a book. The nodes in each BST share the same book category
and are sorted, in ascending order, according to their ISBN numbers. 

\subsection*{Task 3.1 }

Write the object-oriented code for the \texttt{BST} and \texttt{BSTNode}
classes described above. \hfill{}{[}10{]}

A checksum is applied to determine the Hash Value for each \texttt{<Category>},
where the ASCII value of each character in the title is multiplied
by its position in the \texttt{<Category>} string (starting from left
to right), and then summed. 

For example, given the category \textquotedblleft classics\textquotedblright ,
the summed value would thus be: $99\times1+108\times2+97\times3+115\times4+115\times5+105\times6+99\times7+115\times8=3884$.

$(3884\mod19)+1=9$

A weighted modulus 19 operation is then applied and 1 is added to
the remainder to determine the final Hash Value. 

\subsection*{Task 3.2 }

Write the code for the function \texttt{CalcHash(my\_string)}, which
takes in a string argument and returns its resultant Hash Value. \hfill{}{[}4{]}

\subsection*{Task 3.3 }

Write the code to declare and initialise \texttt{HashTable}, an empty
hash table array that may store up to 19 records. {[}2{]} 

\subsection*{Task 3.4 }

\texttt{CATEGORIES.TXT} is a text file containing the book categories.
Read the entire contents of \texttt{CATEGORIES.TXT} and update the
records in the hash table. Collisions are handled using \textbf{linear
probing}. \hfill{}{[}4{]}

\texttt{BOOKS.TXT} holds the details of books in the library. The
format of the data in the file is: \texttt{<Category>},\texttt{ <ISBN>},\texttt{
<Title>}. 

\subsection*{Task 3.5 }

Write program code to: 
\begin{itemize}
\item read the lines from the file, 
\item extract the \texttt{<Category>}, \texttt{<ISBN>} and \texttt{<Title>}
values, and 
\item add each book to the BST of its category. \hfill{}{[}6{]}
\end{itemize}

\subsection*{Task 3.6 }

Write the code to display the ISBN and title of each book belonging
to \texttt{\textquotedblleft classics\textquotedblright{}} category.
The output is sorted according to the ISBN numbers. Ensure your output
uses headings to identify the data displayed. 

Download your program code for Task 3 as 

\texttt{TASK3\_<your class>\_<your name>.ipynb}\hfill{} {[}4{]}

{[}SPLIT\_HERE{]}
\item \textbf{{[}JPJC/PRELIM/9569/2020/P2/Q4{]} }

A computer company has several offices throughout Singapore, each
with several salespersons. Each salesperson is assigned to one office
only. A record of the sales made by each salesperson has been set
up using a relational database. 

The following tables hold the data: 

\texttt{CUSTOMER (}\texttt{\uline{CustomerID}}\texttt{, CustomerName,
Email, Telephone) }

\texttt{OFFICE (}\texttt{\uline{OfficeID}}\texttt{, PostalCode,
Telephone) }

\texttt{SALE (}\texttt{\uline{SalesPersonID{*}}}\texttt{, }\texttt{\uline{CustomerID{*}}}\texttt{,
}\texttt{\uline{SaleDate}}\texttt{, Amount) }

\texttt{SALESPERSON (}\texttt{\uline{SalesPersonID}}\texttt{, SalesPersonName,
OfficeID{*}) }

\textbf{Note}: Underlined field indicates primary key. Asterisk ({*})
indicates a foreign key. 

\subsection*{Task 4.1 }

Write the SQL code to create the four tables in the database named
\texttt{computercompany.db}. 

Save the SQL file as \texttt{TASK4\_1\_<your class>\_<your name>.sql}.
\hfill{}{[}4{]}

\subsection*{Task 4.2 }

The files \texttt{CUSTOMER.CSV}, \texttt{OFFICE.CSV}, \texttt{SALE.CSV}
and \texttt{SALESPERSON.CSV} contain information exported by their
spreadsheets files. Write Python code to migrate them to the database. 

Save your Python code as \texttt{TASK4\_2\_<your class>\_<your name>.py}.
\hfill{} {[}6{]}

\subsection*{Task 4.3 }

Write SQL code to show the \texttt{SaleDate}, \texttt{SalesPersonName},
\texttt{CustomerName} and \texttt{Amount} of all sale transactions
performed at the office with ID \texttt{1}. 

Save the SQL file as \texttt{TASK4\_3\_<your class>\_<your name>.sql}.
\hfill{} {[}4{]}

\subsection*{Task 4.4 }

A report is produced to show the top salesperson in each office each
month. 

Write Python code to: 
\begin{enumerate}
\item[i.]  generate a web form that allows a user to enter the month (\texttt{1}
to \texttt{12}) and year, 
\item[ii.]  use the data submitted by the web form to query the database, and 
\item[iii.]  return a HTML page with the report displayed. 
\end{enumerate}
The following is a sample report for March, 2020. 
\noindent \begin{center}
\begin{tabular}{|lll|}
\hline 
\multicolumn{3}{|c|}{Top Performers in March 2020}\tabularnewline
\textbf{Office ID} & \textbf{Salesperson} & \textbf{Total Amount S\$}\tabularnewline
1 & Low Kok Kheong & 7880\tabularnewline
2 & Mindy Tan & 6935\tabularnewline
3 & Monish Chandr & 10700\tabularnewline
\hline 
\end{tabular}
\par\end{center}

Save your Python program as \texttt{TASK4\_4\_<your class>\_<your
name>.py} with any additional files/ sub-folders as needed in a zipped
folder named 

\texttt{Task4\_<your class>\_<your name>.zip}.\hfill{} {[}12{]}

\subsection*{Task 4.5 }

Deploy the web app on your local host and enter the following data: 
\begin{enumerate}
\item month -- \texttt{8}, and 
\item year -- \texttt{2020}. 
\end{enumerate}
Save the screenshot of table generated as 

\texttt{TASK4\_5\_<your class>\_<your name>.jpg}. \hfill{} {[}2{]}

{[}SPLIT\_HERE{]}
\end{enumerate}
 
\end{document}
