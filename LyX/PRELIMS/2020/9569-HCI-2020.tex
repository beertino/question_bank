%% LyX 2.3.6.1 created this file.  For more info, see http://www.lyx.org/.
%% Do not edit unless you really know what you are doing.
\documentclass[english]{article}
\usepackage[T1]{fontenc}
\usepackage[latin9]{inputenc}
\usepackage{geometry}
\geometry{verbose,tmargin=2.5cm,bmargin=2.5cm,lmargin=2.5cm,rmargin=2.5cm}
\usepackage{array}
\usepackage{multirow}
\PassOptionsToPackage{normalem}{ulem}
\usepackage{ulem}

\makeatletter

%%%%%%%%%%%%%%%%%%%%%%%%%%%%%% LyX specific LaTeX commands.
%% Because html converters don't know tabularnewline
\providecommand{\tabularnewline}{\\}

\makeatother

\usepackage{babel}
\begin{document}
{[}SPLIT\_HERE{]}
\begin{enumerate}
\item \textbf{{[}HCI/PRELIM/9569/2020/P1/Q1{]} }
\begin{enumerate}
\item Describe \textbf{two} characteristics of client-server network. \hfill{}{[}2{]}
\item Give \textbf{one} advantage and \textbf{one} disadvantage of client-server
network compared to peer-to-peer network. \hfill{}{[}2{]}
\item Briefly explain how Domain Name Server (DNS) works. \hfill{}{[}2{]}
\item Describe the DHCP server\textquoteright s process to allocate an IP
address to a client. \hfill{}{[}4{]}
\end{enumerate}
{[}SPLIT\_HERE{]}
\item \textbf{{[}HCI/PRELIM/9569/2020/P1/Q2{]} }

Computing technology was adopted worldwide during the recent pandemic. 
\begin{enumerate}
\item Describe \textbf{one} positive and \textbf{one} negative impact you
have observed.\hfill{} {[}2{]}
\item The national task force sets a code of conduct for all computing professionals
working on contact tracing, data analysis, etc. Describe two rules
that you would expect to be included. For each of your rules, give
an example of the unethical behaviour it is designed to prevent. \hfill{}{[}4{]}
\end{enumerate}
{[}SPLIT\_HERE{]}
\item \textbf{{[}HCI/PRELIM/9569/2020/P1/Q3{]} }
\begin{enumerate}
\item Using real-life examples, explain the terms data validation and data
verification. \hfill{}{[}4{]}
\item Describe how data is transmitted in a packet switching network, and
give two advantages of packet switching over circuit switching network.\hfill{}
{[}4{]}
\item Explain the purpose of layering in TCP/IP model. List the layers in
order and describe each layer\textquoteright s major function. \hfill{}{[}6{]}
\end{enumerate}
{[}SPLIT\_HERE{]}
\item \textbf{{[}HCI/PRELIM/9569/2020/P1/Q4{]} }
\begin{enumerate}
\item The ASCII code in denary for the character \textquoteleft \texttt{1}\textquoteright{}
is \texttt{49}.
\begin{enumerate}
\item Using 7 bits, express the ASCII code for the character \textquoteleft \texttt{4}\textquoteright{}
in binary.\hfill{}{[}1{]}
\item Express the character \textquoteleft \texttt{4}\textquoteright{} as
a hexadecimal number. \hfill{}{[}1{]}
\end{enumerate}
\item Convert \texttt{4B1} hexadecimal number to a binary number stored
as two bytes. \hfill{}{[}2{]}
\item In a restaurant, every membership account number is made up of five
digits followed by a letter e.g. \texttt{36514C} where the letter
is a modulus-eleven check digit for the account number. Each digit
is weighted, with the first digit having a weight of \texttt{7} and
each subsequent digit decreases its weight by \texttt{1}. Valid check
digits are in the range of letter C to letter M, with C corresponding
to the value of 1, D corresponds to 2 and so on.
\begin{enumerate}
\item What is the purpose of including a check digit at the end of each
membership account number? \hfill{}{[}1{]}
\item Write, in \textbf{pseudocode}, an algorithm which checks whether a
membership account number is valid. \hfill{}{[}4{]}
\item Using your algorithm, determine whether a person with the account
number \texttt{47938K} is a member of the restaurant. Explain your
answer. \hfill{}{[}1{]}
\end{enumerate}
\end{enumerate}
{[}SPLIT\_HERE{]}
\item \textbf{{[}HCI/PRELIM/9569/2020/P1/Q5{]} }
\begin{enumerate}
\item Run-length encoding is a simple data compression technique that can
be effective when repeated values occur at adjacent positions within
a string. Compression is achieved by replacing groups of repeated
values with one copy of the value, followed by the number of times
the value should be repeated. For example, \textquotedblleft \texttt{AAAAABBBAAAB}\textquotedblright{}
would be compressed as \textquotedblleft \texttt{A5B3A3B1}\textquotedblright . 

Write, in \textbf{pseudocode}, a function that implements the run-length
compression technique described above. The function will take a string
argument and return the run-length compressed string. \hfill{}{[}6{]}
\item Using \textbf{pseudocode}, write a detailed algorithm for a function
which will take two string values, called \texttt{P} and \texttt{Q},
and will search the string value \texttt{P} for the first occurrence
of the string value \texttt{Q} within it. The value returned is the
start position of the first occurrence of \texttt{Q} in \texttt{P},
or zero if there is no occurrence. Assume the string index starts
at 1. For example, if \texttt{P} is '\texttt{bananas}' and Q is '\texttt{na}'
then the function would give the result 3, because '\texttt{na}' first
occurs in '\texttt{bananas}' starting at character position three.
{[}6{]}
\end{enumerate}
{[}SPLIT\_HERE{]}
\item \textbf{{[}HCI/PRELIM/9569/2020/P1/Q6{]} }
\begin{enumerate}
\item State \textbf{two} key characteristics of a recursive function, and
when is it suitable to be used. {[}3{]}
\item The procedure \texttt{MoveTower(n,i,j)} shown below simulates the
movement of moving \texttt{n} discs from peg \texttt{i} to peg \texttt{j}.

\noindent %
\noindent\begin{minipage}[t]{1\columnwidth}%
\texttt{PROCEDURE MoveTower(n, i, j) }

\texttt{\qquad{}IF n = 1 OUTPUT (\textquotedbl Move disc from peg\textquotedbl ,
i, \textquotedbl to peg\textquotedbl , j) }

\texttt{\qquad{}ELSE }

\texttt{\qquad{}\qquad{}MoveTower(n-1, i, 6-i-j) }

\texttt{\qquad{}\qquad{}OUTPUT (\textquotedbl Move disc from peg\textquotedbl ,
i, \textquotedbl to peg\textquotedbl , j) }

\texttt{\qquad{}\qquad{}MoveTower(n-1, 6-i-j, j) }

\texttt{\qquad{}ENDIF}

\texttt{ENDPROCEDURE}%
\end{minipage}
\begin{enumerate}
\item Dry-run the procedure and show the output that is produced when the
task is to move 3 discs from peg 1 to peg 3. \hfill{}{[}3{]}
\item Assuming a stack is used for passing parameters to the procedure,
show also the contents of the stack, excluding the return address,
after each of the first \textbf{five} procedure calls. \hfill{} {[}3{]}
\end{enumerate}
\item If a procedure is to be able to call itself recursively, it is usual
for the values of any variables used in the procedure to be held in
a stack rather than in fixed storage. Why is this? \hfill{}{[}1{]}
\end{enumerate}
{[}SPLIT\_HERE{]}
\item \textbf{{[}HCI/PRELIM/9569/2020/P1/Q7{]} }

A linked list ADT has the following operations defined: 
\begin{itemize}
\item \texttt{Create(x) }-{}- creates an empty linked list \texttt{x}; 
\item \texttt{Insert(x,item,p)} -{}- inserts new value, \texttt{item}, into
linked list \texttt{x} so that it is at position \texttt{p} in the
linked list; 
\item \texttt{Delete(x,p)} -{}- deletes the item at position \texttt{p}
in the linked list \texttt{x}; 
\item \texttt{Read(x,p)} -{}- returns the item at position \texttt{p} in
the linked list \texttt{x}; 
\item \texttt{Length(x)} -{}- returns the number of items in the linked
list \texttt{x}; 
\item \texttt{IsEmptyList(x)} -{}- returns \texttt{True} if linked list
\texttt{x} is empty, otherwise returns \texttt{False}; 
\item \texttt{Clear(x)} -- empties the linked list \texttt{x};
\end{itemize}
The linked list is implemented by the use of a collection of nodes
that have two parts: the item data and a pointer to the next item
in the list. In addition, there is a \texttt{Start} pointer which
points to the first item in the list.
\begin{enumerate}
\item Write algorithms that could be used to implement the \textquoteleft \texttt{Delete}\textquoteright{}
and \textquoteleft \texttt{Insert}\textquoteright{} operation. \hfill{}{[}8{]}
\end{enumerate}
A stack ADT has the following operations:
\begin{itemize}
\item \texttt{Create()} - creates a new stack; 
\item \texttt{Push(item)} - adds \texttt{item} onto the stack; 
\item \texttt{Pop()} - deletes and returns item from the stack; 
\item \texttt{IsEmpty()} -- if the stack is empty returns True, otherwise
False; 
\item \texttt{Clear()} -- removes all items in the stack;
\end{itemize}
\begin{enumerate}
\item[(b)]  Show how to implement \textquoteleft \texttt{Create}\textquoteright ,
\textquoteleft \texttt{Push}\textquoteright{} and \textquoteleft \texttt{Pop}\textquoteright{}
operation using the list ADT operations.\hfill{} {[}4{]}
\end{enumerate}
The stack implementation above is used to implement the undo/redo
mechanism of a text editor. 

An Undo stack is used to keep the edit history of the editor and the
Redo stack is used to keep the history of the undo operations. The
content of the text editor is stored as a string in the Undo stack
and Redo stack. 

When an undo is invoked, the Undo stack is popped and the content
is pushed into the Redo Stack. When a redo is invoked, the Redo stack
is popped and the content is pushed into the Undo Stack.
\begin{enumerate}
\item[(c)]  Using the stack ADT operations, show how to implement the following
functions which return the contents. Assume that \texttt{undoStack}
and \texttt{redoStack} are created. 
\begin{enumerate}
\item \texttt{FUNCTION Undo() RETURNS STRING} 
\item \texttt{FUNCTION Redo() RETURNS STRING}\hfill{} {[}3{]}
\end{enumerate}
\end{enumerate}
{[}SPLIT\_HERE{]}
\item \textbf{{[}HCI/PRELIM/9569/2020/P1/Q8{]} }

S-Cut offers haircut service at a price based on age group at 60 outlets
located around Singapore. S-Cut has engaged you to develop an application
that runs on kiosk stationed at their outlets. The application uses
a relational database to store the data. Customers need to register
as a regular member with their name, contact number and price. They
can purchase a haircut service using the kiosks provided. The service
record should contain date of service, member information and the
outlet information. Based on the requirements given above, design
the database that consists of 3 tables: Member, Outlet and ServiceRecord.
\begin{enumerate}
\item {}
\begin{enumerate}
\item Draw the Entity-Relationship (E-R) diagram to show the tables in third
normal form (3NF) and their relationships between them. \hfill{}{[}3{]}
\item A table description can be expressed as:

\texttt{TableName( }\texttt{\uline{Attribute1}}\texttt{, Attribute2{*},
Attribute3, \dots ) }

The primary key is indicated by underlining one or more attributes.
Foreign keys are indicated by using a dashed underline/asterisk. Using
the information given, write table descriptions for the tables you
have identified in part \textbf{(a)(i)}. \hfill{}{[}5{]}
\end{enumerate}
\item To attract more customers, some outlets offer 20\% discount on Saturday
and Sunday. Write the table description with the changes to the database
to capture the change in price per haircut on Saturday and Sunday.
\hfill{}{[}1{]}
\item S-Cut offers platinum membership which entice members to additional
benefits. Platinum member needs to credit \$100 to their account which
entitles them to 12 sessions of haircut services, in addition they
will receive a birthday gift on their birthday month. To ensure greater
productivity in development of the membership system, object oriented
design is used. 
\begin{enumerate}
\item Draw a class diagram for member class(es) which exhibit the following: 
\begin{itemize}
\item Suitable classes with appropriate properties and methods
\item Inheritance 
\item Polymorphism \hfill{}{[}6{]}
\end{itemize}
\item Explain the term \textbf{encapsulation} and how it is applied in your
design in (c)(i). \hfill{}{[}2{]}
\item Explain the term \textbf{polymorphism} and how it is applied in your
design in (c)(i). \hfill{}{[}2{]}
\end{enumerate}
\item An alternative way to develop an application that runs on the kiosk
would be a web-based application. State \textbf{two} differences between
a native application and a web-based application. \hfill{}{[}2{]}
\item S-Cut collects data from the customer, describe \textbf{two} data
protection obligations on how S-Cut must comply with the Personal
Data Protection Act (PDPA). \hfill{}{[}2{]}
\end{enumerate}
{[}SPLIT\_HERE{]}
\item \textbf{{[}HCI/PRELIM/9569/2020/P2/Q1{]} }

The file \texttt{COVID19.TXT} stores the cumulative total number of
confirmed covid-19 cases in Singapore from 14 April 2020 till 15 May
2020.

A sample record is: 

\texttt{1704}, \texttt{5050 }

This means that as of 17 April 2020, a cumulative total of 5050 confirmed
covid-19 cases were reported. 

The task is to determine the dates with the highest and lowest number
of the confirmed new covid-19 cases in Singapore and the longest ascending
streak during the period.

Note: You cannot use the built-in \texttt{sort()}, \texttt{min()}
and \texttt{max()} functions.

\subsection*{Task 1.1 }

Write program code to: 
\begin{itemize}
\item read the data from the text file 
\item calculate the number of new covid-19 cases per day (from 15 April
2020 to 15 May 2020) 
\item output the dates with the highest and lowest numbers of new covid-19
cases, excluding the first day on 14 April 2020. 
\item If more than one dates exists with the highest/lowest number of new
covid-19 cases, all dates are reported.
\end{itemize}

\subsubsection*{Sample output:}

\texttt{Highest \# cases (1426) is on 20 April 2020 }

\texttt{Lowest \# cases (447) is on 15 April 2020, 2 May 2020} \hfill{}{[}9{]}

\subsection*{Task 1.2 }

Write program code to:
\begin{itemize}
\item Determine the ascending streak of new covid-19 cases 
\item output the longest ascending streak of new cases across the period
from 15 April 2020 to 15 May 2020 inclusive.
\noindent \begin{center}
\begin{tabular}{|c|c|c|c|}
\hline 
Date & Cumulative Total Cases & New Cases & \tabularnewline
\hline 
\hline 
1404 & 3252 &  & \tabularnewline
\hline 
1504 & 3699 & 447 & \multirow{2}{*}{2-day ascending streak}\tabularnewline
\cline{1-3} \cline{2-3} \cline{3-3} 
1604 & 4427 & 728 & \tabularnewline
\hline 
1704 & 5050 & 623 & \multirow{2}{*}{2-day ascending streak}\tabularnewline
\cline{1-3} \cline{2-3} \cline{3-3} 
1804 & 5992 & 942 & \tabularnewline
\hline 
1904 & 6588 & 596 & \tabularnewline
\hline 
$\vdots$ & $\vdots$ & $\vdots$ & \tabularnewline
\hline 
0305 & 18205 & 657 & \tabularnewline
\hline 
0305 & 18778 & 573 & \multirow{3}{*}{3-day ascending streak}\tabularnewline
\cline{1-3} \cline{2-3} \cline{3-3} 
0305 & 19410 & 632 & \tabularnewline
\cline{1-3} \cline{2-3} \cline{3-3} 
0305 & 20198 & 788 & \tabularnewline
\hline 
0305 & 20939 & 741 & \tabularnewline
\hline 
$\vdots$ & $\vdots$ & $\vdots$ & \tabularnewline
\hline 
1205 & 24671 & 884 & \tabularnewline
\hline 
1305 & 25346 & 675 & \multirow{3}{*}{3-day ascending streak}\tabularnewline
\cline{1-3} \cline{2-3} \cline{3-3} 
1405 & 26098 & 752 & \tabularnewline
\cline{1-3} \cline{2-3} \cline{3-3} 
1505 & 26891 & 793 & \tabularnewline
\hline 
\end{tabular}
\par\end{center}
\end{itemize}

\subsubsection*{Sample output:}

\texttt{Longest ascending streak is 3 days} \hfill{}{[}6{]}

{[}SPLIT\_HERE{]}
\item \textbf{{[}HCI/PRELIM/9569/2020/P2/Q2{]} }

The file \texttt{COUNTRY1.TXT} stores a list of country names.

\subsection*{Task 2.1 }

Using the country name, a hash address is calculated from a hashing
function as follows: 
\begin{itemize}
\item the ASCII code is calculated for each character (in lowercase) within
the country name 
\item the total of ASCII values is calculated 
\item the total is divided by 30 and the remainder is the hash address for
this country 
\end{itemize}
For example, the hash address for \texttt{Brazil} is 14 

Write program code for the hashing function using the following specification 

\texttt{FUNCTION HashKey (Country: STRING): INTEGER }

This function has a single parameter \texttt{Country} and returns
the hash address as an integer. \hfill{}{[}3{]}

\subsection*{Task 2.2 }

The hash table is implemented as a list with 30 elements. Elements
are written to and read from the hash table using the above hash function
with the country name as the input parameter. 

Write program code which does the following: 
\begin{itemize}
\item Read all country names from \texttt{COUNTRY1.TXT} 
\item Use the function \texttt{HashKey} to calculate the hash address for
each country 
\item Store each country name in the hash table 
\end{itemize}
You must ensure that when a collision occurs, your program design
will deal with this situation by searching sequentially from the calculated
hash address for an empty location and storing the country name at
this empty location. The program will generate an error message if
the hash table is full. \hfill{}{[}7{]}

\subsection*{Task 2.3 }

Write program code for a procedure SearchCountry using the following
specification 

\texttt{PROCEDURE SearchCountry (Country: STRING, HashTable: LIST)}

This procedure has two parameters, \texttt{Country} and \texttt{HashTable},
and does the following: 
\begin{itemize}
\item Calculate the hash address for the country 
\item Locate the country name in the hash table 
\item Report whether or not this country name was found. If found, also
output the address of the hash table where the country was found. 
\end{itemize}
You must ensure that your program can deal with collision when searching.
\hfill{}{[}6{]}

Devise a set of \textbf{three} test cases with the country to be used.
\hfill{} {[}3{]}

\subsection*{Task 2.4 }

The file \texttt{COUNTRY2.TXT} stores a list of countries with their
corresponding numbers of confirmed cases and death cases of COVID19
pandemic on May 16, 2020. 

Each record has the following format: 
\noindent \begin{center}
\texttt{<country>,<no. of confirmed cases>,<no. of death case> }
\par\end{center}

A sample record is 
\noindent \begin{center}
\texttt{Brazil, 233142, 15633 }
\par\end{center}

Death rate for each country is calculated as the number of death cases
/ the number of confirmed cases. This rate is output as a percentage
to 1 decimal place. For the example above, the death rate of Brazil
is $15633/233142=6.7\%$.

Write program code which does the following: 
\begin{itemize}
\item Read the data from \texttt{COUNTRY2.TXT }
\item Implement \textbf{bubble sort} to arrange the countries from highest
to lowest death rate 
\item Generate a text file \texttt{RATE.TXT} to display the list of countries
and their death rate. Each record has the format \texttt{<country>,
<death rate>}. 
\end{itemize}
If two countries have the same death rate, their order in the list
does not matter. \hfill{}{[}9{]}

{[}SPLIT\_HERE{]}
\item \textbf{{[}HCI/PRELIM/9569/2020/P2/Q3{]} }

A bakery currently keeps records on paper of all its products it sells
in the shop. It decided to trial a database to manage its products.
It is expected that the database should be normalized. The following
information of the bakery is recorded: 
\begin{itemize}
\item \texttt{ProductCode} -- Unique code of the item 
\item \texttt{Name} -- Name of the item 
\item \texttt{Type} -- The type of product -- \texttt{'Cake'}, \texttt{'Loaf'},
\texttt{'Bun'} 
\item \texttt{Location} -- The location at which the product is made --
\texttt{'North'}, \texttt{'South'}, \texttt{'East'}, \texttt{'West'}
Kitchen 
\item \texttt{Price} -- The selling price of the product 
\end{itemize}
For cakes, the following additional information is recorded: 
\begin{itemize}
\item \texttt{ServingSize} -- the estimated number of servings per cake 
\item \texttt{Shape} -- Shape of the cake -- \texttt{'Square'}, \texttt{'Circle'},
\texttt{'Roll'} 
\end{itemize}
For loaves, the following additional information is recorded: 
\begin{itemize}
\item \texttt{Weight} -- weight of loaf in gram 
\end{itemize}
For buns, the following additional information is recorded:
\begin{itemize}
\item \texttt{PiecesPerPackage} -- number of pieces per package
\end{itemize}
The information is to be stored in four different tables: 

\texttt{Product }

\texttt{Cake }

\texttt{Loaf }

\texttt{Bun}

\subsection*{Task 3.1 }

Create a SQL file named \texttt{Task3\_1\_<your\_name>\_<centre number>\_<index
number>.sql} to show the SQL code to create the database \texttt{bakery.db}
with the four tables. The table, \texttt{Product}, must use the \texttt{ProductCode}
as its \textbf{primary key}. The other tables must refer to the \texttt{ProductCode}
as a \textbf{foreign key}.

Save your SQL code as 

\texttt{Task3\_1\_<your\_name>\_<centre number>\_<index number>.sql}
\hfill{}{[}4{]}

\subsection*{Task 3.2 }

The files \texttt{CAKES.TXT}, \texttt{LOAVES.TXT} and \texttt{BUNS.TXT}
contain information about the bakery\textquoteright s cakes, loaves
and buns respectively for insertion into the bakery database. Each
row in the three files is a comma-separated list of information about
a single product. 

For \texttt{CAKES.TXT}, the information about each cake is given in
the following order: 
\noindent \begin{center}
\texttt{ProductCode, Name, Location, Price, ServingSize, Shape }
\par\end{center}

For \texttt{LOAVES.TXT}, the information about each loaf is given
in the following order: 
\noindent \begin{center}
\texttt{ProductCode, Name, Location, Price, Weight }
\par\end{center}

For \texttt{BUNS.TXT}, the information about each bun is given in
the following order:
\noindent \begin{center}
\texttt{ProductCode, Name, Location, Price, PiecesPerPackage}
\par\end{center}

Write a Python program to insert all information from the three files
into the bakery database, bakery.db. Run the program. 

Save your program code as \texttt{Task3\_2\_<your\_name>\_<centre
number>\_<index number>.py} \hfill{}{[}6{]}

\subsection*{Task 3.3 }

Write SQL code to show the \texttt{ProductCode}, \texttt{Name}, \texttt{Location},
\texttt{Price} and the \texttt{ServingSize} of each cake with \texttt{'Circle'}
Shape. Run the query. 

Save this code as 

\texttt{Task3\_3\_<your\_name>\_<centre number>\_<index number>.sql
}\hfill{}{[}4{]}

\subsection*{Task 3.4 }

The bakery wants to filter the products by \texttt{Location} and display
the results in a web browser. 

Write a Python program and the necessary files to create a web application
that: 
\begin{itemize}
\item receives a \texttt{Location} string from a HTML form, then 
\item creates and returns a HTML document that enables the web browser to
display a table tabulating the details of the product based on the
\texttt{Location} in ascending order of \texttt{Price}. The table
will display the following columns: \texttt{Name}, \texttt{Type},
\texttt{Price}.
\end{itemize}
Save your Python program as 

\texttt{Task3\_4\_<your\_name>\_<centre number>\_<index number>.py }

with any additional files/sub-folders as needed in a folder named
Task3\_4\_<your\_name>\_<centre number>\_<index number>

Run the web application. Save the output of the program when \texttt{'North'}
is entered as the \texttt{Location} as 

\texttt{Task3\_4\_<your\_name>\_<centre number>\_<index number>.html}
\hfill{}{[}10{]}

{[}SPLIT\_HERE{]}
\item \textbf{{[}HCI/PRELIM/9569/2020/P2/Q4{]} }

A programmer is writing a program to manipulate different data structures
using Object-Oriented Programming. 

The superclass, \texttt{LinkedStructure}, will store the following
data:
\begin{itemize}
\item A linear linked list of data items held in an \textbf{array} of size
10. Array index starts at 1. 
\item Head pointer, pointing to the first element in the linked list 
\item Tail pointer, pointing to the last element in the linked list 
\item Free pointer, pointing to the first element in the free list
\end{itemize}
The diagram shows the linked structure after four items have been
added and the unused nodes are linked together. 
\noindent \begin{center}
<INSERT\_IMAGE\_HERE>
\par\end{center}

This superclass has the following methods:
\begin{itemize}
\item \texttt{Initialise()} method sets up an empty linked list. Should
link all nodes to form the free list. Initialise values for head pointer,
tail pointer and free pointer 
\item \texttt{Add(item)} appends the parameter into its correct \textbf{alphabetical}
order in the linked list. 
\item \texttt{Remove(item)} removes the parameter from the ordered linked
list 
\item \texttt{Display()} displays the data items in the linked list in alphabetical
order 
\item \texttt{PrintStructure()} displays the current state of all pointers
and the array contents in index order 
\item \texttt{IsEmpty()} tests for empty linked list 
\item \texttt{IsFull()} tests for no unused nodes
\end{itemize}
The superclass is used to implement a linear queue. 

The subclass \texttt{Queue} has the following methods:
\begin{itemize}
\item \texttt{Add(item)} appends the parameter to the queue and overrides
the \texttt{LinkedStructure} add method . 
\item \texttt{Remove()} returns and removes the next item in the queue
\item \texttt{Display()} method should display the queue contents in order
(e.g. the earliest added item first) and should override the \texttt{LinkedStructure}
display method. 
\end{itemize}
Each method updates its appropriate pointers, and produces suitable
errors (or returns different values) to indicate if the actions are
not possible, e.g. if the structure is empty. 

For each of the sub-tasks, add a comment statement, at the beginning
of the code using the hash symbol \textquoteleft \#\textquoteright ,
to indicate the sub-task the program code belongs to, for example:

\subsection*{Task 4.1 }

Write program code for the superclass \texttt{LinkedStructure}. \hfill{}{[}20{]}

\subsection*{Task 4.2}

Write program code to:
\begin{itemize}
\item create a \texttt{LinkedStructure} object 
\item add the following three data items in the order shown to the ordered
linked list 
\noindent \begin{center}
\texttt{Japan, Singapore, China }
\par\end{center}
\item output all pointers and array contents using the \texttt{PrintStructure()}
method after adding the items 
\item output the current contents of the linked list using the \texttt{Display()}
method 
\item remove two data items \texttt{China}, \texttt{Japan} in that order
from the linked list 
\item output all pointers and array contents using the \texttt{PrintStructure()}
method after the removal of the items. \hfill{} {[}5{]}
\end{itemize}

\subsection*{Task 4.3}

Write program code for the subclass \texttt{Queue}.

Use appropriate inheritance and polymorphism in your designs. \hfill{}{[}5{]}

\subsection*{Task 4.4}

The file \texttt{QUEUE.TXT} stores data to test your program.

Write program code to:
\begin{itemize}
\item create a new queue and add the data in the file QUEUE.TXT to the queue 
\item output the current contents of the queue
\item remove and output two items from the queue 
\item output all pointers and the array contents of the queue after the
removal of the items.
\end{itemize}
All outputs should have appropriate messages to indicate what they
are showing. \hfill{}{[}3{]}

{[}SPLIT\_HERE{]}
\end{enumerate}

\end{document}
