%% LyX 2.3.6.1 created this file.  For more info, see http://www.lyx.org/.
%% Do not edit unless you really know what you are doing.
\documentclass[english]{article}
\usepackage[T1]{fontenc}
\usepackage[latin9]{inputenc}
\usepackage{geometry}
\geometry{verbose,tmargin=2.5cm,bmargin=2.5cm,lmargin=2.5cm,rmargin=2.5cm}
\usepackage{color}
\PassOptionsToPackage{normalem}{ulem}
\usepackage{ulem}

\makeatletter

%%%%%%%%%%%%%%%%%%%%%%%%%%%%%% LyX specific LaTeX commands.
%% Because html converters don't know tabularnewline
\providecommand{\tabularnewline}{\\}

\makeatother

\usepackage{babel}
\begin{document}
{[}SPLIT\_HERE{]}
\begin{enumerate}
\item \textbf{{[}DHS/PRELIM/9569/2020/P1/Q1{]} }

Given an array of numbers \texttt{A}, count the minimum number of
'bubble sort' swaps (swap between pair of consecutive items) needed
to sort the array in ascending order. 

For example, if \texttt{A = {[}3, 2, 1, 4{]}}, we need 3 'bubble sort'
swaps to sort A in ascending order i.e. 

\texttt{swap(3, 2)} to get \texttt{{[}}\texttt{\uline{2, 3}}\texttt{,
1, 4{]} }

\texttt{swap(3, 1)} to get \texttt{{[}2, }\texttt{\uline{1, 3}}\texttt{,
4{]}} 

\texttt{swap(1, 2)} to get \texttt{{[}}\texttt{\uline{1, 2}}\texttt{,
3, 4{]}} 
\begin{enumerate}
\item Devise an $O(n^{2})$ algorithm using bubble sort to count the number
of 'bubble sort' swaps. 
\item Devise an $O(n\log n)$ algorithm to count the number of 'bubble sort'
swaps. 
\end{enumerate}
You should also explain why each algorithm has its efficiency. \hfill{}{[}8{]}
\begin{enumerate}
\item[(c)] Would you expect insertion sort to perform better or worse than bubble
sort in (a)? Explain your answer with respect to the number of comparisons
needed using the above example.\hfill{} {[}5{]}
\item[(d)] Why is quick sort not a suitable algorithm in part (b)? Illustrate
your answer with a suitable example. \hfill{}{[}5{]}
\end{enumerate}
{[}SPLIT\_HERE{]}
\item \textbf{{[}DHS/PRELIM/9569/2020/P1/Q2{]} }

You have been tasked to use a suitable data structure to manage the
preliminary exam results of students in Dunman High School. Each student
is identified by its centre and index numbers, each of which is 4-digit.
For example, Lim Ah Seng's identification number is 30420188. A typical
range of students' identification numbers are from 30420001 to 30420450,
since each graduating cohort will have about 450 students. 

The preliminary examination details to be stored are as follows: 
\begin{itemize}
\item Subject code (4-digit) eg 9569 
\item Subject name eg H2 Computing 
\item Subject grade (1-character eg 'A') You may assume that each student
will have a valid subject grade in the range of {[}'A', 'B', 'C',
'D', 'E', 'S', 'U', 'T', '0'{]}, where 'T' stands for terminated and
'0' stands for absent. 
\end{itemize}
Using suitable examples, evaluate the pros and cons using each of
the following data structure to store the required information: 
\begin{enumerate}
\item dictionary 
\item hash table \hfill{}{[}12{]}
\end{enumerate}
{[}SPLIT\_HERE{]}
\item \textbf{{[}DHS/PRELIM/9569/2020/P1/Q3{]} }

The information in Question 2 will also be stored in a relational
database. 
\begin{enumerate}
\item Why is a relational database model more suitable than a NoSQL database
model for storing the required information? \hfill{}{[}3{]}
\item Draw an ER diagram for a normalised database design. \hfill{}{[}4{]}
\item Produce the specification for the tables. \hfill{}{[}5{]}
\item Using examples in this context, explain the significance of the following
terms: 
\begin{enumerate}
\item primary key \hfill{}{[}2{]}
\item foreign key \hfill{}{[}2{]}
\item 1NF\hfill{} {[}2{]}
\item 2NF\hfill{} {[}2{]}
\item 3NF \hfill{}{[}2{]}
\end{enumerate}
\item What is the relationship between the data structures in Question 2
and a relational database? \hfill{}{[}4{]}
\end{enumerate}
{[}SPLIT\_HERE{]}
\item \textbf{{[}DHS/PRELIM/9569/2020/P1/Q4{]} }

The following diagram shows the contents with some data inserted. 
\begin{enumerate}
\item State two possible insertion orders of data to this BST.\hfill{}
{[}2{]}
\item Generalise how data can be inserted to produce a balanced BST.\hfill{}
{[}3{]}
\item The BST is to be implemented using a 1D array T. Write pseudocode
to show how data can be represented in T with suitable initial values
for an empty BST. \hfill{}{[}4{]}
\item Devise a recursive algorithm to insert to this BST. \hfill{}{[}5{]}
\item Devise a recursive algorithm to check if T is a BST. \hfill{}{[}3{]}
\item Convert the recursive algorithm in part (e) to an iterative one using
a suitable data structure which you should name and justify. \hfill{}{[}5{]}
\item Devise an algorithm to output the items in T that are within a given
range {[}a, b{]} inclusive in ascending order.\hfill{} {[}4{]}
\item Devise an algorithm to output the contents of the leaves of T in descending
order.\hfill{} {[}4{]}
\item Despite its memory overhead, why is recursion often used in BSTs?
\hfill{}{[}3{]}
\item Why is recursion less often used in linked lists?\hfill{} {[}2{]}
\end{enumerate}
{[}SPLIT\_HERE{]}
\item \textbf{{[}DHS/PRELIM/9569/2020/P1/Q5{]} }

A student came up with the following Python program to implement a
linked list data structure:

\noindent %
\noindent\begin{minipage}[t]{1\columnwidth}%
\texttt{01\qquad{}class Node: }

\texttt{02\qquad{}\qquad{}def \_\_init\_\_(self, data): }

\texttt{03\qquad{}\qquad{}self.data = data }

\texttt{04\qquad{}\qquad{}self.link = None }

\texttt{05 }

\texttt{06\qquad{}def insert(data): }

\texttt{07\qquad{}\qquad{}global head }

\texttt{08\qquad{}\qquad{}if head == None: \# empty linked list }

\texttt{09\qquad{}\qquad{}\qquad{}head = Node(data) }

\texttt{10\qquad{}\qquad{}else: \# insert to front }

\texttt{11\qquad{}\qquad{}\qquad{}new\_node = Node(data) }

\texttt{12\qquad{}\qquad{}\qquad{}new\_node.link = head }

\texttt{13\qquad{}\qquad{}\qquad{}head = new\_node }

\texttt{14 }

\texttt{15\qquad{}\qquad{}def display(): }

\texttt{16\qquad{}\qquad{}\qquad{}global head }

\texttt{17\qquad{}\qquad{}\qquad{}curr = head }

\texttt{18\qquad{}\qquad{}\qquad{}while curr: }

\texttt{19\qquad{}\qquad{}\qquad{}\qquad{}print(curr.data) }

\texttt{20\qquad{}\qquad{}\qquad{}\qquad{}curr = curr.link }

\texttt{21 }

\texttt{22\qquad{}\# main }

\texttt{23\qquad{}head = None }

\texttt{24\qquad{}insert(\textquotedbl Ali\textquotedbl ) }

\texttt{25\qquad{}insert(\textquotedbl Tom\textquotedbl ) }

\texttt{26\qquad{}insert(\textquotedbl Mary\textquotedbl ) }

\texttt{27\qquad{}display() }%
\end{minipage}
\begin{enumerate}
\item What will be the output of line 27?\hfill{} {[}1{]}
\item Comment on the programming paradigms used and identify any potential
pitfalls in the above program.\hfill{}{[}5{]}
\item Why is OOP appropriate in the implementation of data structures such
as linked lists?\hfill{}{[}2{]}
\item The above program maintains an unordered linked list. Devise an algorithm
to insert to an ordered linked list.\hfill{}{[}5{]}
\item A linked list can be ordered or unordered. Draw a UML class diagram
to illustrate the concepts of encapsulation, inheritance and polymorphism.\hfill{}
{[}5{]}
\item Explain how polymorphism is applied to the \texttt{insert()} rather
than the \texttt{display()} method in this context.\hfill{} {[}3{]}
\end{enumerate}
{[}SPLIT\_HERE{]}
\item \textbf{{[}DHS/PRELIM/9569/2020/P1/Q6{]} }

On 14 September 2020, it was reported that GrabCar was fined S\$10,000
for a 4th user data privacy violation. The Personal Data Protection
Commission (PDPC) said the update risked the personal data of 21,541
drivers and passengers, including profile pictures, names and vehicle
plate numbers. 

GrabCar rolled back the app to the previous version within about 40
minutes and took other remedial action, PDPC said. 

On Aug 30, 2019, GrabCar notified the PDPC that profile data of 5,651
GrabHitch drivers was exposed to the risk of unauthorised access by
other GrabHitch drivers for a \textquotedbl short period of time
on the same day\textquotedbl{} through the Grab app. 

Grab's investigations traced the cause of the breach to a deployment
of an update to the app on the same day. The purpose of the update
was to address a potential vulnerability discovered within the Grab
app. 

In PDPC's findings, the application programming interface URL which
allowed GrabHitch drivers to access their data, had contained a \textquotedbl userID\textquotedbl{}
portion that could potentially be manipulated to allow access to other
drivers' data. According to GrabCar, there was no evidence that this
vulnerability was exploited. 

To fix the vulnerability, the update removed the \textquotedbl userID\textquotedbl{}
from the URL, which shortened it to a hard-coded \textquotedbl users/profile\textquotedbl .
However, it failed to take into account the URL-based caching mechanism
in the app, which was configured to refresh every 10 seconds. 

The mechanism served cached content in response to data requests,
so as to reduce the load of direct access to GrabCar's database. 

With the update, all URLs in the Grab app ended with \textquotedbl users/profile\textquotedbl .
Without the \textquotedbl userID\textquotedbl{} in the URL, which
directed data requests to the correct GrabHitch driver's accounts,
the caching mechanism could no longer differentiate between drivers. 

Thus, the mechanism provided the same data to all GrabHitch drivers
for 10 seconds before new data was retrieved from GrabCar's database
and cached for the next 10 seconds. 

PDPC said GrabCar did not put in place \textquotedbl sufficiently
robust processes\textquotedbl{} to manage changes to its IT system
that may put personal data it was processing at risk. 

\textquotedbl This was a particularly grave error given that this
is the second time the (GrabCar) is making a similar mistake, albeit
with respect to a different system,\textquotedbl{} he said.

In a statement in response to Reuters' query, Grab said: \textquotedbl To
prevent a recurrence, we have since introduced more robust processes,
especially pertaining to our IT environment testing, along with updated
governance procedures and an architecture review of our legacy application
and source codes.\textquotedbl{} 

In 2019, GrabCar was ordered to pay a financial penalty of S\$16,000
after it sent out more than 120,000 marketing emails to customers
containing the name and mobile phone number of another customer. 

The PDPC had found that GrabCar \textquotedblleft failed to make reasonable
security arrangements\textquotedblright{} to detect the errors in
their database when sending out the emails. 

Source: Reuters/CNA/lk 

Adapted from https://www.channelnewsasia.com/news/business/grab-car-hitch-pdpc-personal-data-risk-fin
e-13108144 
\begin{enumerate}
\item For each of the following, suggest two ways in which the data leaks
could have been exploited by a malicious hacker with reference to 
\begin{enumerate}
\item profile pictures, names and vehicle plate numbers of drivers and passengers\hfill{}
{[}2{]}
\item name and mobile phone number of customers \hfill{}{[}2{]}
\end{enumerate}
\item What could have caused 
\begin{enumerate}
\item the URL related data leak? 
\item the marketing emails related data leak? 
\item the repeated data leaks? 
\end{enumerate}
You should provide a balance of technical and human related reasons.
\hfill{}{[}6{]}
\item How can Grab ensure and assure PDPC that sufficiently robust processes
have been put in place? \hfill{}{[}3{]}
\end{enumerate}
{[}SPLIT\_HERE{]}
\item \textbf{{[}DHS/PRELIM/9569/2020/P2/Q1{]} }

On 28 June 2020, nearly 10,000 TraceTogether tokens were distributed
to vulnerable seniors. The TraceTogether token supplements the TraceTogether
mobile app by extending contact tracing to groups in the community
who do not have smart phones and those whose phones do not work well
with the TraceTogether app. 

The TraceTogether token is designed to capture proximity data based
on Bluetooth signals. Every five minutes, it scans to detect other
TraceTogether users on the token or the app. The more 'hits' between
two TraceTogether users, the more likely they are in close proximity
for an extended period of time. Proximity can be estimated by the
strength of the Bluetooth signal. The closer users are to one another,
the stronger the signal and vice versa. 

There are only four types of data contained in the token: 
\begin{itemize}
\item user's randomised ID 
\item randomised ID of any other user in proximity 
\item Bluetooth signal measured using RSSI{*} 
\item timestamp of the encounter. 
\end{itemize}
{*}Received Signal Strength Indicator (RSSI) is a measure of the power
level at the receiver. A more negative number indicates a device is
further away. For example, a value of -20 to -30 indicates a device
is close while a value of -120 indicates it is near the limit of detection. 

It is important to note that these IDs do not refer to NRIC number,
but randomised and anonymised IDs linked to a personal identifier
like a mobile number. Also, no data is extracted unless a user has
tested positive for COVID-19. From there, MOH has a special software
key that can unlock the device and reveal the data for use in contact
tracing. 

A senior is tested positive for COVID-19 and MOH needs to perform
contact tracing. With the user's permission, data from its TraceTogether
token is retrieved and extracted to the file \texttt{TOKEN.txt} and
has the following structure: 

\texttt{UserRandomisedID, OtherRandomisedID, RSSI, Timestamp} 

Timestamp is in the format YYYY-MM-DD HH:MM:SS. 

\subsection*{Task 1.1 }

Prolonged exposure is currently defined as being in close contact
for at least 15 minutes within a single session. For simplicity, you
may assume close contact as a Bluetooth signal strength of greater
than or equal to -30. Generate the list of close contacts' randomised
IDs which MOH needs to perform contact tracing. \hfill{}{[}10{]}

\subsection*{Task 1.2 }

Another 3 seniors with randomised ID 75348257, 45174591 and 02548147
have also been tested as COVD-19 positive. Write a Boolean function
\texttt{is\_close\_contact(rid1, rid2)} to determine if they are close
contacts of 57345286. If yes, your function should also return the
date(s) they are under prolonged exposure with each other, the start
and end times as well as the total time in hours and minutes they
are in close contact. \hfill{}{[}10{]}

{[}SPLIT\_HERE{]}
\item \textbf{{[}DHS/PRELIM/9569/2020/P2/Q2{]} }

Write a program that sorts a list of student last names, but the sort
only uses the first two letters of the name. Nothing else in the name
is used for sorting. However, if two names have the same first two
letters, they should stay in the same order as in the input (this
is known as a 'stable sort'). Sorting is case sensitive based on ASCII
order (with uppercase letters sorting before lowercase letters i.e.
A<B<\dots <Z<a<b<\dots <z). 

\subsubsection*{Input }

The input file \texttt{LASTNAMES.txt} consists of a sequence of up
to 500 test cases. Each case starts with a line containing an integer
$1\le n\leq200$. After this follow $n$ last names made up of only
letters (a--z, lowercase or uppercase), one name per line. Names
have between 2 and 20 letters. Input ends when $n=0$. 

\subsubsection*{Output }

For each case, print the last names in sort-of-sorted order, one per
line. Print a blank line between cases. 

\subsubsection*{Sample Input }

\noindent %
\noindent\begin{minipage}[t]{1\columnwidth}%
\texttt{3 }

\texttt{Mozart }

\texttt{Beethoven }

\texttt{Bach }

\texttt{5 }

\texttt{Hilbert }

\texttt{Godel}

\texttt{Poincare }

\texttt{Ramanujan }

\texttt{Pochhammmer }

\texttt{0 }%
\end{minipage}

\subsubsection*{Sample Output }

\noindent %
\noindent\begin{minipage}[t]{1\columnwidth}%
\texttt{Bach }

\texttt{Beethoven }

\texttt{Mozart }

\texttt{Godel }

\texttt{Hilbert }

\texttt{Poincare }

\texttt{Pochhammmer}

\texttt{Ramanujan }%
\end{minipage}

You should make use of an appropriate data structure and one or more
suitable sorting algorithms from the syllabus. Indicate as comments
your choice of how you have adapted them or your case for designing
and implementing your own. \hfill{}{[}13{]}

{[}SPLIT\_HERE{]}
\item \textbf{{[}DHS/PRELIM/9569/2020/P2/Q3{]} }

A school needs a web application for teachers to use during English
lessons to record and view students' marks for weekly in-class presentations. 

The teacher should be required to login and should only be able to
input and view any information for students of the classes they teach. 

The maximum score per presentation is 100. Each English lesson is
always taught by the same teacher, each English lesson would only
have one teacher, and each class would only have one English lesson
per day. 

In a training session conducted for teachers, a demo version of the
app would be prepopulated with demo data as shown in the following
table. 
\noindent \begin{center}
\begin{tabular}{|c|c|c|c|c|c|}
\hline 
\textbf{Teacher's username } & \textbf{Teacher's password } & \textbf{Index number of student } & \textbf{Student's class } & \textbf{Date of presentation } & \textbf{Marks obtained}\tabularnewline
\hline 
\hline 
\texttt{mr\_raj } & \texttt{cr53aYJP } & \texttt{3 } & \texttt{19S306 } & \texttt{20200315 } & \texttt{95}\tabularnewline
\hline 
\texttt{mr\_james } & \texttt{8orjqiZc } & \texttt{24 } & \texttt{19S301 } & \texttt{20200315 } & \texttt{60}\tabularnewline
\hline 
\texttt{mdm\_rahayu } & \texttt{7iqndCjW} & \texttt{2 } & \texttt{19S302 } & \texttt{20200315 } & \texttt{35.5}\tabularnewline
\hline 
\texttt{mr\_james } & \texttt{8orjqiZc } & \texttt{11 } & \texttt{19S304 } & \texttt{20200325} & \texttt{60}\tabularnewline
\hline 
\end{tabular}
\par\end{center}

You have been tasked to create this web application. The data file
\texttt{DEMOAPP.txt} contains the demo data. 

\subsection*{Task 3.1 }

Create the user interface using HTML and CSS for the application.
The fields should include: 
\begin{itemize}
\item TeacherUsername 
\item TeacherPassword 
\item StudentIndex 
\item Class 
\item PresentationDate 
\item Marks The information should be presented in a tabular form. \hfill{}{[}7{]}
\end{itemize}

\subsection*{Task 3.2 }

Create a normalised relational database scheme for this application.
Provide the SQL statements for the creation of the tables. \hfill{}{[}6{]}

\subsection*{Task 3.3 }

Populate the table(s) with the demo data described above. Provide
SQL statements for the insertion of the records in \texttt{task3.sql}.\hfill{}
{[}4{]}

\subsection*{Task 3.4 }

Provide the processing logic in \texttt{app.py} and the associated
template file(s). \textcolor{white}{.}\hfill{}{[}13{]}

\subsection*{Task 3.5 }

Using SQL or otherwise, determine and display the total marks scored
by each class to the appropriate template file. \textcolor{white}{.}\hfill{}{[}5{]}

{[}SPLIT\_HERE{]}
\item \textbf{{[}DHS/PRELIM/9569/2020/P2/Q4{]} }

A dental clinic wishes to manage its patients' information using a
NoSQL database. 

There are 2 rooms in the clinic: Room 1 and Room 2. 

There are 3 dentists in the clinic: Doctor 1, Doctor 2, Doctor 3. 

Information about the patients is stored in the data file PATIENTS.txt
in the following format: 

\texttt{PatientID,Name,Appointment Date,Appointment Start Time,Doctor
Assigned,Room Number,Amount Charged }
\begin{itemize}
\item \texttt{PatientID} is an integer 
\item \texttt{Name} is made up of letters and space only 
\item \texttt{Appointment Date} is in DDMMYYYY format 
\item \texttt{Appointment Start} Time is HHMM in 24-hour format 
\item \texttt{Doctor Assigned} is either Doctor 1, Doctor 2 or Doctor 3 
\item \texttt{Room Number} is either Room 1 or Room 2 
\item \texttt{Amount Charged} is a float to 2 decimal places 
\end{itemize}

\subsection*{Task 4.1 }

Write program code to convert and import the data from \texttt{PATIENTS.txt}
into a MongoDB database Clinic under the collection \texttt{Patient}.
\hfill{}{[}4{]}

\subsection*{Task 4.2 }

Write program code to allow the Admin Clerk to add new patients by
requesting for the following information: 
\begin{itemize}
\item New patient's Name 
\item Appointment Date in DDMMYYYY format 
\item Appointment Start Time in 24-hour HHMM format 
\item Doctor to be Assigned 
\end{itemize}
The program should automatically assign a \texttt{PatientID}. Your
program should perform the necessary validation for each field. \hfill{}{[}8{]}

\subsection*{Task 4.3}

The Admin Clerk wants to know if there are issues with appointments.
Write program code to output appointments clashes and double bookings.
If an appointment clash occurs, reschedule the latter appointment
to the nearest available appointment on the same date. If a double
booking occurs, delete the latter duplicate booking. \hfill{}{[}8{]}

\subsection*{Task 4.4 }

Write program code for the Admin Clerk to determine the top 3 paying
patients. \hfill{} {[}4{]}

\subsection*{Task 4.5 }

The Admin Clerk realised that there are patients with the same name.
Write program code to identify patients with the same names and output
their \texttt{PatientID}s. \hfill{} {[}3{]}

{[}SPLIT\_HERE{]}
\item \textbf{{[}DHS/PRELIM/9569/2020/P2/Q5{]} }

The postfix notation is used to represent mathematical expressions.
The expressions written in postfix form are evaluated faster compared
to infix notation as parenthesis are not required in postfix. 

For example, the postfix expression \texttt{231+{*}5-} is the equivalent
of the infix expression\texttt{ 2{*}(3+1) -5}. 

The following algorithm evaluates postfix expressions: 
\begin{enumerate}
\item[1.]  Create a stack to store operands (data values). 
\item[2.]  Scan the given expression and do following for every scanned element. 
\begin{itemize}
\item If the element is a number, push it into the stack. 
\item If the element is a operator, pop operands for the operator from stack.
Evaluate the operator and push the result back to the stack 
\end{itemize}
\item[3. ] When the expression has ended, the number in the stack is the final
answer. 
\end{enumerate}
Example: 

Let the given expression be \texttt{\textquotedbl 231+{*}5-\textquotedbl}.
Scan all elements one by one from left to right. 
\begin{enumerate}
\item[1.]  Scan \texttt{'2'}, it is a number, so push it to stack. Stack contains
\texttt{'2' }
\item[2.]  Scan \texttt{'3'}, again a number, push it to stack, stack now contains
\texttt{'23'} (from bottom to top) 
\item[3.]  Scan \texttt{'1'}, again a number, push it to stack, stack now contains
\texttt{'231'} 
\item[4.]  Scan \texttt{'+'}, it is an operator, pop two operands from stack,
apply the \texttt{+} operator on the operands, we get \texttt{3 +
1} which results in \texttt{4}. We push the result \texttt{'4'} to
stack. Stack now becomes \texttt{'24' }
\item[5.]  Scan \texttt{'{*}'}, it is an operator, pop two operands from the
stack, apply the \texttt{{*}} operator on the operands, we get \texttt{2
{*} 4} which results in \texttt{8}. We push the result \texttt{'8'}
to stack. Stack now becomes \texttt{'8' }
\item[6.]  Scan \texttt{'5'}, it is a number, we push it to the stack. Stack
now becomes \texttt{'85' }
\item[7.]  Scan \texttt{'-'}, it is an operator, pop two operands from stack,
apply the \texttt{-} operator on operands, we get \texttt{8 - 5} which
results in \texttt{3}. We push the result \texttt{'3'} to stack. Stack
now becomes \texttt{'3'} 
\item[8.]  There are no more elements to scan, we return the top element from
the stack (the only element left) as the final result 3. 
\end{enumerate}
Thus the postfix expression is evaluated in a single pass without
the need for parenthesis. 

Implement the evaluation of postfix expression using a stack and OOP.
You may assume that all operands are single digit numbers.\hfill{}
{[}13{]}
\end{enumerate}

\end{document}
