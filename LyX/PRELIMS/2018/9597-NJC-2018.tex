%% LyX 2.3.6.1 created this file.  For more info, see http://www.lyx.org/.
%% Do not edit unless you really know what you are doing.
\documentclass[english]{article}
\usepackage[T1]{fontenc}
\usepackage[latin9]{inputenc}
\usepackage{geometry}
\geometry{verbose,tmargin=2.5cm,bmargin=2.5cm,lmargin=2.5cm,rmargin=2.5cm}
\usepackage{array}
\usepackage{calc}
\usepackage{multirow}
\usepackage{rotating}
\PassOptionsToPackage{normalem}{ulem}
\usepackage{ulem}

\makeatletter

%%%%%%%%%%%%%%%%%%%%%%%%%%%%%% LyX specific LaTeX commands.
%% Because html converters don't know tabularnewline
\providecommand{\tabularnewline}{\\}

\makeatother

\usepackage{babel}
\begin{document}
{[}SPLIT\_HERE{]}
\begin{enumerate}
\item \textbf{{[}NJC/PRELIM/9597/2018/P1/Q1{]} }

The text file \texttt{DATA.TXT} contains the records of 3 floating
point values collected once a day from 9 August 1965 to 9 August 2018
(inclusive). Each set of 3 values (i.e., the set of 3 values for one
day) is stored on one line of the file in the following format: 
\noindent \begin{center}
\texttt{D<Day i>V1<Value 1>V2<Value 2>V3<Value 3>} 
\par\end{center}

Note that the value of \emph{i} from <Day i> ranges from 0 to 19358
(i.e., \emph{i} is the index for each day between 9 August 1965 and
9 August 2018).

Task 1.1 

Write a program to read and store the contents of \texttt{DATA.TXT}.
The 3 values for each day should be stored in a tuple (DATE, VALUE
1, VALUE 2, VALUE 3), where: 
\begin{itemize}
\item DATE corresponds to a string in the format: DDMMYYYY. 
\item VALUE 1, VALUE 2 and VALUE 3 correspond to the 3 values for the given
DATE. 
\end{itemize}

\subsection*{Evidence 1 }

The program code to perform \textbf{Task 1.1}.\hfill{} {[}4{]}

In order to analyse the data, you have been tasked to first sort the
values in descending order based on the mean of VALUE 1, VALUE 2 and
VALUE 3. You must use the Bubble Sort Algorithm to perform the required
sorting. 

\subsection*{Task 1.2 }

Write the function \texttt{bubble\_sort\_means}, whose input corresponds
to the tuples generated in \textbf{Task 1.1}, and which outputs those
tuples sorted in descending order based on the mean of VALUE 1, VALUE
2 and VALUE 3.

\subsection*{Evidence 2 }

The program code for the function \texttt{bubble\_sort\_means}.\hfill{}
{[}4{]}

You are next required to apply the bubble\_sort\_means function in
order to compute the median value over the means of VALUE1, VALUE
2 and VALUE 3. You may not utilise any other form of sorting to compute
this median value.

You should then output the median value using the following sample
format. 

\begin{tabular}{|c|}
\hline 
\texttt{The median value is: 12345.67}\tabularnewline
\hline 
\end{tabular}

\subsection*{Task 1.3 }

Write the code that utilises the function \texttt{bubble\_sort\_means}
to compute the median value and then output it using the specified
formatting. 

\subsection*{Evidence 3 }

The program code to compute and output the median value. \hfill{}{[}3{]}

\subsection*{Evidence 4 }

A screenshot of the output for Task 1.3. \hfill{}{[}1{]}

To check the values that arose on specific days, you are next required
to sort the data in ascending order based on DATE. You must use the
Insertion Sort Algorithm to perform the required sorting. 

\subsection*{Task 1.4 }

Write the function \texttt{insertion\_sort\_dates}, whose input corresponds
to the tuples generated in Task 1.1, and which outputs those tuples
sorted in ascending order based on DATE. 

\subsection*{Evidence 5 }

The program code for the function \texttt{insertion\_sort\_dates}.
\hfill{} {[}4{]}

Finally, you are required to:
\begin{itemize}
\item Repeatedly request a valid date from the user 
\item Perform a binary search using the output from the \texttt{insertion\_sort\_dates}
function to find the specific tuple corresponding to the given date,
and then output the 3 values for that date (i.e., VALUE 1, VALUE 2,
VALUE 3) 
\end{itemize}
The following is a sample of the input and output for the above functionality. 

\noindent\fbox{\begin{minipage}[t]{1\columnwidth - 2\fboxsep - 2\fboxrule}%
\texttt{Please enter a valid YEAR: 2000 }

\texttt{Please enter a valid MONTH: 6 }

\texttt{Please enter a valid DAY: 22}

\bigskip{}

\texttt{The values on 2000-06-22 are: 00111.22, 22222.00, 03333.40 }

\bigskip{}

\texttt{Do you wish to continue? (Y/N): N}%
\end{minipage}} Note that your code must include all relevant exception handling. 

\subsection*{Task 1.5}

Write the program code for the functionality specified above.

\subsection*{Evidence 6 }

The program code for the functionality specified.\hfill{} {[}9{]}

{[}SPLIT\_HERE{]}
\item \textbf{{[}NJC/PRELIM/9597/2018/P1/Q2{]} }

A teacher is interested in exercise questions that perform division
over integers specified using different bases. Assist the teacher
by implementing a function that will allow him to check the answers
of such exercise questions. 

\subsection*{Task 2.1}

Write the function \texttt{universal\_base\_division}, which has the
following parameters (listed in order): 
\begin{itemize}
\item \texttt{dividend: STRING}
\item \texttt{dividend\_base: INTEGER }
\item \texttt{divisor: STRING }
\item \texttt{divisor\_base: INTEGER }
\item \texttt{result\_base: INTEGER }
\end{itemize}
The function will return two values (i.e., strings): 
\begin{itemize}
\item the quotient of the division operation represented in base: \texttt{result\_base}
\item the remainder of the division operation represented in base:\texttt{ result\_base }
\end{itemize}
Note that \texttt{dividend} and \texttt{divisor} both represent integer
values.

Note the following when implementing the above. 
\begin{itemize}
\item Each base must be an integer between 2 and 30. 
\item If any base specified does not satisfy the above, output: 

\texttt{\textquotedblleft You must specify a base between 2 and 30.\textquotedblright{} }
\end{itemize}

\subsection*{Evidence 7 }

The program code for the function \texttt{universal\_base\_division}.
\hfill{}{[}11{]}

\subsection*{Task 2.2 }

Evaluate your code by testing the following:

\texttt{universal\_base\_division(\textquotedbl mma01\textquotedbl ,23,\textquotedbl 30501\textquotedbl ,6,2)}

\subsection*{Evidence 8 }

A screenshot of the output from Task 2.2. \hfill{}{[}1{]}

\subsection*{Task 2.3}

Using the table below, define a list of at least five test cases appropriate
to thoroughly test the \texttt{universal\_base\_division} function. 
\noindent \begin{center}
\begin{tabular}{|c|c|c|}
\hline 
Input & Purpose of the test & Expected output\tabularnewline
\hline 
$\vdots$ & $\vdots$ & $\vdots$\tabularnewline
\hline 
\end{tabular}
\par\end{center}

\subsection*{Evidence 9}

The completed table of test cases. \hfill{}{[}3{]}

{[}SPLIT\_HERE{]}
\item \textbf{{[}NJC/PRELIM/9597/2018/P1/Q3{]} }

You have been tasked to implement a Doubly Linked List and a Binary
Search Tree using object-oriented programming (OOP). 

The implementations of these data structures are to adopt the use
of an array to store their node objects. Consequently, all link information
will correspond to integers representing the indices of the array
that stores these nodes. 

For example, root = 3 implies that actual root node is stored at index
3 of the array of nodes. Note that in order to indicate that no link
has been defined, the value -1 should be utilised. 

More specifically, you are to follow the following OOP design: 
\noindent \begin{center}
\begin{tabular}{cc|c|c|c|cc}
\cline{3-5} \cline{4-5} \cline{5-5} 
 &  & \multicolumn{3}{c|}{\texttt{HybridNode}} &  & \tabularnewline
\cline{3-5} \cline{4-5} \cline{5-5} 
 &  & \multicolumn{3}{l|}{\texttt{-data: OBJECT}} &  & \tabularnewline
 &  & \multicolumn{3}{l|}{\texttt{-link1: INTEGER}} &  & \tabularnewline
 &  & \multicolumn{3}{l|}{\texttt{-link2: INTEGER}} &  & \tabularnewline
\cline{3-5} \cline{4-5} \cline{5-5} 
 &  & \multicolumn{3}{l|}{\texttt{+constructor(OBJECT)}} &  & \tabularnewline
 &  & \multicolumn{3}{l|}{\texttt{+get\_data(): OBJECT}} &  & \tabularnewline
 &  & \multicolumn{3}{l|}{\texttt{+get\_link1(): INTEGER}} &  & \tabularnewline
 &  & \multicolumn{3}{l|}{\texttt{+set\_link1(INTEGER)}} &  & \tabularnewline
 &  & \multicolumn{3}{l|}{\texttt{+get\_link2(): INTEGER}} &  & \tabularnewline
 &  & \multicolumn{3}{l|}{\texttt{+set\_link2(INTEGER)}} &  & \tabularnewline
 &  & \multicolumn{3}{l|}{\texttt{+print()}} &  & \tabularnewline
\cline{3-5} \cline{4-5} \cline{5-5} 
 & \multicolumn{1}{c}{} & \multicolumn{1}{c}{$\uparrow$} & \multicolumn{1}{c}{} & \multicolumn{1}{c}{$\uparrow$} &  & \tabularnewline
\cline{1-3} \cline{2-3} \cline{3-3} \cline{5-7} \cline{6-7} \cline{7-7} 
\multicolumn{3}{|c|}{\texttt{DLLNode}} &  & \multicolumn{3}{c|}{\texttt{BSTNode}}\tabularnewline
\cline{1-3} \cline{2-3} \cline{3-3} \cline{5-7} \cline{6-7} \cline{7-7} 
\multicolumn{3}{|c|}{} &  & \multicolumn{3}{c|}{}\tabularnewline
\cline{1-3} \cline{2-3} \cline{3-3} \cline{5-7} \cline{6-7} \cline{7-7} 
\multicolumn{3}{|c|}{\texttt{+print()}} &  & \multicolumn{3}{c|}{\texttt{+print()}}\tabularnewline
\cline{1-3} \cline{2-3} \cline{3-3} \cline{5-7} \cline{6-7} \cline{7-7} 
\end{tabular}
\par\end{center}

\noindent \begin{center}
\begin{turn}{90}
\begin{tabular}{cc|c|c|c|cc}
\cline{3-5} \cline{4-5} \cline{5-5} 
 &  & \multicolumn{3}{c|}{\texttt{DataStructure}} &  & \tabularnewline
\cline{3-5} \cline{4-5} \cline{5-5} 
 &  & \multicolumn{3}{l|}{\texttt{-capacity: INTEGER}} &  & \tabularnewline
 &  & \multicolumn{3}{l|}{\texttt{-count: INTEGER}} &  & \tabularnewline
 &  & \multicolumn{3}{l|}{\texttt{-nodes: ARRAY <HybridNode>}} &  & \tabularnewline
\cline{3-5} \cline{4-5} \cline{5-5} 
 &  & \multicolumn{3}{l|}{\texttt{+constructor()}} &  & \tabularnewline
 &  & \multicolumn{3}{l|}{\texttt{+is\_empty(): BOOLEAN}} &  & \tabularnewline
 &  & \multicolumn{3}{l|}{\texttt{+is\_full(): BOOLEAN}} &  & \tabularnewline
 &  & \multicolumn{3}{l|}{\texttt{+next\_free(): INTEGER}} &  & \tabularnewline
 &  & \multicolumn{3}{l|}{\texttt{+print()}} &  & \tabularnewline
\cline{3-5} \cline{4-5} \cline{5-5} 
 & \multicolumn{1}{c}{} & \multicolumn{1}{c}{\texttt{$\uparrow$}} & \multicolumn{1}{c}{} & \multicolumn{1}{c}{\texttt{$\uparrow$}} &  & \tabularnewline
\cline{1-3} \cline{2-3} \cline{3-3} \cline{5-7} \cline{6-7} \cline{7-7} 
\multicolumn{3}{|c|}{\texttt{DLL}} &  & \multicolumn{3}{c|}{\texttt{BST}}\tabularnewline
\cline{1-3} \cline{2-3} \cline{3-3} \cline{5-7} \cline{6-7} \cline{7-7} 
\multicolumn{3}{|l|}{\texttt{-head: INTEGER}} &  & \multicolumn{3}{l|}{\texttt{-root: INTEGER}}\tabularnewline
\cline{5-7} \cline{6-7} \cline{7-7} 
\multicolumn{3}{|l|}{\texttt{-tail: INTEGER}} &  & \multicolumn{3}{l|}{\texttt{+constructor()}}\tabularnewline
\cline{1-3} \cline{2-3} \cline{3-3} 
\multicolumn{3}{|l|}{\texttt{+constructor()}} &  & \multicolumn{3}{l|}{\texttt{+insert(OBJECT)}}\tabularnewline
\multicolumn{3}{|l|}{\texttt{+insert\_front(OBJECT)}} &  & \multicolumn{3}{l|}{\texttt{+contains(OBJECT): BOOLEAN}}\tabularnewline
\cline{5-7} \cline{6-7} \cline{7-7} 
\multicolumn{3}{|l|}{\texttt{+insert\_back(OBJECT)}} & \multicolumn{1}{c}{} & \multicolumn{1}{c}{} &  & \tabularnewline
\multicolumn{3}{|l|}{\texttt{+contains(OBJECT): BOOLEAN}} & \multicolumn{1}{c}{} & \multicolumn{1}{c}{} &  & \tabularnewline
\multicolumn{3}{|l|}{\texttt{+delete(OBJECT): BOOLEAN}} & \multicolumn{1}{c}{} & \multicolumn{1}{c}{} &  & \tabularnewline
\cline{1-3} \cline{2-3} \cline{3-3} 
\end{tabular}
\end{turn}
\par\end{center}

Attribute/Method and their Description
\begin{itemize}
\item \texttt{HybridNode.constructor(OBJECT)} - Initialisation of a \texttt{HybridNode}
requires the input of the object to be stored. The given object is
to be stored in the attribute data. 
\item \texttt{HybridNode.print()} - This method should output the \texttt{data},
\texttt{link1} and \texttt{link2} values using the following format:
\texttt{DATA: <value>; LINK1: <value>; LINK2: <value> }
\item \texttt{DLLNode.link1, DLLNode.link2} - For each node in the doubly
linked list (i.e., \texttt{DLLNode}), the attributes \texttt{link1}
and \texttt{link2} are to be treated as the previous and next link
references respectively. 
\item \texttt{DLLNode.print()} - This polymorphed method should output the
\texttt{data}, \texttt{link1} and \texttt{link2} values using the
following format: \texttt{DATA: <value>; PREV: <value>; NEXT: <value> }
\item \texttt{BSTNode.link1, BSTNode.link2} - For each node in the binary
search tree (i.e., BSTNode), the attributes \texttt{link1} and \texttt{link2}
are to be treated as the left child and right chid link references
respectively.
\item \texttt{BSTNode.print()} - This polymorphed method should output the
\texttt{data}, \texttt{link1} and \texttt{link2} values using the
following format: \texttt{DATA: <value>; LEFT: <value>; RIGHT: <value> }
\item \texttt{DataStructure.capacity, DataStructure.count, DataStructure.nodes}
- The attributes of the \texttt{DataStructure} class are meant to
facilitate the use of the array of nodes. The \texttt{capacity} attribute
denotes the total number of nodes that can be stored (this is to be
defined on construction). The \texttt{count} attribute denotes the
number of nodes currently in the structure. And finally, the nodes
attribute corresponds to the array of nodes. Do note that the array
of nodes should be initialised to be empty, using null values, upon
creation -- i.e., any empty node should have a null value. 
\item \texttt{DataStructure. next\_free(): INTEGER} - This method search
the attribute nodes for the first instance of an empty cell (i.e.,
an index housing a null value), and then returns that index. This
is to facilitate the insertion of new nodes. 
\item \texttt{DataStructure. print()} - This methods prints the contents
of the data structure -- i.e., all the nodes, in the order in which
they appear in the array of nodes. Do note that empty cells of the
array should also be appropriately reflected. 
\item \texttt{DLL.head, DLL.tail} - The DLL class maintains both a reference
(i.e., index reference) to the start and end of the doubly linked
list. DLL.contains(OBJECT): BOOLEAN This method will return True if
the specified object can be found in the doubly linked list. Or else,
False is returned. 
\item \texttt{DLL.delete(OBJECT): BOOLEAN} This method will search for the
specified object. If the object is found, it is deleted and True is
returned. If it is not found, False is returned. 
\item \texttt{BST.contains(OBJECT): BOOLEAN} This method will return True
if the specified object can be found in the binary search tree. Or
else, False is returned.
\end{itemize}

\subsection*{Task 3.1}

Write the program code to implement the \texttt{HybridNode}, \texttt{DLLNode}
and \texttt{BSTNode} classes. 

\subsection*{Evidence 10 }

The program code for the \texttt{HybridNode}, \texttt{DLLNode} and
\texttt{BSTNode} classes.\hfill{} {[}5{]}

\subsection*{Task 3.2 }

Write the program code to implement the \texttt{DataStructure} class. 

\subsection*{Evidence 11}

The program code for the \texttt{DataStructure} class. \hfill{}{[}5{]}

\subsection*{Task 3.3 }

Write the program code to implement the \texttt{DLL} class. 

Then test it by executing the following blocks of instructions (in
the sequence specified) on an instance of it: 

\texttt{}%
\begin{tabular}{|c|l|}
\hline 
\multirow{7}{*}{\textbf{Block 1}} & \texttt{insert\_front(30)}\tabularnewline
 & \texttt{insert\_front(20)}\tabularnewline
 & \texttt{insert\_front(10)}\tabularnewline
 & \texttt{insert\_back(40)}\tabularnewline
 & \texttt{insert\_back(50)}\tabularnewline
 & \texttt{insert\_back(60)}\tabularnewline
 & \texttt{print()}\tabularnewline
\hline 
\end{tabular}\texttt{ }

\texttt{}%
\begin{tabular}{|c|l|}
\hline 
\multirow{5}{*}{\textbf{Block 2}} & \texttt{delete(30)}\tabularnewline
 & \texttt{delete(10)}\tabularnewline
 & \texttt{delete(60)}\tabularnewline
 & \texttt{delete(40)}\tabularnewline
 & \texttt{print()}\tabularnewline
\hline 
\end{tabular}\texttt{ }

\texttt{}%
\begin{tabular}{|c|l|}
\hline 
\multirow{7}{*}{\textbf{Block 3}} & \texttt{insert\_front(10)}\tabularnewline
 & \texttt{insert\_back(60)}\tabularnewline
 & \texttt{print(contains(100))}\tabularnewline
 & \texttt{print(contains(10))}\tabularnewline
 & \texttt{print(contains(20))}\tabularnewline
 & \texttt{print(contains(50))}\tabularnewline
 & \texttt{print(contains(60))}\tabularnewline
 & \texttt{print()}\tabularnewline
\hline 
\end{tabular}\texttt{ }

\texttt{}%
\begin{tabular}{|c|l|}
\hline 
\multirow{7}{*}{\textbf{Block 4}} & \texttt{delete(100)}\tabularnewline
 & \texttt{delete(10)}\tabularnewline
 & \texttt{delete(20) }\tabularnewline
 & \texttt{delete(50)}\tabularnewline
 & \texttt{delete(60)}\tabularnewline
 & \texttt{delete(100)}\tabularnewline
 & \texttt{print() }\tabularnewline
\hline 
\end{tabular}\texttt{ }

Note that each block of instructions is to be run in sequence. That
is, run the instructions in block 1, then block 2, etc., without resetting
the instance in between each block of instructions. 

\subsection*{Evidence 12 }

The program code for the \texttt{DLL} class. \hfill{}{[}12{]}

\subsection*{Evidence 13}

4 screenshots; each capturing the output from each of the blocks of
instructions. \hfill{}{[}4{]}

\subsection*{Task 3.4 }

Write the program code to implement the \texttt{BST} class. 

Then test it by executing the following blocks of instructions (in
the sequence specified) on an instance of it: 

\begin{tabular}{|c|l|}
\hline 
\multirow{8}{*}{\textbf{Block 1}} & \texttt{insert(50)}\tabularnewline
 & \texttt{insert(25)}\tabularnewline
 & \texttt{insert(35)}\tabularnewline
 & \texttt{insert(75)}\tabularnewline
 & \texttt{insert(85)}\tabularnewline
 & \texttt{insert(15)}\tabularnewline
 & \texttt{insert(65)}\tabularnewline
 & \texttt{print()}\tabularnewline
\hline 
\end{tabular} 

\begin{tabular}{|c|l|}
\hline 
\multirow{15}{*}{\textbf{Block 2}} & \texttt{print(contains(50))}\tabularnewline
 & \texttt{print(contains(25))}\tabularnewline
 & \texttt{print(contains(75))}\tabularnewline
 & \texttt{print(contains(19))}\tabularnewline
 & \texttt{print(contains(20))}\tabularnewline
 & \texttt{print(contains(21))}\tabularnewline
 & \texttt{print(contains(29))}\tabularnewline
 & \texttt{print(contains(30))}\tabularnewline
 & \texttt{print(contains(31)) }\tabularnewline
 & \texttt{print(contains(69))}\tabularnewline
 & \texttt{print(contains(70))}\tabularnewline
 & \texttt{print(contains(71))}\tabularnewline
 & \texttt{print(contains(79))}\tabularnewline
 & \texttt{print(contains(80))}\tabularnewline
 & \texttt{print(contains(81)) }\tabularnewline
\hline 
\end{tabular} 

Note that each block of instructions is to be run in sequence. That
is, run the instructions in block 1, then block 2, etc., without resetting
the instance in between each block of instructions. 

\subsection*{Evidence 14 }

The program code for the \texttt{BST} class.\hfill{} {[}7{]}

\subsection*{Evidence 15}

2 screenshots; each capturing the output from each of the blocks of
instructions.\hfill{} {[}2{]}

{[}SPLIT\_HERE{]}
\item \textbf{{[}NJC/PRELIM/9597/2018/P1/Q4{]} }

You are given the text file \texttt{STRINGS.TXT}, which contains multiple
entries of the strings \textquotedbl ABC\textquotedbl , \textquotedbl ACB\textquotedbl ,
\textquotedbl BAC\textquotedbl , \textquotedbl BCA\textquotedbl ,
\textquotedbl CAB\textquotedbl , \textquotedbl CBA\textquotedbl ,
with each such string stored on a separate line of the file. 

You must write the code to analyse the frequencies of various substrings
within the file.

The substrings in question correspond to all the possible strings
containing at least 2 letters from \textquotedblleft A\textquotedblright ,
\textquotedblleft B\textquotedblright{} and \textquotedblleft C\textquotedblright ,
such that there are no repeats of each of the letters.

You are to write the code for the function \texttt{generate\_substrings},
which takes in a string of unique letters (e.g., \textquotedblleft XYZ\textquotedblright ),
and then outputs an array containing all the possible strings comprised
of at least 2 letters from the unique letters specified in the input
string -- i.e., if the input is \textquotedblleft XYZ\textquotedblright ,
then the function should output {[}\textquotedblleft XY\textquotedblright ,
\textquotedblleft XZ\textquotedblright , \textquotedblleft YX\textquotedblright ,
\textquotedblleft YZ\textquotedblright , \textquotedblleft ZX\textquotedblright ,
\textquotedblleft ZY\textquotedblright , \textquotedblleft XYZ\textquotedblright ,
\textquotedblleft XZY\textquotedblright , \textquotedblleft YXZ\textquotedblright ,
\textquotedblleft YZX\textquotedblright , \textquotedblleft ZXY\textquotedblright ,
\textquotedblleft ZYX\textquotedblright {]}. 

\subsection*{Task 4.1 }

Write the program code for the function \texttt{generate\_substrings}. 

\subsection*{Evidence 16}

The program code for the function \texttt{generate\_substrings}. \hfill{}{[}10{]}

\subsection*{Evidence 17 }

A screenshot of the output for \texttt{generate\_substrings(\textquotedbl AGMSZ\textquotedbl )}.
\hfill{}{[}1{]}

You are next required to write 3 functions: 
\begin{itemize}
\item \texttt{starts\_with }
\item \texttt{contains }
\item \texttt{ends\_with}
\end{itemize}
Each of these functions takes in 2 string inputs, \texttt{source}
and \texttt{key}, and will search \texttt{source} for \texttt{key}
at the position indicated by the function name. If \texttt{key} is
found in the correct position within \texttt{source}, then the function
will return True, or else, it will return False. 

Your implementation of these 3 functions must be modular. 

\subsection*{Task 4.2 }

Write the program code for the functions \texttt{starts\_with}, \texttt{contains}
and \texttt{ends\_with}. 

\subsection*{Evidence 18 }

The program code for the functions \texttt{starts\_with}, \texttt{con}tains
and \texttt{ends\_with}.\hfill{} {[}10{]}

Use the functions implemented in \textbf{Task 2.1} and \textbf{Task
2.2} to determine the number of strings in the file \texttt{STRINGS.TXT}
that start with each of the valid substrings of \textquotedblleft ABC\textquotedblright{}
(based on the definition of valid substrings stated at the beginning
of this question). Then determine the number of strings in the file
\texttt{STRINGS.TXT} that end with each of the valid substrings of
\textquotedblleft ABC\textquotedblright . 

\subsection*{Task 4.3 }

Write the program code for the functionality specified above. 

\subsection*{Evidence 1}

The program code for the functionality specified above. \hfill{}{[}2{]}

\subsection*{Evidence 20 }

A screenshot of the output for the frequencies of strings in \texttt{STRINGS.TXT}
starting with the various substrings identified. \hfill{}{[}1{]}

\subsection*{Evidence 21}

A screenshot of the output for the frequencies of strings in \texttt{STRINGS.TXT}
ending with the various substrings identified. \hfill{}{[}1{]}

{[}SPLIT\_HERE{]}
\item \textbf{{[}NJC/PRELIM/9597/2018/P2/Q1{]} }

A fast food restaurant chain is considering going cashless by replacing
its cash registers with automated point of sale (POS) terminals. 

Management does realise that most customers still prefer cash transactions.
However, they would like to introduce stored-value smart cards, which
customers can either top-up online or at the actual restaurant outlets. 

As an alternative, they would also allow customers to pay using the
standard electronic payment service provided by banks -- i.e., Network
for Electronic Transfers (NETS). 

The current system typical utilises cash registers, which also allow
for NETS transactions.

Under the existing system, sales reports are generated based on products
(from the menu) that are sold. On occasion, surveys may also be performed
to determine customer satisfaction levels and menu preferences. Management
will then study the various pieces of information acquired in order
to make decisions about changes to the menu, and about any promotional
events and/or schemes. 

By utilising the stored value smart cards, the restaurant chain management
plans to be able to introduce customer loyalty rewards, special offers,
and other discount or promotional schemes. In particular, they would
like to utilise individual customer purchase data to make better decisions
concerning mutually beneficial schemes for both the customers and
the business. 

The management of the restaurant chain has secured the services of
a software development company to handle this project. 
\begin{enumerate}
\item Following an initial feasibility study, the project team is tasked
to analyse the current system in detail. Describe two actions that
the team may undertake in order to define the current system. You
must include all entities and items that would be investigated, and
how the investigation would be performed.\hfill{} {[}4{]}
\item The analysts in the project team must include a data flow diagram
for the existing system as part of the analysis phase. Draw a data
flow diagram for the existing system. \hfill{}{[}5{]}
\item State two other forms of diagrams that are generated during the analysis
phase, and state their purpose.\hfill{} {[}2{]}
\item Gantt and PERT charts are two types of project management tools that
are typically utilised for the planning of resource allocation. With
the use of examples under the current context, explain the benefits
and drawbacks of either of these charts.\hfill{} {[}4{]}
\item Following the analysis phase is the design phase, which will include
the systems requirements specification. Describe four elements of
the systems requirements specification for the system to be implemented.\hfill{}
{[}4{]}
\item Another outcome from the design phase is the testing documentation.
Describe the objective of this document, and its general contents.
\hfill{}{[}2{]}
\item The design team has opted to largely utilise a top-down testing approach
for this project. Describe top-down testing, and explain why this
choice might have been made for the development of this system. \hfill{}{[}2{]}
\item During the course of the project, several points were raised about
both the social and ethical issues associated with the project. Describe
two points that could have been raised. \hfill{}{[}4{]}
\item The final phase of the cycle involves the maintenance of the system.
Describe one relevant form of maintenance for the system described.
Justify your answer. \hfill{}{[}2{]}
\end{enumerate}
Suppose that the management of the restaurant also wished to extend
the system to also handle online orders. 
\begin{enumerate}
\item[(j)]  Explain the difference between a client-server and peer-to-peer
network, and explain which is would be relevant for the proposed online
system.\hfill{} {[}2{]}
\item[(k)]  Should the design team choose to implement a cloud computing solution,
explain how each type of cloud computing, if applicable, would be
relevant. \hfill{}{[}3{]}
\item[(l)]  Describe one security vulnerability there may exist in the online
ordering system, and then describe how it can be mitigated. \hfill{}{[}2{]}
\end{enumerate}
A decision was also made to standardise all user interfaces for orders.
Consequently, all devices used to make orders, including the POS terminals
within restaurants, and computers or mobile devices (anywhere, including
within the restaurant), were all are designed to use the exact same
interface. 
\begin{enumerate}
\item[(m)]  Describe one pro and one con of such a user interface design choice.
\hfill{}{[}2{]}
\item[(n)]  List at least two other user interface design principles that the
designers should adopt. \hfill{}{[}2{]}
\end{enumerate}
{[}SPLIT\_HERE{]}
\item \textbf{{[}NJC/PRELIM/9597/2018/P2/Q2{]} }
\begin{enumerate}
\item Describe the difference between the Bubble Sort and Insertion Sort
algorithms by tracing their use over the following array: 

{[}\textquotedblleft 7\textquotedblright , \textquotedblleft 6\textquotedblright ,
\textquotedblleft 15\textquotedblright , \textquotedblleft 11\textquotedblright ,
\textquotedblleft 1\textquotedblright {]}
\begin{enumerate}
\item Provide the Bubble Sort trace. \hfill{}{[}3{]}
\item Provide the Insertion Sort trace. \hfill{}{[}3{]}
\item Describe the difference between the two sorting algorithms. You should
use the traces performed in (i) and (ii) to do so. You should also
comment on the computational complexity of these algorithms. \hfill{}{[}2{]}
\end{enumerate}
\item A hash table is implemented with linear probing. The following function,
\texttt{search}, returns the object that has the specified \texttt{key}:

\noindent\begin{minipage}[t]{1\columnwidth}%
\texttt{01 FUNCTION search(Table, key) }

\texttt{02 \qquad{}i <- HASH(key)}

\texttt{03 \qquad{}WHILE Table{[}i, 1{]} <> key}

\texttt{04 \qquad{}\qquad{}i <- i + 1 }

\texttt{05 \qquad{}ENDWHILE }

\texttt{06 \qquad{}RETURN Table{[}i, 2{]}}

\texttt{07 ENDFUNCTION }%
\end{minipage}

Assuming that the specified key (and its associated object) exists
in the Hash Table, identify one problem with this implementation,
and then modify the function to fix the issue identified. \hfill{}
{[}2{]}
\item Describe the enqueue and dequeue methods for a circular queue. \hfill{}{[}2{]}
\end{enumerate}
{[}SPLIT\_HERE{]}
\item \textbf{{[}NJC/PRELIM/9597/2018/P2/Q3{]} }

The following is a recursive function.

\noindent\begin{minipage}[t]{1\columnwidth}%
\texttt{01 FUNCTION f(INTEGER a, INTEGER b) }

\texttt{02 \qquad{}IF b = 0 THEN }

\texttt{03 \qquad{}\qquad{}RETURN 1 }

\texttt{04 \qquad{}ELSE IF b = 1 THEN }

\texttt{05 \qquad{}\qquad{}RETURN a }

\texttt{06 \qquad{}ENDIF }

\texttt{07 \qquad{}x <- f(a, b DIV 2)}

\texttt{08 \qquad{}IF (b MOD 2) = 0 THEN }

\texttt{09 \qquad{}\qquad{}RETURN x {*} x }

\texttt{10 \qquad{}ELSE }

\texttt{11 \qquad{}\qquad{}RETURN x {*} a {*} x }

\texttt{12 \qquad{}ENDIF }

\texttt{13 ENDFUNCTION}%
\end{minipage}
\begin{enumerate}
\item Using the above function as an example, define recursion. \hfill{}{[}2{]}
\item Perform a dry-run of the above function using the call \texttt{f(3,
4)}. Show the resultant trace tree.\hfill{} {[}4{]}
\item State the purpose of the specified function. \hfill{}{[}1{]}
\item Identify one flaw in the above algorithm, and explain how this flaw
may be addressed. \hfill{}{[}2{]}
\item Draw a flowchart depicting an iterative version of the specified function.\hfill{}
{[}3{]}
\end{enumerate}
\item \textbf{{[}NJC/PRELIM/9597/2018/P2/Q4{]} }

A student in Malaysia is trying to access the website nationaljc.moe.edu.sg
using a computer in her school. 
\begin{enumerate}
\item Define a DNS server, and explain why it is important for the given
context. \hfill{}{[}2{]}
\item Describe how the web page is fetched, focusing on the various layers/protocols,
network devices and addressing that is utilised throughout this network
communication. Note that you may to use fictitious addresses, but
must use the correct address format for each case. \hfill{}{[}5{]}
\end{enumerate}
While accessing the website, the student also downloads certain information
that is accompanied by a digital signature. 
\begin{enumerate}
\item[(c)]  Explain how a digital signature works.\hfill{} {[}4{]}
\item[(d)]  Provide an example, using the given context, which would require
the use of a digital signature. \hfill{}{[}1{]}
\end{enumerate}
{[}SPLIT\_HERE{]}
\item \textbf{{[}NJC/PRELIM/9597/2018/P2/Q5{]} }

A researcher is compiling a bibliography of the various references
that have been reviewed. These references may either correspond to
conference/workshop papers, journal papers, or textbooks. In each
case, the researcher stores the following information: 
\begin{itemize}
\item Date of publication 
\item Title of publication 
\item Authors 
\end{itemize}
Depending on the type of publication, additional information may also
be stored.

For conference/workshop papers, the following is also stored: 
\begin{itemize}
\item Conference name
\item The relevant pages within the proceedings for the paper in question 
\end{itemize}
For journal papers, the following is also stored: 
\begin{itemize}
\item Journal name
\item The volume and issue numbers 
\item The relevant pages within the issue for the paper in question
\end{itemize}
For textbooks, the following is also stored:
\begin{itemize}
\item The publisher 
\end{itemize}
With each reference, the author also stores: 
\begin{itemize}
\item Meta tags for the paper, based on relevant topics within the paper 
\item A summary of the findings within the paper 
\end{itemize}
When a paper is added to the bibliography, the summary and meta tags
are first left empty, and only later added. All the other information
is available when the paper is added to the bibliography. 
\begin{enumerate}
\item Draw suitable UML class diagrams and relationships to model the scheme
above. You should also include appropriate attributes (with type information)
and methods for each class. \hfill{}{[}4{]}
\item Using the object-oriented programming concepts of encapsulation, inheritance
and polymorphism, explain how your implementation will facilitate
information hiding and, software reuse and code generalisation. \hfill{}{[}3{]}
\item Describe the relationship between objects, classes and instances.
\hfill{} {[}1{]}
\item If the researcher also wishes to link each paper based on citations
(i.e., based on the papers referenced within each paper), suggest
a relevant data structure design that may be utilised over the classes
you have defined. Justify your answer. \hfill{} {[}2{]}
\end{enumerate}
{[}SPLIT\_HERE{]}
\item \textbf{{[}NJC/PRELIM/9597/2018/P2/Q6{]} }

A bookshop has taken its business online and now only sells books
online. 

Each time a sale is made, the following information is captured: 
\begin{itemize}
\item customer reference code -- if the customer is already registered 
\item customer name, address and payment information -- if the customer
is new; a reference code will also be generated
\item books purchased by customer (in the current order) 
\end{itemize}
For each book the company sells, the following information is also
stored: 
\begin{itemize}
\item book ISBN 
\item book title 
\item book authors
\item book publisher 
\item book synopsis
\end{itemize}
Each customer may make as many purchases as he/she desires (even on
the same day). 

For each book, the company keeps track of the quantity in stock. 

Even if a book is no longer sold, the information on the book should
still be stored. 

The company wants to model its sales and inventory using a relational
database.
\begin{enumerate}
\item A database needs a number of tables to store the data for this application. 

Draw the Entity-Relationship (E-R) diagram to show the tables in third
normal form (3NF) and the relationships between them. \hfill{} {[}4{]}
\item A table description can be expressed as: 

\texttt{TableName (}\texttt{\uline{Attribute1}}\texttt{, Attribute2{*},
Attribute3, \dots{} ) }

The primary key is indicated by underlining one or more attributes.
Foreign keys are indicated by using a dashed underline/asterisk. Using
the information given, write table descriptions for the tables you
identified in \textbf{part (a)}. \hfill{}{[}6{]}
\item Like many other online bookshops, the company also provides a list
of recommended books for each customer. Describe the query (or queries)
that would facilitate this. \hfill{}{[}2{]}
\item Relational databases are an improvement over the use of flat files.
One way to explain these improvements are based on data anomalies.
Describe two such anomalies, and exemplify them using the given context.
\hfill{} {[}2{]}
\end{enumerate}
{[}SPLIT\_HERE{]}
\end{enumerate}
 
\end{document}
