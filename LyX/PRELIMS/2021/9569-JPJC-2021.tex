%% LyX 2.3.6.1 created this file.  For more info, see http://www.lyx.org/.
%% Do not edit unless you really know what you are doing.
\documentclass[english]{article}
\usepackage[T1]{fontenc}
\usepackage[latin9]{inputenc}
\usepackage{geometry}
\geometry{verbose,tmargin=2.5cm,bmargin=2.5cm,lmargin=2.5cm,rmargin=2.5cm}

\makeatletter

%%%%%%%%%%%%%%%%%%%%%%%%%%%%%% LyX specific LaTeX commands.
%% Because html converters don't know tabularnewline
\providecommand{\tabularnewline}{\\}

\makeatother

\usepackage{babel}
\begin{document}
{[}SPLIT\_HERE{]}
\begin{enumerate}
\item \textbf{{[}JPJC/PRELIM/9569/2021/P2/Q1{]} }

Your program code and output for Task 1 should be saved in a single
\texttt{.ipynb} file. 

Name your Jupyter Notebook as \texttt{TASK1\_<your name>\_<class>\_<index
number>.ipynb }

The file \texttt{marathon.CSV} contains the full list of athletes
who took part in the 42.195km marathon race. The first line in the
file is the heading for the records. Each subsequent line is a record
of a runner in the form: 
\noindent \begin{center}
\texttt{<name of athlete>,<country code>,<timing in h:mm:ss>} 
\par\end{center}

For example, \texttt{Abdi ABDIRAHMAN,USA,2:18:27} 

Several athletes did not complete the race and their timing is recorded
as 'DNF' to indicate \textquoteleft did not finish\textquoteright . 

For example: \texttt{Alemu BEKELE,BRN,DNF }

\subsubsection*{Task 1.1 }

Write program code to find out the number of athletes who did not
finish the race and output the following three statements: 

\noindent %
\noindent\begin{minipage}[t]{1\columnwidth}%
\texttt{Number of DNF: x }

\texttt{Total number of athletes: y }

\texttt{Percentage of athletes who finished race: z }%
\end{minipage}

\texttt{x} is the number of athletes who did not finish the race, 

\texttt{y} is the total number of athletes who participated in the
marathon race, 

and \texttt{z} is the percentage (rounded to 1 decimal place) of athletes
who finished the race.\hfill{} {[}7{]}

\subsubsection*{Task 1.2 }

Write a function \texttt{insertionSort} that takes an unsorted list
as a parameter, sorts the list using the insertion sort algorithm,
and returns the sorted list. \hfill{}{[}6{]}

\subsubsection*{Task 1.3 }

By making use of the \texttt{insertionSort} function from Task 1.2,
or otherwise, find out the top 20 athletes and list them in order
of rank under the heading (Rank, Country, Name, Timing). \hfill{}{[}5{]} 

Sample Output: 

\noindent\begin{minipage}[t]{1\columnwidth}%
\texttt{Rank Country Name ~~~~~~Timing }

\texttt{1 ~~~ABC ~~~~Harry TAN ~2:08:38 }

\texttt{2 ~~~XYZ ~~~~Andy LEE ~~2:09:58 }

\texttt{3 ~~~... ~~~~... ~~~~~~~... }%
\end{minipage}

{[}SPLIT\_HERE{]}
\item \textbf{{[}JPJC/PRELIM/9569/2021/P2/Q2{]} }

Name your Python file as 

\texttt{TASK2\_<your name>\_<class>\_<index number>.py} 

You will build a simplified, one-player version of the classic board
game Battleship. Refer to \texttt{Task2\_Client\_SampleOutput.JPG}
for the sample output. 

In this version of the game: 
\begin{itemize}
\item The\textbf{ server program} initialises a grid measuring 4 metres
by 5 metres. 
\item The grid is to be represented on the screen by a rectangular grid. 
\item Each square metre of the grid is represented by an \texttt{x}-coordinate
and a \texttt{y}-coordinate. 
\item The top left square metre of the grid display has \texttt{x = 0} and
\texttt{y = 0}. 
\item Use \textquotedbl\texttt{O}\textquotedbl{} to represent an unoccupied
space. 
\item There will be a single ship hidden in a random location. 
\item The ship only occupies one square metre of the grid. 
\item The \textbf{client program} allows the player to input guesses to
sink the ship. After each guess, \textquotedbl\texttt{X}\textquotedbl{}
represents the incorrect position guessed while \textquotedbl\texttt{S}\textquotedbl{}
represents the sunken ship. 
\item The ship will not appear if it has not been sunk. 
\item The game is terminated by the server program when \textbf{three} guesses
are used or the player has guessed the correct position. 
\end{itemize}

\subsubsection*{Task 2.1 }

Write the code for the following functions and procedures for the
\textbf{server program}. \hfill{}{[}8{]}

\begin{tabular}{|l|l|}
\hline 
\textbf{Subroutine Header} & \textbf{Description}\tabularnewline
\hline 
\texttt{InitialiseGrid(): ARRAY{[}0:3, 0:4{]} OF CHAR } & Initialises a 4 by 5 two-dimensional array and returns the array.
Use \textquotedbl\texttt{O}\textquotedbl{} to represent blank.\tabularnewline
\hline 
\texttt{DisplayGrid(arr: ARRAY{[}0:3, 0:4{]} OF CHAR) } & Sends the encoded grid to the client. \tabularnewline
\hline 
\texttt{ValidateRow(row: INTEGER): BOOLEAN } & Returns \texttt{True} if the row is valid and \texttt{False} otherwise.\tabularnewline
\hline 
\texttt{ValidateCol(col: INTEGER): BOOLEAN } & Returns \texttt{True} if the column is valid and \texttt{False} otherwise. \tabularnewline
\hline 
\texttt{CheckResult(row,col:INTEGER): BOOLEAN } & Checks if the guess is correct. If the guess is correct, uses \textquotedbl\texttt{S}\textquotedbl{}
to represent the sunken ship. Otherwise, uses \textquotedbl\texttt{X}\textquotedbl{}
to represent incorrect guess. Returns \texttt{True} for the correct
guess and returns \texttt{False} for the wrong guess.\tabularnewline
\hline 
\end{tabular}

\subsubsection*{Task 2.2 }

The following client program is given in \texttt{Battleship\_Client.py}. 

\noindent\begin{minipage}[t]{1\columnwidth}%
\texttt{import socket }

\texttt{client\_socket = socket.socket() }

\medskip{}

\texttt{address = input('Enter IPv4 address of server: ') }

\texttt{port = int(input('Enter port number of server: ')) }

\texttt{client\_socket.connect((address, port)) }\medskip{}

\texttt{while True: }

\texttt{\qquad{}data = client\_socket.recv(1024)}

\texttt{\qquad{}if b\textquotedbl Enter\textquotedbl{} in data: }

\texttt{\qquad{}\qquad{}choice = input(data.decode()) }

\texttt{\qquad{}\qquad{}client\_socket.sendall(choice.encode()) }

\texttt{\qquad{}\qquad{}print() }

\texttt{\qquad{}else: }

\texttt{\qquad{}\qquad{}print(data.decode()) }

\texttt{\qquad{}\qquad{}if b\textquotedbl GAME OVER\textquotedbl{}
in data or b\textquotedbl YOU WON\textquotedbl{} in data: }

\texttt{\qquad{}\qquad{}\qquad{}break }\medskip{}

\texttt{client\_socket.close() }%
\end{minipage}

Write the corresponding server program that: \hfill{}{[}16{]}
\begin{itemize}
\item Instantiates the server socket. 
\item Binds the socket to localhost and port number 6789. 
\item Listens for incoming request, accepts incoming request and establishes
connection with the client. 
\item Generates a random position for the hidden ship. 
\item Sends a \textquotedblleft Welcome to Battleship!\textquotedblright{}
message to the client. 
\item Uses the subroutines coded in Task 2.1 to play the game. 
\item Sends \textquotedbl\texttt{YOU WON!}\textquotedbl{} to the client
for the correct guess and ends the game. 
\item Sends \textquotedbl\texttt{GAME OVER\dots }\textquotedbl{} to the
client if three guesses are used. 
\item Closes the sockets.
\end{itemize}
{[}SPLIT\_HERE{]}
\item \textbf{{[}JPJC/PRELIM/9569/2021/P2/Q3{]} }

Your program code and output for Task 3 should be saved in a single
\texttt{.ipynb} file. 

Name your Jupyter Notebook as \texttt{TASK3\_<your name>\_<class>\_<index
number>.ipynb }

JP Fitness Club is a gym that keeps details of its members. You are
tasked to help the club manage the members\textquoteright{} details
and store them in a SQL database. 

There are two types of membership: normal and annual. Each member
has a unique membership number, first name, surname, contact number
and last visit date recorded. 

A normal member deposits a selected amount into their account. Each
time the member visits the gym, the entrance fee is deducted from
the amount held in his account. The member may top up the account
any time. 

An annual member pays a fixed fee per year, starting from the date
of registration. He may then visit the gym any number of times for
the whole year without paying the entrance fee. 

Three classes have been identified: \texttt{Member}, \texttt{NormalMember},
\texttt{AnnualMember}. 

The class \texttt{Member} has these attributes and methods defined
on it. 

\begin{tabular}{|c|c|c|}
\hline 
\textbf{Attribute} & \textbf{Data type} & \textbf{Description}\tabularnewline
\hline 
\texttt{memberID} & String  & 8 digit membership number. First four digits represent the year of
joining gym and last four digits are used to make the \texttt{memberID}
unique, e.g. \texttt{20210357}. \tabularnewline
\hline 
\texttt{first\_name } & String  & First name of member, at most 15 characters.\tabularnewline
\hline 
\texttt{surname } & String  & Surname of member, at most 15 characters.\tabularnewline
\hline 
\texttt{contact\_number } & String  & 8 digit contact number.\tabularnewline
\hline 
\texttt{last\_visit } & String  & The date when member last visited the gym, in the format\texttt{ YYYY-MM-DD}.
Initialises to today\textquoteright s date. \tabularnewline
\hline 
\texttt{memberType} & String  & Indicates type of membership, either \textquotedblleft normal\textquotedblright{}
or \textquotedblleft annual\textquotedblright . Initialises to None. \tabularnewline
\hline 
\textbf{Method} & \textbf{Return type} & \textbf{Description}\tabularnewline
\hline 
\texttt{showMember() } & None  & Outputs member\textquoteright s membership number, first name, surname,
contact number and last visit date.\tabularnewline
\hline 
\texttt{isActive() } & Boolean  & Indicates whether a member is active or not. Returns True if the last
visit date is within 30 days, otherwise returns False.\tabularnewline
\hline 
\end{tabular}

The class \texttt{NormalMember} inherits from \texttt{Member} and
has these additional attributes and methods defined on it. 

\begin{tabular}{|c|c|c|}
\hline 
\textbf{Attribute} & \textbf{Data type} & \textbf{Description}\tabularnewline
\hline 
stored\_value & Float & stored in member\textquoteright s account. Display in 2 decimal places.
Initialise to \$0.00.\tabularnewline
\hline 
\textbf{Method} & \textbf{Return type} & \textbf{Description}\tabularnewline
\hline 
\texttt{showMember()} & None & Output \texttt{memberType} in addition to member\textquoteright s
membership number, first name, surname, contact number and last visit
date. \tabularnewline
\hline 
\end{tabular}

The class \texttt{AnnualMember} also inherits from \texttt{Member}
and has these additional attributes and methods defined on it. 

\begin{tabular}{|c|c|c|}
\hline 
\textbf{Attribute} & \textbf{Data type} & \textbf{Description}\tabularnewline
\hline 
\texttt{annual\_fee } & Integer  & Annual fee paid by member. Initialises to \$500\tabularnewline
\hline 
\texttt{date\_register} & String & Date when member joins annual membership, in format \texttt{YYYY-MM-DD}.
Initialises to today\textquoteright s date.\tabularnewline
\hline 
\textbf{Method} & \textbf{Return type} & \textbf{Description}\tabularnewline
\hline 
\texttt{showMember()} & None & Outputs \texttt{memberType} in addition to member\textquoteright s
membership number, first name, surname, contact number and last visit
date. \tabularnewline
\hline 
\end{tabular}

\subsubsection*{Task 3.1 }

Write program code using object-oriented programming for the classes
\texttt{Member}, \texttt{NormalMember}, and \texttt{AnnualMember}.
Include all the identifiers stated and other appropriate methods to
access and modify the attributes. \hfill{}{[}15{]}

\subsubsection*{Task 3.2 }

The text file, \texttt{members.TXT}, contains data items for a number
of members. Each data item is separated by a comma, with each member\textquoteright s
data on a new line as follows: 
\begin{itemize}
\item membership number 
\item first name 
\item surname
\item contact number 
\item member type 
\end{itemize}
Write program code to read in the information from the text file,
\texttt{member.txt}, creating an instance of the appropriate class
for each member, storing each instance in the same list.

Run \texttt{showMember} method to display each of the members\textquoteright{}
details. \hfill{}{[}5{]}

\subsubsection*{Task 3.3 }

The members\textquoteright{} details are to be stored in a SQL database. 

The file \texttt{JPgym.SQL} contains the SQL code to create database
\texttt{JPgym.db} with the single table, \texttt{Member}. The table
will have the following fields: 
\begin{itemize}
\item MemberID -- primary key, text 
\item FirstName -- first name of member, text 
\item Surname -- surname of member, text 
\item ContactNo -- contact number of member, text 
\item LastVisit -- date that member last visited gym, text 
\item MemberType -- indicates \textquoteleft normal\textquoteright{} or
\textquoteleft annual\textquoteright{} membership, text 
\end{itemize}
Copy and paste this SQL code into your Python program to create the
database and table. 

Also, write program code in Python to insert all the information from
the file into the \texttt{JPgym.db} database. 

Run your program and check that all information has been inserted
using SQLite database software. \hfill{}{[}6{]}

\subsubsection*{Task 3.4 }

Write a SQL query code in Python to display all members with \textquotedblleft normal\textquotedblright{}
membership in ascending order of \texttt{FirstName}. 

Display only the following fields from the query: \texttt{FirstName},
\texttt{Surname}, \texttt{ContactNo}. \hfill{} {[}3{]}

{[}SPLIT\_HERE{]}
\item \textbf{{[}JPJC/PRELIM/9569/2021/P2/Q4{]}}

JP Mobile sells mobile phones and manages its inventory using a NoSQL
database. Information about the mobile phones is stored in the JSON
file \texttt{items.JSON}. 

The following fields are recorded: 
\begin{itemize}
\item brand of mobile phone, 
\item model, 
\item colour(s) available, 
\item price in dollar, 
\item quantity in stock. 
\end{itemize}

\subsubsection*{Task 4.1 }

Write program code to import the information from the JSON file into
a MongoDB database. Save the information under the \texttt{phone}
collection in the \texttt{jp\_mobile} database. Ensure that the collection
only stores the information from the JSON file. 

Save your program code as \texttt{TASK4\_<your name>\_<class>\_<index
number>.py }\hfill{}{[}4{]}

\subsubsection*{Task 4.2 }

The shop decides to include one or more free gifts for new batches
of mobile phones it sells. 

Write program code for a user to insert information of a mobile phone
by getting user input of the following: brand, model, colour, price,
quantity, free gift(s). 

Your code should allow user to input one or more free gifts. 

If a phone\textquoteright s brand, model and colour already exists,
add the new quantity to the existing quantity in the database, and
replace the existing price with the new price. 

Run your program and insert the following 2 documents: 
\noindent \begin{center}
\begin{tabular}{|c|c|c|c|c|c|c|}
\hline 
No. & Brand  & Model  & Colour  & Price  & Quantity  & Free gift(s)\tabularnewline
\hline 
1  & orange  & 22  & black  & 900  & 11  & power bank\tabularnewline
\hline 
2  & solo  & A33  & red  & 1300  & 7  & power bank, earbuds\tabularnewline
\hline 
\end{tabular}
\par\end{center}

Add your program code to \texttt{TASK4\_<your name>\_<class>\_<index
number>.py} \hfill{}{[}7{]}

\subsubsection*{Task 4.3 }

Write a function \texttt{display\_all} that will display all the information
in the \texttt{phone} collection under these fields: \texttt{brand},
\texttt{model}, \texttt{colour}, \texttt{price}, \texttt{quantity},
\texttt{free gift(s)}. 

If no free gift comes with the phone, print a \texttt{None} statement.

Include a final statement that shows the total number of documents
in the collection. 

Run the \texttt{display\_all} function to show your output. 

Add your program code to \texttt{TASK4\_<your name>\_<class>\_<index
number>.py} \hfill{}{[}6{]}

\subsubsection*{Task 4.4}

The shop uses a web browser to display the database content. The manager
wants to filter the mobile phones by \texttt{brand} and display the
results in a web browser. 

Write additional Python code and the necessary files to create a web
application that 
\begin{itemize}
\item receives a \texttt{brand} string from a HTML form, then 
\item creates and returns a HTML document that enables the web browser to
display an ordered list of mobile phones sorted by \texttt{price}. 
\end{itemize}
For each document, the web page should include the: 
\begin{itemize}
\item brand as the heading 
\item model, 
\item colour, 
\item price 
\end{itemize}
Save your program as 

\texttt{TASK4\_4\_<your name>\_<class>\_<index number>.py} 

with any additional files / sub-folders as needed in a folder named 

\texttt{TASK4\_4\_<your name>\_<class>\_<index number>} 

Run the web application and input brand as \textquoteleft \texttt{solo}\textquoteright{}
in the webpage.

Save the output of the program as 

TASK4\_4\_<your name>\_<class>\_<index number>.html\hfill{} {[}12{]}

{[}SPLIT\_HERE{]}
\end{enumerate}

\end{document}
