%% LyX 2.3.6.1 created this file.  For more info, see http://www.lyx.org/.
%% Do not edit unless you really know what you are doing.
\documentclass[english]{article}
\usepackage[T1]{fontenc}
\usepackage[latin9]{inputenc}
\usepackage{geometry}
\geometry{verbose,tmargin=2.5cm,bmargin=2.5cm,lmargin=2.5cm,rmargin=2.5cm}
\usepackage{color}
\usepackage{setspace}
\PassOptionsToPackage{normalem}{ulem}
\usepackage{ulem}

\makeatletter

%%%%%%%%%%%%%%%%%%%%%%%%%%%%%% LyX specific LaTeX commands.
%% Because html converters don't know tabularnewline
\providecommand{\tabularnewline}{\\}

\makeatother

\usepackage{babel}
\begin{document}
{[}SPLIT\_HERE{]}
\begin{enumerate}
\item \textbf{{[}ALVL/9569/2021/P1/Q1{]} }

A programmer is writing software for a baker who makes celebration
(wedding and birthday) cakes to order. All cakes are circular.

The programmer wishes to use object-oriented programming to model
the order book and calculate the total price for a cake. 

For every cake. the following items of data are recorded: 
\begin{itemize}
\item an order number 
\item the customer\textquoteright s name \textquoteleft{} 
\item the customer's telephone number 
\item diameter (in cm) 
\item any special requirements (eg. must not contain nuts) 
\item standard price (in S). 
\end{itemize}
For wedding cakes the number of layers, a maximum of four. is also
recorded. The total price is the number of layers multiplied by the
standard price. 

For birthday cakes the number of candles and price of a candle is
recorded. The total price is the standard price plus the number of
candles multiplied by the price of a candle. 
\begin{enumerate}
\item Draw a class diagram that shows the following for the situation described
above:
\begin{itemize}
\item the superclass
\item any subclasses
\item inheritance
\item properties
\item appropriate methods. \hfill{}{[}12{]}
\end{itemize}
\item Name two suitable validation techniques that might be applied to the
number of layers on a wedding cake.\hfill{} {[}2{]}
\item Explain inheritance using examples from this situation. \hfill{}{[}4{]}
\end{enumerate}
{[}SPLIT\_HERE{]}
\item \textbf{{[}ALVL/9569/2021/P1/Q2{]} }

A construction company provides employees to work on projects fro
clients. Each employee has a specific skill.
\begin{itemize}
\item Each project has an ID number and a title.
\item Each employee has a name, ID number and a SINGLE skill.
\item Each employee can work on a number of projects.
\item Each skill has an ID number, name and cost per hour.
\item The cost to the client for each employee is the employee's cost per
hour multiplied by the number of hours worked on the project.
\end{itemize}
This table shows typical data from which server charges can be calculated. 
\noindent \begin{center}
\begin{tabular}{|c|c|c|c|c|c|c|c|}
\hline 
Project ID & Project Title & Employee ID & Employee Name & Skill ID & Skill Name & Cost per Hour (\$) & Hours worked\tabularnewline
\hline 
\hline 
1 & New roof  & 1 & Smith & 1 & Carpentry & 40 & 45\tabularnewline
\hline 
1 & New roof  & 2 & Jones & 1 & Carpentry & 40 & 30\tabularnewline
\hline 
1 & New roof  & 3 & Roberts & 2 & Bricklaying & 45 & 12\tabularnewline
\hline 
2 & Refurbish pool & 4 & Harrison & 3 & Electrical & 50 & 15\tabularnewline
\hline 
2 & Refurbish pool & 5 & Harris & 4 & Plastering & 42 & 20\tabularnewline
\hline 
2 & Refurbish pool & 6 & Patel & 5 & Tiling & 30 & 35\tabularnewline
\hline 
2 & Refurbish pool & 7 & Staples & 50 & Tiling & 30 & 42\tabularnewline
\hline 
3 & Replace kitchen & 2 & Jones & 1 & Carpentry & 40 & 20\tabularnewline
\hline 
3 & Replace kitchen & 5 & Harris & 4 & Plastering & 42 & 14\tabularnewline
\hline 
3 & Replace kitchen & 4 & Harrison & 3 & Electrical & 50 & 30\tabularnewline
\hline 
3 & Replace kitchen & 8 & Charles & 5 & Tiling & 30 & 17\tabularnewline
\hline 
\end{tabular}
\par\end{center}
\begin{enumerate}
\item Explain, giving an example, whether the above table is in first normal
form (1NF). \hfill{}{[}2{]}
\end{enumerate}
The company wants to construct a relational database to store the
data shown in the table on page 3. 

The following tables contain the data:

\begin{tabular}{|c|c|}
\hline 
\multicolumn{2}{|c|}{Region}\tabularnewline
\hline 
Project ID & Project Title\tabularnewline
\hline 
1 & New roof\tabularnewline
\hline 
2 & Refurbish pool\tabularnewline
\hline 
3 & Replace kitchen\tabularnewline
\hline 
\end{tabular} %
\begin{tabular}{|c|c|c|c|c|}
\hline 
\multicolumn{5}{|c|}{Employee}\tabularnewline
\hline 
Employee ID & Employee Name & Skill ID & Skill Name & Cost per Hour (\$)\tabularnewline
\hline 
1 & Smith & 1 & Carpentry & 40\tabularnewline
\hline 
2 & Jones & 1 & Carpentry & 40\tabularnewline
\hline 
3 & Roberts & 2 & Bricklaying & 45\tabularnewline
\hline 
4 & Harrison & 3 & Electrical & 50\tabularnewline
\hline 
5 & Harris & 4 & Plastering & 42\tabularnewline
\hline 
6 & Patel & 5 & Tiling & 30\tabularnewline
\hline 
7 & Staples & 50 & Tiling & 30\tabularnewline
\hline 
8 & Charles & 5 & Tiling & 30\tabularnewline
\hline 
\end{tabular}

\begin{tabular}{|c|c|c|}
\hline 
\multicolumn{3}{|c|}{ProjectEmployee}\tabularnewline
\hline 
Project ID & Employee ID & Hours Worked\tabularnewline
\hline 
1 & 1 & 45\tabularnewline
\hline 
1 & 2 & 30\tabularnewline
\hline 
1 & 3 & 12\tabularnewline
\hline 
2 & 4 & 15\tabularnewline
\hline 
2 & 5 & 20\tabularnewline
\hline 
2 & 6 & 35\tabularnewline
\hline 
2 & 7 & 42\tabularnewline
\hline 
3 & 2 & 20\tabularnewline
\hline 
3 & 5 & 14\tabularnewline
\hline 
3 & 4 & 30\tabularnewline
\hline 
3 & 8 & 17\tabularnewline
\hline 
\end{tabular}
\begin{enumerate}
\item[(b)]  Explain why the table \textbf{Employee} is not in third normal form
(3NF). \hfill{}{[}2{]}
\item[(c)]  A table description can be expressed as:

\texttt{TableName (Attributel, Attribute2, Attribute3, ...)}

The primary key is indicated by underlining one or more attributes.

Write table descriptions for two tables to hold the data from the
\textbf{Employee} table each of which are in third normal form (3NF).
\hfill{}{[}3{]}
\item[(d)]  State the primary key for the table \textbf{ProjectEmployee}.\hfill{}
{[}1{]}
\item[(e)]  Draw an entity-relationship (ER) diagram showing the necessary four
tables and the relationships between them.\hfill{}{[}4{]}
\end{enumerate}
For each project one employee is nominated to be the Project Manager
\begin{enumerate}
\item[(f)]  Explain the change that needs to be made to the existing table design
to allow the Project Manager on each project to be identified.\hfill{}
{[}2{]}
\end{enumerate}
The cost of employing an electrician increases to \$52 per hour. The
client receives an invoice, at the end of the project, showing the
hours worked and charge for each skill. An employee with the skill
Electircal can be called an electrician.
\begin{enumerate}
\item[(g)]  Explain a problem that may arise if the \textbf{Cost per Hour (\$)}
field for Electrical in the \textbf{Employee} table is changed from
\$50 to \$52. \hfill{}{[}2{]}
\end{enumerate}
An employee with the skill Tiling can be called a tiler.
\begin{enumerate}
\item[(h)] Write an SQL query to output the names and running worked of the
tilers who worked on the Refurbish pool project, in descending hours
of \textbf{Hours Worked}.\hfill{}{[}7{]}
\end{enumerate}
The video game company maintains a table of employee names and addresses,
so the company can send letters to them.

The company also maintains a table of employee bank account details,
so monthly payments for the game can be transferred automatically
to their bank accounts.
\begin{enumerate}
\item[(i)] ) State \textbf{four} actions the construction company must take
regarding the collection, disclosure and use of this data under the
Personal Data Protection Act. \hfill{}{[}4{]}
\end{enumerate}
{[}SPLIT\_HERE{]}
\item \textbf{{[}ALVL/9569/2021/P1/Q3{]} }
\begin{enumerate}
\item Explain how a denial of service (DOS) attack can compromise an internet
server. \hfill{}{[}2{]}
\end{enumerate}
A news website posts an article that attracts unusually large worldwide
attention. The monitoring software, running on the news website server.
warns the system administrator that the site might be the victim of
a denial of service attack.
\begin{enumerate}
\item[(b)]  State \textbf{two} reasons why the monitoring software generates
the warning. \hfill{} {[}2{]}
\end{enumerate}
A server connected to the internet provides web hosting, file transfer
and email services. Clients send requests to this server using an
internet protocol.
\begin{enumerate}
\item[(c)]  Explain how requests arriving at this server are handled. \hfill{}{[}2{]}
\end{enumerate}
A firewall is placed between the internet and the sewer.
\begin{enumerate}
\item[(d)]  Explain how the firewall may manage traffic between the server and
the internet. \hfill{}{[}2{]}
\item[(e)]  Describe how a digital signature can be used to give confidence
that a received message has not been altered. \hfill{} {[}6{]}
\item[(f)]  State \textbf{two} authentication techniques to limit access to
a network application. \hfill{}{[}2{]}
\end{enumerate}
{[}SPLIT\_HERE{]}
\item \textbf{{[}ALVL/9569/2021/P1/Q4{]} }

A programmer is writing code to accept, store and process a number
of readings from sensors monitoring an experiment. Depending on conditions,
the experiment can generate between hundreds and hundreds of thousands
of readings. These readings are stored in a data structure. The programmer
has the option of using either a static or a dynamic data structure.
\begin{enumerate}
\item Explain the advantage of selecting a dynamic over static data structure
in terms of memory allocation. \hfill{} {[}2{]}
\item State two problems that might arise if a dynamic data structure is
used in terms of memory allocation. \hfill{}{[}2{]}
\end{enumerate}
The readings are processed in the order they are stored.
\begin{enumerate}
\item[(c)]  Identify a suitable data structure and explain your choice. \hfill{}{[}2{]}
\end{enumerate}
{[}SPLIT\_HERE{]}
\item \textbf{{[}ALVL/9569/2021/P1/Q5{]} }

A hashing algorithm is to be used to locate a record within a hash
table.
\begin{enumerate}
\item State three features that a good hashing algorithm will possess.\hfill{}
{[}3{]}
\item Explain how two different records hashing to the same location can
be managed. \hfill{}{[}2{]}
\end{enumerate}
A social media company has to store and maintain huge quantities of
data.

The social media company hashes a user's ID to locate data for that
user\textquoteleft s account.
\begin{enumerate}
\item[(c)]  Explain the advantage of using a hash table, in this situation.
rather than linear search to locate a record.\hfill{} {[}2{]}
\item[(d)]  Explain the disadvantage of a binary search in this situation.\hfill{}
{[}2{]}
\end{enumerate}
{[}SPLIT\_HERE{]}
\item \textbf{{[}ALVL/9569/2021/P1/Q6{]} }

ASCII allows characters to be stored in memory by associating a number
with each character. In memory. that number is stored as a pattern
of bits. 

The decimal value representing \textquoteleft A\textquoteright{} in
ASCII is 65. 
\begin{enumerate}
\item Represent this value in:
\begin{enumerate}
\item Binary \hfill{}{[}1{]}
\item Hexadecimal. \hfill{}{[}1{]}
\end{enumerate}
\end{enumerate}
The decimal value representing \textquoteleft a\textquoteright{} in
ASCII is 97. 
\begin{enumerate}
\item[(b)]  State the hexadecimal value that must be added to the ASCII code
for \textquoteleft A\textquoteleft{} to convert it to the ASCII code
for 'a\textquoteleft . \hfill{}{[}1{]}
\end{enumerate}
Unicode is another method of encoding characters.
\begin{enumerate}
\item[(c)]  {}
\begin{enumerate}
\item State the values that are common to both ASCII and Unicode.\hfill{}
{[}1{]}
\item Explain what advantage Unicode has over ASCII. \hfill{}{[}2{]}
\end{enumerate}
\end{enumerate}
{[}SPLIT\_HERE{]}
\item \textbf{{[}ALVL/9569/2021/P1/Q7{]} }

The nodes of a binary search tree holding names in alphabetical order
are stored in the elements of an array, \texttt{Names}. 

Each element of the array \texttt{Names} comprises three parts: a
left pointer, the data and a right pointer. 
\noindent \begin{center}
\texttt{}%
\begin{tabular}{|c|c|c|}
\hline 
\texttt{\textbf{LPtr}} & \texttt{\textbf{Data}} & \texttt{\textbf{RPtr}}\tabularnewline
\hline 
\end{tabular} 
\par\end{center}

The pointers contain the array index of a node to either the left
or right of the current node. \textbf{Null} indicates there are no
further nodes in a particular direction. 

An integer variable, \texttt{Root}, holds the index of the root node. 

The contents of the array \texttt{Names} are shown: 
\begin{center}
\textbf{}%
\begin{tabular}{|c|}
\hline 
\texttt{Root}\tabularnewline
\hline 
\textbf{1}\tabularnewline
\hline 
\end{tabular}\textbf{~}%
\begin{tabular}{c|c|c|c|}
\multicolumn{1}{c}{\texttt{Index}} & \multicolumn{1}{c}{\texttt{LPtr}} & \multicolumn{1}{c}{\texttt{Data}} & \multicolumn{1}{c}{\texttt{RPtr}}\tabularnewline
\cline{2-4} \cline{3-4} \cline{4-4} 
0 & \texttt{\textbf{Null}} & \texttt{\textbf{Peter}} & \texttt{\textbf{Null}}\tabularnewline
\cline{2-4} \cline{3-4} \cline{4-4} 
1 & \texttt{\textbf{3}} & \texttt{\textbf{Leona}} & \texttt{\textbf{5}}\tabularnewline
\cline{2-4} \cline{3-4} \cline{4-4} 
2 & \texttt{\textbf{Null}} & \texttt{\textbf{Alice}} & \texttt{\textbf{Null}}\tabularnewline
\cline{2-4} \cline{3-4} \cline{4-4} 
3 & \texttt{\textbf{2}} & \texttt{\textbf{Bobbie}} & \texttt{\textbf{6}}\tabularnewline
\cline{2-4} \cline{3-4} \cline{4-4} 
4 & \texttt{\textbf{Null}} & \texttt{\textbf{Tom}} & \texttt{\textbf{Null}}\tabularnewline
\cline{2-4} \cline{3-4} \cline{4-4} 
5 & \texttt{\textbf{0}} & \texttt{\textbf{Simone}} & \texttt{\textbf{4}}\tabularnewline
\cline{2-4} \cline{3-4} \cline{4-4} 
6 & \texttt{\textbf{Null}} & \texttt{\textbf{David}} & \texttt{\textbf{Null}}\tabularnewline
\cline{2-4} \cline{3-4} \cline{4-4} 
\end{tabular}
\par\end{center}
\begin{enumerate}
\item Draw the binary search tree represented by the data in the array \texttt{Wurdle}
and the value in \texttt{Root}. \hfill{}{[}2{]}
\item A new word, \textbf{Eric}, is to be inserted into the binary search
tree. Show the changes to the array \texttt{Names} after this insertion.
\hfill{}{[}2{]}
\item Using pseudo-code, write a recursive procedure \texttt{P}, together
with line numbers, that takes the value in \texttt{Root} and outputs
the result of an in-order traversal of \texttt{Names}.\hfill{} {[}4{]}
\end{enumerate}
{[}SPLIT\_HERE{]}
\item \textbf{{[}ALVL/9569/2021/P2/Q1{]} }

The IMEI number is a unique value used to identify mobile devices,
such as phones and tablets. The IMEI number has 15 digits: 14 digits
with an extra check digit added to the right-hand Side. 

The check digit is calculated using the following algorithm, on the
left-most 14 digits of an IMEI number: 
\begin{enumerate}
\item starting from the right, the first digit is location number 1 
\item double all digits in the odd numbered positions 
\item sum all the digits, including both the unchanged digits (i.e. those
in the even numbered positions) as well as those doubled (eg. 16 contributes
1 + 6) 
\item the check digit is the value between 0 and 9 that must be added to
the sum to make the result exactly divisible by 10. 
\end{enumerate}
For example, given the 14 digits 14576567654934:

\begin{tabular}{ccccccccccccccc}
{\footnotesize{}Step 1:} & {\footnotesize{}1} & {\footnotesize{}4} & {\footnotesize{}5} & {\footnotesize{}7} & {\footnotesize{}6} & {\footnotesize{}5} & {\footnotesize{}6} & {\footnotesize{}7} & {\footnotesize{}6} & {\footnotesize{}5} & {\footnotesize{}4} & {\footnotesize{}9} & {\footnotesize{}3} & {\footnotesize{}4}\tabularnewline
{\footnotesize{}Step 2:} & {\footnotesize{}1} & {\footnotesize{}8} & {\footnotesize{}5} & {\footnotesize{}14} & {\footnotesize{}6} & {\footnotesize{}10} & {\footnotesize{}6} & {\footnotesize{}14} & {\footnotesize{}6} & {\footnotesize{}10} & {\footnotesize{}4} & {\footnotesize{}18} & {\footnotesize{}3} & {\footnotesize{}8}\tabularnewline
{\footnotesize{}Step 3:} & {\footnotesize{}$1+$} & {\footnotesize{}$(8)+$} & {\footnotesize{}$5+$} & {\footnotesize{}$(1+4)+$} & {\footnotesize{}$6+$} & {\footnotesize{}$(1+0)+$} & {\footnotesize{}$6+$} & {\footnotesize{}$(1+4)+$} & {\footnotesize{}$6+$} & {\footnotesize{}$(1+0)+$} & {\footnotesize{}$4+$} & {\footnotesize{}$(1+8)+$} & {\footnotesize{}$3+$} & {\footnotesize{}$(8)$}\tabularnewline
\end{tabular}

\begin{tabular}{cc}
 & {\footnotesize{}Sum = 68}\tabularnewline
{\footnotesize{}Step 4:} & {\footnotesize{}Check digit = 2}\tabularnewline
\end{tabular}

For each of the sub-tasks, add a comment statement at the beginning
of the code using the hash symbol \textquoteleft \#' to indicate the
sub-task the program code belongs to, for example:

\begin{singlespace}
\noindent \texttt{}%
\begin{tabular}{c|lcccccccccccccccccccccc|}
\cline{2-24} \cline{3-24} \cline{4-24} \cline{5-24} \cline{6-24} \cline{7-24} \cline{8-24} \cline{9-24} \cline{10-24} \cline{11-24} \cline{12-24} \cline{13-24} \cline{14-24} \cline{15-24} \cline{16-24} \cline{17-24} \cline{18-24} \cline{19-24} \cline{20-24} \cline{21-24} \cline{22-24} \cline{23-24} \cline{24-24} 
\texttt{In {[}1{]} :} & \texttt{\#Task 1.1} &  &  &  &  &  &  &  &  &  &  &  &  &  &  &  &  &  &  &  &  &  & \tabularnewline
 & \texttt{Program Code} &  &  &  &  &  &  &  &  &  &  &  &  &  &  &  &  &  &  &  &  &  & \tabularnewline
\cline{2-24} \cline{3-24} \cline{4-24} \cline{5-24} \cline{6-24} \cline{7-24} \cline{8-24} \cline{9-24} \cline{10-24} \cline{11-24} \cline{12-24} \cline{13-24} \cline{14-24} \cline{15-24} \cline{16-24} \cline{17-24} \cline{18-24} \cline{19-24} \cline{20-24} \cline{21-24} \cline{22-24} \cline{23-24} \cline{24-24} 
\multicolumn{1}{c}{} & \texttt{Output:} &  &  &  &  &  &  &  &  &  &  &  &  &  &  &  &  &  &  &  &  &  & \multicolumn{1}{c}{}\tabularnewline
\end{tabular}
\end{singlespace}
\begin{singlespace}

\subsection*{Task 1.1}
\end{singlespace}

Write a function \texttt{task1\_1(input\_value)} that returns an integer.

The function should:
\begin{itemize}
\item validate that the parameter \texttt{input\_value} is either an integer,
or a string containing a valid integer
\item check that it is 14 digits in length
\item return $-1$ if the value received is invalid for any reason
\item calculate the check digit for the given first 14 digits of the IMEI
number
\item return the calculated check digit. \hfill{}{[}6{]}
\end{itemize}
\begin{singlespace}

\subsection*{Task 1.2}
\end{singlespace}

Create \textbf{four} tests that should fully test your function. Ensure
that it validates the inputs accurately and returns the correct expected
result. 

Each of the four tests will be a pair of data items: the input value
and its expected result.

Your four input values should be: 
\begin{itemize}
\item a string containing just a valid integer 
\item a valid integer 
\item a string containing characters that are not numbers 
\item a value of the incorrect length
\end{itemize}
Test your function with your four input values by calling it usrng
the followmg statement:

\texttt{\qquad{}print (task1\_1(input\_value) == expected) }

The four statements should all print \texttt{True}. For example:

\texttt{\qquad{}print(task1\_1(\textquotedbl 14576567654934\textquotedbl )
== 2)}

\textbf{Do not} use the example provided but create your own.\hfill{}{[}4{]}

\subsection*{Task 1.3 }

A full 15-digit IMEI number can be validated by removing the check
digit, calculating the check digit from the remaining 14 digits and
comparing it to the removed digit. 

Write a function \texttt{task1\_3(input\_value)} that returns a Boolean.
The function should: 
\begin{itemize}
\item validate that the parameter \texttt{input\_value} is either an integer,
or a string containing a valid integer 
\item check that it is 15 digits in length 
\item return \texttt{False} if the value received is invalid for any reason 
\item remove the digit on the right-hand side 
\item use your function from Task 1.1 to calculate the check digit from
the remaining 14 digits 
\item return a Boolean representing the comparison between the calculated
check digit and the original removed digit.\hfill{}{[}4{]}
\end{itemize}
Test your function with four input values (these may be based on those
from Task 1.2 or otherwise), two should be correct IMEI numbers, two
should be in error. 

If a value of \texttt{True} is expected, use the following statement: 

\texttt{\qquad{}print(taskl\_3(input\_value)) }

If a value of False is expected to be returned by the comparison,
this can be changed to still output True by using the not keyword,
for example: 

\texttt{\qquad{}print(not task1\_3(\textquotedbl 12345r\textquotedbl )) }

The four statements should all print True. \hfill{}{[}4{]}

Save your Jupyter notebook for Task 1.

{[}SPLIT\_HERE{]}
\item \textbf{{[}ALVL/9569/2021/P2/Q2{]} }

For this question you are provided with three text files, each contains
a valid list of positive integers, one per line:
\begin{itemize}
\item \texttt{TEN.txt} has 10 lines
\item \texttt{HUNDRED.txt} has 100 lines
\item \texttt{THOUSAND.txt} has 1000 lines.
\end{itemize}
For each of the sub-tasks, add a comment statement at the beginning
of the code using the hash symbol \textquoteleft \#' to indicate the
sub-task the program code belongs to, for example:

\begin{singlespace}
\noindent \texttt{}%
\begin{tabular}{c|lcccccccccccccccccccccc|}
\cline{2-24} \cline{3-24} \cline{4-24} \cline{5-24} \cline{6-24} \cline{7-24} \cline{8-24} \cline{9-24} \cline{10-24} \cline{11-24} \cline{12-24} \cline{13-24} \cline{14-24} \cline{15-24} \cline{16-24} \cline{17-24} \cline{18-24} \cline{19-24} \cline{20-24} \cline{21-24} \cline{22-24} \cline{23-24} \cline{24-24} 
\texttt{In {[}1{]} :} & \texttt{\#Task 2.1} &  &  &  &  &  &  &  &  &  &  &  &  &  &  &  &  &  &  &  &  &  & \tabularnewline
 & \texttt{Program Code} &  &  &  &  &  &  &  &  &  &  &  &  &  &  &  &  &  &  &  &  &  & \tabularnewline
\cline{2-24} \cline{3-24} \cline{4-24} \cline{5-24} \cline{6-24} \cline{7-24} \cline{8-24} \cline{9-24} \cline{10-24} \cline{11-24} \cline{12-24} \cline{13-24} \cline{14-24} \cline{15-24} \cline{16-24} \cline{17-24} \cline{18-24} \cline{19-24} \cline{20-24} \cline{21-24} \cline{22-24} \cline{23-24} \cline{24-24} 
\multicolumn{1}{c}{} & \texttt{Output:} &  &  &  &  &  &  &  &  &  &  &  &  &  &  &  &  &  &  &  &  &  & \multicolumn{1}{c}{}\tabularnewline
\end{tabular}
\end{singlespace}

\subsection*{Task 2.1 }

Write a function \texttt{task2\_1(filename)} that: 
\begin{itemize}
\item takes a string filename which represents the name of a text file 
\item reads in the contents of the text file 
\item returns the content as a list of integers. \hfill{}{[}3{]} 
\end{itemize}
Call your function task2\_l with the file TEN.txt, printing the returned
list and its length, using the following statements:

\texttt{\qquad{}result = task2\_l('TEN.txt')}

\texttt{\qquad{}print(result)}

\texttt{\qquad{}print(len(result)) }\hfill{}{[}2{]} 

\subsection*{Task 2.2 }

One method of sorting is the insertion sort.

Write a function \texttt{task2\_2(list\_of\_integers)} that: 
\begin{itemize}
\item takes a list of integers 
\item implements an insertion sort algorithm 
\item returns the sorted list of integers. \hfill{}{[}5{]}
\end{itemize}
Call your function \texttt{task2\_2} with the contents of the file
\texttt{TEN.txt}, printing the returned list, for example, using the
following statement:

\texttt{\qquad{}print(task2\_2(task2\_1('TEN.txt')))}\hfill{}{[}1{]}

\subsection*{Task 2.3 }

Another method of sorting is the quicksort. 

Write a function t\texttt{ask2\_3(list\_of\_integers)} that: 
\begin{itemize}
\item takes a list of integers 
\item implements a quicksort algorithm 
\item returns the sorted list of integers.\hfill{} {[}7{]}
\end{itemize}
Call your function \texttt{task2\_3} with the contents of the file
\texttt{TEN.txt}, printing the returned list. for example, using the
following statement: 

\texttt{\qquad{}print(task2\_3(task2\_1('TEN.txt')))}\hfill{}{[}1{]} 

\subsection*{Task 2.4 }

The \texttt{timeit} library is built into Python and can be used to
time simple function calls. Example code is shown in \texttt{Task2\_timing.py}.
(The sample code assumes that it has access \texttt{task2\_2} function.) 

Using the \texttt{timeit} module, or other evidence, and the three
text files provided with this question, compare and contrast, including
mention of orders of growth, the time complexity of sort and quicksort
algorithms. 

Save your Jupyter notebook for Task 2.\hfill{}{[}5{]}

{[}SPLIT\_HERE{]}
\item \textbf{{[}ALVL/9569/2021/P2/Q3{]} }

A programmer is writing a class, \texttt{LinkedList}, to represent
a linked list of unique integers. A linked list is a collection of
data elements, whose order is not given by their physical placement
in memory. Instead, each element points to the next.

For each of the sub-tasks, add a comment statement at the beginning
of the code using the hash symbol \textquoteleft \#' to indicate the
sub-task the program code belongs to, for example:

\begin{singlespace}
\noindent \texttt{}%
\begin{tabular}{c|lcccccccccccccccccccccc|}
\cline{2-24} \cline{3-24} \cline{4-24} \cline{5-24} \cline{6-24} \cline{7-24} \cline{8-24} \cline{9-24} \cline{10-24} \cline{11-24} \cline{12-24} \cline{13-24} \cline{14-24} \cline{15-24} \cline{16-24} \cline{17-24} \cline{18-24} \cline{19-24} \cline{20-24} \cline{21-24} \cline{22-24} \cline{23-24} \cline{24-24} 
\texttt{In {[}1{]} :} & \texttt{\#Task 3.1} &  &  &  &  &  &  &  &  &  &  &  &  &  &  &  &  &  &  &  &  &  & \tabularnewline
 & \texttt{Program Code} &  &  &  &  &  &  &  &  &  &  &  &  &  &  &  &  &  &  &  &  &  & \tabularnewline
\cline{2-24} \cline{3-24} \cline{4-24} \cline{5-24} \cline{6-24} \cline{7-24} \cline{8-24} \cline{9-24} \cline{10-24} \cline{11-24} \cline{12-24} \cline{13-24} \cline{14-24} \cline{15-24} \cline{16-24} \cline{17-24} \cline{18-24} \cline{19-24} \cline{20-24} \cline{21-24} \cline{22-24} \cline{23-24} \cline{24-24} 
\multicolumn{1}{c}{} & \texttt{Output:} &  &  &  &  &  &  &  &  &  &  &  &  &  &  &  &  &  &  &  &  &  & \multicolumn{1}{c}{}\tabularnewline
\end{tabular}
\end{singlespace}

\subsection*{Task 3.1 }

Write the \texttt{LinkedList} class in Python. Use of a simple Python
list is not sufficient. Include the following methods: 
\begin{itemize}
\item \texttt{insert(integer\_value)} inserts the \texttt{integer\_value}
at the beginning (head) of the list 
\item \texttt{delete(integer\_value)} attempts to delete \texttt{integer\_value}
from the list; if the item was not present, return \texttt{None} 
\item \texttt{search(integer\_value)} returns a Boolean value: \texttt{True}
if \texttt{integer\_value} is in the list, \texttt{False} if not in
the list 
\item \texttt{count()} should return the number of elements in the list,
or zero if empty 
\item \texttt{to\_String()} should return a string containing a suitably
formatted list with the elements separated by a comma and a space,
with square brackets at either end, eg. in the form: 

\texttt{\qquad{}{[}11, 2, 7, 4{]}} \hfill{}{[}8{]}
\end{itemize}
Test \texttt{LinkedList} by using the data in the file \texttt{Task3data.txt}.
Use the \texttt{to\_String()} method to print the resulting contents
of the list. \hfill{}{[}3{]}

\subsection*{Task 3.2 }

Write a Python subclass \texttt{SortedLinkedList} using \texttt{LinkedList}
as its superclass. 

The insert method in the \texttt{SortedLinkedList} subclass should
ensure that the elements are stored in ascending order. \hfill{}{[}5{]}

Test \texttt{SortedLinkedList} by using the data in the file \texttt{Task3data.txt}.
Use the \texttt{to\_String()} method to print the resulting contents
of the list. 

Print the result of searching the SortedLinkedList for the value 94.
\hfill{}{[}2{]}

\subsection*{Task 3.3 }

Write a Python subclass \texttt{Queue} using \texttt{LinkedList} as
its superclass.

Additional \texttt{enqueue} and \texttt{dequeue} methods are to be
defined on the \texttt{Queue} class;
\begin{itemize}
\item \texttt{enqueue(integer\_value)} will insert \texttt{integer\_value}
to the end of the queue
\item \texttt{dequeue()} will return the first element in the queue. If
the queue is empty, return \texttt{None.} \\
\textcolor{white}{.}\hfill{}{[}6{]}
\end{itemize}
Test \texttt{Queue} by using the data in the file \texttt{Task3data.txt}.
Print the first five elements to be dequeued from the list. \hfill{}{[}1{]}

Save your Jupyter notebook for Task 3.

{[}SPLIT\_HERE{]}
\item \textbf{{[}ALVL/9569/2021/P2/Q4{]} }

A competition has three qualifying rounds. Competitors could score
up to 100 points in each round. Each competitor has a unique ID number.
The competitor's ID, name, and the scores for ac round are held in
a database, \texttt{Task4.db}, provided with this question. 

There are two tables: 
\begin{itemize}
\item \texttt{competitor(}\texttt{\uline{id}}\texttt{, name) }
\item \texttt{scores(}\texttt{\uline{id}}\texttt{, }\texttt{\uline{round}}\texttt{,
score)} 
\end{itemize}
Not every competitor competed in all three rounds, but they did all
compete in round 1. 

\subsection*{Task 4.1 }

Write a Python program and the necessary files to create a web application.
The web application offers the following menu options: 
\noindent \begin{center}
\begin{tabular}{|c|}
\hline 
Round 1 Scores\tabularnewline
Round 2 Scores\tabularnewline
Round 3 Scores\tabularnewline
Mean Scores\tabularnewline
Qualifiers\tabularnewline
\hline 
\end{tabular}
\par\end{center}

Round 1 Scores Round 2 Scores Round 3 Scores Mean Scores Qualifiers 

Save your program code as 

\texttt{Task4\_1\_<your name>\_<centre number>\_<index\_\_number>.py }

With any additional files/subfolders as needed in a folder named 

\texttt{Task4\_l\_<your name>\_<centre number>\_\_<index\_number> }

Run the web application and save the output of the program as 

\texttt{Task4\_1\_\_<your name>\_<centre number>\_<index\_number>.html}
\hfill{}{[}3{]}

\subsection*{Task 4.2}

Write an SQL query that, for 3 given round number, shows:
\begin{itemize}
\item competitor names and their scores for that particular round
\item only those who had a score for that round
\item the scores sorted in descending order.
\end{itemize}
The results of the query should be shown on a web page in a table
that:
\begin{itemize}
\item lists the name of each competitor and their score 
\item has the results shown in descending order of the competitor's score.
\end{itemize}
The resulting web page for each round should be accessed from the
corresponding menu option from Task 4.1.

Save all your SQL code as

\texttt{Task4\_2\_<your name>\_<centre number>\_<index\_number>.sql}

With any additional files/subfolders as needed in a folder named

\texttt{Task4\_2\_<your name>\_<centre number>\_<index\_number>} \hfill{}{[}5{]}

Run the web application and save the output of the program as

\texttt{Task4\_2\_<your name>\_<centre number>\_<index\_number>\_1.html}

\texttt{Task4\_2\_<your name>\_<centre number>\_<index\_\_number>\_2.html}

\texttt{Task4\_2\_<your name>\_<centre number>\_<index\_number>\_3.html}
\hfill{}{[}2{]}

\subsection*{Task 4.3 }

Write a SQL query that shows:
\begin{itemize}
\item competitor name
\item the mean score for each competitor, based on the number of rounds
in which they competed. 
\end{itemize}
The query's results should be Shown on a web page in a table that:
\begin{itemize}
\item lists the name of each competitor and their mean score (to 2 decimal
places)
\item has the results shown in ascending alphabetical order of the competitor's
name.
\end{itemize}
The resulting web page should be accessed from the corresponding menu
option from Task 4.1. 

Save your SQL code as

\texttt{Task4\_3\_<your name>\_<centre number>\_<index\_number>.sql}

With any additional files/subfolders as needed in a folder named

\texttt{Task4\_3\_<your name>\_<centre number>\_<index\_number>} \hfill{}{[}6{]}

Run the web application and save the output of the program as

\texttt{Task4\_3\_<your name>\_<centre number>\_<index\_number>.html}
\hfill{}{[}2{]}

\subsection*{Task 4.4 }

In order to qualify for the final of the competition, competitors
need to score over 250, in total, in the first three rounds. 

Write SQL query that shows:
\begin{itemize}
\item competitor\textquoteright s name
\item their total score
\item whether that competitor has qualified for the final.
\end{itemize}
The results of the query should be shown on a web page in a table
that:
\begin{itemize}
\item lists the names of the competitors, their scores and whether they
qualified
\item has the results shown in descending order of the competitor's score.
\end{itemize}
The resulting web page should be accessed from the corresponding menu
option from Task 4.1.

Save your SQL code as

\texttt{Task4\_4\_<your name>\_<centre number>\_<index\_number>.sql}

With any additional files/subfolders as needed in a folder named

\texttt{Task4\_4\_<your name>\_<centre number>\_<index\_number>} \hfill{}{[}5{]}

Run the web application and save the output of the program as

\texttt{Task4\_4\_<your name>\_<centre number>\_<index\_number>.html}
\hfill{}{[}2{]}

{[}SPLIT\_HERE{]}
\end{enumerate}

\end{document}
