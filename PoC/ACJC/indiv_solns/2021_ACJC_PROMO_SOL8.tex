\item {}
\begin{enumerate}
\item[(i)]  {}
\begin{enumerate}
\item[(a)]  At $E$, $y=a(1-\cos\theta)=0$ . Hence $\cos\theta=1\text{ }$ 

$\therefore\theta=\text{2}\pi$

$\therefore x=2a\pi$

Hence $OE=2a\pi$
\item[(b)]  When $y$ is a maximum, 

$\cos\theta=-1$ OR $\frac{\text{d}y}{\text{d}\theta}=a(\sin\theta)=0\text{ }$

$\therefore\theta\text{=}\pi$ and $y=2a$
\end{enumerate}
\item[(ii)]  $\frac{\text{d}y}{\text{d}\theta}=a(\sin\theta)=2a\sin\frac{\theta}{2}\cos\frac{\theta}{2}$
and $\frac{\text{d}x}{\text{d}\theta}=a(1-\cos\theta)=a(1-1+2\sin^{2}\theta)=2a\sin^{2}\frac{\theta}{2}$ 

$\frac{\text{d}y}{\text{d}x}=\frac{\frac{\text{d}y}{\text{d}\theta}}{\frac{\text{d}x}{\text{d}\theta}}=\frac{\cos\frac{\theta}{2}}{\sin\frac{\theta}{2}}=\cot\frac{\theta}{2}$
\item[(iii)]  At $B$, $\frac{\text{d}y}{\text{d}x}=\cot\frac{\beta}{2}=\frac{1}{\tan\frac{\beta}{2}}=\tan\frac{\pi}{6}=\frac{1}{\sqrt{3}}$ 

Hence $\tan\frac{\beta}{2}=\sqrt{3}$. 

\begin{align*}
\frac{\beta}{2} & =\frac{\pi}{3}\\
\beta & =\frac{2\pi}{3} & (\text{shown})
\end{align*}

\item[(iv)]  Since $\frac{\text{d}y}{\text{d}x}=\cot\frac{\theta}{2}$ 

Gradient of normal at point B is $-\tan\frac{\pi}{3}=-\,\sqrt{3}$. 

Equation of normal : 
\begin{align*}
y-\frac{3}{2}a & =-\sqrt{3}\left(x-\left[a\left(\frac{2\pi}{3}-\sin\frac{2\pi}{3}\right)\right]\right)\\
y-\frac{3}{2}a & =-\sqrt{3}\left(x-\left[a\left(\frac{2\pi}{3}-\frac{\sqrt{3}}{2}\right)\right]\right)\\
y & =-\sqrt{3}\left(x-\frac{2\pi a}{3}+\frac{a\sqrt{3}}{2}\right)+\frac{3}{2}a\\
y & =-\sqrt{3}x+\frac{2\pi a}{\sqrt{3}}\\
\sqrt{3}y & =-3x+2\pi a\\
\sqrt{3}y & =-\left(\sqrt{3}\right)x+2\pi a
\end{align*}

\end{enumerate}