\item {}
\begin{enumerate}
\item[(i)]  {}
\begin{enumerate}
\item[(a)]  {}

\begin{align*}
A & =\frac{1}{2}(10\sin\theta)(10+10+2(10\cos\theta))\\
 & =\frac{1}{2}(10\sin\theta)(20+20\cos\theta)\\
 & =(100\sin\theta)(1+\cos\theta)\\
V & =(50)(100\sin\theta)(1+\cos\theta)\\
 & =(5000\sin\theta)(1+\cos\theta)
\end{align*}

\item[(b)]  
\end{enumerate}
\begin{align*}
\frac{dV}{d\theta} & =(5000\cos\theta)(1+\cos\theta)+(5000\sin\theta)(-\sin\theta)\\
 & =5000(\cos\theta+\cos^{2}\theta-\sin^{2}\theta)\\
 & =5000(2\cos^{2}\theta+\cos\theta-1)\\
\frac{\text{d}V}{\text{d}\theta} & =0\\
(2\cos^{2}\theta+\cos\theta-1) & =0\\
(2\cos\theta-1)(\cos\theta+1) & =0
\end{align*}

Since $\theta$ is acute $\cos\theta\ne-1$

$\therefore\theta=\frac{\pi}{3}$

$\frac{\text{d}^{2}V}{\text{d}\theta^{2}}=5000(-4\cos\theta\sin\theta-\sin\theta)=5000(-\frac{3\sqrt{3}}{2})\approx-12990<0$

when $\theta=\frac{\pi}{3}$

$V$ is a maximum when $\theta=\frac{\pi}{3}$

Max $V$ = $(5000\frac{\sqrt{3}}{2})(1+\frac{1}{2})=\frac{15000\sqrt{3}}{4}$

Maximum volume is$\frac{15000\sqrt{3}}{4}\,\text{cm}^{3}$ = $3750\sqrt{3}\text{ cm}^{3}$. 
\item[(ii)]  {}
\begin{enumerate}
\item[(a)]  Volume of water = $V=[\frac{1}{2}h(20+2h\tan\frac{\pi}{4})]50$

\begin{align*}
V & =[h(10+h)]50\\
 & =500h+50h^{2}\\
\frac{\text{d}V}{\text{d}h} & =500+100h.
\end{align*}

When $h=3$ cm, $\frac{\text{d}V}{\text{d}h}=800$

$\frac{\text{d}h}{\text{d}t}=\frac{\text{d}h}{\text{d}V}\times\frac{\text{d}V}{\text{d}t}=\frac{100}{800}\text{cm}\text{s}^{\text{-1}}=\frac{1}{8}\text{cm}\text{s}^{\text{-1}}=0.125\text{cm}\text{s}^{\text{-1}}$
\item[(b)] 

When the depth of the water is $h$ cm, area of water surface $=y=(10+2h\tan\frac{\pi}{4})(50)=500+100h$ 

$\frac{\text{d}y}{\text{d}t}=100\frac{\text{d}h}{\text{d}t}=\frac{100}{8}\text{cm}^{2}\text{s}^{\text{-1}}=12.5\text{cm}^{2}\text{s}^{\text{-1}}$.
\end{enumerate}
\end{enumerate}