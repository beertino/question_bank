%% LyX 2.3.6.1 created this file.  For more info, see http://www.lyx.org/.
%% Do not edit unless you really know what you are doing.
\documentclass[twoside,english]{article}
\usepackage[T1]{fontenc}
\usepackage[latin9]{inputenc}
\usepackage{geometry}
\geometry{verbose,tmargin=2cm,bmargin=2cm,lmargin=2cm,rmargin=2cm}
\usepackage{color}
\usepackage{amsmath}
\usepackage{amssymb}

\makeatletter
%%%%%%%%%%%%%%%%%%%%%%%%%%%%%% User specified LaTeX commands.
\usepackage{helvet}
\renewcommand{\familydefault}{\sfdefault}
\usepackage[T1]{fontenc}
\usepackage[latin9]{inputenc}
\usepackage{geometry}
\geometry{verbose,tmargin=1.8cm,bmargin=4cm,lmargin=1.5cm,rmargin=2cm}
\usepackage{enumitem}
\usepackage{amstext}
\usepackage{amsthm}
\usepackage{amssymb}
\usepackage{setspace}
\usepackage{graphicx}
\doublespacing


\usepackage{enumitem}
\setenumerate[1]{label=\textbf{\arabic*}}
\setenumerate[2]{label=\textbf{(\alph*)}}
\setenumerate[3]{label=\textbf{(\roman*)}}
\setlist[enumerate]{align=right}

\setcounter{page}{2}

%for upright integrals
\usepackage[integrals]{wasysym}

%to be used in conjunction with fancyfoot for last page
\usepackage{zref-totpages}

%fancyhrd settings
\usepackage{fancyhdr}
\pagestyle{fancy}
\fancyhf{}

\fancypagestyle{laststyle}
{
   \fancyhf{}
   \chead{\thepage}
   \fancyfoot[L]{\copyright NJC }
   \fancyfoot[R]{\textbf{END}} %Put \thispagestyle{laststyle} in the last page
}

%%centering page number
\chead{\thepage}

\renewcommand{\headrulewidth}{0pt}
\renewcommand{\footrulewidth}{0pt}

%%footer settings, different footer for ODD and EVEN pages, also for the LASTPAGE
\fancyfoot[LO]{\copyright NJC\hfill \textbf{[Turn Over}}
\fancyfoot[LE]{\copyright NJC }

%%shameless self-plug BRW

\makeatother

\usepackage{babel}
\begin{document}
{[}SPLIT\_HERE{]}
\begin{enumerate}
\item \textbf{{[}ACJC/PROMO/9758/2021/Q1{]} }

State a sequence of transformations that will transform the curve
with equation $y=2\sin(2x+\alpha)\cos(x)$ on to the curve with equation
$y=-2\sin(4x+3\alpha)\cos(2x+\alpha$ ), where $\alpha$ is a positive
constant. \hfill{}{[}3{]}

{[}SPLIT\_HERE{]}
\item \textbf{{[}ACJC/PROMO/9758/2021/Q2{]}} 

Solve algebraically the inequality $\frac{x+3}{x^{2}+x-2}>-1$. \hfill{}{[}3{]}

Hence solve the inequality $\frac{x+3x^{2}}{1+x-2x^{2}}>-1.$ \hfill{}{[}2{]}

{[}SPLIT\_HERE{]}
\item \textbf{{[}ACJC/PROMO/9758/2021/Q3{]}}

A curve $C$ has equation 
\[
\frac{x^{2}-4y^{2}}{x^{2}+xy^{2}+100}=\frac{1}{2}\,,x\in\mathbb{R},x\ne-\,8.
\]
Show that $\frac{\text{d}y}{\text{d}x}=\frac{2x-y^{2}}{2xy+16y}$.
\hfill{}{[}2{]}

Hence, prove that curve $C$ does not have any stationary point. \hfill{}{[}3{]}

{[}SPLIT\_HERE{]}
\item \textbf{{[}ACJC/PROMO/9758/2021/Q4{]}}
\noindent \begin{center}
<INSERT IMAGE HERE>
\par\end{center}

The diagram shows the curve$y=\text{f}(x)$. There are two vertical
asymptotes with equations $x=-2$ and $x=2$ respectively. The curve
crosses the $x$-axis at the point $A$ and has a maximum turning
point at $B$ where it crosses the $y$-axis. 

The curve also has a minimum turning point at $C$. The coordinates
of $A$, $B$ and $C$ are $(a,0)$, $(0,-10)$ and $(p,q)$ respectively,
where $a$, $p$ and $q$ are constants. 

Sketch the following curves and state the equations of the asymptotes,
the coordinates of the turning points and of points where the curve
crosses the axes, if any. Leave your answers in terms of $a$, $p$
or $q$ where necessary. 
\begin{enumerate}
\item[(i)]  $y=\frac{1}{\text{f}(x)}$, and \hfill{}{[}3{]}
\item[(ii)]  $y=\text{f}(2-|x|)$. \hfill{}{[}3{]}
\end{enumerate}
{[}SPLIT\_HERE{]}
\item \textbf{{[}ACJC/PROMO/9758/2021/Q5{]}}

Referred to the origin $O$, points $A$ and $B$ have position vectors
$\mathbf{a}$ and $\mathbf{b}$ respectively. The modulus of $\mathbf{a}$
is 2 and $\mathbf{b}$ is a unit vector. The angle between $\mathbf{a}$
and $\mathbf{b}$ is $60^{\circ}$. Point $C$ lies on $AB$, between
$A$ and $B$, such that $AC=kCB$, where $0<k<1$. 
\begin{enumerate}
\item[(i)]  Express $\overrightarrow{OC}$ in terms of $\mathbf{a}$ and $\mathbf{b}$.
\hfill{}{[}1{]}
\item[(ii)]  Show that the length of projection of $\overrightarrow{OC}$ on
$\overrightarrow{OA}$ is given by $\frac{k+4}{2\left(k+1\right)}$.\hfill{}
{[}3{]}
\item[(iii)]  Find, in terms of $k$, the area of triangle $OAC$.\hfill{} {[}3{]}
\end{enumerate}
{[}SPLIT\_HERE{]}
\item \textbf{{[}ACJC/PROMO/9758/2021/Q6{]}}

The Cartesian equation of line $L_{1}$ is$\frac{x-2}{a}=\frac{y+2}{b}=\frac{z-3}{c}$,
where $a$, $b$, $c$ are constants. The line $L_{2}$ is parallel
to the vector $4\mathbf{i}+3\mathbf{j}$. The line $L_{3}$ asses
through the origin and the point with position vector $\mathbf{j}+\mathbf{k}$. 
\begin{enumerate}
\item[(i)]  Given that $L_{1}$ is perpendicular to $L_{2}$, form an equation
relating $a$ and $b$. \hfill{}{[}1{]}
\item[(ii)]  Given that $L_{1}$ intersects $L_{3}$, show that$5a+2b-2c=0$.
\hfill{}{[}3{]}
\item[(iii)]  Hence express $a$ and $b$ in terms of $c$. \hfill{}{[}1{]}
\item[(iv)]  Find the acute angle between $L_{1}$ and $L_{3}$. \hfill{}{[}2{]}
\end{enumerate}
{[}SPLIT\_HERE{]}
\item \textbf{{[}ACJC/PROMO/9758/2021/Q7{]}}

The functions f and g are defined by 
\begin{align*}
\text{f : }x & \mapsto\frac{1}{\left|1-x^{2}\right|}\ ,x\in\mathbb{R},\text{ }-2\le x<-1,\\
\text{g: }x & \mapsto-(x-2)^{2}+k,x\in\mathbb{R},\text{ }\ x\ge0\,\text{where \ensuremath{k} is a constant}.
\end{align*}

\begin{enumerate}
\item[(i)]  Sketch on the same diagram the graphs of 
\begin{enumerate}
\item[(a)]  $y=\text{f }\left(x\right)$
\item[(b)]  $y=\text{f}^{-1}\left(x\right)$
\item[(c)]  $y=\text{f}^{-1}\text{f}\left(x\right)$ stating the equations of
any asymptotes and the coordinates of any endpoints. \hfill{} {[}3{]}
\end{enumerate}
\item[(ii)]  Find $\text{f}^{-1}$ and state the domain of $\text{f}^{-1}$.
\hfill{}{[}3{]}
\item[(iii)]  Show that the composite function $\text{gf}$ exists and find its
range. \hfill{}{[}2{]}
\end{enumerate}
{[}SPLIT\_HERE{]}
\item \textbf{{[}ACJC/PROMO/9758/2021/Q8{]}}

The figure below shows a cross-section $OBCE$ of a car headlight
whose reflective surface is modelled in suitable units by the curve
with parametric equations

\[
x=a(\theta-\sin\theta),y=a(1-\cos\theta)
\]
 for $0\le\theta\le2\pi$ , where $a$ is a positive constant.
\noindent \begin{center}
<INSERT IMAGE HERE>
\par\end{center}
\begin{enumerate}
\item[(i)]  Find in terms of $a$ 
\begin{enumerate}
\item[(a)]  the length of $OE$, \hfill{} {[}2{]}
\item[(b)]  the maximum height of the curve $OBCE$. \hfill{}{[}1{]}
\end{enumerate}
\item[(ii)]  Show that $\frac{\text{d}y}{\text{d}x}=\cot\frac{\theta}{2}$. \hfill{}
{[}3{]}
\end{enumerate}
Point $B$ lies on the curve and has parameter $\beta$. $TS$ is
tangential to the curve at $B$ and $BC$ is parallel to the $x$-axis.
Given that $\angle TBC=\frac{\pi}{6}$, 
\begin{enumerate}
\item[(iii)]  show that $\beta=\frac{2\pi}{3}$. \hfill{}{[}2{]}
\item[(iv)]  Show that the equation of normal to the curve at the point $B$
is 

\[
ky=-k^{2}x+2\pi a,
\]
 

where $k$ is an exact constant to be determined. \hfill{}{[}3{]}
\end{enumerate}
{[}SPLIT\_HERE{]}
\item \textbf{{[}ACJC/PROMO/9758/2021/Q9{]}}
\begin{enumerate}
\item Given that $\sum\limits _{r=1}^{n}r^{2}=\frac{1}{6}n\left(n+1\right)\left(2n+1\right)$,
find $\sum\limits _{r=7}^{n+1}\left(2^{r}+r^{2}-r\right)$ in terms
of $n$. \hfill{}{[}4{]}
\item {}
\begin{enumerate}
\item Use the method of differences to show that $\sum\limits _{r=2}^{n}\frac{1}{r^{2}-1}=\frac{3}{4}+\frac{A}{n}+\frac{A}{n+1}$,
where $A$ is a constant to be determined. \hfill{}{[}3{]}
\item Explain why the series $\sum\limits _{r=2}^{\infty}\frac{1}{r^{2}-1}$
converges, and write down its value. \hfill{}{[}2{]}
\item Hence deduce that $\frac{2}{2^{2}}+\frac{2}{3^{2}}+\frac{2}{4^{2}}+...$
is less than $\frac{3}{2}$. \hfill{}{[}2{]}
\end{enumerate}
\end{enumerate}
{[}SPLIT\_HERE{]}
\item \textbf{{[}ACJC/PROMO/9758/2021/Q10{]}}
\begin{enumerate}
\item Referred to the origin $O$, the points $A$, $B$ and $C$ have position
vectors $4\mathbf{i}-2\mathbf{j}$, $\alpha\mathbf{i}-\mathbf{j}+2\mathbf{k}$
and $-\mathbf{i}-7\mathbf{j}+\beta\mathbf{k}$ respectively, where
$\alpha$ and $\beta$ are constants. 
\item[(i)]  Given that $A$, $B$ and $C$ are collinear, show that $\alpha=5$,
and find the value of $\beta$. \hfill{}{[}3{]}
\end{enumerate}
The plane $\pi$ contains the line $L$, which has equation $\mathbf{r}=2\mathbf{i}+3\mathbf{j}+\mu(2\mathbf{i}-\mathbf{j}+\mathbf{k}).$
The plane $\pi$ is also parallel to the line that passes through
the points $A$ and $B$. 
\begin{enumerate}
\item[(ii)]  Find the shortest distance from point $A$ to the line $L$. \hfill{}{[}2{]}
\item[(iii)]  Show that the cartesian equation of the plane $\pi$ is $x+y-z=5$.
\hfill{}{[}2{]}
\item[(iv)]  Find the position vector of the foot of the perpendicular from point
$A$ to the plane $\pi$. \hfill{}{[}3{]}
\item[(v)]  Hence find the reflection of the line that passes through points
$A$ and $B$ about the plane $\pi$. \hfill{}{[}2{]}
\end{enumerate}
{[}SPLIT\_HERE{]}
\item \textbf{{[}ACJC/PROMO/9758/2021/Q11{]}}

The figure below shows a container with an open top. The uniform cross
section $ABCD$ of the container is a trapezium with $AB=BC=CD=10$
cm. $AB$ and $CD$ are each inclined to the line $BC$ at an acute
angle of $\theta$ radians. The length of the container is 50 cm and
the container is placed on a horizontal table.
\noindent \begin{center}
<INSERT IMAGE HERE>
\par\end{center}
\begin{enumerate}
\item[(i)]  Show that the volume V of the container is given by 
\[
V=5000\left(\sin\theta\right)(1+\cos\theta)\,\text{cm}^{3}.
\]
\textcolor{white}{\_}\hfill{}{[}2{]}

Hence using differentiation, find the exact maximum value of $V$,
proving that it is a maximum. \hfill{}{[}5{]}
\item[(ii)]  For the remaining part of the question, $\theta$ is fixed at $\frac{\pi}{4}$. 
\end{enumerate}
Water fills the container at a rate of 100 $\text{cm}^{3}\text{s}^{-1}$.
At time $t$ seconds, the depth of the water is $h$ cm. The surface
of the water is a rectangle $PQRS$. When $h=3\text{ cm},$ find the
rate of change of 
\begin{enumerate}
\item[(a)]  the depth of the water, $h$, \hfill{}{[}3{]}
\item[(b)]  the surface area of the water $PQRS$. \hfill{}{[}2{]}
\end{enumerate}
{[}SPLIT\_HERE{]}
\item \textbf{{[}ACJC/PROMO/9758/2021/Q12{]}}

Mrs Tan plans to start a business which requires a start-up capital
of \$700,000. She decided to first save \$200,000 by depositing money
every month into a savings plan. For the remaining \$500,000, she
intends to take a loan from a finance company. 

She deposited \$3000 into the savings plan in the first month and
on the first day of each subsequent month, she deposited \$100 more
than the previous month. Mrs Tan will continue depositing money into
the savings plan until the total amount in her savings plan reaches
\$200,000. It is given that this savings plan pays no interest. 
\begin{enumerate}
\item[(i)]  Find the month in which Mrs Tan\textquoteright s monthly deposit
will exceed \$6,550.\hfill{} {[}2{]}
\item[(ii)]  Find the number of months that it will take for Mrs Tan to save
\$200,000 and hence find the amount that she would have deposited
in the last month.\hfill{} {[}4{]}
\end{enumerate}
After Mrs Tan has saved \$200,000, she took a loan of \$500,000 from
a finance company. To repay the loan from the finance company, Mrs
Tan would have to pay a monthly payment of $\$x$ at the beginning
of each month, starting from the first month. An interest of 0.3\%
per month will be charged on the outstanding loan amount at the end
of the month. 
\begin{enumerate}
\item[(iii)]  Show that the outstanding amount at the end of $n^{\text{th}}$
month, after the interest has been charged, is $A\left(1.003^{n}\right)-Bx\left(1.003^{n}-1\right)$,
where $A$ and $B$ are exact constants to be determined. \hfill{}{[}3{]}
\item[(iv)]  Find the amount of $\$x$, to 2 decimal places, if Mrs Tan wants
to fully repay her loan in 8 years. \hfill{}{[}2{]}
\item[(v)]  Using the value of $x$ found in part (iv), calculate the total
interest that the finance company will earn from Mrs Tan at the end
of 8 years.\hfill{} {[}2{]}
\end{enumerate}
\end{enumerate}

\end{document}
