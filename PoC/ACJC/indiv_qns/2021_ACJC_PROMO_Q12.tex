\item \textbf{{[}ACJC/PROMO/9758/2021/Q12{]}}

Mrs Tan plans to start a business which requires a start-up capital
of \$700,000. She decided to first save \$200,000 by depositing money
every month into a savings plan. For the remaining \$500,000, she
intends to take a loan from a finance company. 

She deposited \$3000 into the savings plan in the first month and
on the first day of each subsequent month, she deposited \$100 more
than the previous month. Mrs Tan will continue depositing money into
the savings plan until the total amount in her savings plan reaches
\$200,000. It is given that this savings plan pays no interest. 
\begin{enumerate}
\item[(i)]  Find the month in which Mrs Tan\textquoteright s monthly deposit
will exceed \$6,550.\hfill{} {[}2{]}
\item[(ii)]  Find the number of months that it will take for Mrs Tan to save
\$200,000 and hence find the amount that she would have deposited
in the last month.\hfill{} {[}4{]}
\end{enumerate}
After Mrs Tan has saved \$200,000, she took a loan of \$500,000 from
a finance company. To repay the loan from the finance company, Mrs
Tan would have to pay a monthly payment of $\$x$ at the beginning
of each month, starting from the first month. An interest of 0.3\%
per month will be charged on the outstanding loan amount at the end
of the month. 
\begin{enumerate}
\item[(iii)]  Show that the outstanding amount at the end of $n^{\text{th}}$
month, after the interest has been charged, is $A\left(1.003^{n}\right)-Bx\left(1.003^{n}-1\right)$,
where $A$ and $B$ are exact constants to be determined. \hfill{}{[}3{]}
\item[(iv)]  Find the amount of $\$x$, to 2 decimal places, if Mrs Tan wants
to fully repay her loan in 8 years. \hfill{}{[}2{]}
\item[(v)]  Using the value of $x$ found in part (iv), calculate the total
interest that the finance company will earn from Mrs Tan at the end
of 8 years.\hfill{} {[}2{]}
\end{enumerate}