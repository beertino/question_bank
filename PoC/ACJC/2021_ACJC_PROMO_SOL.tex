%% LyX 2.3.6.1 created this file.  For more info, see http://www.lyx.org/.
%% Do not edit unless you really know what you are doing.
\documentclass[twoside,english]{article}
\usepackage[T1]{fontenc}
\usepackage[latin9]{inputenc}
\usepackage{geometry}
\geometry{verbose,tmargin=2cm,bmargin=2cm,lmargin=2cm,rmargin=2cm}
\usepackage{amsmath}
\usepackage{amssymb}
\PassOptionsToPackage{normalem}{ulem}
\usepackage{ulem}

\makeatletter

%%%%%%%%%%%%%%%%%%%%%%%%%%%%%% LyX specific LaTeX commands.
\newcommand{\lyxmathsym}[1]{\ifmmode\begingroup\def\b@ld{bold}
  \text{\ifx\math@version\b@ld\bfseries\fi#1}\endgroup\else#1\fi}

%% Because html converters don't know tabularnewline
\providecommand{\tabularnewline}{\\}

%%%%%%%%%%%%%%%%%%%%%%%%%%%%%% User specified LaTeX commands.
\usepackage{helvet}
\renewcommand{\familydefault}{\sfdefault}
\usepackage[T1]{fontenc}
\usepackage[latin9]{inputenc}
\usepackage{geometry}
\geometry{verbose,tmargin=1.8cm,bmargin=4cm,lmargin=1.5cm,rmargin=2cm}
\usepackage{enumitem}
\usepackage{amstext}
\usepackage{amsthm}
\usepackage{amssymb}
\usepackage{setspace}
\usepackage{graphicx}
\doublespacing


\usepackage{enumitem}
\setenumerate[1]{label=\textbf{\arabic*}}
\setenumerate[2]{label=\textbf{(\alph*)}}
\setenumerate[3]{label=\textbf{(\roman*)}}
\setlist[enumerate]{align=right}

\setcounter{page}{2}

%for upright integrals
\usepackage[integrals]{wasysym}

%to be used in conjunction with fancyfoot for last page
\usepackage{zref-totpages}

%fancyhrd settings
\usepackage{fancyhdr}
\pagestyle{fancy}
\fancyhf{}

\fancypagestyle{laststyle}
{
   \fancyhf{}
   \chead{\thepage}
   \fancyfoot[L]{\copyright NJC }
   \fancyfoot[R]{\textbf{END}} %Put \thispagestyle{laststyle} in the last page
}

%%centering page number
\chead{\thepage}

\renewcommand{\headrulewidth}{0pt}
\renewcommand{\footrulewidth}{0pt}

%%footer settings, different footer for ODD and EVEN pages, also for the LASTPAGE
\fancyfoot[LO]{\copyright NJC\hfill \textbf{[Turn Over}}
\fancyfoot[LE]{\copyright NJC }

%%shameless self-plug BRW

\makeatother

\usepackage{babel}
\begin{document}
{[}SPLIT\_HERE{]}
\begin{enumerate}
\item From $y=2\sin(2x+\alpha)\cos(x)$, 

replace $x$ by $x+\alpha$ : 

$y=2\sin(2x+3\alpha)\cos(x+\alpha)$ then 

replace $x$ by $2x$ : $y=2\sin(4x+3\alpha)\cos(2x+\alpha)$.
\begin{enumerate}
\item[1)]  Translation of the graph by $\alpha$ units in the negative $x$-direction,
followed by 
\item[2)]  Scaling parallel to $x$ -- axis by a factor of $\frac{1}{2}a$
. 
\item[3)]  Reflection in the $x$--axis.
\end{enumerate}
OR 
\begin{enumerate}
\item[1)]  Scaling parallel to $x$--axis by a factor of $\frac{1}{2}a$ ,
followed by 
\item[2)]  Translation of the graph by $\frac{\alpha}{2}$ units in the negative
$x$ -- axis direction. 
\item[3)]  Reflection in the $x$ -- axis. 
\end{enumerate}
{[}SPLIT\_HERE{]}
\item {}
\begin{align*}
\frac{x+3}{x^{2}+x-2} & >-1\\
\frac{x+3}{x^{2}+x-2}+1 & >0\\
\frac{x^{2}+2x+1}{x^{2}+x-2} & >0\\
\frac{\left(x+1\right)^{2}}{\left(x+2\right)\left(x-1\right)} & >0
\end{align*}

Since $\left(x+1\right)^{2}\geq0$ for $x\in\mathbb{R}$, $\left(x+2\right)\left(x-1\right)>0$. 

$\therefore x<-2$ or $x>1$.
\begin{align*}
\frac{x+3x^{2}}{1+x-2x^{2}} & >-1\\
\frac{\frac{x+3x^{2}}{x^{2}}}{\frac{1+x-2x^{2}}{x^{2}}} & >-1\\
\frac{\frac{1}{x}+3}{\left(\frac{1}{x}\right)^{2}+\left(\frac{1}{x}\right)-2} & >-1.
\end{align*}

Replace $x$ by $\frac{1}{x}$, $\frac{1}{x}<-2$ or $\frac{1}{x}>1$. 

$\therefore-\frac{1}{2}<x<1$ and $x\ne0$. 

{[}SPLIT\_HERE{]}
\item {}
\begin{align*}
2x^{2}-8y^{2} & =x^{2}+xy^{2}\,+100\\
4x-16y\frac{\text{d}y}{\text{d}x} & =2x+x2y\frac{\text{d}y}{\text{d}x}+y^{2}\\
\frac{\text{d}y}{\text{d}x} & =\frac{2x-y^{2}}{2xy+16y}\,\,\,\,\,\,\,\,\text{(shown)}
\end{align*}
 

Suppose 
\begin{align*}
\frac{\text{d}y}{\text{d}x}=\frac{2x-y^{2}}{2xy+16y} & =0\\
2x & =y^{2}
\end{align*}

$\therefore$
\begin{align*}
\frac{x^{2}-8x}{x^{2}+2x^{2}+100} & =\frac{1}{2}\\
2x^{2}-16x & =3x^{2}\,+100\,\\
x^{2}+16x+100 & =0\\
\left(x+8\right)^{2}+36 & =0
\end{align*}

No solution since $\left(x+8\right)^{2}+36>0$ for $x\in\mathbb{R}$.
{[} Or using discriminant: Since $16^{2}-4\left(1\right)\left(100\right)=-144<0$,
there\textquoteright s no real roots.{]} 

$\therefore$ There is no stationary point since $\frac{\text{d}y}{\text{d}x}\ne0$
for $x\in\mathbb{R}$.

{[}SPLIT\_HERE{]}
\item {} 
\begin{enumerate}
\item (i)
\noindent \begin{center}
<INSERT\_IMAGE\_HERE>
\par\end{center}
\item (ii)
\noindent \begin{center}
<INSERT\_IMAGE\_HERE>
\par\end{center}

\end{enumerate}
{[}SPLIT\_HERE{]}
\item {}
\begin{enumerate}
\item[(i)]  Using Ratio Theorem $\overrightarrow{OC}=\frac{1}{k+1}(\mathbf{a}+k\mathbf{b})$ 
\item[(ii)]  Length of projection of $\overrightarrow{OC}$ onto$\overrightarrow{OA}$
= 

\begin{align*}
\frac{\left|\overrightarrow{OC}\cdot\mathbf{a}\right|}{\left|\mathbf{a}\right|} & =\frac{\left|\frac{1}{k+1}(\mathbf{a}+k\mathbf{b})\cdot\mathbf{a}\right|}{2}\\
 & =\frac{1}{2\left(k+1\right)}\left|(\mathbf{a}+k\mathbf{b})\cdot\mathbf{a}\right| & \text{since }0<k<1\\
 & =\frac{1}{2\left(k+1\right)}\left|(\mathbf{a}\cdot\mathbf{a}+k\mathbf{b}\cdot\mathbf{a})\right|\\
 & =\frac{1}{2\left(k+1\right)}\left|\left|\mathbf{a}\right|^{2}+k\mathbf{b}\cdot\mathbf{a}\right|\\
 & \,=\frac{1}{2\left(k+1\right)}\left|4+k\left|\mathbf{a}\right|\left|\mathbf{b}\right|\cos\left(60^{\circ}\right)\right|\\
 & =\frac{1}{2\left(k+1\right)}\left|4+k\left(2\right)\left(1\right)\left(\frac{1}{2}\right)\right|\\
 & =\frac{4+k}{2\left(k+1\right)} & \text{(proved)}
\end{align*}

\item[(iii)]  Area of triangle OAC =
\begin{align*}
\frac{1}{2}\left|\mathbf{a}\times\text{\ensuremath{\mathbf{c}}}\right| & =\frac{1}{2}\left|\mathbf{a}\times\frac{1}{k+1}\left(\mathbf{a}+k\left|\mathbf{b}\right|\right)\right|\\
 & =\frac{1}{2(k+1)}\left|(\mathbf{a}\times\mathbf{a})+k(\mathbf{a}\times\left|\mathbf{b}\right|)\right| & \text{since }0<k<1\\
 & =\frac{1}{2(k+1)}\left|(k(\mathbf{a}\times\left|\mathbf{b}\right|)\right| & \text{since}(\mathbf{a}\times\mathbf{a})=\mathbf{0}\\
 & =\frac{k}{2(k+1)}\left|\left|\mathbf{a}\right|\left|\mathbf{b}\right|\sin\left(60^{\circ}\right)\right|\\
 & =\frac{k}{2(k+1)}\left|2(1)\frac{\sqrt{3}}{2}\right|\\
 & =\frac{\sqrt{3}k}{2(k+1)}
\end{align*}
\end{enumerate}
{[}SPLIT\_HERE{]}
\item {}
\begin{enumerate}
\item[(i)]  
\begin{align}
L_{1} & :\frac{x-2}{a}=\frac{y+2}{b}=\frac{z-3}{c}\nonumber \\
\mathbf{r} & =\left(\begin{array}{c}
2\\
-2\\
3
\end{array}\right)+\lambda\left(\begin{array}{c}
a\\
b\\
c
\end{array}\right)
\end{align}

$L_{1}$ is perpendicular to $L_{2}$, 

\begin{align*}
\left(\begin{array}{c}
4\\
3\\
0
\end{array}\right)\cdot\left(\begin{array}{c}
a\\
b\\
c
\end{array}\right) & =0\\
4a+3b & =0
\end{align*}

\item[(ii)]  Equation of line $L_{3}$ : 
\begin{equation}
\mathbf{r}=\mu\left(\begin{array}{c}
0\\
1\\
1
\end{array}\right)
\end{equation}
Since $L_{1}$ intersects $L_{3}$, sub (1) into (2):

\begin{align}
2+\lambda a & =0\Rightarrow\lambda=-\frac{2}{a}\\
-2+\lambda b & =\mu\\
3+\lambda c & =\mu
\end{align}

Sub (4) into (5): 

\begin{equation}
-2+\lambda b=3+\lambda c
\end{equation}

Sub (3) into (6): 
\begin{align*}
-2+\left(\frac{-2}{a}\right)b & =3+\left(\frac{-2}{a}\right)c\\
-2a-2b & =3a-2c\\
5a+2b-2c & =0 & \text{ (Shown)}
\end{align*}

\item[(iii)]  Using results in (i) \& (ii), use GC to solve: 
\begin{align*}
4a+3b+0c & =0\\
5a+2b-2c & =0\\
a & =\frac{6}{7}c\\
b & =-\frac{8}{7}c
\end{align*}
\item[(iv)]  Using result in (iii), 
\begin{align*}
L_{1} & :\mathbf{r}=\left(\begin{array}{c}
2\\
-2\\
3
\end{array}\right)+\lambda\left(\begin{array}{c}
a\\
b\\
c
\end{array}\right)=\left(\begin{array}{c}
2\\
-2\\
3
\end{array}\right)+\lambda\left(\begin{array}{c}
\frac{6}{7}c\\
-\frac{8}{7}c\\
c
\end{array}\right)\\
L_{1} & :\mathbf{r}=\left(\begin{array}{c}
2\\
-2\\
3
\end{array}\right)+\lambda\left(\frac{1}{7}\right)\left(\begin{array}{c}
6c\\
-8c\\
7c
\end{array}\right)=\left(\begin{matrix}2\\
-2\\
3
\end{matrix}\right)+\mu\left(\begin{array}{c}
6\\
-8\\
7
\end{array}\right)
\end{align*}

Angle between $L_{1}$ and $L_{3}$: 
\begin{align*}
\cos\theta & =\left|\frac{\left(\begin{matrix}0\\
1\\
1
\end{matrix}\right)\cdot\left(\begin{matrix}6\\
-8\\
7
\end{matrix}\right)}{\sqrt{2}\sqrt{6^{2}+8^{2}+7^{2}}}\right|\\
 & =\left|\frac{-1}{\sqrt{2}\sqrt{149}}\right|\\
\theta & =86.7^{\circ}
\end{align*}

Acute angle between the two planes is $86.7^{\circ}$.
\end{enumerate}
{[}SPLIT\_HERE{]}
\item {}
\begin{enumerate}
\item[(i)]  
\noindent \begin{center}
<INSERT\_IMAGE\_HERE>
\par\end{center}
\item[(ii)]  $y=\frac{1}{\left|1-x^{2}\right|}=\pm\frac{1}{1-x^{2}}$. 

Considering the interval $-2\le x<-1$, 

\begin{align*}
y & =-\frac{1}{1-x^{2}}\\
y & =\frac{1}{x^{2}-1}\\
yx^{2}-y & =1\\
x^{2} & =\frac{1+y}{y}\\
x & =\pm\sqrt{\frac{1+y}{y}}\\
x & =-\sqrt{\frac{1+y}{y}} & (\text{since}-2\le x<-1)\\
\text{f}^{-1}(x) & =-\sqrt{\frac{1+x}{x}}\\
 & =-\sqrt{\frac{1}{x}+1}
\end{align*}

$\text{D}_{\text{f}^{-1}}=\text{R}_{\text{f}}=[\frac{1}{3},\infty)$
\item[(iii)]  Since $R_{\text{f}}=[\tfrac{1}{3},\infty)\subseteq[0,\infty)=D_{\text{g}}$.
<INSERT\_IMAGE\_HERE>

Hence gf exists.

$[-2,-1)\xrightarrow{f}[\tfrac{1}{3},\infty)\xrightarrow{g}(-\infty,k]$

$R_{\text{gf}}=(-\infty,k]$
\end{enumerate}
{[}SPLIT\_HERE{]}
\item {}
\begin{enumerate}
\item[(i)]  {}
\begin{enumerate}
\item[(a)]  At $E$, $y=a(1-\cos\theta)=0$ . Hence $\cos\theta=1\text{ }$ 

$\therefore\theta=\text{2}\pi$

$\therefore x=2a\pi$

Hence $OE=2a\pi$
\item[(b)]  When $y$ is a maximum, 

$\cos\theta=-1$ OR $\frac{\text{d}y}{\text{d}\theta}=a(\sin\theta)=0\text{ }$

$\therefore\theta\text{=}\pi$ and $y=2a$
\end{enumerate}
\item[(ii)]  $\frac{\text{d}y}{\text{d}\theta}=a(\sin\theta)=2a\sin\frac{\theta}{2}\cos\frac{\theta}{2}$
and $\frac{\text{d}x}{\text{d}\theta}=a(1-\cos\theta)=a(1-1+2\sin^{2}\theta)=2a\sin^{2}\frac{\theta}{2}$ 

$\frac{\text{d}y}{\text{d}x}=\frac{\frac{\text{d}y}{\text{d}\theta}}{\frac{\text{d}x}{\text{d}\theta}}=\frac{\cos\frac{\theta}{2}}{\sin\frac{\theta}{2}}=\cot\frac{\theta}{2}$
\item[(iii)]  At $B$, $\frac{\text{d}y}{\text{d}x}=\cot\frac{\beta}{2}=\frac{1}{\tan\frac{\beta}{2}}=\tan\frac{\pi}{6}=\frac{1}{\sqrt{3}}$ 

Hence $\tan\frac{\beta}{2}=\sqrt{3}$. 

\begin{align*}
\frac{\beta}{2} & =\frac{\pi}{3}\\
\beta & =\frac{2\pi}{3} & (\text{shown})
\end{align*}

\item[(iv)]  Since $\frac{\text{d}y}{\text{d}x}=\cot\frac{\theta}{2}$ 

Gradient of normal at point B is $-\tan\frac{\pi}{3}=-\,\sqrt{3}$. 

Equation of normal : 
\begin{align*}
y-\frac{3}{2}a & =-\sqrt{3}\left(x-\left[a\left(\frac{2\pi}{3}-\sin\frac{2\pi}{3}\right)\right]\right)\\
y-\frac{3}{2}a & =-\sqrt{3}\left(x-\left[a\left(\frac{2\pi}{3}-\frac{\sqrt{3}}{2}\right)\right]\right)\\
y & =-\sqrt{3}\left(x-\frac{2\pi a}{3}+\frac{a\sqrt{3}}{2}\right)+\frac{3}{2}a\\
y & =-\sqrt{3}x+\frac{2\pi a}{\sqrt{3}}\\
\sqrt{3}y & =-3x+2\pi a\\
\sqrt{3}y & =-\left(\sqrt{3}\right)x+2\pi a
\end{align*}

\end{enumerate}
{[}SPLIT\_HERE{]}
\item {}
\begin{enumerate}
\item[(a)]  

\begin{align*}
 & \sum\limits _{r=7}^{n+1}\left(2^{r}+r^{2}-r\right)\\
= & \sum\limits _{r=7}^{n+1}\left(2^{r}\right)+\sum\limits _{r=7}^{n+1}\left(r^{2}\right)-\sum\limits _{r=7}^{n+1}\left(r\right)\\
= & \frac{2^{7}\left(2^{n-5}-1\right)}{2-1}+\sum\limits _{r=1}^{n+1}\left(r^{2}\right)-\sum\limits _{r=1}^{6}\left(r^{2}\right)-\left(\frac{n-5}{2}\right)\left(7+n+1\right)\\
= & 2^{7}\left(2^{n-5}-1\right)+\frac{\left(n+1\right)}{6}\left(n+2\right)\left(2n+3\right)-\left(\frac{6}{6}\right)\left(7\right)\left(13\right)-\left(\frac{n-5}{2}\right)\left(8+n\right)\\
= & 2^{7}\left(2^{n-5}-1\right)+\frac{\left(n+1\right)}{6}\left(n+2\right)\left(2n+3\right)-91-\frac{\left(n-5\right)\left(8+n\right)}{2}
\end{align*}

Alternative Method: 
\begin{align*}
 & \sum\limits _{r=7}^{n+1}\left(2^{r}+r^{2}-r\right)\\
= & \sum\limits _{r=1}^{n+1}\left(2^{r}+r^{2}-r\right)-\sum\limits _{r=1}^{6}\left(2^{r}+r^{2}-r\right)\\
= & \frac{2\left(2^{n+1}-1\right)}{2-1}+\frac{1}{6}\left(n+1\right)\left(n+2\right)\left(2n+3\right)-\frac{n+1}{2}\left(1+n+1\right)\\
= & \frac{2\left(2^{6}-1\right)}{2-1}-91+\frac{6}{2}\left(1+6\right)\\
= & 2^{n+2}+\frac{1}{6}\left(n+1\right)\left(n+2\right)\left(2n+3\right)-\frac{\left(n+1\right)\left(n+2\right)}{2}-198\\
= & 2^{n+2}+\left[\frac{\left(n+1\right)\left(n+2\right)}{6}\right]\left(2n+3-3\right)-198\\
= & 2^{n+2}+\frac{1}{3}n\left(n+1\right)\left(n+2\right)-198
\end{align*}

\item[(a)]  
\begin{enumerate}
\item[(i)]  Using partial fractions,

\begin{align*}
\frac{1}{r^{2}-1} & =\frac{1}{2}\left(\frac{1}{r-1}-\frac{1}{r+1}\right)\\
\sum\limits _{r=2}^{n}\frac{1}{r^{2}-1} & =\frac{1}{2}\sum\limits _{r=2}^{n}\left(\frac{1}{r-1}-\frac{1}{r+1}\right)\\
 & =\frac{1}{2}\left[\begin{array}{cccc}
 & \frac{1}{1} & - & \frac{1}{3}\\
+ & \frac{1}{2} & - & \frac{1}{4}\\
+ & \frac{1}{3} & - & \frac{1}{5}\\
 & \vdots\\
+ & \frac{1}{n-3} & - & \frac{1}{n-1}\\
+ & \frac{1}{n-2} & - & \frac{1}{n}\\
+ & \frac{1}{n-1} & - & \frac{1}{n+1}
\end{array}\right]\\
 & =\frac{1}{2}\left(\frac{3}{2}-\frac{1}{n}-\frac{1}{n+1}\right)\\
 & =\frac{3}{4}+\frac{-\frac{1}{2}}{n}+\frac{-\frac{1}{2}}{n+1}
\end{align*}
 

$\therefore A=-\frac{1}{2}$. 
\item[(ii) ] As $n\to\infty$, $\frac{1}{n}\to0$, $\frac{1}{n+1}\to0$, therefore
$\sum\limits _{r=2}^{\infty}\frac{1}{r^{2}-1}$ converges. $\sum\limits _{r=2}^{\infty}\frac{1}{r^{2}-1}=\frac{3}{4}$. 
\item[(iii)]  $\frac{2}{2^{2}}+\frac{2}{3^{2}}+\frac{2}{4^{2}}+...=\sum\limits _{r=2}^{\infty}\frac{2}{r^{2}}$

Since $r^{2}-1<r^{2}$,

$\therefore$. 

\begin{align*}
\frac{2}{r^{2}-1} & >\frac{2}{r^{2}}\\
\sum\limits _{r=2}^{\infty}\frac{2}{r^{2}-1} & >\sum\limits _{r=2}^{\infty}\frac{2}{r^{2}}\\
\sum\limits _{r=2}^{\infty}\frac{2}{r^{2}} & <2\left(\frac{3}{4}\right)\\
\frac{2}{2^{2}}+\frac{2}{3^{2}}+\frac{2}{4^{2}}+.. & <\frac{3}{2} & (\text{shown})
\end{align*}

\end{enumerate}
\end{enumerate}
{[}SPLIT\_HERE{]}
\item {}
\begin{enumerate}
\item[(i)]  $\overrightarrow{OA}=\left(\begin{matrix}4\\
-2\\
0
\end{matrix}\right)\overrightarrow{OB}=\left(\begin{matrix}\alpha\\
-1\\
2
\end{matrix}\right)\overrightarrow{OC}=\left(\begin{matrix}-1\\
-7\\
\beta
\end{matrix}\right)$

$A$, $B$ and $C$ are collinear.

$\therefore\overrightarrow{AB}=k\overrightarrow{AC}$

$\left(\begin{matrix}\alpha-4\\
1\\
2
\end{matrix}\right)=k\left(\begin{matrix}-5\\
-5\\
\beta
\end{matrix}\right)\Rightarrow\left\{ \begin{matrix}\alpha-4=-5k\\
1=-5k\\
2=k\beta
\end{matrix}\right.\Rightarrow\left\{ \begin{matrix}k=-\frac{1}{5}\\
\alpha=5\\
\beta=-10
\end{matrix}\right.$
\item[(ii)]  $\overrightarrow{OP}=\left(\begin{matrix}2\\
3\\
0
\end{matrix}\right)$ is position vector of a point on the line $L$. 

$\overrightarrow{AP}=\left(\begin{matrix}2\\
3\\
0
\end{matrix}\right)-\left(\begin{matrix}4\\
-2\\
0
\end{matrix}\right)=\left(\begin{matrix}-2\\
5\\
0
\end{matrix}\right)$ 

Distance from $A$ to $L$ = 

\begin{align*}
 & \left|\overrightarrow{AP}\times\frac{1}{\sqrt{4+1+1}}\left(\begin{matrix}2\\
-1\\
1
\end{matrix}\right)\right|\\
= & \frac{1}{\sqrt{6}}\left|\left(\begin{matrix}-2\\
5\\
0
\end{matrix}\right)\times\left(\begin{matrix}2\\
-1\\
1
\end{matrix}\right)\right|\\
= & \frac{1}{\sqrt{6}}\left|\left(\begin{matrix}5\\
2\\
-8
\end{matrix}\right)\right|\\
= & \frac{1}{\sqrt{6}}\sqrt{25+4+64}\\
= & \sqrt{\frac{93}{6}}\\
= & \sqrt{\frac{31}{2}}.
\end{align*}

\item[(iii)]  Normal of plane $\pi=\overrightarrow{AB}\times\left(\begin{matrix}2\\
-1\\
1
\end{matrix}\right)=\left(\begin{matrix}1\\
1\\
2
\end{matrix}\right)\times\left(\begin{matrix}2\\
-1\\
1
\end{matrix}\right)=\left(\begin{matrix}3\\
3\\
-3
\end{matrix}\right)=3\left(\begin{matrix}1\\
1\\
-1
\end{matrix}\right)$

Equation of plane $\pi\text{: }\mathbf{r}\cdot\left(\begin{matrix}1\\
1\\
-1
\end{matrix}\right)=\left(\begin{matrix}2\\
3\\
0
\end{matrix}\right)\centerdot\left(\begin{matrix}1\\
1\\
-1
\end{matrix}\right)=2+3=5$

Cartesian equation is $x+y-z=5$
\item[(iv)]  Let $F$ be the foot of perpendicular from $A(4,-2,0)$ to the plane
$\pi$

Equation of line $AF$: $\mathbf{r}=\left(\begin{matrix}4\\
-2\\
0
\end{matrix}\right)+\lambda\left(\begin{matrix}1\\
1\\
-1
\end{matrix}\right)=\left(\begin{matrix}4+\lambda\\
-2+\lambda\\
-\lambda
\end{matrix}\right)$ 

To find the point of intersection of line $AF$ and plane $\pi$,
substitute equation of line into equation of plane $x+y-z=5$. 

$\begin{array}{ccc}
 & 4+\lambda-2+\lambda+\lambda & =5\\
\Rightarrow & 3\lambda+2 & =5\\
\Rightarrow & \lambda & =1
\end{array}$

$\therefore\overrightarrow{OF}=\left(\begin{matrix}4\\
-2\\
0
\end{matrix}\right)+1\left(\begin{matrix}1\\
1\\
-1
\end{matrix}\right)=\left(\begin{matrix}5\\
-1\\
-1
\end{matrix}\right)$

\textbf{Alternative method }

$\overrightarrow{OD}=\left(\begin{matrix}2\\
3\\
0
\end{matrix}\right)$ is a point on the plane $\pi$. $\overrightarrow{DA}=\left(\begin{matrix}2\\
-5\\
0
\end{matrix}\right)$ 

$\overrightarrow{FA}=$ Projected vector of $\overrightarrow{DA}$
on normal of plane. 

$\begin{array}{ccc}
\overrightarrow{FA} & =\left(\overrightarrow{DA}\cdot\mathbf{\hat{n}}\right)\mathbf{\hat{n}} & \text{where }\mathbf{n}\text{ is the normal of }ABC\\
 & =\frac{\left(\left(\begin{matrix}2\\
-5\\
0
\end{matrix}\right)\cdot\left(\begin{matrix}1\\
1\\
-1
\end{matrix}\right)\right)\left(\begin{matrix}1\\
1\\
-1
\end{matrix}\right)}{\sqrt{1+1+1}\sqrt{1+1+1}}\\
 & =\frac{2-5}{3}\left(\begin{matrix}1\\
1\\
-1
\end{matrix}\right)=-\left(\begin{matrix}1\\
1\\
-1
\end{matrix}\right)=\left(\begin{matrix}-1\\
-1\\
1
\end{matrix}\right)
\end{array}$

$\therefore\overrightarrow{OF}=\overrightarrow{OA}-\overrightarrow{FA}=\left(\begin{matrix}4\\
-2\\
0
\end{matrix}\right)-\left(\begin{matrix}-1\\
-1\\
1
\end{matrix}\right)=\left(\begin{matrix}5\\
-1\\
-1
\end{matrix}\right)$ 

Let the reflection of $A$ about plane $\pi$ be $A\lyxmathsym{\textquoteright}(x,y,z)$ 

\begin{align*}
\overrightarrow{AF} & =\overrightarrow{FA'}\\
\overrightarrow{OA'} & =2\overrightarrow{OF}-\overrightarrow{OA}\\
\overrightarrow{OA'} & =2\left(\begin{matrix}5\\
-1\\
-1
\end{matrix}\right)-\left(\begin{matrix}4\\
-2\\
0
\end{matrix}\right)=\left(\begin{matrix}6\\
0\\
-2
\end{matrix}\right)
\end{align*}
 

Alternatively:

\begin{align*}
\frac{4+x}{2} & =5 & \implies x=6\\
\frac{-2+y}{2} & =-1 & \implies y=0\\
\frac{0+z}{2} & =-1 & \implies z=-2
\end{align*}

Equation of reflected line is: 

$\mathbf{r}=\left(\begin{matrix}6\\
0\\
-2
\end{matrix}\right)+\lambda\left(\begin{matrix}1\\
1\\
2
\end{matrix}\right)$
\end{enumerate}
{[}SPLIT\_HERE{]}
\item {}
\begin{enumerate}
\item[(i)]  {}
\begin{enumerate}
\item[(a)]  {}

\begin{align*}
A & =\frac{1}{2}(10\sin\theta)(10+10+2(10\cos\theta))\\
 & =\frac{1}{2}(10\sin\theta)(20+20\cos\theta)\\
 & =(100\sin\theta)(1+\cos\theta)\\
V & =(50)(100\sin\theta)(1+\cos\theta)\\
 & =(5000\sin\theta)(1+\cos\theta)
\end{align*}

\item[(b)]  
\end{enumerate}
\begin{align*}
\frac{dV}{d\theta} & =(5000\cos\theta)(1+\cos\theta)+(5000\sin\theta)(-\sin\theta)\\
 & =5000(\cos\theta+\cos^{2}\theta-\sin^{2}\theta)\\
 & =5000(2\cos^{2}\theta+\cos\theta-1)\\
\frac{\text{d}V}{\text{d}\theta} & =0\\
(2\cos^{2}\theta+\cos\theta-1) & =0\\
(2\cos\theta-1)(\cos\theta+1) & =0
\end{align*}

Since $\theta$ is acute $\cos\theta\ne-1$

$\therefore\theta=\frac{\pi}{3}$

$\frac{\text{d}^{2}V}{\text{d}\theta^{2}}=5000(-4\cos\theta\sin\theta-\sin\theta)=5000(-\frac{3\sqrt{3}}{2})\approx-12990<0$

when $\theta=\frac{\pi}{3}$

$V$ is a maximum when $\theta=\frac{\pi}{3}$

Max $V$ = $(5000\frac{\sqrt{3}}{2})(1+\frac{1}{2})=\frac{15000\sqrt{3}}{4}$

Maximum volume is$\frac{15000\sqrt{3}}{4}\,\text{cm}^{3}$ = $3750\sqrt{3}\text{ cm}^{3}$. 
\item[(ii)]  {}
\begin{enumerate}
\item[(a)]  Volume of water = $V=[\frac{1}{2}h(20+2h\tan\frac{\pi}{4})]50$

\begin{align*}
V & =[h(10+h)]50\\
 & =500h+50h^{2}\\
\frac{\text{d}V}{\text{d}h} & =500+100h.
\end{align*}

When $h=3$ cm, $\frac{\text{d}V}{\text{d}h}=800$

$\frac{\text{d}h}{\text{d}t}=\frac{\text{d}h}{\text{d}V}\times\frac{\text{d}V}{\text{d}t}=\frac{100}{800}\text{cm}\text{s}^{\text{-1}}=\frac{1}{8}\text{cm}\text{s}^{\text{-1}}=0.125\text{cm}\text{s}^{\text{-1}}$
\item[(b)] 

When the depth of the water is $h$ cm, area of water surface $=y=(10+2h\tan\frac{\pi}{4})(50)=500+100h$ 

$\frac{\text{d}y}{\text{d}t}=100\frac{\text{d}h}{\text{d}t}=\frac{100}{8}\text{cm}^{2}\text{s}^{\text{-1}}=12.5\text{cm}^{2}\text{s}^{\text{-1}}$.
\end{enumerate}
\end{enumerate}
{[}SPLIT\_HERE{]}
\item {}
\begin{enumerate}
\item[(i)]  {}

$\begin{array}{cc}
U_{n} & >6550\\
3000+\left(n-1\right)100 & >6550\\
n & >36.5
\end{array}$

$\therefore37^{\text{th}}$ month
\item[(ii)]  {}

\begin{align*}
S_{n} & =\frac{n}{2}\left[6000+\left(n-1\right)100\right]\\
 & =\frac{n}{2}\left[6000+\left(n-1\right)100\right]\\
 & \ge200,000
\end{align*}

\uline{Method 1} 

By GC, 

When $n=40$, $y=198,000<200,000$ 

When $n=41$,$y=205,000>200,000$

\uline{Method 2} 

\begin{align*}
\frac{n}{2}\left[6000+\left(n-1\right)100\right] & \ge200,000\\
100{{n}^{2}}+5900n-400,000 & \ge0\\
\left(n-40.287\right)\left(n+99.287\right) & \ge0\\
n & \le-99.3\text{ (rej) or }n\ge40.3
\end{align*}
 $\therefore41\text{ months}$

$S_{40}=\frac{40}{2}\left[6000+\left(40-1\right)100\right]=\$198,000$

$\$200,000-\$198,000=\$2000$
\item[(iii)]  {}

\begin{tabular}{|c|c|}
\hline 
$n$ & End of the month\tabularnewline
\hline 
\hline 
1 & $1.003\left(500000-x\right)$\tabularnewline
\hline 
2 & $1.003^{2}\left(500000-x\right)-\left(1.003\right)x$\tabularnewline
\hline 
3 & $1.003^{3}\left(500000-x\right)-\left(1.003\right)^{2}x-\left(1.003\right)x$\tabularnewline
\hline 
$\vdots$ & $\vdots$\tabularnewline
\hline 
n & \tabularnewline
\hline 
\end{tabular}

At the end of the $n$th month, the outstanding amount would be 

$\begin{array}{cc}
 & 1.003^{n}\left(500000-x\right)-\left(1.003\right)^{n-1}x-...-\left(1.003\right)x\\
= & 1.003^{n}\left(500000\right)-\left(1.003^{n}\right)x-...-\left(1.003\right)x\\
= & 1.003^{n}\left(500000\right)-x\left[1.003+1.003^{2}+...+1.003^{n}\right]\\
= & 1.003^{n}\left(500000\right)-x\left[\frac{1.003\left(1.003^{n}-1\right)}{1.003-1}\right]\\
= & 1.003^{n}\left(500000\right)-\frac{1003}{3}x\left(1.003^{n}-1\right)
\end{array}$

$\therefore A=500000,B=\frac{1003}{3}$
\item[(iv)]  $1.003^{96}\left(500000\right)-\frac{1003}{3}x\left(1.003^{96}-1\right)=0$

Using GC, $x=\$5984.09$. 
\item[(v)]  Total paid: $\$5984.09\times12\times8=\$574,472.64$

Interest: $\$574,472.64-\$500,000=\$74,472.64$
\end{enumerate}
\end{enumerate}

\end{document}
