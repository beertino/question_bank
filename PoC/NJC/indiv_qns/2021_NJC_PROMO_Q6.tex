\item \textbf{{[}NJC/PROMO/9758/2021/Q6{]}}

A curve \$\{\{C\}\_\{1\}\}\$ has equation

\textbackslash{[}\{\{x\}\textasciicircum\{2\}\}+2\{\{y\}\textasciicircum\{2\}\}=100\textbackslash{]}

and a curve \$\{\{C\}\_\{2\}\}\$ has parametric equations

\textbackslash{[}x=2\{\{\textbackslash text\{e\}\}\textasciicircum\{-t\}\}-4\{\{\textbackslash text\{e\}\}\textasciicircum\{2t\}\},\textbackslash text\{
\}y=3\{\{\textbackslash text\{e\}\}\textasciicircum\{-t\}\}+\{\{\textbackslash text\{e\}\}\textasciicircum\{2t\}\}.\textbackslash{]}
(i) On the same diagram, sketch \$\{\{C\}\_\{1\}\}\$ and \$\{\{C\}\_\{2\}\},\$
labelling the coordinates of the points where both curves cross the
x- and y-axes. {[}5{]} 

(ii) Show that \$\{\{C\}\_\{2\}\}\$ has a Cartesian equation of the
form \$\{\{\textbackslash left( ax+by \textbackslash right)\}\textasciicircum\{2\}\}\textbackslash left(
cx+dy \textbackslash right)=k\$

for some integer constants a, b, c, d and k to be determined. {[}3{]}