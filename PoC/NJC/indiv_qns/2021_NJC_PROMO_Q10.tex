\item \textbf{{[}NJC/PROMO/9758/2021/Q10{]}} 

The curve \$\{\{C\}\_\{1\}\}\$ has equation

\textbackslash{[}\{\{(x-1)\}\textasciicircum\{2\}\}-\{\{(y-4)\}\textasciicircum\{2\}\}=1.\textbackslash{]}

(i) Sketch \$\{\{C\}\_\{1\}\},\$ labelling clearly the equations of
any asymptotes and the coordinates of any vertices and the points
where the curve crosses the x- and y-axes. {[}3{]} 

The curve \$\{\{C\}\_\{2\}\}\$ with equation

\textbackslash{[}y=\textbackslash frac\{\{\{x\}\textasciicircum\{2\}\}+ax\}\{x-1\},\textbackslash{]}

where a is a constant, has two turning points.

(ii) Find the range of possible values of a, showing your working
clearly. {[}3{]}

It is further given that \$\{\{C\}\_\{1\}\}\$ does not intersect \$\{\{C\}\_\{2\}\}.\$

(iii) By finding the equation of the oblique asymptote of \$\{\{C\}\_\{2\}\}\$
in terms of a, find the value of a exactly. {[}2{]} (iv) Assuming
now that a is the value you have found in part (iii), sketch \$\{\{C\}\_\{2\}\}\$
on the same diagram in part (i), labelling clearly the equations of
any asymptotes and the coordinates of any turning points and the points
where the curve crosses the x- and y-axes. {[}3{]}