\item \textbf{{[}NJC/PROMO/9758/2021/Q12{]}} 

In the diagram below, a light source is placed at the point P with
coordinates \$(-2,\textbackslash{} -4,\textbackslash{} -2).\$ A rectangular
glass prism is placed such that the top of the prism is closer to
point P than the bottom of the prism.

It is given that the top of the prism is a part of the plane with
equation

\$\textbackslash mathbf\{r\}=\textbackslash left( \textbackslash begin\{matrix\}
-2 \textbackslash\textbackslash{} 3 \textbackslash\textbackslash{}
-2 \textbackslash\textbackslash{} \textbackslash end\{matrix\}
\textbackslash right)+\textbackslash lambda \textbackslash left(
\textbackslash begin\{matrix\} 1 \textbackslash\textbackslash{}
2 \textbackslash\textbackslash{} -3 \textbackslash\textbackslash{}
\textbackslash end\{matrix\} \textbackslash right)+\textbackslash mu
\textbackslash left( \textbackslash begin\{matrix\} 3 \textbackslash\textbackslash{}
-4 \textbackslash\textbackslash{} 1 \textbackslash\textbackslash{}
\textbackslash end\{matrix\} \textbackslash right),\$

where \$\textbackslash lambda \textbackslash text\{ and \}\textbackslash mu
\$ are parameters.

(i) Show that this plane has a Cartesian equation of the form \$x+y+z=d\$
for some constant d to be determined. {[}3{]}

\quad{} A ray of light is sent in direction \$3\textbackslash mathbf\{i\}+6\textbackslash mathbf\{j\}+2\textbackslash mathbf\{k\}\$
from the light source at P. The light ray enters the prism at point
Q which lies on the top of the prism, as shown in the diagram below.

(ii) Find the exact coordinates of Q. {[}3{]}

The light ray emerges from the prism at point R with coordinates \textbackslash{[}\textbackslash left(
c,\textbackslash{} \textbackslash frac\{53\}\{11\},\textbackslash ,\textbackslash frac\{91\}\{11\}
\textbackslash right),\textbackslash{]} as shown in the diagram
below.

It is known that the plane PQR is perpendicular to the top of the
prism.

(iii) Show that \$c=\textbackslash frac\{87\}\{11\}.\$ {[}3{]}

Snell\textquoteright s Law states that \$\textbackslash sin \textbackslash theta
=k\textbackslash sin \textbackslash phi ,\$ where k is the refractive
index of the prism, is the acute angle between the normal to the top
of the prism and PQ, and is the acute angle between the normal to
the top of the prism and QR.

(iv) Find the value of k. {[}3{]}

(v) Find the exact thickness of the prism measured in the direction
of the normal at Q. {[}2{]}