\item \textbf{{[}NJC/PROMO/9758/2021/Q3{]} }

The diagram shows a sketch of the curve \$y=\textbackslash text\{f\}(x).\$
The curve cuts the x-axis at \textbackslash{[}\textbackslash left(
0,\textbackslash{} \textbackslash text\{0\} \textbackslash right)\textbackslash{]},
\textbackslash{[}\textbackslash left( 2,\textbackslash{} \textbackslash text\{0\}
\textbackslash right)\textbackslash{]} and \textbackslash{[}\textbackslash left(
4,\textbackslash{} \textbackslash text\{0\} \textbackslash right).\textbackslash{]}
It has a stationary point \textbackslash{[}\textbackslash left(
3,\textbackslash{} -\textbackslash frac\{5\}\{2\} \textbackslash right)\textbackslash{]}
and asymptotes with equations \textbackslash{[}x=1\textbackslash{]}
and \textbackslash{[}y=2.\textbackslash{]}

Sketch, on separate diagrams, the curves with the following equations,
stating the equations of any asymptotes and the coordinates of any
turning points and any points where the curves cross the x- and y-axes.

(a) \$y=\textbackslash left| \textbackslash text\{f\}(x) \textbackslash right|,\$
and {[}2{]}

(b) \$y=\textbackslash frac\{1\}\{\textbackslash text\{f\}(x)\}.\$
{[}3{]}