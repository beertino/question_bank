\item \textbf{{[}NJC/PROMO/9758/2021/Q7{]} }

Due to intense rainfall, the Bukit Teemah canal is often filled to
the brim, which causes the surrounding areas to be prone to flooding.
The Ministry of Environment is looking into redesigning the canal
to improve the flow of water by maximising the cross-sectional area,
\$A\textbackslash text\{ \}\{\{\textbackslash text\{m\}\}\textasciicircum\{2\}\},\$of
the canal.

The cross-section of the canal has sides of fixed lengths CD FG IJ
4 m, DE HI 3 m and EF GH 0.5 m. Also, the vertical depth DL IK s m
and \textbackslash{[}\textbackslash angle CDL=\textbackslash angle
JIK=\textbackslash theta \textbackslash{]} radians. 

(i) Show that \textbackslash{[}A=40\textbackslash cos \textbackslash theta
+8\textbackslash sin 2\textbackslash theta +2\textbackslash{]}.
{[}2{]}

(ii) Use differentiation to find the value of which gives a maximum
value of A. {[}4{]}

The National Water Agency conducts regular inspections on the water
quality in the canal. During one such inspection, an officer transfers
water from the canal into a plastic container (as shown in the diagram
below) at a constant rate of 162\$\textbackslash text\{c\}\{\{\textbackslash text\{m\}\}\textasciicircum\{3\}\}\$
per second. 

The plastic container is in the shape of a hollow circular cone with
fixed radius 15 cm and fixed height 35 cm. After t seconds, the depth
of water in the container is h cm and the top surface of the water
has a radius of r cm.

(iii) Find the rate at which h is increasing at the instant when h
21. {[}4{]}