\item {}

(i) Let the plane that contains the top of the prism be \$\{\{p\}\_\{1\}\}.\$
\$\textbackslash begin\{align\} \& \textbackslash left( \textbackslash begin\{matrix\}
1 \textbackslash\textbackslash{} 2 \textbackslash\textbackslash{}
-3 \textbackslash\textbackslash{} \textbackslash end\{matrix\}
\textbackslash right)\textbackslash times \textbackslash left(
\textbackslash begin\{matrix\} 3 \textbackslash\textbackslash{}
-4 \textbackslash\textbackslash{} 1 \textbackslash\textbackslash{}
\textbackslash end\{matrix\} \textbackslash right)=\textbackslash left(
\textbackslash begin\{matrix\} (2)(1)-(-4)(-3) \textbackslash\textbackslash{}
(-3)(3)-(1)(1) \textbackslash\textbackslash{} (1)(-4)-(2)(3) \textbackslash\textbackslash{}
\textbackslash end\{matrix\} \textbackslash right) \textbackslash\textbackslash{}
\& =\textbackslash left( \textbackslash begin\{matrix\} -10 \textbackslash\textbackslash{}
-10 \textbackslash\textbackslash{} -10 \textbackslash\textbackslash{}
\textbackslash end\{matrix\} \textbackslash right) \textbackslash end\{align\}\$
Thus, a vector normal to \$\{\{p\}\_\{1\}\}\$ is \$-\textbackslash frac\{1\}\{10\}\textbackslash left(
\textbackslash begin\{matrix\} -10 \textbackslash\textbackslash{}
-10 \textbackslash\textbackslash{} -10 \textbackslash\textbackslash{}
\textbackslash end\{matrix\} \textbackslash right)=\textbackslash left(
\textbackslash begin\{matrix\} 1 \textbackslash\textbackslash{}
1 \textbackslash\textbackslash{} 1 \textbackslash\textbackslash{}
\textbackslash end\{matrix\} \textbackslash right).\$ Hence a Cartesian
equation of \$\{\{p\}\_\{1\}\}\$ is \$\textbackslash begin\{align\}
\& \textbackslash mathbf\{r\}\textbackslash cdot \textbackslash left(
\textbackslash begin\{matrix\} 1 \textbackslash\textbackslash{}
1 \textbackslash\textbackslash{} 1 \textbackslash\textbackslash{}
\textbackslash end\{matrix\} \textbackslash right)=\textbackslash left(
\textbackslash begin\{matrix\} -2 \textbackslash\textbackslash{}
3 \textbackslash\textbackslash{} -2 \textbackslash\textbackslash{}
\textbackslash end\{matrix\} \textbackslash right)\textbackslash cdot
\textbackslash left( \textbackslash begin\{matrix\} 1 \textbackslash\textbackslash{}
1 \textbackslash\textbackslash{} 1 \textbackslash\textbackslash{}
\textbackslash end\{matrix\} \textbackslash right) \textbackslash\textbackslash{}
\& =-2+3-2=-1 \textbackslash\textbackslash{} \& x+y+z=-1 \textbackslash end\{align\}\$
(ii) Since Q lies on l, 

\textbackslash{[}\textbackslash overset\{\textbackslash xrightarrow\{\{\}\}\}\{\textbackslash mathop\{OQ\}\}\textbackslash ,=\textbackslash left(
\textbackslash begin\{matrix\} -2+3q \textbackslash\textbackslash{}
-4+6q \textbackslash\textbackslash{} -2+2q \textbackslash\textbackslash{}
\textbackslash end\{matrix\} \textbackslash right)\textbackslash{]}
for some \textbackslash{[}q\textbackslash in \textbackslash mathbb\{R\}.\textbackslash{]}

Since Q lies on \$\{\{p\}\_\{1\}\},\$ \textbackslash{[}\textbackslash begin\{align\}
\& (-2+3q)+(-4+6q)+(-2+2q)=-1 \textbackslash\textbackslash{} \&
-8+11q=-1 \textbackslash\textbackslash{} \& q=\textbackslash frac\{7\}\{11\}
\textbackslash end\{align\}\textbackslash{]}

Therefore, \textbackslash{[}\textbackslash overset\{\textbackslash xrightarrow\{\{\}\}\}\{\textbackslash mathop\{OQ\}\}\textbackslash ,=\textbackslash left(
\textbackslash begin\{matrix\} -2+3\textbackslash left( \textbackslash tfrac\{7\}\{11\}
\textbackslash right) \textbackslash\textbackslash{} -4+6\textbackslash left(
\textbackslash tfrac\{7\}\{11\} \textbackslash right) \textbackslash\textbackslash{}
-2+2\textbackslash left( \textbackslash tfrac\{7\}\{11\} \textbackslash right)
\textbackslash\textbackslash{} \textbackslash end\{matrix\} \textbackslash right)=\textbackslash left(
\textbackslash begin\{matrix\} -\{1\}/\{11\}\textbackslash ; \textbackslash\textbackslash{}
-\{2\}/\{11\}\textbackslash ; \textbackslash\textbackslash{} -\{8\}/\{11\}\textbackslash ;
\textbackslash\textbackslash{} \textbackslash end\{matrix\} \textbackslash right)\textbackslash{]}

Therefore, the coordinates of Q are \textbackslash{[}\textbackslash left(
-\textbackslash frac\{1\}\{11\},\textbackslash{} -\textbackslash frac\{2\}\{11\},\textbackslash ,-\textbackslash frac\{8\}\{11\}
\textbackslash right).\textbackslash{]} 

(iii) \$\textbackslash left( \textbackslash begin\{matrix\} 3 \textbackslash\textbackslash{}
6 \textbackslash\textbackslash{} 2 \textbackslash\textbackslash{}
\textbackslash end\{matrix\} \textbackslash right)\textbackslash times
\textbackslash left( \textbackslash begin\{matrix\} 1 \textbackslash\textbackslash{}
1 \textbackslash\textbackslash{} 1 \textbackslash\textbackslash{}
\textbackslash end\{matrix\} \textbackslash right)=\textbackslash left(
\textbackslash begin\{matrix\} (6)(1)-(2)(1) \textbackslash\textbackslash{}
(2)(1)-(3)(1) \textbackslash\textbackslash{} (3)(1)-(6)(1) \textbackslash\textbackslash{}
\textbackslash end\{matrix\} \textbackslash right)=\textbackslash left(
\textbackslash begin\{matrix\} 4 \textbackslash\textbackslash{}
-1 \textbackslash\textbackslash{} -3 \textbackslash\textbackslash{}
\textbackslash end\{matrix\} \textbackslash right)\$ Thus \$\textbackslash left(
\textbackslash begin\{matrix\} 4 \textbackslash\textbackslash{}
-1 \textbackslash\textbackslash{} -3 \textbackslash\textbackslash{}
\textbackslash end\{matrix\} \textbackslash right)\$ is a normal
vector to the plane PQR. Therefore, \$\textbackslash overset\{\textbackslash xrightarrow\{\{\}\}\}\{\textbackslash mathop\{QR\}\}\textbackslash ,\textbackslash{}
\textbackslash cdot \textbackslash left( \textbackslash begin\{matrix\}
4 \textbackslash\textbackslash{} -1 \textbackslash\textbackslash{}
-3 \textbackslash\textbackslash{} \textbackslash end\{matrix\}
\textbackslash right)=0\$ \$\textbackslash begin\{align\} \& \textbackslash left(
\textbackslash begin\{matrix\} c-\textbackslash left( -\textbackslash tfrac\{1\}\{11\}
\textbackslash right) \textbackslash\textbackslash{} \textbackslash tfrac\{53\}\{11\}-\textbackslash left(
-\textbackslash tfrac\{2\}\{11\} \textbackslash right) \textbackslash\textbackslash{}
\textbackslash tfrac\{91\}\{11\}-\textbackslash left( -\textbackslash tfrac\{8\}\{11\}
\textbackslash right) \textbackslash\textbackslash{} \textbackslash end\{matrix\}
\textbackslash right)\textbackslash cdot \textbackslash left( \textbackslash begin\{matrix\}
4 \textbackslash\textbackslash{} -1 \textbackslash\textbackslash{}
-3 \textbackslash\textbackslash{} \textbackslash end\{matrix\}
\textbackslash right)=0 \textbackslash\textbackslash{} \& \textbackslash left(
\textbackslash begin\{matrix\} c+\textbackslash tfrac\{1\}\{11\}
\textbackslash\textbackslash{} 5 \textbackslash\textbackslash{}
9 \textbackslash\textbackslash{} \textbackslash end\{matrix\} \textbackslash right)\textbackslash cdot
\textbackslash left( \textbackslash begin\{matrix\} 4 \textbackslash\textbackslash{}
-1 \textbackslash\textbackslash{} -3 \textbackslash\textbackslash{}
\textbackslash end\{matrix\} \textbackslash right)=0 \textbackslash\textbackslash{}
\& 4c+\textbackslash tfrac\{4\}\{11\}-5-27=0 \textbackslash\textbackslash{}
\& 4c=\textbackslash frac\{348\}\{11\} \textbackslash\textbackslash{}
\& c=\textbackslash frac\{87\}\{11\} \textbackslash end\{align\}\$
(iv) \textbackslash{[}\textbackslash begin\{align\} \& \textbackslash cos
\textbackslash theta =\textbackslash left| \textbackslash frac\{\textbackslash left(
\textbackslash begin\{matrix\} 3 \textbackslash\textbackslash{}
6 \textbackslash\textbackslash{} 2 \textbackslash\textbackslash{}
\textbackslash end\{matrix\} \textbackslash right)\textbackslash cdot
\textbackslash left( \textbackslash begin\{matrix\} 1 \textbackslash\textbackslash{}
1 \textbackslash\textbackslash{} 1 \textbackslash\textbackslash{}
\textbackslash end\{matrix\} \textbackslash right)\}\{\textbackslash sqrt\{\{\{3\}\textasciicircum\{2\}\}+\{\{6\}\textasciicircum\{2\}\}+\{\{2\}\textasciicircum\{2\}\}\}\textbackslash sqrt\{\{\{1\}\textasciicircum\{2\}\}+\{\{1\}\textasciicircum\{2\}\}+\{\{1\}\textasciicircum\{2\}\}\}\}
\textbackslash right| \textbackslash\textbackslash{} \& =\textbackslash left|
\textbackslash frac\{3+6+2\}\{7\textbackslash sqrt\{3\}\} \textbackslash right|
\textbackslash\textbackslash{} \& =\textbackslash frac\{11\}\{7\textbackslash sqrt\{3\}\}\textbackslash Rightarrow
\textbackslash theta =\{\{\textbackslash cos \}\textasciicircum\{-1\}\}\textbackslash left(
\textbackslash frac\{11\}\{7\textbackslash sqrt\{3\}\} \textbackslash right)=24.870\{\}\textasciicircum\textbackslash circ
\textbackslash end\{align\}\textbackslash{]}

\textbackslash{[}\textbackslash begin\{align\} \& \textbackslash cos
\textbackslash phi =\textbackslash left| \textbackslash frac\{\textbackslash left(
\textbackslash begin\{matrix\} 8 \textbackslash\textbackslash{}
5 \textbackslash\textbackslash{} 9 \textbackslash\textbackslash{}
\textbackslash end\{matrix\} \textbackslash right)\textbackslash cdot
\textbackslash left( \textbackslash begin\{matrix\} 1 \textbackslash\textbackslash{}
1 \textbackslash\textbackslash{} 1 \textbackslash\textbackslash{}
\textbackslash end\{matrix\} \textbackslash right)\}\{\textbackslash sqrt\{\{\{8\}\textasciicircum\{2\}\}+\{\{5\}\textasciicircum\{2\}\}+\{\{9\}\textasciicircum\{2\}\}\}\textbackslash sqrt\{\{\{1\}\textasciicircum\{2\}\}+\{\{1\}\textasciicircum\{2\}\}+\{\{1\}\textasciicircum\{2\}\}\}\}
\textbackslash right| \textbackslash\textbackslash{} \& =\textbackslash left|
\textbackslash frac\{8+5+9\}\{\textbackslash sqrt\{510\}\} \textbackslash right|=\textbackslash frac\{22\}\{\textbackslash sqrt\{510\}\}
\textbackslash\textbackslash{} \& \textbackslash theta =\{\{\textbackslash cos
\}\textasciicircum\{-1\}\}\textbackslash left( \textbackslash frac\{22\}\{\textbackslash sqrt\{510\}\}
\textbackslash right)=13.049\{\}\textasciicircum\textbackslash circ
\textbackslash end\{align\}\textbackslash{]}

\textbackslash{[}k=\textbackslash frac\{\textbackslash sin \textbackslash theta
\}\{\textbackslash sin \textbackslash phi \}=\textbackslash frac\{\textbackslash sin
24.870\{\}\textasciicircum\textbackslash circ \}\{\textbackslash sin
13.049\{\}\textasciicircum\textbackslash circ \}=1.86\textbackslash text\{
(to 3 s\}\textbackslash text\{.f\}\textbackslash text\{.)\}\textbackslash{]}
(v) The thickness of prism is the length of projection of QR onto
the normal. Therefore, \textbackslash{[}\textbackslash begin\{align\}
\& \textbackslash text\{Thickness of prism\}=\textbackslash left|
\textbackslash frac\{\textbackslash overset\{\textbackslash xrightarrow\{\{\}\}\}\{\textbackslash mathop\{QR\}\}\textbackslash ,\textbackslash{}
\textbackslash cdot \textbackslash left( \textbackslash begin\{matrix\}
1 \textbackslash\textbackslash{} 1 \textbackslash\textbackslash{}
1 \textbackslash\textbackslash{} \textbackslash end\{matrix\} \textbackslash right)\}\{\textbackslash sqrt\{\{\{1\}\textasciicircum\{2\}\}+\{\{1\}\textasciicircum\{2\}\}+\{\{1\}\textasciicircum\{2\}\}\}\}
\textbackslash right| \textbackslash\textbackslash{} \& =\textbackslash left|
\textbackslash frac\{\textbackslash left( \textbackslash begin\{matrix\}
8 \textbackslash\textbackslash{} 5 \textbackslash\textbackslash{}
9 \textbackslash\textbackslash{} \textbackslash end\{matrix\} \textbackslash right)\textbackslash{}
\textbackslash cdot \textbackslash left( \textbackslash begin\{matrix\}
1 \textbackslash\textbackslash{} 1 \textbackslash\textbackslash{}
1 \textbackslash\textbackslash{} \textbackslash end\{matrix\} \textbackslash right)\}\{\textbackslash sqrt\{3\}\}
\textbackslash right| \textbackslash\textbackslash{} \& =\textbackslash left|
\textbackslash frac\{8+5+9\}\{\textbackslash sqrt\{3\}\} \textbackslash right|
\textbackslash\textbackslash{} \& =\textbackslash frac\{22\}\{\textbackslash sqrt\{3\}\}
\textbackslash\textbackslash{} \& =\textbackslash frac\{22\textbackslash sqrt\{3\}\}\{3\}\textbackslash text\{
units\} \textbackslash end\{align\}\textbackslash{]} Alternatively,
\textbackslash{[}\textbackslash begin\{align\} \& \textbackslash text\{Thickness
of prism\}=QR\textbackslash cos \textbackslash phi \textbackslash\textbackslash{}
\& =\textbackslash sqrt\{\{\{8\}\textasciicircum\{2\}\}+\{\{5\}\textasciicircum\{2\}\}+\{\{9\}\textasciicircum\{2\}\}\}\textbackslash times
\textbackslash frac\{22\}\{\textbackslash sqrt\{510\}\} \textbackslash\textbackslash{}
\& =\textbackslash sqrt\{170\}\textbackslash times \textbackslash frac\{22\}\{\textbackslash sqrt\{510\}\}
\textbackslash\textbackslash{} \& =\textbackslash frac\{22\}\{\textbackslash sqrt\{3\}\}
\textbackslash\textbackslash{} \& =\textbackslash frac\{22\textbackslash sqrt\{3\}\}\{3\}\textbackslash text\{
units\} \textbackslash end\{align\}\textbackslash{]} .