\item {}

(i) \$\textbackslash overset\{\textbackslash xrightarrow\{\{\}\}\}\{\textbackslash mathop\{AB\}\}\textbackslash ,=\textbackslash left(
\textbackslash mathbf\{u\}+2\textbackslash mathbf\{v\} \textbackslash right)-\textbackslash left(
2\textbackslash mathbf\{u\}-3\textbackslash mathbf\{v\} \textbackslash right)=-\textbackslash mathbf\{u\}+5\textbackslash mathbf\{v\}\$
\$\textbackslash overset\{\textbackslash xrightarrow\{\{\}\}\}\{\textbackslash mathop\{AC\}\}\textbackslash ,=\textbackslash left(
3\textbackslash mathbf\{u\}-2\textbackslash mathbf\{v\} \textbackslash right)-\textbackslash left(
2\textbackslash mathbf\{u\}-3\textbackslash mathbf\{v\} \textbackslash right)=\textbackslash mathbf\{u\}+\textbackslash mathbf\{v\}\$
Therefore, area of triangle ABC \$\textbackslash begin\{align\} \&
=\textbackslash frac\{1\}\{2\}\textbackslash left| \textbackslash overset\{\textbackslash xrightarrow\{\{\}\}\}\{\textbackslash mathop\{AB\}\}\textbackslash ,\textbackslash{}
\textbackslash times \textbackslash overset\{\textbackslash xrightarrow\{\{\}\}\}\{\textbackslash mathop\{AC\}\}\textbackslash ,
\textbackslash right| \textbackslash\textbackslash{} \& =\textbackslash frac\{1\}\{2\}\textbackslash left|
\textbackslash left( -\textbackslash mathbf\{u\}+5\textbackslash mathbf\{v\}
\textbackslash right)\textbackslash{} \textbackslash times \textbackslash left(
\textbackslash mathbf\{u\}+\textbackslash mathbf\{v\} \textbackslash right)\textbackslash{}
\textbackslash right| \textbackslash\textbackslash{} \& =\textbackslash frac\{1\}\{2\}\textbackslash left|
-\textbackslash mathbf\{u\}\textbackslash times \textbackslash mathbf\{u\}+5\textbackslash mathbf\{v\}\textbackslash times
\textbackslash mathbf\{u\}-\textbackslash mathbf\{u\}\textbackslash times
\textbackslash mathbf\{v\}+5\textbackslash mathbf\{v\}\textbackslash times
\textbackslash mathbf\{v\} \textbackslash right| \textbackslash\textbackslash{}
\& =\textbackslash frac\{1\}\{2\}\textbackslash left| \textbackslash mathbf\{0\}-5\textbackslash mathbf\{u\}\textbackslash times
\textbackslash mathbf\{v\}-\textbackslash mathbf\{u\}\textbackslash times
\textbackslash mathbf\{v\}+\textbackslash mathbf\{0\} \textbackslash right|
\textbackslash\textbackslash{} \& =\textbackslash frac\{1\}\{2\}\textbackslash left|
-6\textbackslash mathbf\{u\}\textbackslash times \textbackslash mathbf\{v\}
\textbackslash right| \textbackslash\textbackslash{} \& =\textbackslash frac\{6\}\{2\}\textbackslash left|
\textbackslash mathbf\{u\}\textbackslash times \textbackslash mathbf\{v\}
\textbackslash right| \textbackslash\textbackslash{} \& =3\textbackslash left|
\textbackslash mathbf\{u\}\textbackslash times \textbackslash mathbf\{v\}
\textbackslash right|\textbackslash text\{ (shown)\} \textbackslash end\{align\}\$
(ii) \$\textbackslash mathbf\{u\}\textbackslash times \textbackslash mathbf\{v\}\$
\$\textbackslash begin\{align\} \& =\textbackslash left( \textbackslash begin\{matrix\}
2 \textbackslash\textbackslash{} -1 \textbackslash\textbackslash{}
2 \textbackslash\textbackslash{} \textbackslash end\{matrix\} \textbackslash right)\textbackslash times
\textbackslash left( \textbackslash begin\{matrix\} 3 \textbackslash\textbackslash{}
-4 \textbackslash\textbackslash{} -p \textbackslash\textbackslash{}
\textbackslash end\{matrix\} \textbackslash right) \textbackslash\textbackslash{}
\& =\textbackslash left( \textbackslash begin\{matrix\} \textbackslash left(
-1 \textbackslash right)\textbackslash left( -p \textbackslash right)-\textbackslash left(
2 \textbackslash right)\textbackslash left( -4 \textbackslash right)
\textbackslash\textbackslash{} \textbackslash left( 2 \textbackslash right)\textbackslash left(
3 \textbackslash right)-\textbackslash left( 2 \textbackslash right)\textbackslash left(
-p \textbackslash right) \textbackslash\textbackslash{} \textbackslash left(
2 \textbackslash right)\textbackslash left( -4 \textbackslash right)-\textbackslash left(
-1 \textbackslash right)\textbackslash left( 3 \textbackslash right)
\textbackslash\textbackslash{} \textbackslash end\{matrix\} \textbackslash right)
\textbackslash\textbackslash{} \& =\textbackslash left( \textbackslash begin\{matrix\}
p+8 \textbackslash\textbackslash{} 2p+6 \textbackslash\textbackslash{}
-5 \textbackslash\textbackslash{} \textbackslash end\{matrix\}
\textbackslash right) \textbackslash end\{align\}\$

Since the area of triangle ABC is \$15\textbackslash sqrt\{53\}\$
square units, \$\textbackslash begin\{align\} \& 3\textbackslash sqrt\{\{\{(p+8)\}\textasciicircum\{2\}\}+\{\{(2p+6)\}\textasciicircum\{2\}\}+\{\{5\}\textasciicircum\{2\}\}\}=15\textbackslash sqrt\{53\}
\textbackslash\textbackslash{} \& \{\{(p+8)\}\textasciicircum\{2\}\}+\{\{(2p+6)\}\textasciicircum\{2\}\}+\{\{5\}\textasciicircum\{2\}\}=25\textbackslash times
53 \textbackslash\textbackslash{} \& 5\{\{p\}\textasciicircum\{2\}\}+40p-1200=0
\textbackslash\textbackslash{} \& \{\{p\}\textasciicircum\{2\}\}+8p-240=0
\textbackslash\textbackslash{} \& (p+20)(p-12)=0 \textbackslash end\{align\}\$
Therefore, \$p=-20\$ or \$p=12.\$ \quad{} (iii) \$\textbackslash mathbf\{u\}\textbackslash cdot
\textbackslash mathbf\{v\}=\textbackslash left( \textbackslash begin\{matrix\}
2 \textbackslash\textbackslash{} -1 \textbackslash\textbackslash{}
2 \textbackslash\textbackslash{} \textbackslash end\{matrix\} \textbackslash right)\textbackslash cdot
\textbackslash left( \textbackslash begin\{matrix\} 3 \textbackslash\textbackslash{}
-4 \textbackslash\textbackslash{} -p \textbackslash\textbackslash{}
\textbackslash end\{matrix\} \textbackslash right)=10-2p\$ For \$p=-20,\$
\$\textbackslash mathbf\{u\}\textbackslash cdot \textbackslash mathbf\{v\}=\textbackslash underline\{10-2(-20)\}=50\textbackslash{}
\textbackslash underline\{>0\}\$ For \$p=12,\$ \$\textbackslash mathbf\{u\}\textbackslash cdot
\textbackslash mathbf\{v\}=\textbackslash underline\{10-2(12)\}=-14\textbackslash{}
\textbackslash underline\{<0\}\$

Therefore, \$p=-20\$ is the only possible case for the angle between
u and v to be acute. Let this angle be .

\textbackslash{[}\textbackslash begin\{align\} \& \textbackslash cos
\textbackslash theta =\textbackslash frac\{\textbackslash left(
\textbackslash begin\{matrix\} 2 \textbackslash\textbackslash{}
-1 \textbackslash\textbackslash{} 2 \textbackslash\textbackslash{}
\textbackslash end\{matrix\} \textbackslash right)\textbackslash cdot
\textbackslash left( \textbackslash begin\{matrix\} 3 \textbackslash\textbackslash{}
-4 \textbackslash\textbackslash{} 20 \textbackslash\textbackslash{}
\textbackslash end\{matrix\} \textbackslash right)\}\{\textbackslash sqrt\{\{\{2\}\textasciicircum\{2\}\}+\{\{1\}\textasciicircum\{2\}\}+\{\{2\}\textasciicircum\{2\}\}\}\textbackslash sqrt\{\{\{3\}\textasciicircum\{2\}\}+\{\{4\}\textasciicircum\{2\}\}+\{\{20\}\textasciicircum\{2\}\}\}\}
\textbackslash\textbackslash{} \& =\textbackslash frac\{50\}\{3\textbackslash sqrt\{425\}\}=\textbackslash frac\{10\}\{3\textbackslash sqrt\{17\}\}
\textbackslash\textbackslash{} \& \textbackslash theta =36.1\{\}\textasciicircum\textbackslash circ
\textbackslash text\{ (to 1 d\}\textbackslash text\{.p\}\textbackslash text\{.)\}
\textbackslash end\{align\}\textbackslash{]}