\item {}

(i) \$\textbackslash text\{f\}(21)=\textbackslash text\{f\}(5\textbackslash times
4+1)=\textbackslash text\{f\}(1)=\{\{(1-2)\}\textasciicircum\{2\}\}=1.\$ 

(ii) 

(iii) When \$0\textbackslash le \textbackslash text\{g\}\textbackslash left(
x \textbackslash right)<2\$, \$4\textbackslash le x<8\$. This corresponds
to \$\textbackslash text\{fg\}(x)=\textbackslash text\{f\}\textbackslash left(
\textbackslash sqrt\{x-4\} \textbackslash right)=\{\{\textbackslash left(
\textbackslash sqrt\{x-4\}-2 \textbackslash right)\}\textasciicircum\{2\}\}\$.
When \$2\textbackslash le \textbackslash text\{g\}\textbackslash left(
x \textbackslash right)\textbackslash le 4\$, \$8\textbackslash le
x\textbackslash le 20\$. This corresponds to \$\textbackslash text\{fg\}(x)=\textbackslash text\{f\}\textbackslash left(
\textbackslash sqrt\{x-4\} \textbackslash right)=2\textbackslash sqrt\{x-4\}-4\$.

Therefore, \$\textbackslash text\{fg\}(x)=\textbackslash left\textbackslash\{
\textbackslash begin\{matrix\} \{\{\textbackslash left( \textbackslash sqrt\{x-4\}-2
\textbackslash right)\}\textasciicircum\{2\}\}\textbackslash text\{,
\} \& 4\textbackslash le x<8 \textbackslash\textbackslash{} 2\textbackslash sqrt\{x-4\}-4,\textbackslash text\{
\} \& 8\textbackslash le x\textbackslash le 20 \textbackslash\textbackslash{}
\textbackslash end\{matrix\} \textbackslash right.\$