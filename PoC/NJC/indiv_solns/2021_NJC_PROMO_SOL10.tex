\item {}

(i), (iv) 

(ii) At the stationary points of \$\{\{C\}\_\{1\}\},\$ Thus, for \$\{\{C\}\_\{1\}\}\$
to have 2 turning points, \$\textbackslash begin\{align\} \& \{\{(-2)\}\textasciicircum\{2\}\}-4(1)(-a)>0
\textbackslash\textbackslash{} \& 4+4a>0 \textbackslash\textbackslash{}
\& a>-1 \textbackslash end\{align\}\$ (iii) \$\{\{C\}\_\{1\}\}\$
is a hyperbola with vertices lying on the horizontal line \textbackslash{[}y=4.\textbackslash{]}
\$\{\{C\}\_\{2\}\}:\$\$y=\textbackslash frac\{\{\{x\}\textasciicircum\{2\}\}+ax\}\{x-1\}=x+(a+1)+\textbackslash frac\{a+1\}\{x-1\}\$
Through observation from the sketch in part (i), we see that the only
way for the two curves to have no point of intersection is for both
curves to share the same asymptote with positive gradient. Thus the
oblique asymptote of \$\{\{C\}\_\{1\}\}\$ with positive gradient is
\textbackslash{[}y-4=(1)(x-1)\textbackslash Rightarrow y=x+3.\textbackslash{]}
Hence, \textbackslash{[}a+1=3\textbackslash Rightarrow a=2.\textbackslash{]}

Alternatively, sub \textbackslash{[}x=1,\textbackslash{]} \textbackslash{[}y=4\textbackslash{]}into
the equation\$y=x+(a+1)\$ to obtain \$4=1+(a+1)\textbackslash Rightarrow
a=2\$