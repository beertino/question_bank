%% LyX 2.3.6.1 created this file.  For more info, see http://www.lyx.org/.
%% Do not edit unless you really know what you are doing.
\documentclass[twoside,english]{article}
\usepackage[T1]{fontenc}
\usepackage[latin9]{inputenc}
\usepackage{geometry}
\geometry{verbose,tmargin=2cm,bmargin=2cm,lmargin=2cm,rmargin=2cm}

\makeatletter
%%%%%%%%%%%%%%%%%%%%%%%%%%%%%% User specified LaTeX commands.
\usepackage{helvet}
\renewcommand{\familydefault}{\sfdefault}
\usepackage[T1]{fontenc}
\usepackage[latin9]{inputenc}
\usepackage{geometry}
\geometry{verbose,tmargin=1.8cm,bmargin=4cm,lmargin=1.5cm,rmargin=2cm}
\usepackage{enumitem}
\usepackage{amstext}
\usepackage{amsthm}
\usepackage{amssymb}
\usepackage{setspace}
\usepackage{graphicx}
\doublespacing


\usepackage{enumitem}
\setenumerate[1]{label=\textbf{\arabic*}}
\setenumerate[2]{label=\textbf{(\alph*)}}
\setenumerate[3]{label=\textbf{(\roman*)}}
\setlist[enumerate]{align=right}

\setcounter{page}{2}

%for upright integrals
\usepackage[integrals]{wasysym}

%to be used in conjunction with fancyfoot for last page
\usepackage{zref-totpages}

%fancyhrd settings
\usepackage{fancyhdr}
\pagestyle{fancy}
\fancyhf{}

\fancypagestyle{laststyle}
{
   \fancyhf{}
   \chead{\thepage}
   \fancyfoot[L]{\copyright NJC }
   \fancyfoot[R]{\textbf{END}} %Put \thispagestyle{laststyle} in the last page
}

%%centering page number
\chead{\thepage}

\renewcommand{\headrulewidth}{0pt}
\renewcommand{\footrulewidth}{0pt}

%%footer settings, different footer for ODD and EVEN pages, also for the LASTPAGE
\fancyfoot[LO]{\copyright NJC\hfill \textbf{[Turn Over}}
\fancyfoot[LE]{\copyright NJC }

%%shameless self-plug BRW

\makeatother

\usepackage{babel}
\begin{document}
{[}SPLIT\_HERE{]}
\begin{enumerate}
\item \textbf{{[}NJC/PROMO/9758/2021/Q1{]} }

A curve \$y=\textbackslash frac\{2\}\{\{\{x\}\textasciicircum\{2\}\}-3\}\$
undergoes, in succession, the following transformations, where a and
b are positive constants. A: A translation of a units in the negative
x-direction. B: A reflection about the y-axis. C: A scaling parallel
to the y-axis with scale factor of b. The equation of the resultant
curve is \$y=\textbackslash frac\{6\}\{\{\{x\}\textasciicircum\{2\}\}-10x+22\}\$.
Find the values of a and b. {[}3{]}

{[}SPLIT\_HERE{]}
\item \textbf{{[}NJC/PROMO/9758/2021/Q2{]} }

A candy shop is having a Halloween sale. The items that are on promotion
are chocolate bars, gummy bears and lollipops. There is a 20\% discount
for every chocolate bar purchased, a \$2 discount for every 3 bags
of gummy bears purchased and every 6 lollipops can be purchased at
the price of 5 lollipops.

Hannah, Jo and Pete are preparing for a Halloween party. The table
below shows the total bill and the number of chocolate bars, the number
of bags of gummy bears and the number of lollipops bought from the
candy shop. Chocolate Bar Gummy bears Lollipops Total Bill (\$) Jo
5 3 36 65.20 Hannah 4 14 24 119.84 Pete 17 5 20 89.12

Calculate the original selling price for each of a chocolate bar,
a bag of gummy bears and a lollipop. {[}4{]}

{[}SPLIT\_HERE{]}
\item \textbf{{[}NJC/PROMO/9758/2021/Q3{]} }

The diagram shows a sketch of the curve \$y=\textbackslash text\{f\}(x).\$
The curve cuts the x-axis at \textbackslash{[}\textbackslash left(
0,\textbackslash{} \textbackslash text\{0\} \textbackslash right)\textbackslash{]},
\textbackslash{[}\textbackslash left( 2,\textbackslash{} \textbackslash text\{0\}
\textbackslash right)\textbackslash{]} and \textbackslash{[}\textbackslash left(
4,\textbackslash{} \textbackslash text\{0\} \textbackslash right).\textbackslash{]}
It has a stationary point \textbackslash{[}\textbackslash left(
3,\textbackslash{} -\textbackslash frac\{5\}\{2\} \textbackslash right)\textbackslash{]}
and asymptotes with equations \textbackslash{[}x=1\textbackslash{]}
and \textbackslash{[}y=2.\textbackslash{]}

Sketch, on separate diagrams, the curves with the following equations,
stating the equations of any asymptotes and the coordinates of any
turning points and any points where the curves cross the x- and y-axes.

(a) \$y=\textbackslash left| \textbackslash text\{f\}(x) \textbackslash right|,\$
and {[}2{]}

(b) \$y=\textbackslash frac\{1\}\{\textbackslash text\{f\}(x)\}.\$
{[}3{]}

{[}SPLIT\_HERE{]}
\item \textbf{{[}NJC/PROMO/9758/2021/Q4{]} }
\begin{enumerate}
\item 4 (i) Solve the inequality \$20x\textbackslash ge \textbackslash frac\{17\{\{x\}\textasciicircum\{2\}\}+81x-6\}\{5-3x\},\$
leaving your answer in exact form. {[}4{]}
\item (ii) Hence solve the inequality \$\textbackslash frac\{20\}\{x\}\textbackslash ge
\textbackslash frac\{17+81x-6\{\{x\}\textasciicircum\{2\}\}\}\{5\{\{x\}\textasciicircum\{2\}\}-3x\}\$
exactly. {[}3{]}
\end{enumerate}
{[}SPLIT\_HERE{]}
\item \textbf{{[}NJC/PROMO/9758/2021/Q5{]} }

\textbackslash{[}\textbackslash text\{f\}(x)=\textbackslash left\textbackslash\{
\textbackslash begin\{matrix\} \{\{\textbackslash left( x-2 \textbackslash right)\}\textasciicircum\{2\}\},
\& 0\textbackslash le x<2 \textbackslash\textbackslash{} 2x-4,
\& 2\textbackslash le x\textbackslash le 4 \textbackslash\textbackslash{}
\textbackslash end\{matrix\} \textbackslash right.\textbackslash{]}

and that \$\textbackslash text\{f\}(x)=\textbackslash text\{f\}(x+4)\$
for all real values of x. (i) State the value of \$\textbackslash text\{f\}(21).\$
{[}1{]}

(ii) Sketch the graph of \$y=\textbackslash text\{f\}(x)\$ for \$-6\textbackslash le
x\textbackslash le 6.\$ {[}2{]}

The function g is defined by

\$\textbackslash text\{g\}(x)=\textbackslash sqrt\{x-4\},\textbackslash quad
4\textbackslash le x\textbackslash le 20.\$

(iii) Find fg in a similar form as f. {[}4{]}

{[}SPLIT\_HERE{]}
\item \textbf{{[}NJC/PROMO/9758/2021/Q6{]}}

A curve \$\{\{C\}\_\{1\}\}\$ has equation

\textbackslash{[}\{\{x\}\textasciicircum\{2\}\}+2\{\{y\}\textasciicircum\{2\}\}=100\textbackslash{]}

and a curve \$\{\{C\}\_\{2\}\}\$ has parametric equations

\textbackslash{[}x=2\{\{\textbackslash text\{e\}\}\textasciicircum\{-t\}\}-4\{\{\textbackslash text\{e\}\}\textasciicircum\{2t\}\},\textbackslash text\{
\}y=3\{\{\textbackslash text\{e\}\}\textasciicircum\{-t\}\}+\{\{\textbackslash text\{e\}\}\textasciicircum\{2t\}\}.\textbackslash{]}
(i) On the same diagram, sketch \$\{\{C\}\_\{1\}\}\$ and \$\{\{C\}\_\{2\}\},\$
labelling the coordinates of the points where both curves cross the
x- and y-axes. {[}5{]} 

(ii) Show that \$\{\{C\}\_\{2\}\}\$ has a Cartesian equation of the
form \$\{\{\textbackslash left( ax+by \textbackslash right)\}\textasciicircum\{2\}\}\textbackslash left(
cx+dy \textbackslash right)=k\$

for some integer constants a, b, c, d and k to be determined. {[}3{]}

{[}SPLIT\_HERE{]}
\item \textbf{{[}NJC/PROMO/9758/2021/Q7{]} }

Due to intense rainfall, the Bukit Teemah canal is often filled to
the brim, which causes the surrounding areas to be prone to flooding.
The Ministry of Environment is looking into redesigning the canal
to improve the flow of water by maximising the cross-sectional area,
\$A\textbackslash text\{ \}\{\{\textbackslash text\{m\}\}\textasciicircum\{2\}\},\$of
the canal.

The cross-section of the canal has sides of fixed lengths CD FG IJ
4 m, DE HI 3 m and EF GH 0.5 m. Also, the vertical depth DL IK s m
and \textbackslash{[}\textbackslash angle CDL=\textbackslash angle
JIK=\textbackslash theta \textbackslash{]} radians. 

(i) Show that \textbackslash{[}A=40\textbackslash cos \textbackslash theta
+8\textbackslash sin 2\textbackslash theta +2\textbackslash{]}.
{[}2{]}

(ii) Use differentiation to find the value of which gives a maximum
value of A. {[}4{]}

The National Water Agency conducts regular inspections on the water
quality in the canal. During one such inspection, an officer transfers
water from the canal into a plastic container (as shown in the diagram
below) at a constant rate of 162\$\textbackslash text\{c\}\{\{\textbackslash text\{m\}\}\textasciicircum\{3\}\}\$
per second. 

The plastic container is in the shape of a hollow circular cone with
fixed radius 15 cm and fixed height 35 cm. After t seconds, the depth
of water in the container is h cm and the top surface of the water
has a radius of r cm.

(iii) Find the rate at which h is increasing at the instant when h
21. {[}4{]} 

{[}SPLIT\_HERE{]}
\item \textbf{{[}NJC/PROMO/9758/2021/Q8{]}} 

(a) A curve C has parametric equations

\$x=2\textbackslash theta -\textbackslash sin 2\textbackslash theta
,\textbackslash text\{ \}y=1-\textbackslash cos 2\textbackslash theta
,\$ for \$0\textbackslash le \textbackslash theta \textbackslash le
\textbackslash frac\{\textbackslash text\{ \}\textbackslash !\textbackslash !\textbackslash pi\textbackslash !\textbackslash !\textbackslash text\{
\}\}\{2\}\$.

Find the exact area of the region bounded by C, the line \$x=\textbackslash text\{
\}\textbackslash !\textbackslash !\textbackslash pi\textbackslash !\textbackslash !\textbackslash text\{
\}\$ and the x-axis. {[}5{]} 

(b) The region bounded by the curve \$y=\textbackslash tan x\$, the
line \$y=1\$ and the y-axis is rotated about the x-axis through \$2\textbackslash text\{
\}\textbackslash !\textbackslash !\textbackslash pi\textbackslash !\textbackslash !\textbackslash text\{
\}\$ radians. Find the exact volume of the solid formed. {[}5{]}

{[}SPLIT\_HERE{]}
\item \textbf{{[}NJC/PROMO/9758/2021/Q9{]}} 

Relative to the origin O, the points A, B and C have position vectors
\textbackslash{[}2\textbackslash mathbf\{u\}-3\textbackslash mathbf\{v\},\textbackslash{]}\textbackslash{[}\textbackslash mathbf\{u\}+2\textbackslash mathbf\{v\}\textbackslash{]}and
\textbackslash{[}3\textbackslash mathbf\{u\}-2\textbackslash mathbf\{v\}\textbackslash{]}respectively,
where u and v are two non-zero and non-parallel vectors. (i) Show
that the area of triangle ABC is given by \$3\textbackslash left|
\textbackslash mathbf\{u\}\textbackslash times \textbackslash mathbf\{v\}
\textbackslash right|.\$ {[}3{]}

It is given that \$\textbackslash mathbf\{u\}=2\textbackslash mathbf\{i\}-\textbackslash mathbf\{j\}+2\textbackslash mathbf\{k\}\$
and \$\textbackslash mathbf\{v\}=3\textbackslash mathbf\{i\}-4\textbackslash mathbf\{j\}-p\textbackslash mathbf\{k\},\$
where p is a constant. 

(ii) If the area of triangle ABC is \$15\textbackslash sqrt\{53\}\$
square units, find the possible values of p exactly. {[}4{]}

(iii) It is given that the angle between vectors u and v is acute.
Find this angle, giving your answer in degrees. Explain why there
is only one answer. {[}3{]}

{[}SPLIT\_HERE{]}
\item \textbf{{[}NJC/PROMO/9758/2021/Q10{]}} 

The curve \$\{\{C\}\_\{1\}\}\$ has equation

\textbackslash{[}\{\{(x-1)\}\textasciicircum\{2\}\}-\{\{(y-4)\}\textasciicircum\{2\}\}=1.\textbackslash{]}

(i) Sketch \$\{\{C\}\_\{1\}\},\$ labelling clearly the equations of
any asymptotes and the coordinates of any vertices and the points
where the curve crosses the x- and y-axes. {[}3{]} 

The curve \$\{\{C\}\_\{2\}\}\$ with equation

\textbackslash{[}y=\textbackslash frac\{\{\{x\}\textasciicircum\{2\}\}+ax\}\{x-1\},\textbackslash{]}

where a is a constant, has two turning points.

(ii) Find the range of possible values of a, showing your working
clearly. {[}3{]}

It is further given that \$\{\{C\}\_\{1\}\}\$ does not intersect \$\{\{C\}\_\{2\}\}.\$

(iii) By finding the equation of the oblique asymptote of \$\{\{C\}\_\{2\}\}\$
in terms of a, find the value of a exactly. {[}2{]} (iv) Assuming
now that a is the value you have found in part (iii), sketch \$\{\{C\}\_\{2\}\}\$
on the same diagram in part (i), labelling clearly the equations of
any asymptotes and the coordinates of any turning points and the points
where the curve crosses the x- and y-axes. {[}3{]} 

{[}SPLIT\_HERE{]}
\item \textbf{{[}NJC/PROMO/9758/2021/Q11{]}} 

(a) Find \$\textbackslash int\{\textbackslash frac\{x\}\{\{\{x\}\textasciicircum\{4\}\}+6\{\{x\}\textasciicircum\{2\}\}+9\}\textbackslash text\{
d\}x\}.\$ {[}3{]}

(b) Find \$\textbackslash int\{\{\{x\}\textasciicircum\{2\}\}\textbackslash ln
\textbackslash left( x+2 \textbackslash right)\}\textbackslash text\{
d\}x.\$ {[}3{]} 

(c) Using the substitution \$u=\textbackslash frac\{1\}\{x\}\$, show
that 

\textbackslash{[}\textbackslash int\_\{\textbackslash sqrt\{2\}\}\textasciicircum\{2\}\{\textbackslash frac\{1\}\{x\textbackslash sqrt\{\{\{x\}\textasciicircum\{2\}\}-2\}\}\}\textbackslash text\{
d\}x=\textbackslash frac\{\textbackslash sqrt\{2\}\}\{k\}\textbackslash text\{
\}\textbackslash !\textbackslash !\textbackslash pi\textbackslash !\textbackslash !\textbackslash text\{
\}\textbackslash{]},

where k is a constant to be determined. {[}5{]}

{[}SPLIT\_HERE{]}
\item \textbf{{[}NJC/PROMO/9758/2021/Q12{]}} 

In the diagram below, a light source is placed at the point P with
coordinates \$(-2,\textbackslash{} -4,\textbackslash{} -2).\$ A rectangular
glass prism is placed such that the top of the prism is closer to
point P than the bottom of the prism.

It is given that the top of the prism is a part of the plane with
equation

\$\textbackslash mathbf\{r\}=\textbackslash left( \textbackslash begin\{matrix\}
-2 \textbackslash\textbackslash{} 3 \textbackslash\textbackslash{}
-2 \textbackslash\textbackslash{} \textbackslash end\{matrix\}
\textbackslash right)+\textbackslash lambda \textbackslash left(
\textbackslash begin\{matrix\} 1 \textbackslash\textbackslash{}
2 \textbackslash\textbackslash{} -3 \textbackslash\textbackslash{}
\textbackslash end\{matrix\} \textbackslash right)+\textbackslash mu
\textbackslash left( \textbackslash begin\{matrix\} 3 \textbackslash\textbackslash{}
-4 \textbackslash\textbackslash{} 1 \textbackslash\textbackslash{}
\textbackslash end\{matrix\} \textbackslash right),\$

where \$\textbackslash lambda \textbackslash text\{ and \}\textbackslash mu
\$ are parameters.

(i) Show that this plane has a Cartesian equation of the form \$x+y+z=d\$
for some constant d to be determined. {[}3{]}

\quad{} A ray of light is sent in direction \$3\textbackslash mathbf\{i\}+6\textbackslash mathbf\{j\}+2\textbackslash mathbf\{k\}\$
from the light source at P. The light ray enters the prism at point
Q which lies on the top of the prism, as shown in the diagram below.

(ii) Find the exact coordinates of Q. {[}3{]}

The light ray emerges from the prism at point R with coordinates \textbackslash{[}\textbackslash left(
c,\textbackslash{} \textbackslash frac\{53\}\{11\},\textbackslash ,\textbackslash frac\{91\}\{11\}
\textbackslash right),\textbackslash{]} as shown in the diagram
below.

It is known that the plane PQR is perpendicular to the top of the
prism.

(iii) Show that \$c=\textbackslash frac\{87\}\{11\}.\$ {[}3{]}

Snell\textquoteright s Law states that \$\textbackslash sin \textbackslash theta
=k\textbackslash sin \textbackslash phi ,\$ where k is the refractive
index of the prism, is the acute angle between the normal to the top
of the prism and PQ, and is the acute angle between the normal to
the top of the prism and QR.

(iv) Find the value of k. {[}3{]}

(v) Find the exact thickness of the prism measured in the direction
of the normal at Q. {[}2{]} 
\end{enumerate}

\end{document}
