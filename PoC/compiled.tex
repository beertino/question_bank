%% LyX 2.3.6.1 created this file.  For more info, see http://www.lyx.org/.
%% Do not edit unless you really know what you are doing.
\documentclass[twoside,english]{article}
\usepackage[T1]{fontenc}
\usepackage[latin9]{inputenc}
\usepackage{geometry}
\geometry{verbose,tmargin=2cm,bmargin=2cm,lmargin=2cm,rmargin=2cm}
\usepackage{color}
\usepackage{amsmath}
\usepackage{amssymb}

\makeatletter
%%%%%%%%%%%%%%%%%%%%%%%%%%%%%% User specified LaTeX commands.
\usepackage{helvet}
\renewcommand{\familydefault}{\sfdefault}
\usepackage[T1]{fontenc}
\usepackage[latin9]{inputenc}
\usepackage{geometry}
\geometry{verbose,tmargin=1.8cm,bmargin=4cm,lmargin=1.5cm,rmargin=2cm}
\usepackage{enumitem}
\usepackage{amstext}
\usepackage{amsthm}
\usepackage{amssymb}
\usepackage{setspace}
\usepackage{graphicx}
\doublespacing


\usepackage{enumitem}
\setenumerate[1]{label=\textbf{\arabic*}}
\setenumerate[2]{label=\textbf{(\alph*)}}
\setenumerate[3]{label=\textbf{(\roman*)}}
\setlist[enumerate]{align=right}

\setcounter{page}{2}

%for upright integrals
\usepackage[integrals]{wasysym}

%to be used in conjunction with fancyfoot for last page
\usepackage{zref-totpages}

%fancyhrd settings
\usepackage{fancyhdr}
\pagestyle{fancy}
\fancyhf{}

\fancypagestyle{laststyle}
{
   \fancyhf{}
   \chead{\thepage}
   \fancyfoot[L]{\copyright NJC }
   \fancyfoot[R]{\textbf{END}} %Put \thispagestyle{laststyle} in the last page
}

%%centering page number
\chead{\thepage}

\renewcommand{\headrulewidth}{0pt}
\renewcommand{\footrulewidth}{0pt}

%%footer settings, different footer for ODD and EVEN pages, also for the LASTPAGE
\fancyfoot[LO]{Generated with Python Script by BRW\hfill \textbf{[Turn Over}}
\fancyfoot[LE]{Generated with Python Script by BRW }

%%shameless self-plug BRW

\makeatother

\usepackage{babel}
\begin{document}

\title{2021 H2 Math PROMO Compilation - APGP (2 Qns with Ans)}
\maketitle
\begin{enumerate}\item \textbf{{[}NJC/PROMO/9758/2021/Q1{]} }

A curve \$y=\textbackslash frac\{2\}\{\{\{x\}\textasciicircum\{2\}\}-3\}\$
undergoes, in succession, the following transformations, where a and
b are positive constants. A: A translation of a units in the negative
x-direction. B: A reflection about the y-axis. C: A scaling parallel
to the y-axis with scale factor of b. The equation of the resultant
curve is \$y=\textbackslash frac\{6\}\{\{\{x\}\textasciicircum\{2\}\}-10x+22\}\$.
Find the values of a and b. {[}3{]}

SOLUTION

\item \$y=\textbackslash frac\{2\}\{\{\{x\}\textasciicircum\{2\}\}-3\}\textbackslash xrightarrow\{A\}y=\textbackslash frac\{2\}\{\{\{(x+a)\}\textasciicircum\{2\}\}-3\}\textbackslash xrightarrow\{B\}y=\textbackslash frac\{2\}\{\{\{(-x+a)\}\textasciicircum\{2\}\}-3\}\$
\$\textbackslash xrightarrow\{C\}\textbackslash frac\{y\}\{b\}=\textbackslash frac\{2\}\{\{\{(-x+a)\}\textasciicircum\{2\}\}-3\}\textbackslash Rightarrow
y=\textbackslash frac\{2b\}\{\{\{x\}\textasciicircum\{2\}\}-2ax+\{\{a\}\textasciicircum\{2\}\}-3\}\$
Comparing with\$y=\textbackslash frac\{6\}\{\{\{x\}\textasciicircum\{2\}\}-10x+22\},\$
\$\textbackslash begin\{align\} \& -2a=-10\textbackslash Rightarrow
a=5 \textbackslash\textbackslash{} \& b=3 \textbackslash\textbackslash{}
\textbackslash end\{align\}\$

Alternatively, \$y=\textbackslash frac\{6\}\{\{\{x\}\textasciicircum\{2\}\}-10x+22\}=\textbackslash frac\{6\}\{\{\{(5-x)\}\textasciicircum\{2\}\}-3\}\textbackslash Rightarrow
\textbackslash frac\{y\}\{3\}=\textbackslash frac\{2\}\{\{\{(5-x)\}\textasciicircum\{2\}\}-3\}\$
\$\textbackslash therefore a=5,\textbackslash text\{ \}b=3\$

 \newpage 

\item \textbf{{[}NJC/PROMO/9758/2021/Q10{]}} 

The curve \$\{\{C\}\_\{1\}\}\$ has equation

\textbackslash{[}\{\{(x-1)\}\textasciicircum\{2\}\}-\{\{(y-4)\}\textasciicircum\{2\}\}=1.\textbackslash{]}

(i) Sketch \$\{\{C\}\_\{1\}\},\$ labelling clearly the equations of
any asymptotes and the coordinates of any vertices and the points
where the curve crosses the x- and y-axes. {[}3{]} 

The curve \$\{\{C\}\_\{2\}\}\$ with equation

\textbackslash{[}y=\textbackslash frac\{\{\{x\}\textasciicircum\{2\}\}+ax\}\{x-1\},\textbackslash{]}

where a is a constant, has two turning points.

(ii) Find the range of possible values of a, showing your working
clearly. {[}3{]}

It is further given that \$\{\{C\}\_\{1\}\}\$ does not intersect \$\{\{C\}\_\{2\}\}.\$

(iii) By finding the equation of the oblique asymptote of \$\{\{C\}\_\{2\}\}\$
in terms of a, find the value of a exactly. {[}2{]} (iv) Assuming
now that a is the value you have found in part (iii), sketch \$\{\{C\}\_\{2\}\}\$
on the same diagram in part (i), labelling clearly the equations of
any asymptotes and the coordinates of any turning points and the points
where the curve crosses the x- and y-axes. {[}3{]}

SOLUTION

\item {}

(i), (iv) 

(ii) At the stationary points of \$\{\{C\}\_\{1\}\},\$ Thus, for \$\{\{C\}\_\{1\}\}\$
to have 2 turning points, \$\textbackslash begin\{align\} \& \{\{(-2)\}\textasciicircum\{2\}\}-4(1)(-a)>0
\textbackslash\textbackslash{} \& 4+4a>0 \textbackslash\textbackslash{}
\& a>-1 \textbackslash end\{align\}\$ (iii) \$\{\{C\}\_\{1\}\}\$
is a hyperbola with vertices lying on the horizontal line \textbackslash{[}y=4.\textbackslash{]}
\$\{\{C\}\_\{2\}\}:\$\$y=\textbackslash frac\{\{\{x\}\textasciicircum\{2\}\}+ax\}\{x-1\}=x+(a+1)+\textbackslash frac\{a+1\}\{x-1\}\$
Through observation from the sketch in part (i), we see that the only
way for the two curves to have no point of intersection is for both
curves to share the same asymptote with positive gradient. Thus the
oblique asymptote of \$\{\{C\}\_\{1\}\}\$ with positive gradient is
\textbackslash{[}y-4=(1)(x-1)\textbackslash Rightarrow y=x+3.\textbackslash{]}
Hence, \textbackslash{[}a+1=3\textbackslash Rightarrow a=2.\textbackslash{]}

Alternatively, sub \textbackslash{[}x=1,\textbackslash{]} \textbackslash{[}y=4\textbackslash{]}into
the equation\$y=x+(a+1)\$ to obtain \$4=1+(a+1)\textbackslash Rightarrow
a=2\$

 \newpage 

\item \textbf{{[}NJC/PROMO/9758/2021/Q11{]}} 

(a) Find \$\textbackslash int\{\textbackslash frac\{x\}\{\{\{x\}\textasciicircum\{4\}\}+6\{\{x\}\textasciicircum\{2\}\}+9\}\textbackslash text\{
d\}x\}.\$ {[}3{]}

(b) Find \$\textbackslash int\{\{\{x\}\textasciicircum\{2\}\}\textbackslash ln
\textbackslash left( x+2 \textbackslash right)\}\textbackslash text\{
d\}x.\$ {[}3{]} 

(c) Using the substitution \$u=\textbackslash frac\{1\}\{x\}\$, show
that 

\textbackslash{[}\textbackslash int\_\{\textbackslash sqrt\{2\}\}\textasciicircum\{2\}\{\textbackslash frac\{1\}\{x\textbackslash sqrt\{\{\{x\}\textasciicircum\{2\}\}-2\}\}\}\textbackslash text\{
d\}x=\textbackslash frac\{\textbackslash sqrt\{2\}\}\{k\}\textbackslash text\{
\}\textbackslash !\textbackslash !\textbackslash pi\textbackslash !\textbackslash !\textbackslash text\{
\}\textbackslash{]},

where k is a constant to be determined. {[}5{]}

SOLUTION

\item {}

(a) \$\textbackslash int\{\textbackslash frac\{x\}\{\{\{x\}\textasciicircum\{4\}\}+6\{\{x\}\textasciicircum\{2\}\}+9\}\textbackslash text\{
d\}x\}\$ \$\textbackslash begin\{align\} \& =\textbackslash frac\{1\}\{2\}\textbackslash int\{\textbackslash frac\{2x\}\{\{\{\textbackslash left(
\{\{x\}\textasciicircum\{2\}\}+3 \textbackslash right)\}\textasciicircum\{2\}\}\}\textbackslash text\{
d\}x\} \textbackslash\textbackslash{} \& =\textbackslash frac\{1\}\{2\}\textbackslash times
\textbackslash frac\{\{\{\textbackslash left( \{\{x\}\textasciicircum\{2\}\}+3
\textbackslash right)\}\textasciicircum\{-1\}\}\}\{-1\}+c \textbackslash\textbackslash{}
\& =-\textbackslash frac\{1\}\{2\textbackslash left( \{\{x\}\textasciicircum\{2\}\}+3
\textbackslash right)\}+c \textbackslash end\{align\}\$ (b) \textbackslash{[}\textbackslash begin\{align\}
\& \textbackslash int\{\{\{x\}\textasciicircum\{2\}\}\textbackslash ln
\textbackslash left( x+2 \textbackslash right)\}\textbackslash text\{
d\}x \textbackslash\textbackslash{} \& =\textbackslash frac\{\{\{x\}\textasciicircum\{3\}\}\}\{3\}\textbackslash ln
\textbackslash left( x+2 \textbackslash right)-\textbackslash int\{\textbackslash frac\{\{\{x\}\textasciicircum\{3\}\}\}\{3\}\textbackslash left(
\textbackslash frac\{1\}\{x+2\} \textbackslash right)\}\textbackslash text\{
d\}x \textbackslash\textbackslash{} \& =\textbackslash frac\{\{\{x\}\textasciicircum\{3\}\}\}\{3\}\textbackslash ln
\textbackslash left( x+2 \textbackslash right)-\textbackslash frac\{1\}\{3\}\textbackslash int\{\textbackslash frac\{\{\{x\}\textasciicircum\{3\}\}\}\{x+2\}\}\textbackslash text\{
d\}x \textbackslash\textbackslash{} \& =\textbackslash frac\{\{\{x\}\textasciicircum\{3\}\}\}\{3\}\textbackslash ln
\textbackslash left( x+2 \textbackslash right)-\textbackslash frac\{1\}\{3\}\textbackslash int\{\{\{x\}\textasciicircum\{2\}\}-2x+4-\textbackslash frac\{8\}\{x+2\}\}\textbackslash text\{
d\}x \textbackslash\textbackslash{} \& =\textbackslash frac\{\{\{x\}\textasciicircum\{3\}\}\}\{3\}\textbackslash ln
\textbackslash left( x+2 \textbackslash right)-\textbackslash frac\{1\}\{3\}\textbackslash left{[}
\textbackslash frac\{\{\{x\}\textasciicircum\{3\}\}\}\{3\}-\textbackslash frac\{2\{\{x\}\textasciicircum\{2\}\}\}\{2\}+4x-8\textbackslash ln
\textbackslash left( x+2 \textbackslash right) \textbackslash right{]}+c
\textbackslash\textbackslash{} \& =\textbackslash frac\{\{\{x\}\textasciicircum\{3\}\}\}\{3\}\textbackslash ln
\textbackslash left( x+2 \textbackslash right)-\textbackslash frac\{1\}\{3\}\textbackslash left{[}
\textbackslash frac\{\{\{x\}\textasciicircum\{3\}\}\}\{3\}-\{\{x\}\textasciicircum\{2\}\}+4x-8\textbackslash ln
\textbackslash left( x+2 \textbackslash right) \textbackslash right{]}+c\textbackslash text\{
or\} \textbackslash\textbackslash{} \& \textbackslash frac\{\{\{x\}\textasciicircum\{3\}\}\}\{3\}\textbackslash ln
\textbackslash left( x+2 \textbackslash right)-\textbackslash frac\{\{\{x\}\textasciicircum\{3\}\}\}\{9\}+\textbackslash frac\{\{\{x\}\textasciicircum\{2\}\}\}\{3\}-\textbackslash frac\{4\}\{3\}x+\textbackslash frac\{8\}\{3\}\textbackslash ln
\textbackslash left( x+2 \textbackslash right)+c\textbackslash text\{
or\} \textbackslash\textbackslash{} \& \textbackslash left( \textbackslash frac\{\{\{x\}\textasciicircum\{3\}\}+8\}\{3\}
\textbackslash right)\textbackslash ln \textbackslash left( x+2
\textbackslash right)-\textbackslash frac\{\{\{x\}\textasciicircum\{3\}\}\}\{9\}+\textbackslash frac\{\{\{x\}\textasciicircum\{2\}\}\}\{3\}-\textbackslash frac\{4\}\{3\}x+c\textbackslash{}
\textbackslash text\{or\} \textbackslash\textbackslash{} \& \textbackslash frac\{1\}\{9\}\textbackslash left{[}
3\textbackslash left( \{\{x\}\textasciicircum\{3\}\}+8 \textbackslash right)\textbackslash ln
\textbackslash left( x+2 \textbackslash right)-\{\{x\}\textasciicircum\{3\}\}+3\{\{x\}\textasciicircum\{2\}\}-12x
\textbackslash right{]}+c \textbackslash end\{align\}\textbackslash{]} 

\quad{} (c) \$\textbackslash text\{Let \}u=\textbackslash frac\{1\}\{x\}\textbackslash Rightarrow
x=\textbackslash frac\{1\}\{u\}\textbackslash Rightarrow \textbackslash frac\{\textbackslash text\{d\}x\}\{\textbackslash text\{d\}u\}=-\textbackslash frac\{1\}\{\{\{u\}\textasciicircum\{2\}\}\}\$
\$\textbackslash text\{when \}x=\textbackslash sqrt\{2\},\textbackslash text\{
\}u=\textbackslash frac\{\textbackslash sqrt\{2\}\}\{2\};\textbackslash text\{
when \}x=2,\textbackslash text\{ \}u=\textbackslash frac\{1\}\{2\}\$
\textbackslash{[}\textbackslash begin\{align\} \& \textbackslash int\_\{\textbackslash sqrt\{2\}\}\textasciicircum\{2\}\{\textbackslash frac\{1\}\{x\textbackslash sqrt\{\{\{x\}\textasciicircum\{2\}\}-2\}\}\}\textbackslash text\{
d\}x=\textbackslash int\_\{\textbackslash frac\{\textbackslash sqrt\{2\}\}\{2\}\}\textasciicircum\{\textbackslash frac\{1\}\{2\}\}\{\textbackslash frac\{u\}\{\textbackslash sqrt\{\textbackslash frac\{1\}\{\{\{u\}\textasciicircum\{2\}\}\}-2\}\}\}\textbackslash left(
-\textbackslash frac\{1\}\{\{\{u\}\textasciicircum\{2\}\}\} \textbackslash right)\textbackslash text\{
d\}u \textbackslash\textbackslash{} \& =\textbackslash int\_\{\textbackslash frac\{\textbackslash sqrt\{2\}\}\{2\}\}\textasciicircum\{\textbackslash frac\{1\}\{2\}\}\{\textbackslash frac\{u\}\{\textbackslash sqrt\{\textbackslash frac\{1-2\{\{u\}\textasciicircum\{2\}\}\}\{\{\{u\}\textasciicircum\{2\}\}\}\}\}\}\textbackslash left(
-\textbackslash frac\{1\}\{\{\{u\}\textasciicircum\{2\}\}\} \textbackslash right)\textbackslash text\{
d\}u \textbackslash\textbackslash{} \& =-\textbackslash int\_\{\textbackslash frac\{\textbackslash sqrt\{2\}\}\{2\}\}\textasciicircum\{\textbackslash frac\{1\}\{2\}\}\{\textbackslash frac\{1\}\{\textbackslash sqrt\{1-2\{\{u\}\textasciicircum\{2\}\}\}\}\}\textbackslash text\{
d\}u \textbackslash\textbackslash{} \& =-\textbackslash frac\{1\}\{\textbackslash sqrt\{2\}\}\textbackslash int\_\{\textbackslash frac\{\textbackslash sqrt\{2\}\}\{2\}\}\textasciicircum\{\textbackslash frac\{1\}\{2\}\}\{\textbackslash frac\{\textbackslash sqrt\{2\}\}\{\textbackslash sqrt\{1-\{\{\textbackslash left(
\textbackslash sqrt\{2\}u \textbackslash right)\}\textasciicircum\{2\}\}\}\}\}\textbackslash text\{
d\}u \textbackslash\textbackslash{} \& =-\textbackslash frac\{1\}\{\textbackslash sqrt\{2\}\}\textbackslash left{[}
\{\{\textbackslash sin \}\textasciicircum\{-1\}\}\textbackslash left(
\textbackslash sqrt\{2\}u \textbackslash right) \textbackslash right{]}\_\{\textbackslash frac\{\textbackslash sqrt\{2\}\}\{2\}\}\textasciicircum\{\textbackslash frac\{1\}\{2\}\}
\textbackslash\textbackslash{} \& =-\textbackslash frac\{1\}\{\textbackslash sqrt\{2\}\}\textbackslash left{[}
\textbackslash frac\{\textbackslash text\{ \}\textbackslash !\textbackslash !\textbackslash pi\textbackslash !\textbackslash !\textbackslash text\{
\}\}\{4\}-\textbackslash frac\{\textbackslash text\{ \}\textbackslash !\textbackslash !\textbackslash pi\textbackslash !\textbackslash !\textbackslash text\{
\}\}\{2\} \textbackslash right{]} \textbackslash\textbackslash{}
\& =\textbackslash frac\{1\}\{\textbackslash sqrt\{2\}\}\textbackslash left(
\textbackslash frac\{\textbackslash text\{ \}\textbackslash !\textbackslash !\textbackslash pi\textbackslash !\textbackslash !\textbackslash text\{
\}\}\{4\} \textbackslash right) \textbackslash\textbackslash{}
\& =\textbackslash frac\{\textbackslash sqrt\{2\}\}\{8\}\textbackslash text\{
\}\textbackslash !\textbackslash !\textbackslash pi\textbackslash !\textbackslash !\textbackslash text\{
\} \textbackslash end\{align\}\textbackslash{]}

 \newpage 

\item \textbf{{[}NJC/PROMO/9758/2021/Q12{]}} 

In the diagram below, a light source is placed at the point P with
coordinates \$(-2,\textbackslash{} -4,\textbackslash{} -2).\$ A rectangular
glass prism is placed such that the top of the prism is closer to
point P than the bottom of the prism.

It is given that the top of the prism is a part of the plane with
equation

\$\textbackslash mathbf\{r\}=\textbackslash left( \textbackslash begin\{matrix\}
-2 \textbackslash\textbackslash{} 3 \textbackslash\textbackslash{}
-2 \textbackslash\textbackslash{} \textbackslash end\{matrix\}
\textbackslash right)+\textbackslash lambda \textbackslash left(
\textbackslash begin\{matrix\} 1 \textbackslash\textbackslash{}
2 \textbackslash\textbackslash{} -3 \textbackslash\textbackslash{}
\textbackslash end\{matrix\} \textbackslash right)+\textbackslash mu
\textbackslash left( \textbackslash begin\{matrix\} 3 \textbackslash\textbackslash{}
-4 \textbackslash\textbackslash{} 1 \textbackslash\textbackslash{}
\textbackslash end\{matrix\} \textbackslash right),\$

where \$\textbackslash lambda \textbackslash text\{ and \}\textbackslash mu
\$ are parameters.

(i) Show that this plane has a Cartesian equation of the form \$x+y+z=d\$
for some constant d to be determined. {[}3{]}

\quad{} A ray of light is sent in direction \$3\textbackslash mathbf\{i\}+6\textbackslash mathbf\{j\}+2\textbackslash mathbf\{k\}\$
from the light source at P. The light ray enters the prism at point
Q which lies on the top of the prism, as shown in the diagram below.

(ii) Find the exact coordinates of Q. {[}3{]}

The light ray emerges from the prism at point R with coordinates \textbackslash{[}\textbackslash left(
c,\textbackslash{} \textbackslash frac\{53\}\{11\},\textbackslash ,\textbackslash frac\{91\}\{11\}
\textbackslash right),\textbackslash{]} as shown in the diagram
below.

It is known that the plane PQR is perpendicular to the top of the
prism.

(iii) Show that \$c=\textbackslash frac\{87\}\{11\}.\$ {[}3{]}

Snell\textquoteright s Law states that \$\textbackslash sin \textbackslash theta
=k\textbackslash sin \textbackslash phi ,\$ where k is the refractive
index of the prism, is the acute angle between the normal to the top
of the prism and PQ, and is the acute angle between the normal to
the top of the prism and QR.

(iv) Find the value of k. {[}3{]}

(v) Find the exact thickness of the prism measured in the direction
of the normal at Q. {[}2{]}

SOLUTION

\item {}

(i) Let the plane that contains the top of the prism be \$\{\{p\}\_\{1\}\}.\$
\$\textbackslash begin\{align\} \& \textbackslash left( \textbackslash begin\{matrix\}
1 \textbackslash\textbackslash{} 2 \textbackslash\textbackslash{}
-3 \textbackslash\textbackslash{} \textbackslash end\{matrix\}
\textbackslash right)\textbackslash times \textbackslash left(
\textbackslash begin\{matrix\} 3 \textbackslash\textbackslash{}
-4 \textbackslash\textbackslash{} 1 \textbackslash\textbackslash{}
\textbackslash end\{matrix\} \textbackslash right)=\textbackslash left(
\textbackslash begin\{matrix\} (2)(1)-(-4)(-3) \textbackslash\textbackslash{}
(-3)(3)-(1)(1) \textbackslash\textbackslash{} (1)(-4)-(2)(3) \textbackslash\textbackslash{}
\textbackslash end\{matrix\} \textbackslash right) \textbackslash\textbackslash{}
\& =\textbackslash left( \textbackslash begin\{matrix\} -10 \textbackslash\textbackslash{}
-10 \textbackslash\textbackslash{} -10 \textbackslash\textbackslash{}
\textbackslash end\{matrix\} \textbackslash right) \textbackslash end\{align\}\$
Thus, a vector normal to \$\{\{p\}\_\{1\}\}\$ is \$-\textbackslash frac\{1\}\{10\}\textbackslash left(
\textbackslash begin\{matrix\} -10 \textbackslash\textbackslash{}
-10 \textbackslash\textbackslash{} -10 \textbackslash\textbackslash{}
\textbackslash end\{matrix\} \textbackslash right)=\textbackslash left(
\textbackslash begin\{matrix\} 1 \textbackslash\textbackslash{}
1 \textbackslash\textbackslash{} 1 \textbackslash\textbackslash{}
\textbackslash end\{matrix\} \textbackslash right).\$ Hence a Cartesian
equation of \$\{\{p\}\_\{1\}\}\$ is \$\textbackslash begin\{align\}
\& \textbackslash mathbf\{r\}\textbackslash cdot \textbackslash left(
\textbackslash begin\{matrix\} 1 \textbackslash\textbackslash{}
1 \textbackslash\textbackslash{} 1 \textbackslash\textbackslash{}
\textbackslash end\{matrix\} \textbackslash right)=\textbackslash left(
\textbackslash begin\{matrix\} -2 \textbackslash\textbackslash{}
3 \textbackslash\textbackslash{} -2 \textbackslash\textbackslash{}
\textbackslash end\{matrix\} \textbackslash right)\textbackslash cdot
\textbackslash left( \textbackslash begin\{matrix\} 1 \textbackslash\textbackslash{}
1 \textbackslash\textbackslash{} 1 \textbackslash\textbackslash{}
\textbackslash end\{matrix\} \textbackslash right) \textbackslash\textbackslash{}
\& =-2+3-2=-1 \textbackslash\textbackslash{} \& x+y+z=-1 \textbackslash end\{align\}\$
(ii) Since Q lies on l, 

\textbackslash{[}\textbackslash overset\{\textbackslash xrightarrow\{\{\}\}\}\{\textbackslash mathop\{OQ\}\}\textbackslash ,=\textbackslash left(
\textbackslash begin\{matrix\} -2+3q \textbackslash\textbackslash{}
-4+6q \textbackslash\textbackslash{} -2+2q \textbackslash\textbackslash{}
\textbackslash end\{matrix\} \textbackslash right)\textbackslash{]}
for some \textbackslash{[}q\textbackslash in \textbackslash mathbb\{R\}.\textbackslash{]}

Since Q lies on \$\{\{p\}\_\{1\}\},\$ \textbackslash{[}\textbackslash begin\{align\}
\& (-2+3q)+(-4+6q)+(-2+2q)=-1 \textbackslash\textbackslash{} \&
-8+11q=-1 \textbackslash\textbackslash{} \& q=\textbackslash frac\{7\}\{11\}
\textbackslash end\{align\}\textbackslash{]}

Therefore, \textbackslash{[}\textbackslash overset\{\textbackslash xrightarrow\{\{\}\}\}\{\textbackslash mathop\{OQ\}\}\textbackslash ,=\textbackslash left(
\textbackslash begin\{matrix\} -2+3\textbackslash left( \textbackslash tfrac\{7\}\{11\}
\textbackslash right) \textbackslash\textbackslash{} -4+6\textbackslash left(
\textbackslash tfrac\{7\}\{11\} \textbackslash right) \textbackslash\textbackslash{}
-2+2\textbackslash left( \textbackslash tfrac\{7\}\{11\} \textbackslash right)
\textbackslash\textbackslash{} \textbackslash end\{matrix\} \textbackslash right)=\textbackslash left(
\textbackslash begin\{matrix\} -\{1\}/\{11\}\textbackslash ; \textbackslash\textbackslash{}
-\{2\}/\{11\}\textbackslash ; \textbackslash\textbackslash{} -\{8\}/\{11\}\textbackslash ;
\textbackslash\textbackslash{} \textbackslash end\{matrix\} \textbackslash right)\textbackslash{]}

Therefore, the coordinates of Q are \textbackslash{[}\textbackslash left(
-\textbackslash frac\{1\}\{11\},\textbackslash{} -\textbackslash frac\{2\}\{11\},\textbackslash ,-\textbackslash frac\{8\}\{11\}
\textbackslash right).\textbackslash{]} 

(iii) \$\textbackslash left( \textbackslash begin\{matrix\} 3 \textbackslash\textbackslash{}
6 \textbackslash\textbackslash{} 2 \textbackslash\textbackslash{}
\textbackslash end\{matrix\} \textbackslash right)\textbackslash times
\textbackslash left( \textbackslash begin\{matrix\} 1 \textbackslash\textbackslash{}
1 \textbackslash\textbackslash{} 1 \textbackslash\textbackslash{}
\textbackslash end\{matrix\} \textbackslash right)=\textbackslash left(
\textbackslash begin\{matrix\} (6)(1)-(2)(1) \textbackslash\textbackslash{}
(2)(1)-(3)(1) \textbackslash\textbackslash{} (3)(1)-(6)(1) \textbackslash\textbackslash{}
\textbackslash end\{matrix\} \textbackslash right)=\textbackslash left(
\textbackslash begin\{matrix\} 4 \textbackslash\textbackslash{}
-1 \textbackslash\textbackslash{} -3 \textbackslash\textbackslash{}
\textbackslash end\{matrix\} \textbackslash right)\$ Thus \$\textbackslash left(
\textbackslash begin\{matrix\} 4 \textbackslash\textbackslash{}
-1 \textbackslash\textbackslash{} -3 \textbackslash\textbackslash{}
\textbackslash end\{matrix\} \textbackslash right)\$ is a normal
vector to the plane PQR. Therefore, \$\textbackslash overset\{\textbackslash xrightarrow\{\{\}\}\}\{\textbackslash mathop\{QR\}\}\textbackslash ,\textbackslash{}
\textbackslash cdot \textbackslash left( \textbackslash begin\{matrix\}
4 \textbackslash\textbackslash{} -1 \textbackslash\textbackslash{}
-3 \textbackslash\textbackslash{} \textbackslash end\{matrix\}
\textbackslash right)=0\$ \$\textbackslash begin\{align\} \& \textbackslash left(
\textbackslash begin\{matrix\} c-\textbackslash left( -\textbackslash tfrac\{1\}\{11\}
\textbackslash right) \textbackslash\textbackslash{} \textbackslash tfrac\{53\}\{11\}-\textbackslash left(
-\textbackslash tfrac\{2\}\{11\} \textbackslash right) \textbackslash\textbackslash{}
\textbackslash tfrac\{91\}\{11\}-\textbackslash left( -\textbackslash tfrac\{8\}\{11\}
\textbackslash right) \textbackslash\textbackslash{} \textbackslash end\{matrix\}
\textbackslash right)\textbackslash cdot \textbackslash left( \textbackslash begin\{matrix\}
4 \textbackslash\textbackslash{} -1 \textbackslash\textbackslash{}
-3 \textbackslash\textbackslash{} \textbackslash end\{matrix\}
\textbackslash right)=0 \textbackslash\textbackslash{} \& \textbackslash left(
\textbackslash begin\{matrix\} c+\textbackslash tfrac\{1\}\{11\}
\textbackslash\textbackslash{} 5 \textbackslash\textbackslash{}
9 \textbackslash\textbackslash{} \textbackslash end\{matrix\} \textbackslash right)\textbackslash cdot
\textbackslash left( \textbackslash begin\{matrix\} 4 \textbackslash\textbackslash{}
-1 \textbackslash\textbackslash{} -3 \textbackslash\textbackslash{}
\textbackslash end\{matrix\} \textbackslash right)=0 \textbackslash\textbackslash{}
\& 4c+\textbackslash tfrac\{4\}\{11\}-5-27=0 \textbackslash\textbackslash{}
\& 4c=\textbackslash frac\{348\}\{11\} \textbackslash\textbackslash{}
\& c=\textbackslash frac\{87\}\{11\} \textbackslash end\{align\}\$
(iv) \textbackslash{[}\textbackslash begin\{align\} \& \textbackslash cos
\textbackslash theta =\textbackslash left| \textbackslash frac\{\textbackslash left(
\textbackslash begin\{matrix\} 3 \textbackslash\textbackslash{}
6 \textbackslash\textbackslash{} 2 \textbackslash\textbackslash{}
\textbackslash end\{matrix\} \textbackslash right)\textbackslash cdot
\textbackslash left( \textbackslash begin\{matrix\} 1 \textbackslash\textbackslash{}
1 \textbackslash\textbackslash{} 1 \textbackslash\textbackslash{}
\textbackslash end\{matrix\} \textbackslash right)\}\{\textbackslash sqrt\{\{\{3\}\textasciicircum\{2\}\}+\{\{6\}\textasciicircum\{2\}\}+\{\{2\}\textasciicircum\{2\}\}\}\textbackslash sqrt\{\{\{1\}\textasciicircum\{2\}\}+\{\{1\}\textasciicircum\{2\}\}+\{\{1\}\textasciicircum\{2\}\}\}\}
\textbackslash right| \textbackslash\textbackslash{} \& =\textbackslash left|
\textbackslash frac\{3+6+2\}\{7\textbackslash sqrt\{3\}\} \textbackslash right|
\textbackslash\textbackslash{} \& =\textbackslash frac\{11\}\{7\textbackslash sqrt\{3\}\}\textbackslash Rightarrow
\textbackslash theta =\{\{\textbackslash cos \}\textasciicircum\{-1\}\}\textbackslash left(
\textbackslash frac\{11\}\{7\textbackslash sqrt\{3\}\} \textbackslash right)=24.870\{\}\textasciicircum\textbackslash circ
\textbackslash end\{align\}\textbackslash{]}

\textbackslash{[}\textbackslash begin\{align\} \& \textbackslash cos
\textbackslash phi =\textbackslash left| \textbackslash frac\{\textbackslash left(
\textbackslash begin\{matrix\} 8 \textbackslash\textbackslash{}
5 \textbackslash\textbackslash{} 9 \textbackslash\textbackslash{}
\textbackslash end\{matrix\} \textbackslash right)\textbackslash cdot
\textbackslash left( \textbackslash begin\{matrix\} 1 \textbackslash\textbackslash{}
1 \textbackslash\textbackslash{} 1 \textbackslash\textbackslash{}
\textbackslash end\{matrix\} \textbackslash right)\}\{\textbackslash sqrt\{\{\{8\}\textasciicircum\{2\}\}+\{\{5\}\textasciicircum\{2\}\}+\{\{9\}\textasciicircum\{2\}\}\}\textbackslash sqrt\{\{\{1\}\textasciicircum\{2\}\}+\{\{1\}\textasciicircum\{2\}\}+\{\{1\}\textasciicircum\{2\}\}\}\}
\textbackslash right| \textbackslash\textbackslash{} \& =\textbackslash left|
\textbackslash frac\{8+5+9\}\{\textbackslash sqrt\{510\}\} \textbackslash right|=\textbackslash frac\{22\}\{\textbackslash sqrt\{510\}\}
\textbackslash\textbackslash{} \& \textbackslash theta =\{\{\textbackslash cos
\}\textasciicircum\{-1\}\}\textbackslash left( \textbackslash frac\{22\}\{\textbackslash sqrt\{510\}\}
\textbackslash right)=13.049\{\}\textasciicircum\textbackslash circ
\textbackslash end\{align\}\textbackslash{]}

\textbackslash{[}k=\textbackslash frac\{\textbackslash sin \textbackslash theta
\}\{\textbackslash sin \textbackslash phi \}=\textbackslash frac\{\textbackslash sin
24.870\{\}\textasciicircum\textbackslash circ \}\{\textbackslash sin
13.049\{\}\textasciicircum\textbackslash circ \}=1.86\textbackslash text\{
(to 3 s\}\textbackslash text\{.f\}\textbackslash text\{.)\}\textbackslash{]}
(v) The thickness of prism is the length of projection of QR onto
the normal. Therefore, \textbackslash{[}\textbackslash begin\{align\}
\& \textbackslash text\{Thickness of prism\}=\textbackslash left|
\textbackslash frac\{\textbackslash overset\{\textbackslash xrightarrow\{\{\}\}\}\{\textbackslash mathop\{QR\}\}\textbackslash ,\textbackslash{}
\textbackslash cdot \textbackslash left( \textbackslash begin\{matrix\}
1 \textbackslash\textbackslash{} 1 \textbackslash\textbackslash{}
1 \textbackslash\textbackslash{} \textbackslash end\{matrix\} \textbackslash right)\}\{\textbackslash sqrt\{\{\{1\}\textasciicircum\{2\}\}+\{\{1\}\textasciicircum\{2\}\}+\{\{1\}\textasciicircum\{2\}\}\}\}
\textbackslash right| \textbackslash\textbackslash{} \& =\textbackslash left|
\textbackslash frac\{\textbackslash left( \textbackslash begin\{matrix\}
8 \textbackslash\textbackslash{} 5 \textbackslash\textbackslash{}
9 \textbackslash\textbackslash{} \textbackslash end\{matrix\} \textbackslash right)\textbackslash{}
\textbackslash cdot \textbackslash left( \textbackslash begin\{matrix\}
1 \textbackslash\textbackslash{} 1 \textbackslash\textbackslash{}
1 \textbackslash\textbackslash{} \textbackslash end\{matrix\} \textbackslash right)\}\{\textbackslash sqrt\{3\}\}
\textbackslash right| \textbackslash\textbackslash{} \& =\textbackslash left|
\textbackslash frac\{8+5+9\}\{\textbackslash sqrt\{3\}\} \textbackslash right|
\textbackslash\textbackslash{} \& =\textbackslash frac\{22\}\{\textbackslash sqrt\{3\}\}
\textbackslash\textbackslash{} \& =\textbackslash frac\{22\textbackslash sqrt\{3\}\}\{3\}\textbackslash text\{
units\} \textbackslash end\{align\}\textbackslash{]} Alternatively,
\textbackslash{[}\textbackslash begin\{align\} \& \textbackslash text\{Thickness
of prism\}=QR\textbackslash cos \textbackslash phi \textbackslash\textbackslash{}
\& =\textbackslash sqrt\{\{\{8\}\textasciicircum\{2\}\}+\{\{5\}\textasciicircum\{2\}\}+\{\{9\}\textasciicircum\{2\}\}\}\textbackslash times
\textbackslash frac\{22\}\{\textbackslash sqrt\{510\}\} \textbackslash\textbackslash{}
\& =\textbackslash sqrt\{170\}\textbackslash times \textbackslash frac\{22\}\{\textbackslash sqrt\{510\}\}
\textbackslash\textbackslash{} \& =\textbackslash frac\{22\}\{\textbackslash sqrt\{3\}\}
\textbackslash\textbackslash{} \& =\textbackslash frac\{22\textbackslash sqrt\{3\}\}\{3\}\textbackslash text\{
units\} \textbackslash end\{align\}\textbackslash{]} .

 \newpage 

\item \textbf{{[}NJC/PROMO/9758/2021/Q2{]} }

A candy shop is having a Halloween sale. The items that are on promotion
are chocolate bars, gummy bears and lollipops. There is a 20\% discount
for every chocolate bar purchased, a \$2 discount for every 3 bags
of gummy bears purchased and every 6 lollipops can be purchased at
the price of 5 lollipops.

Hannah, Jo and Pete are preparing for a Halloween party. The table
below shows the total bill and the number of chocolate bars, the number
of bags of gummy bears and the number of lollipops bought from the
candy shop. Chocolate Bar Gummy bears Lollipops Total Bill (\$) Jo
5 3 36 65.20 Hannah 4 14 24 119.84 Pete 17 5 20 89.12

Calculate the original selling price for each of a chocolate bar,
a bag of gummy bears and a lollipop. {[}4{]}

SOLUTION

\item Let \$x, \$y and \$z be the cost of a chocolate bar, a bag of gummy
bears and a lollipop respectively.

\$\textbackslash begin\{align\} \& 5(0.8)x+(3y-2)+30z=65.20 \textbackslash\textbackslash{}
\& 4x+3y+30z=67.20\textbackslash text\{ -{}-{}-{}-{}- (1)\} \textbackslash\textbackslash{}
\& 4(0.8)x+(14y-8)+20z=119.84 \textbackslash\textbackslash{} \&
3.2x+14y+20z=127.84\textbackslash text\{ -{}-{}-{}-{}- (2) \} \textbackslash\textbackslash{}
\& \textbackslash text\{17(0\}\textbackslash text\{.8)\}x+(5y-2)+17z=89.12\textbackslash text\{
\} \textbackslash\textbackslash{} \& \textbackslash text\{ 13\}\textbackslash text\{.6\}x+5y+17z\textbackslash text\{
\}=91.12\textbackslash text\{ -{}-{}-{}-{}- (3) \} \textbackslash end\{align\}\$

From GC, \$x=2.70,\textbackslash text\{ \}y=6.80\textbackslash text\{
and \}z=1.20\$ The cost of a chocolate bar, a bag of gummy bears and
a lollipop is \$2.70, \$6.80 and \$1.20 respectively.

 \newpage 

\item \textbf{{[}NJC/PROMO/9758/2021/Q3{]} }

The diagram shows a sketch of the curve \$y=\textbackslash text\{f\}(x).\$
The curve cuts the x-axis at \textbackslash{[}\textbackslash left(
0,\textbackslash{} \textbackslash text\{0\} \textbackslash right)\textbackslash{]},
\textbackslash{[}\textbackslash left( 2,\textbackslash{} \textbackslash text\{0\}
\textbackslash right)\textbackslash{]} and \textbackslash{[}\textbackslash left(
4,\textbackslash{} \textbackslash text\{0\} \textbackslash right).\textbackslash{]}
It has a stationary point \textbackslash{[}\textbackslash left(
3,\textbackslash{} -\textbackslash frac\{5\}\{2\} \textbackslash right)\textbackslash{]}
and asymptotes with equations \textbackslash{[}x=1\textbackslash{]}
and \textbackslash{[}y=2.\textbackslash{]}

Sketch, on separate diagrams, the curves with the following equations,
stating the equations of any asymptotes and the coordinates of any
turning points and any points where the curves cross the x- and y-axes.

(a) \$y=\textbackslash left| \textbackslash text\{f\}(x) \textbackslash right|,\$
and {[}2{]}

(b) \$y=\textbackslash frac\{1\}\{\textbackslash text\{f\}(x)\}.\$
{[}3{]}

SOLUTION

\item {}

(a) 

(b)

 \newpage 

\item \textbf{{[}NJC/PROMO/9758/2021/Q4{]} }
\begin{enumerate}
\item 4 (i) Solve the inequality \$20x\textbackslash ge \textbackslash frac\{17\{\{x\}\textasciicircum\{2\}\}+81x-6\}\{5-3x\},\$
leaving your answer in exact form. {[}4{]}
\item (ii) Hence solve the inequality \$\textbackslash frac\{20\}\{x\}\textbackslash ge
\textbackslash frac\{17+81x-6\{\{x\}\textasciicircum\{2\}\}\}\{5\{\{x\}\textasciicircum\{2\}\}-3x\}\$
exactly. {[}3{]}
\end{enumerate}

SOLUTION

\item (i) \textbackslash{[}20x\textbackslash ge \textbackslash frac\{17\{\{x\}\textasciicircum\{2\}\}+81x-6\}\{5-3x\}\textbackslash text\{
and \}x\textbackslash ne \textbackslash frac\{5\}\{3\}\textbackslash{]}
\textbackslash{[}\textbackslash begin\{align\} \& \textbackslash frac\{17\{\{x\}\textasciicircum\{2\}\}+81x-6\}\{5-3x\}-20x\textbackslash le
0 \textbackslash\textbackslash{} \& \textbackslash frac\{17\{\{x\}\textasciicircum\{2\}\}+81x-6-20x\textbackslash left(
5-3x \textbackslash right)\}\{5-3x\}\textbackslash le 0 \textbackslash\textbackslash{}
\& \textbackslash frac\{17\{\{x\}\textasciicircum\{2\}\}+81x-6-100x+60\{\{x\}\textasciicircum\{2\}\}\}\{5-3x\}\textbackslash le
0 \textbackslash\textbackslash{} \& \textbackslash frac\{77\{\{x\}\textasciicircum\{2\}\}-19x-6\}\{5-3x\}\textbackslash le
0 \textbackslash\textbackslash{} \& \textbackslash frac\{\textbackslash left(
7x-3 \textbackslash right)\textbackslash left( 11x+2 \textbackslash right)\}\{5-3x\}\textbackslash le
0 \textbackslash end\{align\}\textbackslash{]}

\$-\textbackslash frac\{2\}\{11\}\textbackslash le x\textbackslash le
\textbackslash frac\{3\}\{7\}\textbackslash text\{ or\}\textasciitilde\textasciitilde x>\textbackslash frac\{5\}\{3\}\$
(ii) Replacing x with \textbackslash{[}\textbackslash frac\{1\}\{x\}\textbackslash{]}
in the original inequality, \textbackslash{[}\textbackslash begin\{align\}
\& 20\textbackslash left( \textbackslash frac\{1\}\{x\} \textbackslash right)\textbackslash ge
\textbackslash frac\{17\{\{\textbackslash left( \textbackslash frac\{1\}\{x\}
\textbackslash right)\}\textasciicircum\{2\}\}+81\textbackslash left(
\textbackslash frac\{1\}\{x\} \textbackslash right)-6\}\{5-3\textbackslash left(
\textbackslash frac\{1\}\{x\} \textbackslash right)\}\textbackslash times
\textbackslash frac\{\{\{x\}\textasciicircum\{2\}\}\}\{\{\{x\}\textasciicircum\{2\}\}\}
\textbackslash\textbackslash{} \& \textbackslash frac\{20\}\{x\}\textbackslash ge
\textbackslash frac\{17+81x-6\{\{x\}\textasciicircum\{2\}\}\}\{5\{\{x\}\textasciicircum\{2\}\}-3x\}
\textbackslash end\{align\}\textbackslash{]} 

Thus, from the first part,

\$\textbackslash begin\{align\} \& -\textbackslash frac\{2\}\{11\}\textbackslash le
\textbackslash frac\{1\}\{x\}\textbackslash le \textbackslash frac\{3\}\{7\}\textbackslash text\{
or\}\textasciitilde\textbackslash{} \textbackslash frac\{1\}\{x\}>\textbackslash frac\{5\}\{3\}
\textbackslash\textbackslash{} \& x\textbackslash le -\textbackslash frac\{11\}\{2\}\textbackslash text\{
or \}x\textbackslash ge \textbackslash frac\{7\}\{3\}\textbackslash text\{
or\}\textasciitilde\textbackslash{} 0<x<\textbackslash frac\{3\}\{5\}
\textbackslash\textbackslash{} \textbackslash end\{align\}\$

 \newpage 

\item \textbf{{[}NJC/PROMO/9758/2021/Q5{]} }

\textbackslash{[}\textbackslash text\{f\}(x)=\textbackslash left\textbackslash\{
\textbackslash begin\{matrix\} \{\{\textbackslash left( x-2 \textbackslash right)\}\textasciicircum\{2\}\},
\& 0\textbackslash le x<2 \textbackslash\textbackslash{} 2x-4,
\& 2\textbackslash le x\textbackslash le 4 \textbackslash\textbackslash{}
\textbackslash end\{matrix\} \textbackslash right.\textbackslash{]}

and that \$\textbackslash text\{f\}(x)=\textbackslash text\{f\}(x+4)\$
for all real values of x. (i) State the value of \$\textbackslash text\{f\}(21).\$
{[}1{]}

(ii) Sketch the graph of \$y=\textbackslash text\{f\}(x)\$ for \$-6\textbackslash le
x\textbackslash le 6.\$ {[}2{]}

The function g is defined by

\$\textbackslash text\{g\}(x)=\textbackslash sqrt\{x-4\},\textbackslash quad
4\textbackslash le x\textbackslash le 20.\$

(iii) Find fg in a similar form as f. {[}4{]}

SOLUTION

\item {}

(i) \$\textbackslash text\{f\}(21)=\textbackslash text\{f\}(5\textbackslash times
4+1)=\textbackslash text\{f\}(1)=\{\{(1-2)\}\textasciicircum\{2\}\}=1.\$ 

(ii) 

(iii) When \$0\textbackslash le \textbackslash text\{g\}\textbackslash left(
x \textbackslash right)<2\$, \$4\textbackslash le x<8\$. This corresponds
to \$\textbackslash text\{fg\}(x)=\textbackslash text\{f\}\textbackslash left(
\textbackslash sqrt\{x-4\} \textbackslash right)=\{\{\textbackslash left(
\textbackslash sqrt\{x-4\}-2 \textbackslash right)\}\textasciicircum\{2\}\}\$.
When \$2\textbackslash le \textbackslash text\{g\}\textbackslash left(
x \textbackslash right)\textbackslash le 4\$, \$8\textbackslash le
x\textbackslash le 20\$. This corresponds to \$\textbackslash text\{fg\}(x)=\textbackslash text\{f\}\textbackslash left(
\textbackslash sqrt\{x-4\} \textbackslash right)=2\textbackslash sqrt\{x-4\}-4\$.

Therefore, \$\textbackslash text\{fg\}(x)=\textbackslash left\textbackslash\{
\textbackslash begin\{matrix\} \{\{\textbackslash left( \textbackslash sqrt\{x-4\}-2
\textbackslash right)\}\textasciicircum\{2\}\}\textbackslash text\{,
\} \& 4\textbackslash le x<8 \textbackslash\textbackslash{} 2\textbackslash sqrt\{x-4\}-4,\textbackslash text\{
\} \& 8\textbackslash le x\textbackslash le 20 \textbackslash\textbackslash{}
\textbackslash end\{matrix\} \textbackslash right.\$

 \newpage 

\item \textbf{{[}NJC/PROMO/9758/2021/Q6{]}}

A curve \$\{\{C\}\_\{1\}\}\$ has equation

\textbackslash{[}\{\{x\}\textasciicircum\{2\}\}+2\{\{y\}\textasciicircum\{2\}\}=100\textbackslash{]}

and a curve \$\{\{C\}\_\{2\}\}\$ has parametric equations

\textbackslash{[}x=2\{\{\textbackslash text\{e\}\}\textasciicircum\{-t\}\}-4\{\{\textbackslash text\{e\}\}\textasciicircum\{2t\}\},\textbackslash text\{
\}y=3\{\{\textbackslash text\{e\}\}\textasciicircum\{-t\}\}+\{\{\textbackslash text\{e\}\}\textasciicircum\{2t\}\}.\textbackslash{]}
(i) On the same diagram, sketch \$\{\{C\}\_\{1\}\}\$ and \$\{\{C\}\_\{2\}\},\$
labelling the coordinates of the points where both curves cross the
x- and y-axes. {[}5{]} 

(ii) Show that \$\{\{C\}\_\{2\}\}\$ has a Cartesian equation of the
form \$\{\{\textbackslash left( ax+by \textbackslash right)\}\textasciicircum\{2\}\}\textbackslash left(
cx+dy \textbackslash right)=k\$

for some integer constants a, b, c, d and k to be determined. {[}3{]}

SOLUTION

\item {}

(i) For \$\{\{C\}\_\{2\}\},\$ when \textbackslash{[}x=0,\textbackslash{]}

\textbackslash{[}\textbackslash begin\{align\} \& 2\{\{\textbackslash text\{e\}\}\textasciicircum\{-t\}\}-4\{\{\textbackslash text\{e\}\}\textasciicircum\{2t\}\}=0
\textbackslash\textbackslash{} \& 4\{\{\textbackslash text\{e\}\}\textasciicircum\{2t\}\}=2\{\{\textbackslash text\{e\}\}\textasciicircum\{-t\}\}
\textbackslash\textbackslash{} \& \{\{\textbackslash text\{e\}\}\textasciicircum\{3t\}\}=\textbackslash frac\{1\}\{2\}
\textbackslash\textbackslash{} \& t=\textbackslash frac\{1\}\{3\}\textbackslash ln
\textbackslash frac\{1\}\{2\}=-\textbackslash frac\{1\}\{3\}\textbackslash ln
2 \textbackslash end\{align\}\textbackslash{]}

Therefore,

\textbackslash{[}y=3\{\{\textbackslash text\{e\}\}\textasciicircum\{\textbackslash frac\{1\}\{3\}\textbackslash ln
2\}\}+\{\{\textbackslash text\{e\}\}\textasciicircum\{-\textbackslash frac\{2\}\{3\}\textbackslash ln
2\}\}=4.41\textbackslash text\{ (to 3 s\}\textbackslash text\{.f\}\textbackslash text\{.)\}\textbackslash{]}

(ii) \textbackslash{[}x=2\{\{\textbackslash text\{e\}\}\textasciicircum\{-t\}\}-4\{\{\textbackslash text\{e\}\}\textasciicircum\{2t\}\}(1)\textbackslash{]}
\textbackslash{[}y=3\{\{\textbackslash text\{e\}\}\textasciicircum\{-t\}\}+\{\{\textbackslash text\{e\}\}\textasciicircum\{2t\}\}(2)\textbackslash{]}
\$2\textbackslash times (2)-3\textbackslash times (1):2y-3x=14\{\{\textbackslash text\{e\}\}\textasciicircum\{2t\}\}(3)\$
\$\textbackslash begin\{align\} \& 4\textbackslash times (2)+1\textbackslash times
(1):4y+x=14\{\{\textbackslash text\{e\}\}\textasciicircum\{-t\}\}
\textbackslash\textbackslash{} \& \{\{\textbackslash text\{e\}\}\textasciicircum\{t\}\}=\textbackslash frac\{14\}\{x+4y\}(4)
\textbackslash\textbackslash{} \textbackslash end\{align\}\$ Substituting
(4) into (3), \$2y-3x=14\{\{\textbackslash left( \textbackslash frac\{14\}\{x+4y\}
\textbackslash right)\}\textasciicircum\{2\}\}\$ \$\{\{\textbackslash left(
x+4y \textbackslash right)\}\textasciicircum\{2\}\}\textbackslash left(
2y-3x \textbackslash right)=2744\$

 \newpage 

\item \textbf{{[}NJC/PROMO/9758/2021/Q7{]} }

Due to intense rainfall, the Bukit Teemah canal is often filled to
the brim, which causes the surrounding areas to be prone to flooding.
The Ministry of Environment is looking into redesigning the canal
to improve the flow of water by maximising the cross-sectional area,
\$A\textbackslash text\{ \}\{\{\textbackslash text\{m\}\}\textasciicircum\{2\}\},\$of
the canal.

The cross-section of the canal has sides of fixed lengths CD FG IJ
4 m, DE HI 3 m and EF GH 0.5 m. Also, the vertical depth DL IK s m
and \textbackslash{[}\textbackslash angle CDL=\textbackslash angle
JIK=\textbackslash theta \textbackslash{]} radians. 

(i) Show that \textbackslash{[}A=40\textbackslash cos \textbackslash theta
+8\textbackslash sin 2\textbackslash theta +2\textbackslash{]}.
{[}2{]}

(ii) Use differentiation to find the value of which gives a maximum
value of A. {[}4{]}

The National Water Agency conducts regular inspections on the water
quality in the canal. During one such inspection, an officer transfers
water from the canal into a plastic container (as shown in the diagram
below) at a constant rate of 162\$\textbackslash text\{c\}\{\{\textbackslash text\{m\}\}\textasciicircum\{3\}\}\$
per second. 

The plastic container is in the shape of a hollow circular cone with
fixed radius 15 cm and fixed height 35 cm. After t seconds, the depth
of water in the container is h cm and the top surface of the water
has a radius of r cm.

(iii) Find the rate at which h is increasing at the instant when h
21. {[}4{]}

SOLUTION

\item {}

(i) \textbackslash{[}A=\textbackslash left( 3+4+3 \textbackslash right)s+4\textbackslash left(
0.5 \textbackslash right)+2\textbackslash left( \textbackslash frac\{1\}\{2\}
\textbackslash right)\textbackslash left( 4 \textbackslash right)\textbackslash left(
s \textbackslash right)\textbackslash sin \textbackslash theta
\textbackslash{]} \textbackslash{[}=10s+2+4s\textbackslash sin
\textbackslash theta \textbackslash{]} \$\textbackslash text\{Since
\}\textbackslash cos \textbackslash theta =\textbackslash frac\{s\}\{4\}\textbackslash Rightarrow
s=4\textbackslash cos \textbackslash theta ,\$ \$\textbackslash begin\{align\}
\& A=10\textbackslash left( 4\textbackslash cos \textbackslash theta
\textbackslash right)+2+4\textbackslash sin \textbackslash theta
\textbackslash left( 4\textbackslash cos \textbackslash theta \textbackslash right)
\textbackslash\textbackslash{} \& =40\textbackslash cos \textbackslash theta
+8\textbackslash left( 2\textbackslash sin \textbackslash theta
\textbackslash cos \textbackslash theta \textbackslash right)+2
\textbackslash\textbackslash{} \& =40\textbackslash cos \textbackslash theta
+8\textbackslash sin 2\textbackslash theta +2 \textbackslash end\{align\}\$
(ii) \$\textbackslash frac\{\textbackslash text\{d\}A\}\{\textbackslash text\{d\}\textbackslash theta
\}=40\textbackslash left( -\textbackslash sin \textbackslash theta
\textbackslash right)+8\textbackslash left( \textbackslash cos
2\textbackslash theta \textbackslash right)\textbackslash left(
2 \textbackslash right)\$ \$=16\textbackslash cos 2\textbackslash theta
-40\textbackslash sin \textbackslash theta \$

\$\textbackslash text\{For \}A\textbackslash text\{ to be maximum,
\}\textbackslash frac\{\textbackslash text\{d\}A\}\{\textbackslash text\{d\}\textbackslash theta
\}=0,\$ \$\textbackslash begin\{align\} \& 16\textbackslash cos
2\textbackslash theta -40\textbackslash sin \textbackslash theta
=0 \textbackslash\textbackslash{} \& 16\textbackslash left( 1-2\{\{\textbackslash sin
\}\textasciicircum\{2\}\}\textbackslash theta \textbackslash right)-40\textbackslash sin
\textbackslash theta =0 \textbackslash\textbackslash{} \& 32\{\{\textbackslash sin
\}\textasciicircum\{2\}\}\textbackslash theta +40\textbackslash sin
\textbackslash theta -16=0 \textbackslash\textbackslash{} \& 4\{\{\textbackslash sin
\}\textasciicircum\{2\}\}\textbackslash theta +5\textbackslash sin
\textbackslash theta -2=0 \textbackslash\textbackslash{} \textbackslash end\{align\}\$
\$\textbackslash begin\{align\} \& \textbackslash sin \textbackslash theta
=\textbackslash frac\{-5\textbackslash pm \textbackslash sqrt\{25-4\textbackslash left(
4 \textbackslash right)\textbackslash left( -2 \textbackslash right)\}\}\{8\}
\textbackslash\textbackslash{} \& =\textbackslash frac\{-5\textbackslash pm
\textbackslash sqrt\{57\}\}\{8\} \textbackslash end\{align\}\$ \$\textbackslash begin\{align\}
\& \textbackslash text\{Since \}0<\textbackslash theta <\textbackslash frac\{\textbackslash text\{
\}\textbackslash !\textbackslash !\textbackslash pi\textbackslash !\textbackslash !\textbackslash text\{
\}\}\{2\},\textbackslash text\{ \}\textbackslash sin \textbackslash theta
>0, \textbackslash\textbackslash{} \& \textbackslash therefore
\textbackslash sin \textbackslash theta =\textbackslash frac\{-5+\textbackslash sqrt\{57\}\}\{8\}=0.31873
\textbackslash\textbackslash{} \textbackslash end\{align\}\$ \$\textbackslash theta
=0.32439\$ 

\$\textbackslash begin\{align\} \& \textbackslash frac\{\{\{\textbackslash text\{d\}\}\textasciicircum\{2\}\}A\}\{\textbackslash text\{d\}\{\{\textbackslash theta
\}\textasciicircum\{2\}\}\}=16\textbackslash left( -\textbackslash sin
2\textbackslash theta \textbackslash right)\textbackslash left(
2 \textbackslash right)-40\textbackslash cos \textbackslash theta
\textbackslash\textbackslash{} \& =-32\textbackslash sin 2\textbackslash theta
-40\textbackslash cos \textbackslash theta \textbackslash end\{align\}\$ 

\$\textbackslash begin\{align\} \& \textbackslash text\{When \}\textbackslash theta
=0.32439, \textbackslash\textbackslash{} \& \textbackslash frac\{\{\{\textbackslash text\{d\}\}\textasciicircum\{2\}\}A\}\{\textbackslash text\{d\}\{\{\textbackslash theta
\}\textasciicircum\{2\}\}\}=-32\textbackslash sin \textbackslash left(
0.64878 \textbackslash right)-40\textbackslash cos \textbackslash left(
0.32439 \textbackslash right) \textbackslash\textbackslash{} \&
=-57.249<0 \textbackslash end\{align\}\$ 

Thus, A is a maximum when \$\textbackslash theta =0.324\textbackslash{}
\textbackslash text\{(to 3 s\}\textbackslash text\{.f\}\textbackslash text\{.)\}\textbackslash text\{.\}\$ 

\quad{} (iii) Let the volume of water in the container be V\$\textbackslash text\{c\}\{\{\textbackslash text\{m\}\}\textasciicircum\{3\}\}.\$
\$\textbackslash begin\{align\} \& V=\textbackslash frac\{1\}\{3\}\textbackslash text\{
\}\textbackslash !\textbackslash !\textbackslash pi\textbackslash !\textbackslash !\textbackslash text\{
\}\{\{r\}\textasciicircum\{2\}\}h \textbackslash\textbackslash{}
\& \textbackslash text\{Using similar triangles, \}\textbackslash frac\{r\}\{h\}=\textbackslash frac\{15\}\{35\},\textbackslash text\{
\} \textbackslash\textbackslash{} \& \textbackslash therefore r=\textbackslash frac\{3\}\{7\}h
\textbackslash\textbackslash{} \textbackslash end\{align\}\$ \$\textbackslash begin\{align\}
\& V=\textbackslash frac\{1\}\{3\}\textbackslash text\{ \}\textbackslash !\textbackslash !\textbackslash pi\textbackslash !\textbackslash !\textbackslash text\{
\}\{\{\textbackslash left( \textbackslash frac\{3\}\{7\}h \textbackslash right)\}\textasciicircum\{2\}\}h
\textbackslash\textbackslash{} \& =\textbackslash frac\{3\textbackslash text\{
\}\textbackslash !\textbackslash !\textbackslash pi\textbackslash !\textbackslash !\textbackslash text\{
\}\}\{49\}\{\{h\}\textasciicircum\{3\}\} \textbackslash\textbackslash{}
\& \textbackslash frac\{\textbackslash text\{d\}V\}\{\textbackslash text\{d\}h\}=\textbackslash frac\{3\textbackslash text\{
\}\textbackslash !\textbackslash !\textbackslash pi\textbackslash !\textbackslash !\textbackslash text\{
\}\}\{49\}\textbackslash left( 3 \textbackslash right)\{\{h\}\textasciicircum\{2\}\}
\textbackslash\textbackslash{} \& =\textbackslash frac\{9\textbackslash text\{
\}\textbackslash !\textbackslash !\textbackslash pi\textbackslash !\textbackslash !\textbackslash text\{
\}\}\{49\}\{\{h\}\textasciicircum\{2\}\} \textbackslash end\{align\}\$
\textbackslash{[}\textbackslash begin\{align\} \& \textbackslash frac\{\textbackslash text\{d\}h\}\{\textbackslash text\{d\}t\}=\textbackslash frac\{\textbackslash text\{d\}h\}\{\textbackslash text\{d\}V\}\textbackslash times
\textbackslash frac\{\textbackslash text\{d\}V\}\{\textbackslash text\{d\}t\}
\textbackslash\textbackslash{} \& =\textbackslash frac\{49\}\{9\textbackslash text\{
\}\textbackslash !\textbackslash !\textbackslash pi\textbackslash !\textbackslash !\textbackslash text\{
\}\{\{h\}\textasciicircum\{2\}\}\}\textbackslash left( 162 \textbackslash right)
\textbackslash\textbackslash{} \& =\textbackslash frac\{882\}\{\textbackslash text\{
\}\textbackslash !\textbackslash !\textbackslash pi\textbackslash !\textbackslash !\textbackslash text\{
\}\{\{h\}\textasciicircum\{2\}\}\} \textbackslash end\{align\}\textbackslash{]}
\textbackslash{[}\textbackslash begin\{align\} \& \textbackslash text\{When
\}h=21, \textbackslash\textbackslash{} \& \textbackslash frac\{\textbackslash text\{d\}h\}\{\textbackslash text\{d\}t\}=\textbackslash frac\{882\}\{\textbackslash text\{
\}\textbackslash !\textbackslash !\textbackslash pi\textbackslash !\textbackslash !\textbackslash text\{
\}\{\{\textbackslash left( 21 \textbackslash right)\}\textasciicircum\{2\}\}\}=0.637\textbackslash text\{
or \}\textbackslash frac\{2\}\{\textbackslash text\{ \}\textbackslash !\textbackslash !\textbackslash pi\textbackslash !\textbackslash !\textbackslash text\{
\}\} \textbackslash\textbackslash{} \textbackslash end\{align\}\textbackslash{]}
h is increasing at a rate of 0.637 or \$\textbackslash frac\{2\}\{\textbackslash text\{
\}\textbackslash !\textbackslash !\textbackslash pi\textbackslash !\textbackslash !\textbackslash text\{
\}\}\textbackslash text\{ cm per second\}\textbackslash text\{.\}\$

 \newpage 

\item \textbf{{[}NJC/PROMO/9758/2021/Q8{]}} 

(a) A curve C has parametric equations

\$x=2\textbackslash theta -\textbackslash sin 2\textbackslash theta
,\textbackslash text\{ \}y=1-\textbackslash cos 2\textbackslash theta
,\$ for \$0\textbackslash le \textbackslash theta \textbackslash le
\textbackslash frac\{\textbackslash text\{ \}\textbackslash !\textbackslash !\textbackslash pi\textbackslash !\textbackslash !\textbackslash text\{
\}\}\{2\}\$.

Find the exact area of the region bounded by C, the line \$x=\textbackslash text\{
\}\textbackslash !\textbackslash !\textbackslash pi\textbackslash !\textbackslash !\textbackslash text\{
\}\$ and the x-axis. {[}5{]} 

(b) The region bounded by the curve \$y=\textbackslash tan x\$, the
line \$y=1\$ and the y-axis is rotated about the x-axis through \$2\textbackslash text\{
\}\textbackslash !\textbackslash !\textbackslash pi\textbackslash !\textbackslash !\textbackslash text\{
\}\$ radians. Find the exact volume of the solid formed. {[}5{]}

SOLUTION

\item {}

(a) Area of the bounded region \textbackslash{[}\textbackslash begin\{align\}
\& =\textbackslash int\_\{0\}\textasciicircum\{\textbackslash text\{
\}\textbackslash !\textbackslash !\textbackslash pi\textbackslash !\textbackslash !\textbackslash text\{
\}\}\{y\}\textbackslash text\{ d\}x \textbackslash\textbackslash{}
\& =\textbackslash int\_\{0\}\textasciicircum\{\textbackslash frac\{\textbackslash text\{
\}\textbackslash !\textbackslash !\textbackslash pi\textbackslash !\textbackslash !\textbackslash text\{
\}\}\{2\}\}\{\textbackslash left( 1-\textbackslash cos 2\textbackslash theta
\textbackslash right)\textbackslash left( \textbackslash frac\{\textbackslash text\{d\}x\}\{\textbackslash text\{d\}\textbackslash theta
\} \textbackslash right)\}\textbackslash text\{ d\}\textbackslash theta
\textbackslash\textbackslash{} \& =\textbackslash int\_\{0\}\textasciicircum\{\textbackslash frac\{\textbackslash text\{
\}\textbackslash !\textbackslash !\textbackslash pi\textbackslash !\textbackslash !\textbackslash text\{
\}\}\{2\}\}\{\textbackslash left( 1-\textbackslash cos 2\textbackslash theta
\textbackslash right)\textbackslash left( 2-2\textbackslash cos
2\textbackslash theta \textbackslash right)\}\textbackslash text\{
d\}\textbackslash theta \textbackslash\textbackslash{} \& =2\textbackslash int\_\{0\}\textasciicircum\{\textbackslash frac\{\textbackslash text\{
\}\textbackslash !\textbackslash !\textbackslash pi\textbackslash !\textbackslash !\textbackslash text\{
\}\}\{2\}\}\{\{\{\textbackslash left( 1-\textbackslash cos 2\textbackslash theta
\textbackslash right)\}\textasciicircum\{2\}\}\}\textbackslash text\{
d\}\textbackslash theta \textbackslash\textbackslash{} \& =2\textbackslash int\_\{0\}\textasciicircum\{\textbackslash frac\{\textbackslash text\{
\}\textbackslash !\textbackslash !\textbackslash pi\textbackslash !\textbackslash !\textbackslash text\{
\}\}\{2\}\}\{1-2\textbackslash cos 2\textbackslash theta +\{\{\textbackslash cos
\}\textasciicircum\{2\}\}2\textbackslash theta \}\textbackslash text\{
d\}\textbackslash theta \textbackslash\textbackslash{} \& =\textbackslash int\_\{0\}\textasciicircum\{\textbackslash frac\{\textbackslash text\{
\}\textbackslash !\textbackslash !\textbackslash pi\textbackslash !\textbackslash !\textbackslash text\{
\}\}\{2\}\}\{2-4\textbackslash cos 2\textbackslash theta +\textbackslash left(
\textbackslash cos 4\textbackslash theta +1 \textbackslash right)\}\textbackslash text\{
d\}\textbackslash theta \textbackslash\textbackslash{} \& =\textbackslash int\_\{0\}\textasciicircum\{\textbackslash frac\{\textbackslash text\{
\}\textbackslash !\textbackslash !\textbackslash pi\textbackslash !\textbackslash !\textbackslash text\{
\}\}\{2\}\}\{3-4\textbackslash cos 2\textbackslash theta +\textbackslash cos
4\textbackslash theta \}\textbackslash text\{ d\}\textbackslash theta
\textbackslash\textbackslash{} \& \textbackslash text\{=\}\textbackslash left{[}
3\textbackslash theta -2\textbackslash sin 2\textbackslash theta
+\textbackslash frac\{1\}\{4\}\textbackslash sin 4\textbackslash theta
\textbackslash right{]}\_\{0\}\textasciicircum\{\textbackslash frac\{\textbackslash text\{
\}\textbackslash !\textbackslash !\textbackslash pi\textbackslash !\textbackslash !\textbackslash text\{
\}\}\{2\}\} \textbackslash\textbackslash{} \& =\textbackslash left{[}
\textbackslash frac\{\textbackslash text\{3 \}\textbackslash !\textbackslash !\textbackslash pi\textbackslash !\textbackslash !\textbackslash text\{
\}\}\{2\}-\textbackslash sin \textbackslash text\{ \}\textbackslash !\textbackslash !\textbackslash pi\textbackslash !\textbackslash !\textbackslash text\{
\}+\textbackslash frac\{1\}\{4\}\textbackslash sin 2\textbackslash text\{
\}\textbackslash !\textbackslash !\textbackslash pi\textbackslash !\textbackslash !\textbackslash text\{
\} \textbackslash right{]}=\textbackslash frac\{3\textbackslash text\{
\}\textbackslash !\textbackslash !\textbackslash pi\textbackslash !\textbackslash !\textbackslash text\{
\}\}\{2\}\textbackslash text\{ unit\}\{\{\textbackslash text\{s\}\}\textasciicircum\{2\}\}
\textbackslash end\{align\}\textbackslash{]} 

(b) When \$\textbackslash tan x=1,\$ \$x=\textbackslash frac\{\textbackslash text\{
\}\textbackslash !\textbackslash !\textbackslash pi\textbackslash !\textbackslash !\textbackslash text\{
\}\}\{4\}\$ Volume of solid formed \textbackslash{[}\textbackslash begin\{align\}
\& =\textbackslash text\{ \}\textbackslash !\textbackslash !\textbackslash pi\textbackslash !\textbackslash !\textbackslash text\{
\}\textbackslash left( \{\{1\}\textasciicircum\{2\}\} \textbackslash right)\textbackslash left(
\textbackslash frac\{\textbackslash text\{ \}\textbackslash !\textbackslash !\textbackslash pi\textbackslash !\textbackslash !\textbackslash text\{
\}\}\{4\} \textbackslash right)-\textbackslash text\{ \}\textbackslash !\textbackslash !\textbackslash pi\textbackslash !\textbackslash !\textbackslash text\{
\}\textbackslash int\_\{0\}\textasciicircum\{\textbackslash frac\{\textbackslash text\{
\}\textbackslash !\textbackslash !\textbackslash pi\textbackslash !\textbackslash !\textbackslash text\{
\}\}\{4\}\}\{\{\{y\}\textasciicircum\{2\}\}\}\textbackslash text\{
d\}x \textbackslash\textbackslash{} \& =\textbackslash frac\{\{\{\textbackslash text\{
\}\textbackslash !\textbackslash !\textbackslash pi\textbackslash !\textbackslash !\textbackslash text\{
\}\}\textasciicircum\{2\}\}\}\{4\}-\textbackslash text\{ \}\textbackslash !\textbackslash !\textbackslash pi\textbackslash !\textbackslash !\textbackslash text\{
\}\textbackslash int\_\{0\}\textasciicircum\{\textbackslash frac\{\textbackslash text\{
\}\textbackslash !\textbackslash !\textbackslash pi\textbackslash !\textbackslash !\textbackslash text\{
\}\}\{4\}\}\{\{\{\textbackslash tan \}\textasciicircum\{2\}\}x\}\textbackslash text\{
d\}x \textbackslash\textbackslash{} \& =\textbackslash frac\{\{\{\textbackslash text\{
\}\textbackslash !\textbackslash !\textbackslash pi\textbackslash !\textbackslash !\textbackslash text\{
\}\}\textasciicircum\{2\}\}\}\{4\}-\textbackslash text\{ \}\textbackslash !\textbackslash !\textbackslash pi\textbackslash !\textbackslash !\textbackslash text\{
\}\textbackslash int\_\{0\}\textasciicircum\{\textbackslash frac\{\textbackslash text\{
\}\textbackslash !\textbackslash !\textbackslash pi\textbackslash !\textbackslash !\textbackslash text\{
\}\}\{4\}\}\{\{\{\textbackslash sec \}\textasciicircum\{2\}\}x-1\}\textbackslash text\{
d\}x \textbackslash\textbackslash{} \& =\textbackslash frac\{\{\{\textbackslash text\{
\}\textbackslash !\textbackslash !\textbackslash pi\textbackslash !\textbackslash !\textbackslash text\{
\}\}\textasciicircum\{2\}\}\}\{4\}-\textbackslash text\{ \}\textbackslash !\textbackslash !\textbackslash pi\textbackslash !\textbackslash !\textbackslash text\{
\}\textbackslash left{[} \textbackslash tan x-x \textbackslash right{]}\_\{0\}\textasciicircum\{\textbackslash frac\{\textbackslash text\{
\}\textbackslash !\textbackslash !\textbackslash pi\textbackslash !\textbackslash !\textbackslash text\{
\}\}\{4\}\} \textbackslash\textbackslash{} \& =\textbackslash frac\{\{\{\textbackslash text\{
\}\textbackslash !\textbackslash !\textbackslash pi\textbackslash !\textbackslash !\textbackslash text\{
\}\}\textasciicircum\{2\}\}\}\{4\}-\textbackslash text\{ \}\textbackslash !\textbackslash !\textbackslash pi\textbackslash !\textbackslash !\textbackslash text\{
\}\textbackslash left{[} \textbackslash tan \textbackslash frac\{\textbackslash text\{
\}\textbackslash !\textbackslash !\textbackslash pi\textbackslash !\textbackslash !\textbackslash text\{
\}\}\{4\}-\textbackslash frac\{\textbackslash text\{ \}\textbackslash !\textbackslash !\textbackslash pi\textbackslash !\textbackslash !\textbackslash text\{
\}\}\{4\} \textbackslash right{]} \textbackslash\textbackslash{}
\& =\textbackslash frac\{\{\{\textbackslash text\{ \}\textbackslash !\textbackslash !\textbackslash pi\textbackslash !\textbackslash !\textbackslash text\{
\}\}\textasciicircum\{2\}\}\}\{2\}-\textbackslash text\{ \}\textbackslash !\textbackslash !\textbackslash pi\textbackslash !\textbackslash !\textbackslash text\{
unit\}\{\{\textbackslash text\{s\}\}\textasciicircum\{3\}\} \textbackslash\textbackslash{}
\textbackslash end\{align\}\textbackslash{]}

 \newpage 

\item \textbf{{[}NJC/PROMO/9758/2021/Q9{]}} 

Relative to the origin O, the points A, B and C have position vectors
\textbackslash{[}2\textbackslash mathbf\{u\}-3\textbackslash mathbf\{v\},\textbackslash{]}\textbackslash{[}\textbackslash mathbf\{u\}+2\textbackslash mathbf\{v\}\textbackslash{]}and
\textbackslash{[}3\textbackslash mathbf\{u\}-2\textbackslash mathbf\{v\}\textbackslash{]}respectively,
where u and v are two non-zero and non-parallel vectors. (i) Show
that the area of triangle ABC is given by \$3\textbackslash left|
\textbackslash mathbf\{u\}\textbackslash times \textbackslash mathbf\{v\}
\textbackslash right|.\$ {[}3{]}

It is given that \$\textbackslash mathbf\{u\}=2\textbackslash mathbf\{i\}-\textbackslash mathbf\{j\}+2\textbackslash mathbf\{k\}\$
and \$\textbackslash mathbf\{v\}=3\textbackslash mathbf\{i\}-4\textbackslash mathbf\{j\}-p\textbackslash mathbf\{k\},\$
where p is a constant. 

(ii) If the area of triangle ABC is \$15\textbackslash sqrt\{53\}\$
square units, find the possible values of p exactly. {[}4{]}

(iii) It is given that the angle between vectors u and v is acute.
Find this angle, giving your answer in degrees. Explain why there
is only one answer. {[}3{]}

SOLUTION

\item {}

(i) \$\textbackslash overset\{\textbackslash xrightarrow\{\{\}\}\}\{\textbackslash mathop\{AB\}\}\textbackslash ,=\textbackslash left(
\textbackslash mathbf\{u\}+2\textbackslash mathbf\{v\} \textbackslash right)-\textbackslash left(
2\textbackslash mathbf\{u\}-3\textbackslash mathbf\{v\} \textbackslash right)=-\textbackslash mathbf\{u\}+5\textbackslash mathbf\{v\}\$
\$\textbackslash overset\{\textbackslash xrightarrow\{\{\}\}\}\{\textbackslash mathop\{AC\}\}\textbackslash ,=\textbackslash left(
3\textbackslash mathbf\{u\}-2\textbackslash mathbf\{v\} \textbackslash right)-\textbackslash left(
2\textbackslash mathbf\{u\}-3\textbackslash mathbf\{v\} \textbackslash right)=\textbackslash mathbf\{u\}+\textbackslash mathbf\{v\}\$
Therefore, area of triangle ABC \$\textbackslash begin\{align\} \&
=\textbackslash frac\{1\}\{2\}\textbackslash left| \textbackslash overset\{\textbackslash xrightarrow\{\{\}\}\}\{\textbackslash mathop\{AB\}\}\textbackslash ,\textbackslash{}
\textbackslash times \textbackslash overset\{\textbackslash xrightarrow\{\{\}\}\}\{\textbackslash mathop\{AC\}\}\textbackslash ,
\textbackslash right| \textbackslash\textbackslash{} \& =\textbackslash frac\{1\}\{2\}\textbackslash left|
\textbackslash left( -\textbackslash mathbf\{u\}+5\textbackslash mathbf\{v\}
\textbackslash right)\textbackslash{} \textbackslash times \textbackslash left(
\textbackslash mathbf\{u\}+\textbackslash mathbf\{v\} \textbackslash right)\textbackslash{}
\textbackslash right| \textbackslash\textbackslash{} \& =\textbackslash frac\{1\}\{2\}\textbackslash left|
-\textbackslash mathbf\{u\}\textbackslash times \textbackslash mathbf\{u\}+5\textbackslash mathbf\{v\}\textbackslash times
\textbackslash mathbf\{u\}-\textbackslash mathbf\{u\}\textbackslash times
\textbackslash mathbf\{v\}+5\textbackslash mathbf\{v\}\textbackslash times
\textbackslash mathbf\{v\} \textbackslash right| \textbackslash\textbackslash{}
\& =\textbackslash frac\{1\}\{2\}\textbackslash left| \textbackslash mathbf\{0\}-5\textbackslash mathbf\{u\}\textbackslash times
\textbackslash mathbf\{v\}-\textbackslash mathbf\{u\}\textbackslash times
\textbackslash mathbf\{v\}+\textbackslash mathbf\{0\} \textbackslash right|
\textbackslash\textbackslash{} \& =\textbackslash frac\{1\}\{2\}\textbackslash left|
-6\textbackslash mathbf\{u\}\textbackslash times \textbackslash mathbf\{v\}
\textbackslash right| \textbackslash\textbackslash{} \& =\textbackslash frac\{6\}\{2\}\textbackslash left|
\textbackslash mathbf\{u\}\textbackslash times \textbackslash mathbf\{v\}
\textbackslash right| \textbackslash\textbackslash{} \& =3\textbackslash left|
\textbackslash mathbf\{u\}\textbackslash times \textbackslash mathbf\{v\}
\textbackslash right|\textbackslash text\{ (shown)\} \textbackslash end\{align\}\$
(ii) \$\textbackslash mathbf\{u\}\textbackslash times \textbackslash mathbf\{v\}\$
\$\textbackslash begin\{align\} \& =\textbackslash left( \textbackslash begin\{matrix\}
2 \textbackslash\textbackslash{} -1 \textbackslash\textbackslash{}
2 \textbackslash\textbackslash{} \textbackslash end\{matrix\} \textbackslash right)\textbackslash times
\textbackslash left( \textbackslash begin\{matrix\} 3 \textbackslash\textbackslash{}
-4 \textbackslash\textbackslash{} -p \textbackslash\textbackslash{}
\textbackslash end\{matrix\} \textbackslash right) \textbackslash\textbackslash{}
\& =\textbackslash left( \textbackslash begin\{matrix\} \textbackslash left(
-1 \textbackslash right)\textbackslash left( -p \textbackslash right)-\textbackslash left(
2 \textbackslash right)\textbackslash left( -4 \textbackslash right)
\textbackslash\textbackslash{} \textbackslash left( 2 \textbackslash right)\textbackslash left(
3 \textbackslash right)-\textbackslash left( 2 \textbackslash right)\textbackslash left(
-p \textbackslash right) \textbackslash\textbackslash{} \textbackslash left(
2 \textbackslash right)\textbackslash left( -4 \textbackslash right)-\textbackslash left(
-1 \textbackslash right)\textbackslash left( 3 \textbackslash right)
\textbackslash\textbackslash{} \textbackslash end\{matrix\} \textbackslash right)
\textbackslash\textbackslash{} \& =\textbackslash left( \textbackslash begin\{matrix\}
p+8 \textbackslash\textbackslash{} 2p+6 \textbackslash\textbackslash{}
-5 \textbackslash\textbackslash{} \textbackslash end\{matrix\}
\textbackslash right) \textbackslash end\{align\}\$

Since the area of triangle ABC is \$15\textbackslash sqrt\{53\}\$
square units, \$\textbackslash begin\{align\} \& 3\textbackslash sqrt\{\{\{(p+8)\}\textasciicircum\{2\}\}+\{\{(2p+6)\}\textasciicircum\{2\}\}+\{\{5\}\textasciicircum\{2\}\}\}=15\textbackslash sqrt\{53\}
\textbackslash\textbackslash{} \& \{\{(p+8)\}\textasciicircum\{2\}\}+\{\{(2p+6)\}\textasciicircum\{2\}\}+\{\{5\}\textasciicircum\{2\}\}=25\textbackslash times
53 \textbackslash\textbackslash{} \& 5\{\{p\}\textasciicircum\{2\}\}+40p-1200=0
\textbackslash\textbackslash{} \& \{\{p\}\textasciicircum\{2\}\}+8p-240=0
\textbackslash\textbackslash{} \& (p+20)(p-12)=0 \textbackslash end\{align\}\$
Therefore, \$p=-20\$ or \$p=12.\$ \quad{} (iii) \$\textbackslash mathbf\{u\}\textbackslash cdot
\textbackslash mathbf\{v\}=\textbackslash left( \textbackslash begin\{matrix\}
2 \textbackslash\textbackslash{} -1 \textbackslash\textbackslash{}
2 \textbackslash\textbackslash{} \textbackslash end\{matrix\} \textbackslash right)\textbackslash cdot
\textbackslash left( \textbackslash begin\{matrix\} 3 \textbackslash\textbackslash{}
-4 \textbackslash\textbackslash{} -p \textbackslash\textbackslash{}
\textbackslash end\{matrix\} \textbackslash right)=10-2p\$ For \$p=-20,\$
\$\textbackslash mathbf\{u\}\textbackslash cdot \textbackslash mathbf\{v\}=\textbackslash underline\{10-2(-20)\}=50\textbackslash{}
\textbackslash underline\{>0\}\$ For \$p=12,\$ \$\textbackslash mathbf\{u\}\textbackslash cdot
\textbackslash mathbf\{v\}=\textbackslash underline\{10-2(12)\}=-14\textbackslash{}
\textbackslash underline\{<0\}\$

Therefore, \$p=-20\$ is the only possible case for the angle between
u and v to be acute. Let this angle be .

\textbackslash{[}\textbackslash begin\{align\} \& \textbackslash cos
\textbackslash theta =\textbackslash frac\{\textbackslash left(
\textbackslash begin\{matrix\} 2 \textbackslash\textbackslash{}
-1 \textbackslash\textbackslash{} 2 \textbackslash\textbackslash{}
\textbackslash end\{matrix\} \textbackslash right)\textbackslash cdot
\textbackslash left( \textbackslash begin\{matrix\} 3 \textbackslash\textbackslash{}
-4 \textbackslash\textbackslash{} 20 \textbackslash\textbackslash{}
\textbackslash end\{matrix\} \textbackslash right)\}\{\textbackslash sqrt\{\{\{2\}\textasciicircum\{2\}\}+\{\{1\}\textasciicircum\{2\}\}+\{\{2\}\textasciicircum\{2\}\}\}\textbackslash sqrt\{\{\{3\}\textasciicircum\{2\}\}+\{\{4\}\textasciicircum\{2\}\}+\{\{20\}\textasciicircum\{2\}\}\}\}
\textbackslash\textbackslash{} \& =\textbackslash frac\{50\}\{3\textbackslash sqrt\{425\}\}=\textbackslash frac\{10\}\{3\textbackslash sqrt\{17\}\}
\textbackslash\textbackslash{} \& \textbackslash theta =36.1\{\}\textasciicircum\textbackslash circ
\textbackslash text\{ (to 1 d\}\textbackslash text\{.p\}\textbackslash text\{.)\}
\textbackslash end\{align\}\textbackslash{]}

 \newpage 



 \end{enumerate}

 \end{document}

