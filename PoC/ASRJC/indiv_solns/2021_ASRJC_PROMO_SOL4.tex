\item (i) \$\{\{y\}\textasciicircum\{3\}\}-2x\{\{y\}\textasciicircum\{2\}\}+3\{\{x\}\textasciicircum\{2\}\}-3=0\$
Differentiate with respect to x: \$3\{\{y\}\textasciicircum\{2\}\}\textbackslash frac\{\textbackslash text\{d\}y\}\{\textbackslash text\{d\}x\}-(2\{\{y\}\textasciicircum\{2\}\}+4xy\textbackslash frac\{\textbackslash text\{d\}y\}\{\textbackslash text\{d\}x\})+6x=0\$
\textbackslash{[}\textbackslash begin\{align\} \& (3\{\{y\}\textasciicircum\{2\}\}-4xy)\textbackslash frac\{\textbackslash text\{d\}y\}\{\textbackslash text\{d\}x\}=2\{\{y\}\textasciicircum\{2\}\}-6x
\textbackslash\textbackslash{} \& \textbackslash frac\{\textbackslash text\{d\}y\}\{\textbackslash text\{d\}x\}=\textbackslash frac\{2\{\{y\}\textasciicircum\{2\}\}-6x\}\{3\{\{y\}\textasciicircum\{2\}\}-4xy\}
\textbackslash\textbackslash{} \textbackslash end\{align\}\textbackslash{]}
(ii) Gradient of tangent to curve at P(2, 3) =\textbackslash{[}\textbackslash frac\{2\{\{(3)\}\textasciicircum\{2\}\}-6(2)\}\{3\{\{(3)\}\textasciicircum\{2\}\}-4(2)(3)\}=2\textbackslash{]}
Gradient of normal to curve at P = \$-\textbackslash frac\{1\}\{2\}\$
Equation of normal at P: \$\textbackslash begin\{align\} \& y-3=-\textbackslash tfrac\{1\}\{2\}(x-2)
\textbackslash\textbackslash{} \& \textbackslash Rightarrow y=-\textbackslash tfrac\{1\}\{2\}x+4
\textbackslash\textbackslash{} \textbackslash end\{align\}\$ (iii)
When x = 0, normal cuts y-axis at A(0, 4); C: \$\{\{y\}\textasciicircum\{3\}\}-2x\{\{y\}\textasciicircum\{2\}\}+3\{\{x\}\textasciicircum\{2\}\}-3=0\$
When x = 0, \$\{\{y\}\textasciicircum\{3\}\}\$= 3 y = \$\textbackslash sqrt{[}3{]}\{3\}\$
C meets y-axis at R(0, \$\textbackslash sqrt{[}3{]}\{3\}\$). Area
of triangle APR = \$\textbackslash frac\{1\}\{2\}\textbackslash times
2\textbackslash times (4-\textbackslash sqrt{[}3{]}\{3\})\$ = \$4-\textbackslash sqrt{[}3{]}\{3\}\$
where a = 4, b = 3