\item (ai) Since \textbackslash{[}x=2\textbackslash{]} is a vertical
asymptote, \$s=4\$ \textbackslash{[}\textbackslash text\{At the
point \}(0,\textbackslash tfrac\{1\}\{2\})\textbackslash text\{
\}\textbackslash frac\{4\{\{(0)\}\textasciicircum\{2\}\}+p(0)-q\}\{\{\{(0)\}\textasciicircum\{2\}\}-4\}=\textbackslash frac\{1\}\{2\}\textbackslash{]}
\$\textbackslash therefore q=2\$ (ii) When y = 1, \$1=\textbackslash frac\{4\{\{x\}\textasciicircum\{2\}\}+px-2\}\{\{\{x\}\textasciicircum\{2\}\}-4\}\$
\$\{\{x\}\textasciicircum\{2\}\}-4=4\{\{x\}\textasciicircum\{2\}\}+px-2\$
\$3\{\{x\}\textasciicircum\{2\}\}+px+2=0\$ Since y = 1 is a tangent
to C, Discriminant = 0 \$\{\{p\}\textasciicircum\{2\}\}-4\textbackslash left(
3 \textbackslash right)\textbackslash left( 2 \textbackslash right)=0\$
\$p=-2\textbackslash sqrt\{6\}\textbackslash left( \textbackslash because
p<0 \textbackslash right)\$ 

(bi)

(bii) \textbackslash{[}\textbackslash begin\{align\} \& 10\{\{(x+2)\}\textasciicircum\{2\}\}=3\textbackslash left(
10-\{\{\textbackslash left( \textbackslash frac\{4\{\{x\}\textasciicircum\{2\}\}+4x+1\}\{\{\{x\}\textasciicircum\{2\}\}-1\}
\textbackslash right)\}\textasciicircum\{2\}\} \textbackslash right)
\textbackslash\textbackslash{} \& 10\{\{(x+2)\}\textasciicircum\{2\}\}+3\{\{\textbackslash left(
\textbackslash frac\{4\{\{x\}\textasciicircum\{2\}\}+4x+1\}\{\{\{x\}\textasciicircum\{2\}\}-1\}
\textbackslash right)\}\textasciicircum\{2\}\}=30 \textbackslash\textbackslash{}
\& \textbackslash frac\{\{\{(x+2)\}\textasciicircum\{2\}\}\}\{3\}+\textbackslash frac\{\{\{\textbackslash left(
\textbackslash frac\{4\{\{x\}\textasciicircum\{2\}\}+4x+1\}\{\{\{x\}\textasciicircum\{2\}\}-1\}
\textbackslash right)\}\textasciicircum\{2\}\}\}\{10\}=1 \textbackslash\textbackslash{}
\textbackslash end\{align\}\textbackslash{]} The equation of the
additional curve is \textbackslash{[}\textbackslash frac\{\{\{(x+2)\}\textasciicircum\{2\}\}\}\{3\}+\textbackslash frac\{\{\{y\}\textasciicircum\{2\}\}\}\{10\}=1\textbackslash{]}