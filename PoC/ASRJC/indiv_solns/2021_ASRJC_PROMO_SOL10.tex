\item (ai) Height of cylinder = \$\textbackslash sqrt\{\{\{8\}\textasciicircum\{2\}\}-\{\{x\}\textasciicircum\{2\}\}\}\$\$\textbackslash times
2\$ = 2\$\textbackslash sqrt\{\{\{8\}\textasciicircum\{2\}\}-\{\{x\}\textasciicircum\{2\}\}\}\$
\$\textbackslash begin\{align\} \& S=2\textbackslash pi rh \textbackslash\textbackslash{}
\& \textbackslash ,\textbackslash ,\textbackslash ,\textbackslash ,\textbackslash ,=2\textbackslash pi
\textbackslash times \textbackslash left( \textbackslash frac\{x\}\{2\}
\textbackslash right)\textbackslash times 2\textbackslash sqrt\{64-\{\{x\}\textasciicircum\{2\}\}\}
\textbackslash\textbackslash{} \textbackslash end\{align\}\$ \$S\$=
\$2\textbackslash pi x\textbackslash sqrt\{64-\{\{x\}\textasciicircum\{2\}\}\}\$
c (shown) (ii)\textbackslash{[}\textbackslash frac\{\textbackslash text\{d\}S\}\{\textbackslash text\{d\}x\}=2\textbackslash pi
\textbackslash sqrt\{64-\{\{x\}\textasciicircum\{2\}\}\}+2\textbackslash pi
x\textbackslash left( \textbackslash frac\{-2x\}\{2\textbackslash sqrt\{64-\{\{x\}\textasciicircum\{2\}\}\}\}
\textbackslash right)\textbackslash{]} = \textbackslash{[}2\textbackslash pi
\textbackslash sqrt\{64-\{\{x\}\textasciicircum\{2\}\}\}-\textbackslash frac\{2\textbackslash pi
\{\{x\}\textasciicircum\{2\}\}\}\{\textbackslash sqrt\{64-\{\{x\}\textasciicircum\{2\}\}\}\}\textbackslash{]}
= \$\textbackslash frac\{4\textbackslash pi (32-\{\{x\}\textasciicircum\{2\}\})\}\{\textbackslash sqrt\{64-\{\{x\}\textasciicircum\{2\}\}\}\}\$
For stationary point, \textbackslash{[}\textbackslash frac\{\textbackslash text\{d\}S\}\{\textbackslash text\{d\}x\}\textbackslash{]}=
0 \textbackslash{[}\textbackslash frac\{4\textbackslash pi (32-\{\{x\}\textasciicircum\{2\}\})\}\{\textbackslash sqrt\{64-\{\{x\}\textasciicircum\{2\}\}\}\}=0\textbackslash{]}
\$x=\textbackslash sqrt\{32\}=4\textbackslash sqrt\{2\}\$ (since
x > 0) x : 2\$\textbackslash sqrt\{64-\{\{x\}\textasciicircum\{2\}\}\}\$=
1 : k \$\textbackslash sqrt\{32\}\$: 2\$\textbackslash sqrt\{32\}\$
= 1 : k k = 2 First derivative test: x \$\textbackslash sqrt\{32\}\{\{\textbackslash ,\}\textasciicircum\{-\}\}\$
\$\textbackslash sqrt\{32\}\$ \$\textbackslash sqrt\{32\}\{\{\textbackslash ,\}\textasciicircum\{+\}\}\$
\$\textbackslash frac\{\textbackslash text\{d\}S\}\{\textbackslash text\{d\}x\}=\textbackslash frac\{4\textbackslash pi
(32-\{\{x\}\textasciicircum\{2\}\})\}\{\textbackslash sqrt\{64-\{\{x\}\textasciicircum\{2\}\}\}\}\$
\$\textbackslash frac\{(+)\}\{(+)\}=(+)\$ 0 \$\textbackslash frac\{(-)\}\{(+)\}=(-)\$
\$\textbackslash frac\{\{\}\}\{\{\}\}\$ Shape 

\$\textbackslash therefore \$ S is maximum when x = \$4\textbackslash sqrt\{2\}\$.
Second derivative test: \textbackslash{[}\textbackslash frac\{\{\{\textbackslash text\{d\}\}\textasciicircum\{2\}\}S\}\{\textbackslash text\{d\}\{\{x\}\textasciicircum\{2\}\}\}=\textbackslash frac\{\textbackslash text\{d\}\textbackslash left(
\textbackslash frac\{4\textbackslash pi (32-\{\{x\}\textasciicircum\{2\}\})\}\{\textbackslash sqrt\{64-\{\{x\}\textasciicircum\{2\}\}\}\}
\textbackslash right)\}\{\textbackslash text\{d\}x\}=\textbackslash frac\{4\textbackslash pi
{[}\textbackslash sqrt\{64-\{\{x\}\textasciicircum\{2\}\}\})(-2x)-(32-\{\{x\}\textasciicircum\{2\}\})\textbackslash frac\{1\}\{2\}\{\{\textbackslash left(
64-\{\{x\}\textasciicircum\{2\}\} \textbackslash right)\}\textasciicircum\{-\textbackslash tfrac\{1\}\{2\}\}\}(-2x){]}\}\{64-\{\{x\}\textasciicircum\{2\}\}\}\textbackslash{]}
= \$\textbackslash frac\{4\textbackslash pi (-x){[}2(64-\{\{x\}\textasciicircum\{2\}\})-\textbackslash left(
32-\{\{x\}\textasciicircum\{2\}\} \textbackslash right){]}\}\{\{\{(64-\{\{x\}\textasciicircum\{2\}\})\}\textasciicircum\{3/2\}\}\}=\textbackslash frac\{4\textbackslash pi
x(\{\{x\}\textasciicircum\{2\}\}-96)\}\{\{\{(64-\{\{x\}\textasciicircum\{2\}\})\}\textasciicircum\{3/2\}\}\}\$
When \$x=4\textbackslash sqrt\{2\},\$ \textbackslash{[}\textbackslash frac\{\{\{\textbackslash text\{d\}\}\textasciicircum\{2\}\}S\}\{\textbackslash text\{d\}\{\{x\}\textasciicircum\{2\}\}\}=\textbackslash{]}\$-8\textbackslash pi
\$ < 0 S is maximum when \$x=4\textbackslash sqrt\{2\}\$. Max S =
\$2\textbackslash pi \textbackslash left( 4\textbackslash sqrt\{2\}
\textbackslash right)\textbackslash sqrt\{64-32\}=64\textbackslash pi
\textbackslash text\{ c\}\{\{\textbackslash text\{m\}\}\textasciicircum\{2\}\}\$
(b) \$y=2\{\{\textbackslash sin \}\textasciicircum\{-1\}\}(3x),\textbackslash ,\textbackslash ,\textbackslash ,\textbackslash ,-\textbackslash frac\{1\}\{3\}\textbackslash le
x\textbackslash le \textbackslash frac\{1\}\{3\}.\$ \textbackslash{[}\textbackslash frac\{\textbackslash text\{d\}y\}\{\textbackslash text\{d\}x\}=\textbackslash frac\{2(3)\}\{\textbackslash sqrt\{1-9\{\{x\}\textasciicircum\{2\}\}\}\}=\textbackslash frac\{6\}\{\textbackslash sqrt\{1-9\{\{x\}\textasciicircum\{2\}\}\}\}\textbackslash{]}
\$y=2\{\{\textbackslash sin \}\textasciicircum\{-1\}\}(3x)\$ \$\textbackslash sin
\textbackslash frac\{y\}\{2\}=(3x)\$ When \$y\$=\$\textbackslash frac\{\textbackslash pi
\}\{3\}\$, 3x = sin\$\textbackslash frac\{\textbackslash pi \}\{6\}\$=
\$\textbackslash frac\{1\}\{2\}\$ x = \$\textbackslash frac\{1\}\{6\}\$
\textbackslash{[}\textbackslash frac\{\textbackslash text\{d\}x\}\{\textbackslash text\{d\}t\}=\textbackslash frac\{\textbackslash text\{d\}x\}\{\textbackslash text\{d\}y\}\textbackslash times
\textbackslash frac\{\textbackslash text\{d\}y\}\{\textbackslash text\{d\}t\}\textbackslash{]}
= \$\textbackslash frac\{\textbackslash sqrt\{1-9\{\{x\}\textasciicircum\{2\}\}\}\}\{6\}\textbackslash times
(-2)\$ = \$\textbackslash frac\{\textbackslash sqrt\{1-9\{\{\textbackslash left(
\textbackslash frac\{1\}\{6\} \textbackslash right)\}\textasciicircum\{2\}\}\}\}\{6\}\textbackslash times
(-2)\$ \textbackslash{[}\textbackslash frac\{\textbackslash text\{d\}x\}\{\textbackslash text\{d\}t\}=-\textbackslash frac\{\textbackslash sqrt\{3\}\}\{2\}\textbackslash times
\textbackslash frac\{1\}\{6\}\textbackslash times (-2)=-\textbackslash frac\{\textbackslash sqrt\{3\}\}\{6\}\textbackslash{]}units/s