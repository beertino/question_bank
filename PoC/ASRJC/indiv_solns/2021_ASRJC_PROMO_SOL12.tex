\item Let F be the foot of perpendicular from A to the plane p1. \textbackslash{[}\{\{\textbackslash ell
\}\_\{AF\}\}:\textbackslash text\{r\}=\textbackslash left( \textbackslash begin\{matrix\}
0 \textbackslash\textbackslash{} -1 \textbackslash\textbackslash{}
2 \textbackslash\textbackslash{} \textbackslash end\{matrix\} \textbackslash right)+\textbackslash lambda
\textbackslash left( \textbackslash begin\{matrix\} 1 \textbackslash\textbackslash{}
-1 \textbackslash\textbackslash{} 0 \textbackslash\textbackslash{}
\textbackslash end\{matrix\} \textbackslash right),\textbackslash lambda
\textbackslash in \textbackslash mathbb\{R\}\textbackslash{]}
Since F lies on line AF \textbackslash{[}\textbackslash overrightarrow\{OF\}=\textbackslash left(
\textbackslash begin\{matrix\} 0 \textbackslash\textbackslash{}
-1 \textbackslash\textbackslash{} 2 \textbackslash\textbackslash{}
\textbackslash end\{matrix\} \textbackslash right)+\textbackslash lambda
\textbackslash left( \textbackslash begin\{matrix\} 1 \textbackslash\textbackslash{}
-1 \textbackslash\textbackslash{} 0 \textbackslash\textbackslash{}
\textbackslash end\{matrix\} \textbackslash right)\textbackslash{]}
for some \textbackslash{[}\textbackslash lambda \textbackslash in
\textbackslash mathbb\{R\}\textbackslash{]} As F is also on the
plane p1, \textbackslash{[}\textbackslash left( \textbackslash begin\{matrix\}
\textbackslash lambda \textbackslash\textbackslash{} -1-\textbackslash lambda
\textbackslash\textbackslash{} 2 \textbackslash\textbackslash{}
\textbackslash end\{matrix\} \textbackslash right).\textbackslash left(
\textbackslash begin\{matrix\} 1 \textbackslash\textbackslash{}
-1 \textbackslash\textbackslash{} 0 \textbackslash\textbackslash{}
\textbackslash end\{matrix\} \textbackslash right)=3\textbackslash{]}
\textbackslash{[}\textbackslash begin\{align\} \& \textbackslash lambda
+1+\textbackslash lambda =3 \textbackslash\textbackslash{} \& \textbackslash lambda
=1 \textbackslash\textbackslash{} \textbackslash end\{align\}\textbackslash{]}
\textbackslash{[}\textbackslash overrightarrow\{OF\}=\textbackslash left(
\textbackslash begin\{matrix\} 1 \textbackslash\textbackslash{}
-2 \textbackslash\textbackslash{} 2 \textbackslash\textbackslash{}
\textbackslash end\{matrix\} \textbackslash right)\textbackslash{]}
(ii) Let \textbackslash{[}\textbackslash theta \textbackslash{]}be
the acute angle between the plane p1 and the line l. \textbackslash{[}\textbackslash sin
\textbackslash theta =\textbackslash frac\{\textbackslash left|
\textbackslash left( \textbackslash begin\{matrix\} 1 \textbackslash\textbackslash{}
-1 \textbackslash\textbackslash{} 0 \textbackslash\textbackslash{}
\textbackslash end\{matrix\} \textbackslash right)\textbackslash bullet
\textbackslash left( \textbackslash begin\{matrix\} 0 \textbackslash\textbackslash{}
1 \textbackslash\textbackslash{} 1 \textbackslash\textbackslash{}
\textbackslash end\{matrix\} \textbackslash right) \textbackslash right|\}\{\textbackslash sqrt\{2\}\textbackslash sqrt\{2\}\}\textbackslash{]}
\textbackslash{[}\textbackslash theta =\{\{30\}\textasciicircum\{\textbackslash circ
\}\}\textbackslash{]} (iii) Let point D and E be \textbackslash{[}\textbackslash left(
0,-1,2 \textbackslash right)\textbackslash{]}and \textbackslash{[}\textbackslash left(
3,0,0 \textbackslash right)\textbackslash{]}respectively. \textbackslash{[}\textbackslash frac\{\textbackslash left|
\textbackslash overrightarrow\{BE\}\textbackslash bullet \{\{\{\textbackslash underset\{\textbackslash scriptscriptstyle\textbackslash thicksim\}\{n\}\}\}\_\{1\}\}
\textbackslash right|\}\{\textbackslash left| \{\{\{\textbackslash underset\{\textbackslash scriptscriptstyle\textbackslash thicksim\}\{n\}\}\}\_\{1\}\}
\textbackslash right|\}=\textbackslash frac\{\textbackslash left|
\textbackslash overrightarrow\{BD\}\textbackslash times \{\{\{\textbackslash underset\{\textbackslash scriptscriptstyle\textbackslash thicksim\}\{d\}\}\}\_\{\textbackslash ell
\}\} \textbackslash right|\}\{\textbackslash left| \{\{\{\textbackslash underset\{\textbackslash scriptscriptstyle\textbackslash thicksim\}\{d\}\}\}\_\{\textbackslash ell
\}\} \textbackslash right|\}\textbackslash{]} \textbackslash{[}\textbackslash frac\{\textbackslash left|
\textbackslash left( \textbackslash begin\{matrix\} 3+\textbackslash alpha
\textbackslash\textbackslash{} -2 \textbackslash\textbackslash{}
-\textbackslash alpha \textbackslash\textbackslash{} \textbackslash end\{matrix\}
\textbackslash right)\textbackslash bullet \textbackslash left(
\textbackslash begin\{matrix\} 1 \textbackslash\textbackslash{}
-1 \textbackslash\textbackslash{} 0 \textbackslash\textbackslash{}
\textbackslash end\{matrix\} \textbackslash right) \textbackslash right|\}\{\textbackslash sqrt\{2\}\}=\textbackslash frac\{\textbackslash left|
\textbackslash left( \textbackslash begin\{matrix\} \textbackslash alpha
\textbackslash\textbackslash{} -3 \textbackslash\textbackslash{}
2-\textbackslash alpha \textbackslash\textbackslash{} \textbackslash end\{matrix\}
\textbackslash right)\textbackslash times \textbackslash left(
\textbackslash begin\{matrix\} 0 \textbackslash\textbackslash{}
1 \textbackslash\textbackslash{} 1 \textbackslash\textbackslash{}
\textbackslash end\{matrix\} \textbackslash right) \textbackslash right|\}\{\textbackslash sqrt\{2\}\}\textbackslash{]}
\textbackslash{[}\textbackslash left| 3+\textbackslash alpha +2
\textbackslash right|=\textbackslash left| \textbackslash left(
\textbackslash begin\{matrix\} \textbackslash alpha -5 \textbackslash\textbackslash{}
-\textbackslash alpha \textbackslash\textbackslash{} \textbackslash alpha
\textbackslash\textbackslash{} \textbackslash end\{matrix\} \textbackslash right)
\textbackslash right|\textbackslash{]} \textbackslash{[}\textbackslash left|
\textbackslash alpha +5 \textbackslash right|=\textbackslash sqrt\{2\{\{\textbackslash alpha
\}\textasciicircum\{2\}\}+\{\{\textbackslash left( \textbackslash alpha
-5 \textbackslash right)\}\textasciicircum\{2\}\}\}\textbackslash{]}
\textbackslash{[}\{\{\textbackslash alpha \}\textasciicircum\{2\}\}+10\textbackslash alpha
+25=3\{\{\textbackslash alpha \}\textasciicircum\{2\}\}-10\textbackslash alpha
+25\textbackslash{]} \textbackslash{[}\textbackslash alpha \textbackslash left(
\textbackslash alpha -10 \textbackslash right)=0\textbackslash{]}
\textbackslash{[}\textbackslash therefore \textbackslash alpha
=0\textbackslash text\{ \}\textbackslash mathrm\{or \}10\textbackslash{]}
(iv) \textbackslash{[}\textbackslash left( \textbackslash begin\{matrix\}
4 \textbackslash\textbackslash{} 1 \textbackslash\textbackslash{}
\textbackslash beta \textbackslash\textbackslash{} \textbackslash end\{matrix\}
\textbackslash right)\textbackslash bullet \textbackslash left(
\textbackslash begin\{matrix\} 1 \textbackslash\textbackslash{}
-1 \textbackslash\textbackslash{} 0 \textbackslash\textbackslash{}
\textbackslash end\{matrix\} \textbackslash right)=4\textbackslash left(
1 \textbackslash right)+1\textbackslash left( -1 \textbackslash right)+\textbackslash beta
\textbackslash left( 0 \textbackslash right)=3\textbackslash{]}
And \textbackslash{[}\textbackslash left( \textbackslash begin\{matrix\}
4 \textbackslash\textbackslash{} 1 \textbackslash\textbackslash{}
\textbackslash beta \textbackslash\textbackslash{} \textbackslash end\{matrix\}
\textbackslash right)\textbackslash bullet \textbackslash left(
\textbackslash begin\{matrix\} 1 \textbackslash\textbackslash{}
1 \textbackslash\textbackslash{} -1 \textbackslash\textbackslash{}
\textbackslash end\{matrix\} \textbackslash right)=4\textbackslash left(
1 \textbackslash right)+1\textbackslash left( 1 \textbackslash right)+\textbackslash beta
\textbackslash left( -1 \textbackslash right)=5-\textbackslash beta
\textbackslash{]} Hence C lies on both p1 and p2. \textbackslash{[}\textbackslash left(
\textbackslash begin\{matrix\} 1 \textbackslash\textbackslash{}
-1 \textbackslash\textbackslash{} 0 \textbackslash\textbackslash{}
\textbackslash end\{matrix\} \textbackslash right)\textbackslash times
\textbackslash left( \textbackslash begin\{matrix\} 1 \textbackslash\textbackslash{}
1 \textbackslash\textbackslash{} -1 \textbackslash\textbackslash{}
\textbackslash end\{matrix\} \textbackslash right)=\textbackslash left(
\textbackslash begin\{matrix\} 1 \textbackslash\textbackslash{}
1 \textbackslash\textbackslash{} 2 \textbackslash\textbackslash{}
\textbackslash end\{matrix\} \textbackslash right)\textbackslash{]}
Eqn of the line of intersection is \textbackslash{[}\textbackslash text\{r\}=\textbackslash left(
\textbackslash begin\{matrix\} 4 \textbackslash\textbackslash{}
1 \textbackslash\textbackslash{} \textbackslash beta \textbackslash\textbackslash{}
\textbackslash end\{matrix\} \textbackslash right)+\textbackslash gamma
\textbackslash left( \textbackslash begin\{matrix\} 1 \textbackslash\textbackslash{}
1 \textbackslash\textbackslash{} 2 \textbackslash\textbackslash{}
\textbackslash end\{matrix\} \textbackslash right),\textbackslash gamma
\textbackslash in \textbackslash mathbb\{R\}\textbackslash{]}