%% LyX 2.3.6.1 created this file.  For more info, see http://www.lyx.org/.
%% Do not edit unless you really know what you are doing.
\documentclass[twoside,english]{article}
\usepackage[T1]{fontenc}
\usepackage[latin9]{inputenc}
\usepackage{geometry}
\geometry{verbose,tmargin=2cm,bmargin=2cm,lmargin=2cm,rmargin=2cm}

\makeatletter
%%%%%%%%%%%%%%%%%%%%%%%%%%%%%% User specified LaTeX commands.
\usepackage{helvet}
\renewcommand{\familydefault}{\sfdefault}
\usepackage[T1]{fontenc}
\usepackage[latin9]{inputenc}
\usepackage{geometry}
\geometry{verbose,tmargin=1.8cm,bmargin=4cm,lmargin=1.5cm,rmargin=2cm}
\usepackage{enumitem}
\usepackage{amstext}
\usepackage{amsthm}
\usepackage{amssymb}
\usepackage{setspace}
\usepackage{graphicx}
\doublespacing


\usepackage{enumitem}
\setenumerate[1]{label=\textbf{\arabic*}}
\setenumerate[2]{label=\textbf{(\alph*)}}
\setenumerate[3]{label=\textbf{(\roman*)}}
\setlist[enumerate]{align=right}

\setcounter{page}{2}

%for upright integrals
\usepackage[integrals]{wasysym}

%to be used in conjunction with fancyfoot for last page
\usepackage{zref-totpages}

%fancyhrd settings
\usepackage{fancyhdr}
\pagestyle{fancy}
\fancyhf{}

\fancypagestyle{laststyle}
{
   \fancyhf{}
   \chead{\thepage}
   \fancyfoot[L]{\copyright NJC }
   \fancyfoot[R]{\textbf{END}} %Put \thispagestyle{laststyle} in the last page
}

%%centering page number
\chead{\thepage}

\renewcommand{\headrulewidth}{0pt}
\renewcommand{\footrulewidth}{0pt}

%%footer settings, different footer for ODD and EVEN pages, also for the LASTPAGE
\fancyfoot[LO]{\copyright NJC\hfill \textbf{[Turn Over}}
\fancyfoot[LE]{\copyright NJC }

%%shameless self-plug BRW

\makeatother

\usepackage{babel}
\begin{document}
{[}SPLIT\_HERE{]}
\begin{enumerate}
\item \textbackslash{[}\textbackslash begin\{align\} \& \{\{y\}\textasciicircum\{2\}\}=\textbackslash sin
x+\textbackslash cos x \textbackslash\textbackslash{} \& 2y\textbackslash frac\{\textbackslash text\{d\}y\}\{\textbackslash text\{d\}x\}=\textbackslash cos
x-\textbackslash sin x \textbackslash\textbackslash{} \textbackslash end\{align\}\textbackslash{]}
\textbackslash{[}2\{\{\textbackslash left( \textbackslash frac\{\textbackslash text\{d\}y\}\{\textbackslash text\{d\}x\}
\textbackslash right)\}\textasciicircum\{2\}\}+2y\textbackslash frac\{\{\{\textbackslash text\{d\}\}\textasciicircum\{2\}\}y\}\{\textbackslash text\{d\}\{\{x\}\textasciicircum\{2\}\}\}=-\textbackslash sin
x-\textbackslash cos x\textbackslash{]} \textbackslash{[}2\{\{\textbackslash left(
\textbackslash frac\{\textbackslash text\{d\}y\}\{\textbackslash text\{d\}x\}
\textbackslash right)\}\textasciicircum\{2\}\}+2y\textbackslash frac\{\{\{\textbackslash text\{d\}\}\textasciicircum\{2\}\}y\}\{\textbackslash text\{d\}\{\{x\}\textasciicircum\{2\}\}\}=-\{\{y\}\textasciicircum\{2\}\}\textbackslash{]}
\textbackslash{[}4\textbackslash frac\{\textbackslash text\{d\}y\}\{\textbackslash text\{d\}x\}\textbackslash frac\{\{\{\textbackslash text\{d\}\}\textasciicircum\{2\}\}y\}\{\textbackslash text\{d\}\{\{x\}\textasciicircum\{2\}\}\}+2y\textbackslash frac\{\{\{\textbackslash text\{d\}\}\textasciicircum\{3\}\}y\}\{\textbackslash text\{d\}\{\{x\}\textasciicircum\{3\}\}\}+2\textbackslash frac\{\textbackslash text\{d\}y\}\{\textbackslash text\{d\}x\}\textbackslash frac\{\{\{\textbackslash text\{d\}\}\textasciicircum\{2\}\}y\}\{\textbackslash text\{d\}\{\{x\}\textasciicircum\{2\}\}\}=-2y\textbackslash frac\{\textbackslash text\{d\}y\}\{\textbackslash text\{d\}x\}\textbackslash{]}
\textbackslash{[}y\textbackslash frac\{\{\{\textbackslash text\{d\}\}\textasciicircum\{3\}\}y\}\{\textbackslash text\{d\}\{\{x\}\textasciicircum\{3\}\}\}+3\textbackslash frac\{\textbackslash text\{d\}y\}\{\textbackslash text\{d\}x\}\textbackslash frac\{\{\{\textbackslash text\{d\}\}\textasciicircum\{2\}\}y\}\{\textbackslash text\{d\}\{\{x\}\textasciicircum\{2\}\}\}+y\textbackslash frac\{\textbackslash text\{d\}y\}\{\textbackslash text\{d\}x\}=0\textbackslash{]}
A = 3

{[}SPLIT\_HERE{]}
\item \textbackslash{[}\textbackslash begin\{align\} \& y=\textbackslash text\{f\}\textbackslash left(
x \textbackslash right)=\textbackslash frac\{3\}\{\{\{x\}\textasciicircum\{2\}\}+2\}
\textbackslash\textbackslash{} \& \textbackslash text\{ \}\textbackslash downarrow
\textbackslash text\{after \}A\textbackslash left( \textbackslash text\{replace
\}x\textbackslash text\{ by \}x+1 \textbackslash right) \textbackslash\textbackslash{}
\& y=\textbackslash text\{f\}\textbackslash left( x+1 \textbackslash right)
\textbackslash\textbackslash{} \textbackslash end\{align\}\textbackslash{]}
\textbackslash{[}\textbackslash begin\{align\} \& \textbackslash text\{
\}\textbackslash downarrow \textbackslash text\{after \}B\textbackslash left(
\textbackslash text\{replace \}x\textbackslash text\{ by \}3x \textbackslash right)
\textbackslash\textbackslash{} \& y=\textbackslash text\{f\}\textbackslash left(
3x+1 \textbackslash right) \textbackslash\textbackslash{} \textbackslash end\{align\}\textbackslash{]}
\textbackslash{[}\textbackslash begin\{align\} \& \textbackslash text\{
\}\textbackslash downarrow \textbackslash text\{after \}C\textbackslash left(
\textbackslash text\{replace \}x\textbackslash text\{ by \}-x \textbackslash right)
\textbackslash\textbackslash{} \& y=\textbackslash text\{f\}\textbackslash left(
-3x+1 \textbackslash right) \textbackslash\textbackslash{} \textbackslash end\{align\}\textbackslash{]}
\$y=\textbackslash text\{f\}\textbackslash left( -3x+1 \textbackslash right)=\textbackslash frac\{3\}\{\{\{\textbackslash left(
-3x+1 \textbackslash right)\}\textasciicircum\{2\}\}+2\}\$ \$=\textbackslash frac\{3\}\{9\{\{x\}\textasciicircum\{2\}\}-6x+3\}\$
\$y=\textbackslash frac\{1\}\{3\{\{x\}\textasciicircum\{2\}\}-2x+1\}\$ 

{[}SPLIT\_HERE{]}
\item (a) \textbackslash{[}\textbackslash int\{\{\{x\}\textasciicircum\{2\}\}\{\{\textbackslash tan
\}\textasciicircum\{-1\}\}\textbackslash left( 2x \textbackslash right)\textbackslash text\{
\}\}\textbackslash text\{d\}x=\textbackslash frac\{\{\{x\}\textasciicircum\{3\}\}\}\{3\}\{\{\textbackslash tan
\}\textasciicircum\{-1\}\}\textbackslash left( 2x \textbackslash right)-\textbackslash int\{\textbackslash left(
\textbackslash frac\{\{\{x\}\textasciicircum\{3\}\}\}\{3\} \textbackslash right)\textbackslash left(
\textbackslash frac\{2\}\{1+4\{\{x\}\textasciicircum\{2\}\}\} \textbackslash right)\textbackslash text\{
\}\}\textbackslash text\{d\}x\textbackslash{]} \textbackslash{[}=\textbackslash frac\{\{\{x\}\textasciicircum\{3\}\}\}\{3\}\{\{\textbackslash tan
\}\textasciicircum\{-1\}\}\textbackslash left( 2x \textbackslash right)-\textbackslash frac\{2\}\{3\}\textbackslash int\{\textbackslash left(
\textbackslash frac\{1\}\{4\}x-\textbackslash frac\{x\}\{4\textbackslash left(
1+4\{\{x\}\textasciicircum\{2\}\} \textbackslash right)\} \textbackslash right)\textbackslash text\{
\}\}\textbackslash text\{d\}x\textbackslash{]} \textbackslash{[}=\textbackslash frac\{\{\{x\}\textasciicircum\{3\}\}\}\{3\}\{\{\textbackslash tan
\}\textasciicircum\{-1\}\}\textbackslash left( 2x \textbackslash right)-\textbackslash frac\{1\}\{12\}\{\{x\}\textasciicircum\{2\}\}+\textbackslash frac\{1\}\{6\}\textbackslash int\{\textbackslash frac\{x\}\{1+4\{\{x\}\textasciicircum\{2\}\}\}\textbackslash text\{
\}\}\textbackslash text\{d\}x\textbackslash{]} \textbackslash{[}=\textbackslash frac\{\{\{x\}\textasciicircum\{3\}\}\}\{3\}\{\{\textbackslash tan
\}\textasciicircum\{-1\}\}\textbackslash left( 2x \textbackslash right)-\textbackslash frac\{1\}\{12\}\{\{x\}\textasciicircum\{2\}\}+\textbackslash frac\{1\}\{48\}\textbackslash int\{\textbackslash frac\{8x\}\{1+4\{\{x\}\textasciicircum\{2\}\}\}\textbackslash text\{
\}\}\textbackslash text\{d\}x\textbackslash{]} \textbackslash{[}=\textbackslash frac\{\{\{x\}\textasciicircum\{3\}\}\}\{3\}\{\{\textbackslash tan
\}\textasciicircum\{-1\}\}\textbackslash left( 2x \textbackslash right)-\textbackslash frac\{1\}\{12\}\{\{x\}\textasciicircum\{2\}\}+\textbackslash frac\{1\}\{48\}\textbackslash ln
\textbackslash left( 1+4\{\{x\}\textasciicircum\{2\}\} \textbackslash right)+c\textbackslash{]}
(b) \textbackslash{[}\textbackslash int\{\textbackslash frac\{x\}\{\textbackslash sqrt\{8-2x-\{\{x\}\textasciicircum\{2\}\}\}\}\textbackslash text\{
\}\}\textbackslash text\{d\}x=-\textbackslash frac\{1\}\{2\}\textbackslash int\{\textbackslash frac\{-2x-2\}\{\textbackslash sqrt\{8-2x-\{\{x\}\textasciicircum\{2\}\}\}\}\textbackslash text\{
\}\}\textbackslash text\{d\}x-\textbackslash int\{\textbackslash frac\{1\}\{\textbackslash sqrt\{8-2x-\{\{x\}\textasciicircum\{2\}\}\}\}\textbackslash text\{
\}\}\textbackslash text\{d\}x\textbackslash{]} \textbackslash{[}=-\textbackslash sqrt\{8-2x-\{\{x\}\textasciicircum\{2\}\}\}-\textbackslash int\{\textbackslash frac\{1\}\{\textbackslash sqrt\{9-\{\{\textbackslash left(
x+1 \textbackslash right)\}\textasciicircum\{2\}\}\}\}\textbackslash text\{
\}\}\textbackslash text\{d\}x\textbackslash{]} \textbackslash{[}=-\textbackslash sqrt\{8-2x-\{\{x\}\textasciicircum\{2\}\}\}-\{\{\textbackslash sin
\}\textasciicircum\{-1\}\}\textbackslash left( \textbackslash frac\{x+1\}\{3\}
\textbackslash right)+c\textbackslash{]} 

{[}SPLIT\_HERE{]}
\item (i) \$\{\{y\}\textasciicircum\{3\}\}-2x\{\{y\}\textasciicircum\{2\}\}+3\{\{x\}\textasciicircum\{2\}\}-3=0\$
Differentiate with respect to x: \$3\{\{y\}\textasciicircum\{2\}\}\textbackslash frac\{\textbackslash text\{d\}y\}\{\textbackslash text\{d\}x\}-(2\{\{y\}\textasciicircum\{2\}\}+4xy\textbackslash frac\{\textbackslash text\{d\}y\}\{\textbackslash text\{d\}x\})+6x=0\$
\textbackslash{[}\textbackslash begin\{align\} \& (3\{\{y\}\textasciicircum\{2\}\}-4xy)\textbackslash frac\{\textbackslash text\{d\}y\}\{\textbackslash text\{d\}x\}=2\{\{y\}\textasciicircum\{2\}\}-6x
\textbackslash\textbackslash{} \& \textbackslash frac\{\textbackslash text\{d\}y\}\{\textbackslash text\{d\}x\}=\textbackslash frac\{2\{\{y\}\textasciicircum\{2\}\}-6x\}\{3\{\{y\}\textasciicircum\{2\}\}-4xy\}
\textbackslash\textbackslash{} \textbackslash end\{align\}\textbackslash{]}
(ii) Gradient of tangent to curve at P(2, 3) =\textbackslash{[}\textbackslash frac\{2\{\{(3)\}\textasciicircum\{2\}\}-6(2)\}\{3\{\{(3)\}\textasciicircum\{2\}\}-4(2)(3)\}=2\textbackslash{]}
Gradient of normal to curve at P = \$-\textbackslash frac\{1\}\{2\}\$
Equation of normal at P: \$\textbackslash begin\{align\} \& y-3=-\textbackslash tfrac\{1\}\{2\}(x-2)
\textbackslash\textbackslash{} \& \textbackslash Rightarrow y=-\textbackslash tfrac\{1\}\{2\}x+4
\textbackslash\textbackslash{} \textbackslash end\{align\}\$ (iii)
When x = 0, normal cuts y-axis at A(0, 4); C: \$\{\{y\}\textasciicircum\{3\}\}-2x\{\{y\}\textasciicircum\{2\}\}+3\{\{x\}\textasciicircum\{2\}\}-3=0\$
When x = 0, \$\{\{y\}\textasciicircum\{3\}\}\$= 3 y = \$\textbackslash sqrt{[}3{]}\{3\}\$
C meets y-axis at R(0, \$\textbackslash sqrt{[}3{]}\{3\}\$). Area
of triangle APR = \$\textbackslash frac\{1\}\{2\}\textbackslash times
2\textbackslash times (4-\textbackslash sqrt{[}3{]}\{3\})\$ = \$4-\textbackslash sqrt{[}3{]}\{3\}\$
where a = 4, b = 3

{[}SPLIT\_HERE{]}
\item (a) <INSERT\_IMAGE\_HERE> (b) <INSERT\_IMAGE\_HERE>

{[}SPLIT\_HERE{]}
\item (i) \textbackslash{[}\textbackslash begin\{align\} \& \{\{\textbackslash text\{R\}\}\_\{\textbackslash text\{g\}\}\}=\textbackslash left(
1,8 \textbackslash right{]}\textbackslash text\{ \}\{\{\textbackslash text\{D\}\}\_\{\textbackslash text\{f\}\}\}=\textbackslash mathbb\{R\}
\textbackslash\textbackslash{} \& \textbackslash text\{Since \}\{\{\textbackslash text\{R\}\}\_\{\textbackslash text\{g\}\}\}\textbackslash subseteq
\{\{\textbackslash text\{D\}\}\_\{\textbackslash text\{f\}\}\},\textbackslash text\{
fg exists\}\textbackslash text\{.\} \textbackslash\textbackslash{}
\textbackslash end\{align\}\textbackslash{]} \textbackslash{[}\textbackslash text\{f\}\textbackslash left(
x \textbackslash right)=-\{\{\textbackslash left( x-\textbackslash frac\{5\}\{2\}
\textbackslash right)\}\textasciicircum\{2\}\}+\textbackslash frac\{49\}\{4\}\textbackslash{]}
\textbackslash{[}\textbackslash text\{f\}\textbackslash left(
8 \textbackslash right)=-\{\{\textbackslash left( 8-\textbackslash frac\{5\}\{2\}
\textbackslash right)\}\textasciicircum\{2\}\}+\textbackslash frac\{49\}\{4\}=-18\textbackslash{]}
\textbackslash{[}\textbackslash begin\{align\} \& \textbackslash left{[}
0,\textbackslash infty \textbackslash right)\textbackslash xrightarrow{[}\textbackslash text\{g\}{]}\{\}\textbackslash left(
1,8 \textbackslash right{]}\textbackslash xrightarrow{[}\textbackslash text\{f\}{]}\{\}\textbackslash left{[}
-18,\textbackslash frac\{49\}\{4\} \textbackslash right{]} \textbackslash\textbackslash{}
\& \textbackslash therefore \{\{\textbackslash text\{R\}\}\_\{\textbackslash text\{fg\}\}\}=\textbackslash left{[}
-18,\textbackslash frac\{49\}\{4\} \textbackslash right{]} \textbackslash\textbackslash{}
\textbackslash end\{align\}\textbackslash{]} (ii) \textbackslash{[}\textbackslash text\{Let
\}y=-\{\{\textbackslash left( x-\textbackslash frac\{\textbackslash lambda
\}\{2\} \textbackslash right)\}\textasciicircum\{2\}\}+\textbackslash left(
6+\textbackslash frac\{\{\{\textbackslash lambda \}\textasciicircum\{2\}\}\}\{4\}
\textbackslash right)\textbackslash{]} \textbackslash{[}\textbackslash begin\{align\}
\& \textbackslash text\{ \}x=\textbackslash frac\{\textbackslash lambda
\}\{2\}\textbackslash pm \textbackslash sqrt\{6+\textbackslash frac\{\{\{\textbackslash lambda
\}\textasciicircum\{2\}\}\}\{4\}-y\} \textbackslash\textbackslash{}
\& \textbackslash therefore x=\textbackslash frac\{\textbackslash lambda
\}\{2\}-\textbackslash sqrt\{6+\textbackslash frac\{\{\{\textbackslash lambda
\}\textasciicircum\{2\}\}\}\{4\}-y\}\textbackslash left( \textbackslash because
x\textbackslash le \textbackslash frac\{\textbackslash lambda \}\{2\}
\textbackslash right) \textbackslash\textbackslash{} \textbackslash end\{align\}\textbackslash{]}
\textbackslash{[}\{\{\textbackslash text\{f\}\}\textasciicircum\{-1\}\}:x\textbackslash mapsto
\textbackslash frac\{\textbackslash lambda \}\{2\}-\textbackslash sqrt\{6+\textbackslash frac\{\{\{\textbackslash lambda
\}\textasciicircum\{2\}\}\}\{4\}-x\},\textbackslash ,\textbackslash ,\textbackslash ,\textbackslash ,\textbackslash ,x\textbackslash in
\textbackslash text\{ \}\textbackslash mathbb\{R\},\textbackslash text\{
\}x\textbackslash le 6+\textbackslash frac\{\{\{\textbackslash lambda
\}\textasciicircum\{2\}\}\}\{4\}\textbackslash{]} 

{[}SPLIT\_HERE{]}
\item (a)\textbackslash{[}\textbackslash frac\{\textbackslash text\{d\}\}\{\textbackslash text\{d\}x\}\textbackslash left(
\{\{\textbackslash tan \}\textasciicircum\{3\}\}x \textbackslash right)=3\{\{\textbackslash tan
\}\textasciicircum\{2\}\}x\textbackslash text\{ se\}\{\{\textbackslash text\{c\}\}\textasciicircum\{2\}\}x\textbackslash{]}
\textbackslash{[}\textbackslash int\{\textbackslash text\{se\}\{\{\textbackslash text\{c\}\}\textasciicircum\{4\}\}x\textbackslash text\{
\}\}\textbackslash text\{d\}x=\textbackslash int\{\textbackslash text\{se\}\{\{\textbackslash text\{c\}\}\textasciicircum\{2\}\}x\textbackslash left(
1+\{\{\textbackslash tan \}\textasciicircum\{2\}\}x \textbackslash right)\textbackslash text\{
\}\}\textbackslash text\{d\}x\textbackslash{]} \textbackslash{[}=\textbackslash int\{\textbackslash left(
\textbackslash text\{se\}\{\{\textbackslash text\{c\}\}\textasciicircum\{2\}\}x+\textbackslash text\{se\}\{\{\textbackslash text\{c\}\}\textasciicircum\{2\}\}x\{\{\textbackslash tan
\}\textasciicircum\{2\}\}x \textbackslash right)\textbackslash text\{
\}\}\textbackslash text\{d\}x\textbackslash{]} \textbackslash{[}=\textbackslash tan
x+\textbackslash frac\{1\}\{3\}\{\{\textbackslash tan \}\textasciicircum\{3\}\}x+c\textbackslash{]}
(b) \textbackslash{[}\textbackslash int\{\textbackslash sin x\textbackslash left(
\textbackslash sin x+\textbackslash sin 3x \textbackslash right)\textbackslash text\{
\}\}\textbackslash text\{d\}x=\textbackslash int\{\textbackslash left(
\{\{\textbackslash sin \}\textasciicircum\{2\}\}x+\textbackslash sin
x\textbackslash sin 3x \textbackslash right)\textbackslash text\{
\}\}\textbackslash text\{d\}x\textbackslash{]} \textbackslash{[}=\textbackslash int\{\textbackslash left(
\textbackslash frac\{1-\textbackslash cos 2x\}\{2\}-\textbackslash frac\{\textbackslash cos
4x-\textbackslash cos 2x\}\{2\} \textbackslash right)\textbackslash text\{
\}\}\textbackslash text\{d\}x\textbackslash{]} \textbackslash{[}\textbackslash begin\{align\}
\& =\textbackslash int\{\textbackslash left( \textbackslash frac\{1-\textbackslash cos
4x\}\{2\} \textbackslash right)\textbackslash text\{ \}\}\textbackslash text\{d\}x
\textbackslash\textbackslash{} \& =\textbackslash frac\{1\}\{2\}x-\textbackslash frac\{\textbackslash sin
4x\}\{8\}+c \textbackslash\textbackslash{} \textbackslash end\{align\}\textbackslash{]}
(c) \textbackslash{[}\textbackslash int\_\{0\}\textasciicircum\{p\}\{x\textbackslash left|
2-x \textbackslash right|\}\textbackslash text\{ d\}x=\textbackslash int\_\{0\}\textasciicircum\{2\}\{x\textbackslash left(
2-x \textbackslash right)\}\textbackslash text\{ d\}x+\textbackslash int\_\{2\}\textasciicircum\{p\}\{x\textbackslash left(
x-2 \textbackslash right)\}\textbackslash text\{ d\}x\textbackslash{]}
\textbackslash{[}\textbackslash begin\{align\} \& =\textbackslash left{[}
\{\{x\}\textasciicircum\{2\}\}-\textbackslash frac\{\{\{x\}\textasciicircum\{3\}\}\}\{3\}
\textbackslash right{]}\_\{0\}\textasciicircum\{2\}+\textbackslash left{[}
\textbackslash frac\{\{\{x\}\textasciicircum\{3\}\}\}\{3\}-\{\{x\}\textasciicircum\{2\}\}
\textbackslash right{]}\_\{2\}\textasciicircum\{p\} \textbackslash\textbackslash{}
\& =4-\textbackslash frac\{8\}\{3\}+\textbackslash frac\{\{\{p\}\textasciicircum\{3\}\}\}\{3\}-\{\{p\}\textasciicircum\{2\}\}-\textbackslash left(
\textbackslash frac\{8\}\{3\}-4 \textbackslash right) \textbackslash\textbackslash{}
\& =\textbackslash frac\{8\}\{3\}+\textbackslash frac\{\{\{p\}\textasciicircum\{3\}\}\}\{3\}-\{\{p\}\textasciicircum\{2\}\}
\textbackslash end\{align\}\textbackslash{]} 

{[}SPLIT\_HERE{]}
\item (i) By Ratio Theorem, \textbackslash{[}\textbackslash begin\{align\}
\& \textbackslash text\{a\}=\textbackslash frac\{3\textbackslash text\{c\}+2\textbackslash text\{b\}\}\{5\}\textbackslash text\{
\} \textbackslash\textbackslash{} \& \textbackslash Rightarrow
\textbackslash text\{c\}=\textbackslash frac\{5\textbackslash text\{a\}-2\textbackslash text\{b\}\}\{3\}
\textbackslash\textbackslash{} \textbackslash end\{align\}\textbackslash{]}
(ii) \$\textbackslash mathbf\{c\}\textbackslash cdot \textbackslash mathbf\{b\}=0\$
\textbackslash{[}\textbackslash frac\{5\textbackslash text\{a\}-2\textbackslash text\{b\}\}\{3\}\textbackslash cdot
\textbackslash text\{b\}=0\textbackslash{]} \textbackslash{[}\textbackslash frac\{5\textbackslash text\{a\}\textbackslash cdot
\textbackslash text\{b\}\}\{3\}=\textbackslash frac\{2\textbackslash text\{b\}\textbackslash cdot
\textbackslash text\{b\}\}\{3\}\textbackslash{]} \textbackslash{[}\textbackslash text\{a\}\textbackslash cdot
\textbackslash text\{b\}=\textbackslash frac\{2\}\{5\}\{\{\textbackslash left|
\textbackslash text\{b\} \textbackslash right|\}\textasciicircum\{2\}\}\textbackslash{]}
(iii) Area \textbackslash{[}=\textbackslash frac\{1\}\{2\}\textbackslash left|
\textbackslash overrightarrow\{BP\}\textbackslash times \textbackslash overrightarrow\{BC\}
\textbackslash right|\textbackslash{]} \textbackslash{[}=\textbackslash frac\{1\}\{2\}\textbackslash left|
\textbackslash left{[} \textbackslash text\{b\}-\textbackslash left(
-\textbackslash text\{b\} \textbackslash right) \textbackslash right{]}\textbackslash times
\textbackslash left{[} \textbackslash frac\{5\textbackslash text\{a\}-2\textbackslash text\{b\}\}\{3\}-\textbackslash text\{b\}
\textbackslash right{]} \textbackslash right|\textbackslash{]}
\textbackslash{[}=\textbackslash frac\{1\}\{2\}\textbackslash left|
2\textbackslash text\{b\}\textbackslash times \textbackslash left(
\textbackslash frac\{5\textbackslash text\{a\}\}\{3\}-\textbackslash frac\{5\textbackslash text\{b\}\}\{3\}
\textbackslash right) \textbackslash right|\textbackslash{]} \textbackslash{[}=\textbackslash frac\{5\}\{3\}\textbackslash left|
\textbackslash text\{b\}\textbackslash times \textbackslash left(
\textbackslash text\{a\}-\textbackslash text\{b\} \textbackslash right)
\textbackslash right|\textbackslash{]} \textbackslash{[}=\textbackslash frac\{5\}\{3\}\textbackslash left|
\textbackslash text\{b\}\textbackslash times \textbackslash text\{a\}
\textbackslash right|\textbackslash left( \textbackslash because
\textbackslash text\{b\}\textbackslash times \textbackslash text\{b\}=\textbackslash text\{0\}
\textbackslash right)\textbackslash{]} \textbackslash{[}=\textbackslash frac\{5\}\{3\}\textbackslash left|
-\textbackslash text\{a\}\textbackslash times \textbackslash text\{b\}
\textbackslash right|=\textbackslash frac\{5\}\{3\}\textbackslash left|
\textbackslash text\{a\}\textbackslash times \textbackslash text\{b\}
\textbackslash right|\textbackslash left( \textbackslash because
\textbackslash text\{a\}\textbackslash times \textbackslash text\{b\}=-\textbackslash text\{b\}\textbackslash times
\textbackslash text\{a\} \textbackslash right)\textbackslash{]}
OR Area \textbackslash{[}=\textbackslash frac\{1\}\{2\}\textbackslash left|
\textbackslash overrightarrow\{PC\}\textbackslash times \textbackslash overrightarrow\{BC\}
\textbackslash right|\textbackslash{]} \textbackslash{[}=\textbackslash frac\{1\}\{2\}\textbackslash left|
\textbackslash left{[} \textbackslash frac\{5\textbackslash text\{a\}-2\textbackslash text\{b\}\}\{3\}-\textbackslash left(
-\textbackslash text\{b\} \textbackslash right) \textbackslash right{]}\textbackslash times
\textbackslash left{[} \textbackslash frac\{5\textbackslash text\{a\}-2\textbackslash text\{b\}\}\{3\}-\textbackslash text\{b\}
\textbackslash right{]} \textbackslash right|\textbackslash{]}
\textbackslash{[}=\textbackslash frac\{1\}\{2\}\textbackslash left|
\textbackslash frac\{5\textbackslash text\{a\}+\textbackslash text\{b\}\}\{3\}\textbackslash times
\textbackslash frac\{5\}\{3\}\textbackslash left( \textbackslash text\{a\}-\textbackslash text\{b\}
\textbackslash right) \textbackslash right|\textbackslash{]} \textbackslash{[}=\textbackslash frac\{5\}\{18\}\textbackslash left|
5\textbackslash text\{a\}\textbackslash times \textbackslash text\{a\}-5\textbackslash text\{a\}\textbackslash times
\textbackslash text\{b\}+\textbackslash text\{b\}\textbackslash times
\textbackslash text\{a\}-\textbackslash text\{b\}\textbackslash times
\textbackslash text\{b\} \textbackslash right|\textbackslash{]}
\textbackslash{[}=\textbackslash frac\{5\}\{18\}\textbackslash left|
-5\textbackslash text\{a\}\textbackslash times \textbackslash text\{b\}-\textbackslash text\{a\}\textbackslash times
\textbackslash text\{b\} \textbackslash right|\textbackslash{]}\textbackslash{[}\textbackslash left(
\textbackslash because \textbackslash text\{a\}\textbackslash times
\textbackslash text\{a\}=\textbackslash text\{0 \}\textbackslash mathbf\{and\}\textbackslash text\{
b\}\textbackslash times \textbackslash text\{b\}=\textbackslash text\{0\}
\textbackslash right)\textbackslash{]} \textbackslash{[}=\textbackslash frac\{5\}\{3\}\textbackslash left|
\textbackslash text\{a\}\textbackslash times \textbackslash text\{b\}
\textbackslash right|\textbackslash{]} 

(iv) Since F is on the line AB, \textbackslash{[}\textbackslash overrightarrow\{OF\}=\textbackslash text\{a\}+\textbackslash lambda
\textbackslash left( \textbackslash text\{b\}-\textbackslash text\{a\}
\textbackslash right)\textbackslash{]} for some \$\textbackslash lambda
\textbackslash in \textbackslash mathbb\{R\}\$ \textbackslash{[}\textbackslash overrightarrow\{PF\}\textbackslash cdot
\textbackslash overrightarrow\{AB\}=0\textbackslash{]} \textbackslash{[}\textbackslash left{[}
\textbackslash text\{a\}+\textbackslash lambda \textbackslash left(
\textbackslash text\{b\}-\textbackslash text\{a\} \textbackslash right)-\textbackslash left(
-\textbackslash text\{b\} \textbackslash right) \textbackslash right{]}\textbackslash bullet
\textbackslash left( \textbackslash text\{b\}-\textbackslash text\{a\}
\textbackslash right)=0\textbackslash{]} \textbackslash{[}\textbackslash left{[}
\textbackslash left( 1-\textbackslash lambda \textbackslash right)\textbackslash text\{a\}+\textbackslash left(
1+\textbackslash lambda \textbackslash right)\textbackslash text\{b\}
\textbackslash right{]}\textbackslash bullet \textbackslash left(
\textbackslash text\{b\}-\textbackslash text\{a\} \textbackslash right)=0\textbackslash{]}
\textbackslash{[}\textbackslash left( \textbackslash lambda -1
\textbackslash right)\{\{\textbackslash left| \textbackslash text\{a\}
\textbackslash right|\}\textasciicircum\{2\}\}+\textbackslash left(
1+\textbackslash lambda \textbackslash right)\{\{\textbackslash left|
\textbackslash text\{b\} \textbackslash right|\}\textasciicircum\{2\}\}+\textbackslash left(
1-\textbackslash lambda \textbackslash right)\textbackslash text\{a\}\textbackslash bullet
\textbackslash text\{b\}-\textbackslash left( 1+\textbackslash lambda
\textbackslash right)\textbackslash text\{a\}\textbackslash bullet
\textbackslash text\{b\}=0\textbackslash{]} \textbackslash{[}\textbackslash left(
\textbackslash lambda -1 \textbackslash right)\textbackslash frac\{7\}\{25\}\{\{\textbackslash left|
\textbackslash text\{b\} \textbackslash right|\}\textasciicircum\{2\}\}+\textbackslash left(
1+\textbackslash lambda \textbackslash right)\{\{\textbackslash left|
\textbackslash text\{b\} \textbackslash right|\}\textasciicircum\{2\}\}-2\textbackslash lambda
\textbackslash left( \textbackslash frac\{2\}\{5\}\{\{\textbackslash left|
\textbackslash text\{b\} \textbackslash right|\}\textasciicircum\{2\}\}
\textbackslash right)=0\textbackslash{]} \textbackslash{[}\textbackslash left(
\textbackslash frac\{7\}\{25\}\textbackslash lambda -\textbackslash frac\{7\}\{25\}+1+\textbackslash lambda
-\textbackslash frac\{4\}\{5\}\textbackslash lambda \textbackslash right)\{\{\textbackslash left|
\textbackslash text\{b\} \textbackslash right|\}\textasciicircum\{2\}\}=0\textbackslash{]}
\textbackslash{[}\textbackslash frac\{12\}\{25\}\textbackslash lambda
+\textbackslash frac\{18\}\{25\}=0\textbackslash{]} \textbackslash{[}\textbackslash left(
\textbackslash because \textbackslash text\{b \}\textbackslash mathbf\{is\}\textbackslash text\{
\}\textbackslash mathbf\{a\}\textbackslash text\{ \}\textbackslash mathbf\{non
zero vector\} \textbackslash right)\textbackslash{]} \textbackslash{[}\textbackslash lambda
=-\textbackslash frac\{3\}\{2\}\textbackslash{]} \textbackslash{[}\textbackslash overrightarrow\{OF\}=\textbackslash text\{a\}-\textbackslash frac\{3\}\{2\}\textbackslash left(
\textbackslash text\{b\}-\textbackslash text\{a\} \textbackslash right)=\textbackslash frac\{5\textbackslash text\{a\}-3\textbackslash text\{b\}\}\{2\}\textbackslash{]} 

{[}SPLIT\_HERE{]}
\item (ai) Since \textbackslash{[}x=2\textbackslash{]} is a vertical
asymptote, \$s=4\$ \textbackslash{[}\textbackslash text\{At the
point \}(0,\textbackslash tfrac\{1\}\{2\})\textbackslash text\{
\}\textbackslash frac\{4\{\{(0)\}\textasciicircum\{2\}\}+p(0)-q\}\{\{\{(0)\}\textasciicircum\{2\}\}-4\}=\textbackslash frac\{1\}\{2\}\textbackslash{]}
\$\textbackslash therefore q=2\$ (ii) When y = 1, \$1=\textbackslash frac\{4\{\{x\}\textasciicircum\{2\}\}+px-2\}\{\{\{x\}\textasciicircum\{2\}\}-4\}\$
\$\{\{x\}\textasciicircum\{2\}\}-4=4\{\{x\}\textasciicircum\{2\}\}+px-2\$
\$3\{\{x\}\textasciicircum\{2\}\}+px+2=0\$ Since y = 1 is a tangent
to C, Discriminant = 0 \$\{\{p\}\textasciicircum\{2\}\}-4\textbackslash left(
3 \textbackslash right)\textbackslash left( 2 \textbackslash right)=0\$
\$p=-2\textbackslash sqrt\{6\}\textbackslash left( \textbackslash because
p<0 \textbackslash right)\$ 

(bi)

(bii) \textbackslash{[}\textbackslash begin\{align\} \& 10\{\{(x+2)\}\textasciicircum\{2\}\}=3\textbackslash left(
10-\{\{\textbackslash left( \textbackslash frac\{4\{\{x\}\textasciicircum\{2\}\}+4x+1\}\{\{\{x\}\textasciicircum\{2\}\}-1\}
\textbackslash right)\}\textasciicircum\{2\}\} \textbackslash right)
\textbackslash\textbackslash{} \& 10\{\{(x+2)\}\textasciicircum\{2\}\}+3\{\{\textbackslash left(
\textbackslash frac\{4\{\{x\}\textasciicircum\{2\}\}+4x+1\}\{\{\{x\}\textasciicircum\{2\}\}-1\}
\textbackslash right)\}\textasciicircum\{2\}\}=30 \textbackslash\textbackslash{}
\& \textbackslash frac\{\{\{(x+2)\}\textasciicircum\{2\}\}\}\{3\}+\textbackslash frac\{\{\{\textbackslash left(
\textbackslash frac\{4\{\{x\}\textasciicircum\{2\}\}+4x+1\}\{\{\{x\}\textasciicircum\{2\}\}-1\}
\textbackslash right)\}\textasciicircum\{2\}\}\}\{10\}=1 \textbackslash\textbackslash{}
\textbackslash end\{align\}\textbackslash{]} The equation of the
additional curve is \textbackslash{[}\textbackslash frac\{\{\{(x+2)\}\textasciicircum\{2\}\}\}\{3\}+\textbackslash frac\{\{\{y\}\textasciicircum\{2\}\}\}\{10\}=1\textbackslash{]} 

{[}SPLIT\_HERE{]}
\item (ai) Height of cylinder = \$\textbackslash sqrt\{\{\{8\}\textasciicircum\{2\}\}-\{\{x\}\textasciicircum\{2\}\}\}\$\$\textbackslash times
2\$ = 2\$\textbackslash sqrt\{\{\{8\}\textasciicircum\{2\}\}-\{\{x\}\textasciicircum\{2\}\}\}\$
\$\textbackslash begin\{align\} \& S=2\textbackslash pi rh \textbackslash\textbackslash{}
\& \textbackslash ,\textbackslash ,\textbackslash ,\textbackslash ,\textbackslash ,=2\textbackslash pi
\textbackslash times \textbackslash left( \textbackslash frac\{x\}\{2\}
\textbackslash right)\textbackslash times 2\textbackslash sqrt\{64-\{\{x\}\textasciicircum\{2\}\}\}
\textbackslash\textbackslash{} \textbackslash end\{align\}\$ \$S\$=
\$2\textbackslash pi x\textbackslash sqrt\{64-\{\{x\}\textasciicircum\{2\}\}\}\$
c (shown) (ii)\textbackslash{[}\textbackslash frac\{\textbackslash text\{d\}S\}\{\textbackslash text\{d\}x\}=2\textbackslash pi
\textbackslash sqrt\{64-\{\{x\}\textasciicircum\{2\}\}\}+2\textbackslash pi
x\textbackslash left( \textbackslash frac\{-2x\}\{2\textbackslash sqrt\{64-\{\{x\}\textasciicircum\{2\}\}\}\}
\textbackslash right)\textbackslash{]} = \textbackslash{[}2\textbackslash pi
\textbackslash sqrt\{64-\{\{x\}\textasciicircum\{2\}\}\}-\textbackslash frac\{2\textbackslash pi
\{\{x\}\textasciicircum\{2\}\}\}\{\textbackslash sqrt\{64-\{\{x\}\textasciicircum\{2\}\}\}\}\textbackslash{]}
= \$\textbackslash frac\{4\textbackslash pi (32-\{\{x\}\textasciicircum\{2\}\})\}\{\textbackslash sqrt\{64-\{\{x\}\textasciicircum\{2\}\}\}\}\$
For stationary point, \textbackslash{[}\textbackslash frac\{\textbackslash text\{d\}S\}\{\textbackslash text\{d\}x\}\textbackslash{]}=
0 \textbackslash{[}\textbackslash frac\{4\textbackslash pi (32-\{\{x\}\textasciicircum\{2\}\})\}\{\textbackslash sqrt\{64-\{\{x\}\textasciicircum\{2\}\}\}\}=0\textbackslash{]}
\$x=\textbackslash sqrt\{32\}=4\textbackslash sqrt\{2\}\$ (since
x > 0) x : 2\$\textbackslash sqrt\{64-\{\{x\}\textasciicircum\{2\}\}\}\$=
1 : k \$\textbackslash sqrt\{32\}\$: 2\$\textbackslash sqrt\{32\}\$
= 1 : k k = 2 First derivative test: x \$\textbackslash sqrt\{32\}\{\{\textbackslash ,\}\textasciicircum\{-\}\}\$
\$\textbackslash sqrt\{32\}\$ \$\textbackslash sqrt\{32\}\{\{\textbackslash ,\}\textasciicircum\{+\}\}\$
\$\textbackslash frac\{\textbackslash text\{d\}S\}\{\textbackslash text\{d\}x\}=\textbackslash frac\{4\textbackslash pi
(32-\{\{x\}\textasciicircum\{2\}\})\}\{\textbackslash sqrt\{64-\{\{x\}\textasciicircum\{2\}\}\}\}\$
\$\textbackslash frac\{(+)\}\{(+)\}=(+)\$ 0 \$\textbackslash frac\{(-)\}\{(+)\}=(-)\$
\$\textbackslash frac\{\{\}\}\{\{\}\}\$ Shape 

\$\textbackslash therefore \$ S is maximum when x = \$4\textbackslash sqrt\{2\}\$.
Second derivative test: \textbackslash{[}\textbackslash frac\{\{\{\textbackslash text\{d\}\}\textasciicircum\{2\}\}S\}\{\textbackslash text\{d\}\{\{x\}\textasciicircum\{2\}\}\}=\textbackslash frac\{\textbackslash text\{d\}\textbackslash left(
\textbackslash frac\{4\textbackslash pi (32-\{\{x\}\textasciicircum\{2\}\})\}\{\textbackslash sqrt\{64-\{\{x\}\textasciicircum\{2\}\}\}\}
\textbackslash right)\}\{\textbackslash text\{d\}x\}=\textbackslash frac\{4\textbackslash pi
{[}\textbackslash sqrt\{64-\{\{x\}\textasciicircum\{2\}\}\})(-2x)-(32-\{\{x\}\textasciicircum\{2\}\})\textbackslash frac\{1\}\{2\}\{\{\textbackslash left(
64-\{\{x\}\textasciicircum\{2\}\} \textbackslash right)\}\textasciicircum\{-\textbackslash tfrac\{1\}\{2\}\}\}(-2x){]}\}\{64-\{\{x\}\textasciicircum\{2\}\}\}\textbackslash{]}
= \$\textbackslash frac\{4\textbackslash pi (-x){[}2(64-\{\{x\}\textasciicircum\{2\}\})-\textbackslash left(
32-\{\{x\}\textasciicircum\{2\}\} \textbackslash right){]}\}\{\{\{(64-\{\{x\}\textasciicircum\{2\}\})\}\textasciicircum\{3/2\}\}\}=\textbackslash frac\{4\textbackslash pi
x(\{\{x\}\textasciicircum\{2\}\}-96)\}\{\{\{(64-\{\{x\}\textasciicircum\{2\}\})\}\textasciicircum\{3/2\}\}\}\$
When \$x=4\textbackslash sqrt\{2\},\$ \textbackslash{[}\textbackslash frac\{\{\{\textbackslash text\{d\}\}\textasciicircum\{2\}\}S\}\{\textbackslash text\{d\}\{\{x\}\textasciicircum\{2\}\}\}=\textbackslash{]}\$-8\textbackslash pi
\$ < 0 S is maximum when \$x=4\textbackslash sqrt\{2\}\$. Max S =
\$2\textbackslash pi \textbackslash left( 4\textbackslash sqrt\{2\}
\textbackslash right)\textbackslash sqrt\{64-32\}=64\textbackslash pi
\textbackslash text\{ c\}\{\{\textbackslash text\{m\}\}\textasciicircum\{2\}\}\$
(b) \$y=2\{\{\textbackslash sin \}\textasciicircum\{-1\}\}(3x),\textbackslash ,\textbackslash ,\textbackslash ,\textbackslash ,-\textbackslash frac\{1\}\{3\}\textbackslash le
x\textbackslash le \textbackslash frac\{1\}\{3\}.\$ \textbackslash{[}\textbackslash frac\{\textbackslash text\{d\}y\}\{\textbackslash text\{d\}x\}=\textbackslash frac\{2(3)\}\{\textbackslash sqrt\{1-9\{\{x\}\textasciicircum\{2\}\}\}\}=\textbackslash frac\{6\}\{\textbackslash sqrt\{1-9\{\{x\}\textasciicircum\{2\}\}\}\}\textbackslash{]}
\$y=2\{\{\textbackslash sin \}\textasciicircum\{-1\}\}(3x)\$ \$\textbackslash sin
\textbackslash frac\{y\}\{2\}=(3x)\$ When \$y\$=\$\textbackslash frac\{\textbackslash pi
\}\{3\}\$, 3x = sin\$\textbackslash frac\{\textbackslash pi \}\{6\}\$=
\$\textbackslash frac\{1\}\{2\}\$ x = \$\textbackslash frac\{1\}\{6\}\$
\textbackslash{[}\textbackslash frac\{\textbackslash text\{d\}x\}\{\textbackslash text\{d\}t\}=\textbackslash frac\{\textbackslash text\{d\}x\}\{\textbackslash text\{d\}y\}\textbackslash times
\textbackslash frac\{\textbackslash text\{d\}y\}\{\textbackslash text\{d\}t\}\textbackslash{]}
= \$\textbackslash frac\{\textbackslash sqrt\{1-9\{\{x\}\textasciicircum\{2\}\}\}\}\{6\}\textbackslash times
(-2)\$ = \$\textbackslash frac\{\textbackslash sqrt\{1-9\{\{\textbackslash left(
\textbackslash frac\{1\}\{6\} \textbackslash right)\}\textasciicircum\{2\}\}\}\}\{6\}\textbackslash times
(-2)\$ \textbackslash{[}\textbackslash frac\{\textbackslash text\{d\}x\}\{\textbackslash text\{d\}t\}=-\textbackslash frac\{\textbackslash sqrt\{3\}\}\{2\}\textbackslash times
\textbackslash frac\{1\}\{6\}\textbackslash times (-2)=-\textbackslash frac\{\textbackslash sqrt\{3\}\}\{6\}\textbackslash{]}units/s 

{[}SPLIT\_HERE{]}
\item (a)(i) Year / n Outstanding amount at the end of year 2025 1 \textbackslash{[}\textbackslash begin\{align\}
\& \textbackslash left{[} 40000-12\textbackslash left( 550 \textbackslash right)
\textbackslash right{]}1.04 \textbackslash\textbackslash{} \& =40000\textbackslash left(
1.04 \textbackslash right)-12\textbackslash left( 550 \textbackslash right)\textbackslash left(
1.04 \textbackslash right) \textbackslash\textbackslash{} \textbackslash end\{align\}\textbackslash{]}
2026 2 \textbackslash{[}\textbackslash begin\{align\} \& \textbackslash left{[}
40000\textbackslash left( 1.04 \textbackslash right)-12\textbackslash left(
550 \textbackslash right)\textbackslash left( 1.04 \textbackslash right)-12\textbackslash left(
550 \textbackslash right) \textbackslash right{]}1.04 \textbackslash\textbackslash{}
\& =40000\{\{\textbackslash left( 1.04 \textbackslash right)\}\textasciicircum\{2\}\}-12\textbackslash left(
550 \textbackslash right)\{\{\textbackslash left( 1.04 \textbackslash right)\}\textasciicircum\{2\}\}-12\textbackslash left(
550 \textbackslash right)\textbackslash left( 1.04 \textbackslash right)
\textbackslash\textbackslash{} \textbackslash end\{align\}\textbackslash{]}
2027 3 \textbackslash{[}\textbackslash begin\{align\} \& \textbackslash left{[}
40000\{\{\textbackslash left( 1.04 \textbackslash right)\}\textasciicircum\{2\}\}-12\textbackslash left(
550 \textbackslash right)\{\{\textbackslash left( 1.04 \textbackslash right)\}\textasciicircum\{2\}\}-12\textbackslash left(
550 \textbackslash right)\textbackslash left( 1.04 \textbackslash right)-12(550)
\textbackslash right{]}1.04 \textbackslash\textbackslash{} \& =40000\{\{\textbackslash left(
1.04 \textbackslash right)\}\textasciicircum\{3\}\}-12\textbackslash left(
550 \textbackslash right)\{\{\textbackslash left( 1.04 \textbackslash right)\}\textasciicircum\{3\}\}-12\textbackslash left(
550 \textbackslash right)\{\{\textbackslash left( 1.04 \textbackslash right)\}\textasciicircum\{2\}\}-12\textbackslash left(
550 \textbackslash right)\textbackslash left( 1.04 \textbackslash right)
\textbackslash\textbackslash{} \textbackslash end\{align\}\textbackslash{]}
\dots{}

Amt owe at the end of n years = \textbackslash{[}40000\{\{\textbackslash left(
1.04 \textbackslash right)\}\textasciicircum\{n\}\}-12\textbackslash left(
550 \textbackslash right)\{\{\textbackslash left( 1.04 \textbackslash right)\}\textasciicircum\{n\}\}-12\textbackslash left(
550 \textbackslash right)\{\{\textbackslash left( 1.04 \textbackslash right)\}\textasciicircum\{n-1\}\}-\textbackslash cdots
-12\textbackslash left( 550 \textbackslash right)\textbackslash left(
1.04 \textbackslash right)\textbackslash{]} \textbackslash{[}\textbackslash begin\{align\}
\& =40000\{\{\textbackslash left( 1.04 \textbackslash right)\}\textasciicircum\{n\}\}-12\textbackslash left(
550 \textbackslash right)\textbackslash left{[} 1.04+\textbackslash cdots
+\{\{1.04\}\textasciicircum\{n\}\} \textbackslash right{]} \textbackslash\textbackslash{}
\& =40000\{\{\textbackslash left( 1.04 \textbackslash right)\}\textasciicircum\{n\}\}-171600\textbackslash left(
\{\{1.04\}\textasciicircum\{n\}\}-1 \textbackslash right) \textbackslash\textbackslash{}
\& =171600-131600\{\{\textbackslash left( 1.04 \textbackslash right)\}\textasciicircum\{n\}\}
\textbackslash\textbackslash{} \textbackslash end\{align\}\textbackslash{]}
(ii) At the end of 2030, n = 6 Outstanding amount at end of 2030 \textbackslash{[}\textbackslash begin\{align\}
\& =171600-131600\{\{\textbackslash left( 1.04 \textbackslash right)\}\textasciicircum\{6\}\}
\textbackslash\textbackslash{} \& =5084.02>0 \textbackslash\textbackslash{}
\textbackslash end\{align\}\textbackslash{]} She will not be able
to finish repaying the loan by the end of 2030. 

(iii) Let m be the monthly loan repayment. To utilise fully the loan
repayment period, n = 15 \textbackslash{[}40000\{\{\textbackslash left(
1.04 \textbackslash right)\}\textasciicircum\{15\}\}-12m\textbackslash left{[}
\textbackslash frac\{1.04\textbackslash left( \{\{1.04\}\textasciicircum\{15\}\}-1
\textbackslash right)\}\{1.04-1\} \textbackslash right{]}\textbackslash le
0\textbackslash{]} \textbackslash{[}m\textbackslash ge \textbackslash frac\{40000\{\{\textbackslash left(
1.04 \textbackslash right)\}\textasciicircum\{15\}\}\}\{12\}\textbackslash left{[}
\textbackslash frac\{0.04\}\{1.04\textbackslash left( \{\{1.04\}\textasciicircum\{15\}\}-1
\textbackslash right)\} \textbackslash right{]}=288.27276\textbackslash{]}
\$\textbackslash therefore \$ Minimum monthly repayment = \$288.28 

(b)(i) n Amount at end of month

1 \$200+\textbackslash left( \textbackslash frac\{d\}\{100\} \textbackslash right)\textbackslash left(
200 \textbackslash right)=200+2d\$ 2 \$200+2d+2d=200+2d\textbackslash left(
2 \textbackslash right)\$ 3 \$200+2d\textbackslash left( 3 \textbackslash right)\$
. . n \$200+2d\textbackslash left( n \textbackslash right)\$

At end of 2022, n = 14 \$\textbackslash begin\{align\} \& \textbackslash text\{Interest
\}=2d\textbackslash left( 14 \textbackslash right) \textbackslash\textbackslash{}
\& =28d \textbackslash end\{align\}\$ 

(b)(ii) n Amount at start of month Amount at end of month 1 \$200\$
\$200+2d\$ 2 \$\textbackslash begin\{align\} \& 200+2d+200 \textbackslash\textbackslash{}
\& =2\textbackslash left( 200 \textbackslash right)+2d \textbackslash\textbackslash{}
\textbackslash end\{align\}\$ \$\textbackslash begin\{align\} \&
2\textbackslash left( 200 \textbackslash right)+2d+2d(2) \textbackslash\textbackslash{}
\& =2\textbackslash left( 200 \textbackslash right)+2d(1+2) \textbackslash\textbackslash{}
\textbackslash end\{align\}\$ 3 \$\textbackslash begin\{align\}
\& 2\textbackslash left( 200 \textbackslash right)+2d(1+2)+200 \textbackslash\textbackslash{}
\& =3\textbackslash left( 200 \textbackslash right)+2d(1+2) \textbackslash\textbackslash{}
\textbackslash end\{align\}\$ \$\textbackslash begin\{align\} \&
3\textbackslash left( 200 \textbackslash right)+2d(1+2)+2d(3) \textbackslash\textbackslash{}
\& =3\textbackslash left( 200 \textbackslash right)+2d(1+2+3) \textbackslash\textbackslash{}
\textbackslash end\{align\}\$ .. \dots{} \dots{} n \dots{} \$\textbackslash begin\{align\}
\& n\textbackslash left( 200 \textbackslash right)+2d(1+2+\textbackslash cdots
+n) \textbackslash\textbackslash{} \& =200n+2d\textbackslash left{[}
\textbackslash frac\{n\}\{2\}\textbackslash left( 1+n \textbackslash right)
\textbackslash right{]} \textbackslash\textbackslash{} \& =200n+dn\textbackslash left(
1+n \textbackslash right) \textbackslash\textbackslash{} \textbackslash end\{align\}\$

\$200\textbackslash left( 36 \textbackslash right)+d\textbackslash left(
36 \textbackslash right)\textbackslash left( 1+36 \textbackslash right)>10000\$
\$d>2.1021\$ \$\textbackslash therefore \$ Least d = 2.11 (3sf) 

{[}SPLIT\_HERE{]}
\item Let F be the foot of perpendicular from A to the plane p1. \textbackslash{[}\{\{\textbackslash ell
\}\_\{AF\}\}:\textbackslash text\{r\}=\textbackslash left( \textbackslash begin\{matrix\}
0 \textbackslash\textbackslash{} -1 \textbackslash\textbackslash{}
2 \textbackslash\textbackslash{} \textbackslash end\{matrix\} \textbackslash right)+\textbackslash lambda
\textbackslash left( \textbackslash begin\{matrix\} 1 \textbackslash\textbackslash{}
-1 \textbackslash\textbackslash{} 0 \textbackslash\textbackslash{}
\textbackslash end\{matrix\} \textbackslash right),\textbackslash lambda
\textbackslash in \textbackslash mathbb\{R\}\textbackslash{]}
Since F lies on line AF \textbackslash{[}\textbackslash overrightarrow\{OF\}=\textbackslash left(
\textbackslash begin\{matrix\} 0 \textbackslash\textbackslash{}
-1 \textbackslash\textbackslash{} 2 \textbackslash\textbackslash{}
\textbackslash end\{matrix\} \textbackslash right)+\textbackslash lambda
\textbackslash left( \textbackslash begin\{matrix\} 1 \textbackslash\textbackslash{}
-1 \textbackslash\textbackslash{} 0 \textbackslash\textbackslash{}
\textbackslash end\{matrix\} \textbackslash right)\textbackslash{]}
for some \textbackslash{[}\textbackslash lambda \textbackslash in
\textbackslash mathbb\{R\}\textbackslash{]} As F is also on the
plane p1, \textbackslash{[}\textbackslash left( \textbackslash begin\{matrix\}
\textbackslash lambda \textbackslash\textbackslash{} -1-\textbackslash lambda
\textbackslash\textbackslash{} 2 \textbackslash\textbackslash{}
\textbackslash end\{matrix\} \textbackslash right).\textbackslash left(
\textbackslash begin\{matrix\} 1 \textbackslash\textbackslash{}
-1 \textbackslash\textbackslash{} 0 \textbackslash\textbackslash{}
\textbackslash end\{matrix\} \textbackslash right)=3\textbackslash{]}
\textbackslash{[}\textbackslash begin\{align\} \& \textbackslash lambda
+1+\textbackslash lambda =3 \textbackslash\textbackslash{} \& \textbackslash lambda
=1 \textbackslash\textbackslash{} \textbackslash end\{align\}\textbackslash{]}
\textbackslash{[}\textbackslash overrightarrow\{OF\}=\textbackslash left(
\textbackslash begin\{matrix\} 1 \textbackslash\textbackslash{}
-2 \textbackslash\textbackslash{} 2 \textbackslash\textbackslash{}
\textbackslash end\{matrix\} \textbackslash right)\textbackslash{]}
(ii) Let \textbackslash{[}\textbackslash theta \textbackslash{]}be
the acute angle between the plane p1 and the line l. \textbackslash{[}\textbackslash sin
\textbackslash theta =\textbackslash frac\{\textbackslash left|
\textbackslash left( \textbackslash begin\{matrix\} 1 \textbackslash\textbackslash{}
-1 \textbackslash\textbackslash{} 0 \textbackslash\textbackslash{}
\textbackslash end\{matrix\} \textbackslash right)\textbackslash bullet
\textbackslash left( \textbackslash begin\{matrix\} 0 \textbackslash\textbackslash{}
1 \textbackslash\textbackslash{} 1 \textbackslash\textbackslash{}
\textbackslash end\{matrix\} \textbackslash right) \textbackslash right|\}\{\textbackslash sqrt\{2\}\textbackslash sqrt\{2\}\}\textbackslash{]}
\textbackslash{[}\textbackslash theta =\{\{30\}\textasciicircum\{\textbackslash circ
\}\}\textbackslash{]} (iii) Let point D and E be \textbackslash{[}\textbackslash left(
0,-1,2 \textbackslash right)\textbackslash{]}and \textbackslash{[}\textbackslash left(
3,0,0 \textbackslash right)\textbackslash{]}respectively. \textbackslash{[}\textbackslash frac\{\textbackslash left|
\textbackslash overrightarrow\{BE\}\textbackslash bullet \{\{\{\textbackslash underset\{\textbackslash scriptscriptstyle\textbackslash thicksim\}\{n\}\}\}\_\{1\}\}
\textbackslash right|\}\{\textbackslash left| \{\{\{\textbackslash underset\{\textbackslash scriptscriptstyle\textbackslash thicksim\}\{n\}\}\}\_\{1\}\}
\textbackslash right|\}=\textbackslash frac\{\textbackslash left|
\textbackslash overrightarrow\{BD\}\textbackslash times \{\{\{\textbackslash underset\{\textbackslash scriptscriptstyle\textbackslash thicksim\}\{d\}\}\}\_\{\textbackslash ell
\}\} \textbackslash right|\}\{\textbackslash left| \{\{\{\textbackslash underset\{\textbackslash scriptscriptstyle\textbackslash thicksim\}\{d\}\}\}\_\{\textbackslash ell
\}\} \textbackslash right|\}\textbackslash{]} \textbackslash{[}\textbackslash frac\{\textbackslash left|
\textbackslash left( \textbackslash begin\{matrix\} 3+\textbackslash alpha
\textbackslash\textbackslash{} -2 \textbackslash\textbackslash{}
-\textbackslash alpha \textbackslash\textbackslash{} \textbackslash end\{matrix\}
\textbackslash right)\textbackslash bullet \textbackslash left(
\textbackslash begin\{matrix\} 1 \textbackslash\textbackslash{}
-1 \textbackslash\textbackslash{} 0 \textbackslash\textbackslash{}
\textbackslash end\{matrix\} \textbackslash right) \textbackslash right|\}\{\textbackslash sqrt\{2\}\}=\textbackslash frac\{\textbackslash left|
\textbackslash left( \textbackslash begin\{matrix\} \textbackslash alpha
\textbackslash\textbackslash{} -3 \textbackslash\textbackslash{}
2-\textbackslash alpha \textbackslash\textbackslash{} \textbackslash end\{matrix\}
\textbackslash right)\textbackslash times \textbackslash left(
\textbackslash begin\{matrix\} 0 \textbackslash\textbackslash{}
1 \textbackslash\textbackslash{} 1 \textbackslash\textbackslash{}
\textbackslash end\{matrix\} \textbackslash right) \textbackslash right|\}\{\textbackslash sqrt\{2\}\}\textbackslash{]}
\textbackslash{[}\textbackslash left| 3+\textbackslash alpha +2
\textbackslash right|=\textbackslash left| \textbackslash left(
\textbackslash begin\{matrix\} \textbackslash alpha -5 \textbackslash\textbackslash{}
-\textbackslash alpha \textbackslash\textbackslash{} \textbackslash alpha
\textbackslash\textbackslash{} \textbackslash end\{matrix\} \textbackslash right)
\textbackslash right|\textbackslash{]} \textbackslash{[}\textbackslash left|
\textbackslash alpha +5 \textbackslash right|=\textbackslash sqrt\{2\{\{\textbackslash alpha
\}\textasciicircum\{2\}\}+\{\{\textbackslash left( \textbackslash alpha
-5 \textbackslash right)\}\textasciicircum\{2\}\}\}\textbackslash{]}
\textbackslash{[}\{\{\textbackslash alpha \}\textasciicircum\{2\}\}+10\textbackslash alpha
+25=3\{\{\textbackslash alpha \}\textasciicircum\{2\}\}-10\textbackslash alpha
+25\textbackslash{]} \textbackslash{[}\textbackslash alpha \textbackslash left(
\textbackslash alpha -10 \textbackslash right)=0\textbackslash{]}
\textbackslash{[}\textbackslash therefore \textbackslash alpha
=0\textbackslash text\{ \}\textbackslash mathrm\{or \}10\textbackslash{]}
(iv) \textbackslash{[}\textbackslash left( \textbackslash begin\{matrix\}
4 \textbackslash\textbackslash{} 1 \textbackslash\textbackslash{}
\textbackslash beta \textbackslash\textbackslash{} \textbackslash end\{matrix\}
\textbackslash right)\textbackslash bullet \textbackslash left(
\textbackslash begin\{matrix\} 1 \textbackslash\textbackslash{}
-1 \textbackslash\textbackslash{} 0 \textbackslash\textbackslash{}
\textbackslash end\{matrix\} \textbackslash right)=4\textbackslash left(
1 \textbackslash right)+1\textbackslash left( -1 \textbackslash right)+\textbackslash beta
\textbackslash left( 0 \textbackslash right)=3\textbackslash{]}
And \textbackslash{[}\textbackslash left( \textbackslash begin\{matrix\}
4 \textbackslash\textbackslash{} 1 \textbackslash\textbackslash{}
\textbackslash beta \textbackslash\textbackslash{} \textbackslash end\{matrix\}
\textbackslash right)\textbackslash bullet \textbackslash left(
\textbackslash begin\{matrix\} 1 \textbackslash\textbackslash{}
1 \textbackslash\textbackslash{} -1 \textbackslash\textbackslash{}
\textbackslash end\{matrix\} \textbackslash right)=4\textbackslash left(
1 \textbackslash right)+1\textbackslash left( 1 \textbackslash right)+\textbackslash beta
\textbackslash left( -1 \textbackslash right)=5-\textbackslash beta
\textbackslash{]} Hence C lies on both p1 and p2. \textbackslash{[}\textbackslash left(
\textbackslash begin\{matrix\} 1 \textbackslash\textbackslash{}
-1 \textbackslash\textbackslash{} 0 \textbackslash\textbackslash{}
\textbackslash end\{matrix\} \textbackslash right)\textbackslash times
\textbackslash left( \textbackslash begin\{matrix\} 1 \textbackslash\textbackslash{}
1 \textbackslash\textbackslash{} -1 \textbackslash\textbackslash{}
\textbackslash end\{matrix\} \textbackslash right)=\textbackslash left(
\textbackslash begin\{matrix\} 1 \textbackslash\textbackslash{}
1 \textbackslash\textbackslash{} 2 \textbackslash\textbackslash{}
\textbackslash end\{matrix\} \textbackslash right)\textbackslash{]}
Eqn of the line of intersection is \textbackslash{[}\textbackslash text\{r\}=\textbackslash left(
\textbackslash begin\{matrix\} 4 \textbackslash\textbackslash{}
1 \textbackslash\textbackslash{} \textbackslash beta \textbackslash\textbackslash{}
\textbackslash end\{matrix\} \textbackslash right)+\textbackslash gamma
\textbackslash left( \textbackslash begin\{matrix\} 1 \textbackslash\textbackslash{}
1 \textbackslash\textbackslash{} 2 \textbackslash\textbackslash{}
\textbackslash end\{matrix\} \textbackslash right),\textbackslash gamma
\textbackslash in \textbackslash mathbb\{R\}\textbackslash{]} 
\end{enumerate}

\end{document}
