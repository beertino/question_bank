\item \textbf{{[}ASRJC/PROMO/9758/2021/Q11{]}}
\begin{enumerate}
\item Ivy took a \$40 000 tuition fee loan for her 4-year university course
that commences on 1st January 2021. The loan is interest-free during
the period of study. Immediately after graduation, interest is charged
at 4\% per annum of the outstanding amount owe at the end of each
year. The maximum loan repayment period is at most 15 years upon graduation.
Ivy took a 550 every month upon graduation.
\begin{enumerate}
\item Show that the amount she owes at the end of the n years after graduation
is $\$171600-131600\left(1.04\right)^{n}$. \hfill{}{[}3{]}
\item Will she be able to finish repaying the loan by the end of 2030? Justify
your answer clearly. \hfill{}{[}2{]}
\item Find the minimum monthly repayment Ivy should make if she intends
to utilize fully the loan repayment period. \hfill{} {[}2{]}
\end{enumerate}
\item To save for her tuition fee loan repayment, Ivy wishes to start a
new savings plan on the first day of November 2021. In this plan,
she needs to invest \$200 into the account on the first day of each
month. Every \$200 invested earns a fixed interest of $d$\% of \$200
at the end of each month until a withdrawal is made from the account.
The interest is added to the account and does not accumulate further
interest. 
\begin{enumerate}
\item How much interest, in terms of $d$, will the first \$200 deposited
earn at the end of 2022? \hfill{}{[}2{]}
\item Find the least value of $d$ such that the total amount in the account
exceed \$10 000 at the end of 36 months. \hfill{} {[}3{]}
\end{enumerate}
\end{enumerate}