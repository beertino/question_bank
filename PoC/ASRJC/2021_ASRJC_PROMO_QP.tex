%% LyX 2.3.6.1 created this file.  For more info, see http://www.lyx.org/.
%% Do not edit unless you really know what you are doing.
\documentclass[english]{article}
\usepackage[T1]{fontenc}
\usepackage[latin9]{inputenc}
\usepackage{amsmath}
\usepackage{amssymb}
\usepackage{babel}
\begin{document}
{[}SPLIT\_HERE{]}
\begin{enumerate}
\item \textbf{{[}ASRJC/PROMO/9758/2021/Q1{]} }

It is given that $y^{2}=\sin x+\cos x$. Show that$y\frac{\text{d}^{3}y}{\text{d}x^{3}}+A\frac{\text{d}y}{\text{d}x}\frac{\text{d}^{2}y}{\text{d}x^{2}}+y\frac{\text{d}y}{\text{d}x}=0$,
where $A$ is a real constant to be determined. \hfill{}{[}4{]}

{[}SPLIT\_HERE{]}
\item \textbf{{[}ASRJC/PROMO/9758/2021/Q2{]} }

A graph with the equation $y=\frac{3}{x^{2}+2}$ undergoes, in succession,
the following transformations: 

$A$: A translation of 1 unit in the direction of the negative $x$-axis. 

$B$: A scaling parallel to the $x$-axis by a scale factor of $\frac{1}{3}$. 

$C$: A reflection in the $y$-axis. 

Determine the equation of the resulting curve. \hfill{}{[}4{]}

{[}SPLIT\_HERE{]}
\item \textbf{{[}ASRJC/PROMO/9758/2021/Q3{]}}
\begin{enumerate}
\item Find $\int x^{2}\tan^{-1}\left(2x\right)\text{d}x$. \hfill{}{[}3{]}
\item Find $\int\frac{x}{\sqrt{8-2x-x^{2}}}\text{d}x$. \hfill{}{[}3{]}
\end{enumerate}
{[}SPLIT\_HERE{]}
\item \textbf{{[}ASRJC/PROMO/9758/2021/Q4{]} }

A curve C has equation $y^{3}-2xy^{2}+3x^{2}-3=0$. 
\begin{enumerate}
\item[(i)]  Find $\frac{\text{d}y}{\text{d}x}$ in terms of $x$ and $y$. \hfill{}{[}2{]}
\item[(ii)]  Find the equation of the normal to the curve at the point $P(2,3)$.
\hfill{}{[}2{]}
\item[(iii)]  Given that $C$ meets the $y$-axis at the point $R$ and the normal
in (ii) meets the y-axis at the point $A$, find the area of triangle
$APR$ in the form $a-\sqrt[3]{b}$, where $a$ and $b$ are integers
to be determined. \hfill{}{[}3{]}
\end{enumerate}
{[}SPLIT\_HERE{]}
\item \textbf{{[}ASRJC/PROMO/9758/2021/Q5{]} }
\noindent \begin{center}
<INSERT\_IMAGE\_HERE>
\par\end{center}

The diagram above shows the graph of $y=\text{f}(x)$. It has a maximum
point at $(4,a)$, where $a>0$, and meets the axes at $(1,0)$ and
$(0,0.5)$. The curve has asymptotes with equations $y=0$ and $x=-1$. 

On separate diagrams, sketch the graphs of 
\begin{enumerate}
\item $y=\frac{1}{\text{f}\left(x\right)}$\} ; \hfill{}{[}3{]}
\item $y=\text{ f }'\left(x\right)$,\hfill{} {[}3{]}
\end{enumerate}
stating the equation(s) of any asymptotes and where possible, the
coordinates of any turning point(s) and axial intercept(s).

{[}SPLIT\_HERE{]}
\item \textbf{{[}ASRJC/PROMO/9758/2021/Q6{]} }

The functions $\text{f}$ and $\text{g}$ are defined by 
\begin{align*}
\ensuremath{\text{f}:x} & \mapsto6+\lambda x-x^{2} & \text{, }x\in\mathbb{R},\\
\ensuremath{\text{g}:x} & \mapsto1+7\text{e}^{-x} & ,\ensuremath{x\ge0}.
\end{align*}

\begin{enumerate}
\item[(i)]  For this part of the question, it is given that $\lambda=5$. Show
that the composite function fg exists. Hence find the range of fg.
\hfill{} {[}3{]}
\item[(ii)]  It is now given that $\lambda$ is a real constant that is not necessary
equals to 5. If the domain of f is restricted to $x\le\frac{\lambda}{2}$,
find $\text{f}^{-1}$ in a similar form.\hfill{} {[}4{]}
\end{enumerate}
{[}SPLIT\_HERE{]}
\item \textbf{{[}ASRJC/PROMO/9758/2021/Q7{]} }
\begin{enumerate}
\item Find $\frac{\text{d}}{\text{d}x}\left(\tan^{3}x\right)$. \hfill{}{[}1{]}

Hence find $\int\text{sec}^{4}x\,\text{d}x$. \hfill{}{[}2{]}
\item Find $\int\sin x\left(\sin x+\sin3x\right)\text{d}x$. \hfill{} {[}3{]}
\item For $p>2$, find the value of $\int_{0}^{p}x\left|2-x\right|\text{ d}x$
in terms of $p$. \hfill{}{[}3{]}
\end{enumerate}
{[}SPLIT\_HERE{]}
\item \textbf{{[}ASRJC/PROMO/9758/2021/Q8{]} }

Relative to origin $O$, the points $A$ and $B$ have position vectors
$\mathbf{a}$ and $\mathbf{b}$ respectively, where $\mathbf{a}$
and $\mathbf{b}$ are non-zero vectors. The point $C$ is on $BA$
produced such that $BA:BC=3:5$ and $OC$ is perpendicular to $OB$. 
\begin{enumerate}
\item[(i)]  Find $\overrightarrow{OC}$ in terms of $\mathbf{a}$ and $\mathbf{b}$.\hfill{}
{[}1{]}
\item[(ii)]  Show that $\mathbf{a}\cdot\mathbf{b}=\frac{2}{5}\left|\mathbf{b}\right|^{2}$.\hfill{}
{[}2{]}
\end{enumerate}
The point $P$ is on the line $OB$ such that it is the image of $B$
in the line $OC$. 
\begin{enumerate}
\item[(iii)]  Find the area of triangle $PCB$. Leave your answer in the form
of $k\left|\mathbf{a}\times\mathbf{b}\right|$, where $k$ is an exact
real constant. \hfill{}{[}3{]}
\end{enumerate}
The point $F$ is the foot of perpendicular of $P$ to the line $AB$. 
\begin{enumerate}
\item[(iv)]  Given that $\left|\mathbf{a}\right|^{2}=\frac{7}{25}\left|\mathbf{b}\right|^{2}$,
find the position vector of $F$ in terms of $\mathbf{a}$ and $\mathbf{b}$.
\hfill{}{[}4{]}
\end{enumerate}
{[}SPLIT\_HERE{]}
\item \textbf{{[}ASRJC/PROMO/9758/2021/Q9{]} }

The curve C has equation 

\[
y=\frac{4x^{2}+px-q}{x^{2}-s},\text{ }x\in\mathbb{R},\text{ }x^{2}\ne s
\]

where $p$, $q$ and $s$ are non-zero constants. 
\begin{enumerate}
\item It is given that $C$ passes through the point $\left(0,\frac{1}{2}\right)$
and has a vertical asymptote $x=2$. 
\begin{enumerate}
\item State the value of $s$ and show that the value of $q$ is 2. \hfill{}{[}2{]}
\item It is given further that the line $y=1$ is a tangent to $C$ and
it does not meet the curve again. Find the exact value of $p$ if
$p$ is a negative real value. \hfill{}{[}3{]}
\end{enumerate}
\item It is now given instead that $p=4,\text{ }q=-1\text{ and }s=1$. 
\begin{enumerate}
\item Sketch the curve $C$, showing clearly the coordinates of any turning
point(s), equations of any asymptotes and the coordinates of any points
of intersection with the axes. \hfill{}{[}3{]}
\item Find the equation of the additional curve that needs to be added to
the curve sketched in (b)(i) to determine the number of distinct real
roots for the equation $10(x+2)^{2}=3\left(10-\left(\frac{(4x^{2}+4x+1)}{(x^{2}-1)}\right)^{2}\right)$.
\hfill{}{[}2{]}
\end{enumerate}
\end{enumerate}
{[}SPLIT\_HERE{]}
\item \textbf{{[}ASRJC/PROMO/9758/2021/Q10{]} }
\begin{enumerate}
\item {}
\noindent \begin{center}
<INSERT\_IMAGE\_HERE>
\par\end{center}

Fig. 1 shows the cylindrical-shaped water pipe, with negligible thickness
and open on both ends, inscribed in a hemisphere with fixed radius
8 cm. The cross sectional view of the pipe and the hemisphere is shown
in Fig. 2. 
\begin{enumerate}
\item If the diameter of the pipe is $x$ cm, show that the curved surface
area, S, of the pipe is $2\pi x\sqrt{64-x^{2}}$ $\text{cm}^{2}$.
\hfill{}{[}2{]}
\item It is given that as $x$ varies, the maximum value of $S$ occurs
when the ratio of the diameter of the pipe to its height is $1:k$.
Find the exact value of $k$ and the exact maximum value of $S$.\hfill{}
{[}6{]}
\end{enumerate}
\item A particle is moving on the curve with equation $y=2\sin^{-1}(3x),-\frac{1}{3}\le x\le\frac{1}{3},$
where $(x,y)$ is the coordinate of the particle at time $t$ relative
to a fixed point $O$. The $x$ and $y$ values represent the horizontal
and vertical displacements. When the $y$-coordinate of the particle
is $\frac{\pi}{3}$, the rate at which the $y$-coordinate is decreasing
with time $t$ is 2 units per second. At this instant, find the exact
rate at which the $x$-coordinate of the particle changes with time.
\hfill{} {[}4{]}
\end{enumerate}
{[}SPLIT\_HERE{]}
\item \textbf{{[}ASRJC/PROMO/9758/2021/Q11{]}}
\begin{enumerate}
\item Ivy took a \$40 000 tuition fee loan for her 4-year university course
that commences on 1st January 2021. The loan is interest-free during
the period of study. Immediately after graduation, interest is charged
at 4\% per annum of the outstanding amount owe at the end of each
year. The maximum loan repayment period is at most 15 years upon graduation.
Ivy took a 550 every month upon graduation.
\begin{enumerate}
\item Show that the amount she owes at the end of the n years after graduation
is $\$171600-131600\left(1.04\right)^{n}$. \hfill{}{[}3{]}
\item Will she be able to finish repaying the loan by the end of 2030? Justify
your answer clearly. \hfill{}{[}2{]}
\item Find the minimum monthly repayment Ivy should make if she intends
to utilize fully the loan repayment period. \hfill{} {[}2{]}
\end{enumerate}
\item To save for her tuition fee loan repayment, Ivy wishes to start a
new savings plan on the first day of November 2021. In this plan,
she needs to invest \$200 into the account on the first day of each
month. Every \$200 invested earns a fixed interest of $d$\% of \$200
at the end of each month until a withdrawal is made from the account.
The interest is added to the account and does not accumulate further
interest. 
\begin{enumerate}
\item How much interest, in terms of $d$, will the first \$200 deposited
earn at the end of 2022? \hfill{}{[}2{]}
\item Find the least value of $d$ such that the total amount in the account
exceed \$10 000 at the end of 36 months. \hfill{} {[}3{]}
\end{enumerate}
\end{enumerate}
{[}SPLIT\_HERE{]}
\item \textbf{{[}ASRJC/PROMO/9758/2021/Q12{]}}
\begin{enumerate}
\item Relative to the origin $O$, a point $A$ has position vector $-\mathbf{j}+2\mathbf{k}$.
The plane $p_{1}$ has equation $\mathbf{r}\cdot\left(\begin{array}{c}
1\\
-1\\
0
\end{array}\right)=3$. 
\item[(i)]  Find the position vector of the foot of perpendicular from point
$A$ to the plane $p_{1}$.\hfill{} {[}4{]}
\end{enumerate}
The line $l$ has equation $\mathbf{r}=\left(\begin{array}{c}
0\\
-1\\
2
\end{array}\right)+\mu\left(\begin{array}{c}
0\\
1\\
1
\end{array}\right)$, $\mu\in\mathbb{R}$.
\begin{enumerate}
\item[(ii)]  Find the acute angle between the plane $p_{1}$ and the line $l$.
\hfill{}{[}2{]}
\item[(iii)]  The point $B\left(-\alpha,2,\alpha\right)$ is equidistant from
the plane $p_{1}$ and the line $l$. Find the possible values of
$\alpha$. {[}4{]}
\end{enumerate}
The plane $p_{2}$ has equation $\mathbf{r}\cdot\left(\begin{array}{c}
1\\
1\\
-1
\end{array}\right)=5-\beta$ , $\beta\ne8$. 
\begin{enumerate}
\item[(iv)]  Show that the point $C\left(4,1,\beta\right)$ lies on both $p_{1}$
and $p_{2}$. Hence find the vector equation of the line of intersection
between the planes $p_{1}$ and $p_{2}$.\hfill{} {[}3{]}
\end{enumerate}
\end{enumerate}

\end{document}
