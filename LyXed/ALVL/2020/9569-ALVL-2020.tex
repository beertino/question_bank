%% LyX 2.3.6.1 created this file.  For more info, see http://www.lyx.org/.
%% Do not edit unless you really know what you are doing.
\documentclass[english]{article}
\usepackage[T1]{fontenc}
\usepackage[latin9]{inputenc}
\usepackage{geometry}
\geometry{verbose,tmargin=2.5cm,bmargin=2.5cm,lmargin=2.5cm,rmargin=2.5cm}
\usepackage{array}
\usepackage{float}
\usepackage{multirow}
\usepackage{amstext}
\usepackage{setspace}

\makeatletter

%%%%%%%%%%%%%%%%%%%%%%%%%%%%%% LyX specific LaTeX commands.
%% Because html converters don't know tabularnewline
\providecommand{\tabularnewline}{\\}
\floatstyle{ruled}
\newfloat{algorithm}{tbp}{loa}
\providecommand{\algorithmname}{Algorithm}
\floatname{algorithm}{\protect\algorithmname}

\makeatother

\usepackage{babel}
\begin{document}
{[}SPLIT\_HERE{]}
\begin{enumerate}
\item \textbf{{[}ALVL/9569/2020/P1/Q1{]} }

A software company is writing a program for a vehicle hire business.
Both cars and vans are available for hire. 

For all vehicles, the data that will be stored include: 

\begin{singlespace}
~~~~Vehicle Registration Number (VRN) 

~~~~Model 

~~~~Total distance travelled (km) 

~~~~Date hired 

~~~~Date of return 

~~~~Cost per day 

~~~~Available for hire 
\end{singlespace}

For cars. the additional data stored include: 

\begin{singlespace}
~~~~Number of seats 

~~~~Fuel type (petrol, diesel. electric. hybrid) 
\end{singlespace}

For vans. the additional data stored include: 

\begin{singlespace}
~~~~Load volume ($\text{m}^{3}$) 

~~~~Maximum load (kg) 
\end{singlespace}

The odometer in the vehicle displays the total distance the vehicle
has travelled since manufacture. 

When a vehicle is hired: 
\begin{itemize}
\begin{singlespace}
\item total distance travelled is set to the odometer's value
\item date hired is set to the current date
\item return date is set to the date the vehicle is expected to be returned 
\item available for hire is set to \texttt{FALSE}. 
\end{singlespace}
\end{itemize}
When a vehicle is returned: 
\begin{itemize}
\begin{singlespace}
\item hire cost is returned as the cost per day multiplied by the number
of days the vehicle was hired 
\item total distance travelled is set to the odometer value 
\item date returned is set to the current date 
\item available for hire is set to \texttt{TRUE}.
\end{singlespace}
\end{itemize}
Object-oriented programming will be used to model vehicles. 
\begin{enumerate}
\item Draw a class diagram that shows the following for the situation described
above. 
\begin{itemize}
\begin{singlespace}
\item the superclass
\item any subclasses
\item inheritance 
\item properties 
\item appropriate methods \hfill{} {[}12{]}
\end{singlespace}
\end{itemize}
\item State the purpose of a superciass. Give an example of a superciass
from the vehicle hire example. \hfill{} {[}2{]}
\end{enumerate}
Objects provide encapsulation of properties and methods.
\begin{enumerate}
\item[\textbf{(c)}] State the purposes of encapsulation. \hfill{} {[}2{]}
\end{enumerate}
The business wants to change the way the hire cost is calculated for
a car. As well as charging per day, an additional charge of \$0.05
is to be made per km travelled during this hire.
\begin{enumerate}
\item[\textbf{(d)}] Suggest a change to the class diagram to enable the new charging
scheme to be used for cars. \hfill{} {[}1{]}
\item[\textbf{(e)}] State the purpose of polymorphism. \hfill{} {[}1{]}
\end{enumerate}
{[}SPLIT\_HERE{]}
\item \textbf{{[}ALVL/9569/2020/P1/Q2{]} }

Quicksort is an algorithm to arrange data items into ascending or
descending order. The algorithm selects a pivot from the data set.
The data set is divided into two subsets around the pivot.
\begin{enumerate}
\item {}
\begin{enumerate}
\item State the ideal pivot for the quicksort algorithm to execute most
efficiently. \hfill{}{[}1{]} 
\item State the difficulty in locating this ideal pivot. \hfill{}{[}1{]}
\end{enumerate}
\end{enumerate}
Sometimes the item in the first or last position in the data set is
used as the pivot. An alternative is to pick the pivot at random.

A given data set is largely sorted.
\begin{enumerate}
\item[\textbf{(b)}]  Explain what advantage random selection has over selecting the item
in the first or last position. \hfill{}{[}2{]}
\item[\textbf{(c)}]  Explain why a programmer might choose to use an insertion sort rather
than quicksort in this situation. \hfill{}{[}4{]}
\end{enumerate}
{[}SPLIT\_HERE{]}
\item \textbf{{[}ALVL/9569/2020/P1/Q3{]} }

A binary search tree is implemented using an array, \texttt{b\_tree}.
Each element of the array comprises three parts: \texttt{l\_ptr} and
\texttt{r\_ptr} are of data type integer and \texttt{data\_item} is
of data type char. 
\begin{center}
\begin{tabular}{|c|c|c|}
\hline 
\texttt{l\_ptr} & \texttt{data\_item} & \texttt{r\_ptr}\tabularnewline
\hline 
\end{tabular}
\par\end{center}

The root of the binary search tree is stored in an integer variable,
\texttt{root}. The contents of \texttt{b\_tree} is shown below. $-1$
represents the null pointer. 
\begin{center}
\begin{tabular}{c|c|c|c|}
\cline{2-4} \cline{3-4} \cline{4-4} 
Index & \texttt{l\_ptr} & \texttt{data\_item} & \texttt{r\_ptr}\tabularnewline
\cline{2-4} \cline{3-4} \cline{4-4} 
$0$ & $-1$ & A & $-1$\tabularnewline
\cline{2-4} \cline{3-4} \cline{4-4} 
$1$ & $0$ & + & $2$\tabularnewline
\cline{2-4} \cline{3-4} \cline{4-4} 
$2$ & $-1$ & B & $-1$\tabularnewline
\cline{2-4} \cline{3-4} \cline{4-4} 
$3$ & $1$ & $\star$ & $5$\tabularnewline
\cline{2-4} \cline{3-4} \cline{4-4} 
$4$ & $-1$ & C & $-1$\tabularnewline
\cline{2-4} \cline{3-4} \cline{4-4} 
$5$ & $4$ & $-$ & $6$\tabularnewline
\cline{2-4} \cline{3-4} \cline{4-4} 
$6$ & $-1$ & D & $-1$\tabularnewline
\cline{2-4} \cline{3-4} \cline{4-4} 
\end{tabular}\qquad{}\qquad{} %
\begin{tabular}{|c|}
\hline 
\texttt{root}\tabularnewline
\hline 
\hline 
$3$\tabularnewline
\hline 
\end{tabular}
\par\end{center}
\begin{enumerate}
\item Draw the binary search tree represented by the array \texttt{b\_tree}.\hfill{}
{[}2{]}
\item State the index of a leaf node in \texttt{b\_tree}.\hfill{} {[}1{]}
\end{enumerate}
A procedure, \texttt{P}, uses recursion.

\begin{algorithm}[H]
\texttt{01 PROCEDURE P (Index: INTEGER)}

\texttt{02 ~~IF b\_tree{[}Index{]}.l\_ptr <> -1 THEN}

\texttt{03 ~~~~~P(b\_tree{[}lndex{]}.1\_pt.r)}

\texttt{04 ~~ENDIF}

\texttt{05 ~~IF b\_tree{[}Index{]}.r\_ptr <> \textemdash 1 THEN}

\texttt{O6 ~~~~~P(b\_tree{[}lndex{]}.r\_ptr)}

\texttt{07 ~~ENDIF}

\texttt{08 ~~OUTPUT b\_tree{[}Index{]}.data\_item}

\texttt{09 ENDPROCEDURE}
\end{algorithm}

\begin{enumerate}
\item[\textbf{(c)}] {}
\begin{enumerate}
\item State what is meant by \textbf{recursion}. \hfill{}{[}1{]} 
\item State the line numbers that indicate procedure \texttt{P} is recursive.
\hfill{}{[}1{]} 
\item State the significance of lines \texttt{02} and \texttt{05}. \hfill{}{[}1{]} 
\end{enumerate}
\end{enumerate}
Procedure \texttt{P} is called with the parameter value of 1. 
\begin{enumerate}
\item[\textbf{(d)}] Copy and then complete the trace table for procedure \texttt{P} showing
the values of \texttt{Index} and the output. 
\begin{center}
\begin{tabular}{|c|c|}
\hline 
\texttt{Index} & Output\tabularnewline
\hline 
\hline 
\texttt{1} & \tabularnewline
\hline 
 & \tabularnewline
\end{tabular}
\par\end{center}

\hfill{}{[}5{]}
\item[\textbf{(e)}] Explain the use of a stack when the recursive procedure \texttt{P}
executes. \hfill{}{[}3{]}
\item[\textbf{(f)}] Identify the type of tree traversal that procedure \texttt{P} performs.
\hfill{}{[}1{]}
\end{enumerate}
A 1-dimensional array is used to hold a queue. 
\begin{enumerate}
\item[\textbf{(g)}] Explain the concept of a queue. \hfill{}{[}2{]}
\end{enumerate}
Queues can be either linear or circular. 
\begin{enumerate}
\item[\textbf{(h)}] State \textbf{two} differences between a circular queue and a linear
queue.\hfill{} {[}2{]}
\end{enumerate}
{[}SPLIT\_HERE{]}
\item \textbf{{[}ALVL/9569/2020/P1/Q4{]} }

A computer on LAN A wants to send data to a computer on a remote LAN
B. 

The internet is used to provide a data path between the two LANs. 
\begin{enumerate}
\item {}
\begin{enumerate}
\item State \textbf{two} ways that a particular device can be identified
on a LAN. \hfill{}{[}2{]}
\item State \textbf{two} reasons why LANs need communication protocols.
\hfill{}{[}2{]}
\end{enumerate}
\end{enumerate}
IP is the protocol used to transfer packets of data between hosts
and routers on the intemet. 

The intemet is a packet-switched network. 
\begin{enumerate}
\item[\textbf{(b)}] {}
\begin{enumerate}
\item Explain the term \textbf{packet-switching}. \hfill{}{[}3{]}
\item Describe a disadvantage of packet-switching and how the problem can
be handled. \hfill{}{[}3{]}
\item State how a packet-switched network can cope with a broken cable on
part of the network. \hfill{}{[}2{]}
\end{enumerate}
\end{enumerate}
When using a web browser. most users do not know the IP address of
the server hosting the desired web page. So users enter the domain
name instead. which the browser sends to a local domain name server
(DNS). 
\begin{enumerate}
\item[\textbf{(c)}] Describe the actions that would be carried out by the local DNS on
receiving this request.\hfill{} {[}4{]}
\item[\textbf{(d)}] State the security feature that can be used as a precautionary measure
when sensitive data is sent across a network in each of the following
situations: 
\begin{enumerate}
\item No one other than the intended recipient of the message should be
able to read it.\hfill{} {[}1{]}
\item The intended recipient must be confident that the message is from
the identified sender. \hfill{}{[}1{]}
\end{enumerate}
\end{enumerate}
{[}SPLIT\_HERE{]}
\item \textbf{{[}ALVL/9569/2020/P1/Q5{]} }

Validation and verification are used in data entry.
\begin{enumerate}
\item {}
\begin{enumerate}
\item State the purpose of verification. \hfill{} {[}1{]}
\item State \textbf{one} method of verification. \hfill{} {[}1{]}
\end{enumerate}
\end{enumerate}
The use of check digits is one validation technique.
\begin{enumerate}
\item[\textbf{(b)}] {}
\begin{enumerate}
\item State the purpose of validation. \hfill{}{[}1{]}
\item State \textbf{three} other validation techniques. \hfill{}{[}3{]}
\item Name \textbf{two} types of error that check digits usually detect.
\hfill{}{[}2{]}
\end{enumerate}
\end{enumerate}
A check digit is to be added to the end of 02757 using Modulus 11.
The weight of each digit, starting with the first digit (0) is 6,
5, 4, 3, 2.
\begin{enumerate}
\item[\textbf{(c)}] Showing your working. determine the check digit for 02757. \hfill{}{[}3{]}
\item[\textbf{(d)}] Give \textbf{two} reasons why the data type of a field storing 02757
with a Modulus 11 check digit should be a string rather than an integer.
\hfill{} {[}2{]}
\end{enumerate}
{[}SPLIT\_HERE{]}
\item \textbf{{[}ALVL/9569/2020/P1/Q6{]} }

A college is designing a database to store data about:
\begin{itemize}
\begin{singlespace}
\item students
\item courses
\item subjects
\item teachers
\item classrooms.
\end{singlespace}
\end{itemize}
The designers are told that:
\begin{itemize}
\begin{singlespace}
\item each student takes four courses
\item each teacher delivers all their lessons in one room 
\item each room may be used by more than one teacher 
\item each teacher may teach more than one course
\item a course can only be taught by one teacher
\item a subject may be taught by more than one teacher. 
\end{singlespace}
\end{itemize}
A first attempt is represented by the following table:

\begin{tabular}{|c|c|c|c|c|c|c|c|}
\hline 
\multirow{2}{*}{\textbf{Student ID}} & \multirow{2}{*}{\textbf{First Name}} & \multirow{2}{*}{\textbf{Last Name}} & \textbf{Course} & \multirow{2}{*}{\textbf{Subject}} & \textbf{Teacher} & \textbf{Teacher} & \textbf{Room}\tabularnewline
 &  &  & \textbf{ID} &  & \textbf{ID} & \textbf{Name} & \textbf{Number}\tabularnewline
\hline 
\hline 
1279 & Joe & Smith & 934 & Geography & 334 & Mansoor & 12\tabularnewline
\hline 
 &  &  & 926 & Maths & 451 & Yang & 16\tabularnewline
\hline 
 &  &  & 882 & Physics & 628 & Lee & 12\tabularnewline
\hline 
 &  &  & 425 & Computing & 329 & James & 14\tabularnewline
\hline 
1395 & Muhammad & Hilmi & 934 & Geography & 334 & Mansoor & 12\tabularnewline
\hline 
 &  &  & 927 & Maths & 723 & Morris & 8\tabularnewline
\hline 
 &  &  & 883 & Physics & 534 & Weston & 10\tabularnewline
\hline 
 &  &  & 586 & French & 271 & Dubois & 16\tabularnewline
\hline 
2883 & Sumiko & Chong & 425 & Computing & 329 & James & 14\tabularnewline
\hline 
 &  &  & 882 & Physics & 628 & Lee & 12\tabularnewline
\hline 
 &  &  & 934 & Geography & 334 & Mansoor & 12\tabularnewline
\hline 
 &  &  & 586 & French & 271 & Dubois & 16\tabularnewline
\hline 
\end{tabular}
\begin{enumerate}
\item Explain why this table is not in first normal form (1NF). \hfill{}{[}2{]}
\end{enumerate}
The following is an attempt to reduce data redundancy. 

Student

\begin{tabular}{|c|c|c|}
\hline 
\textbf{Student ID} & \textbf{First Name} & \textbf{Last Name}\tabularnewline
\hline 
\hline 
1279 & Joe & Smith\tabularnewline
\hline 
1395 & Muhammad & Hilmi\tabularnewline
\hline 
2883 & Sumiko & Chong\tabularnewline
\hline 
\end{tabular}

Course

\begin{tabular}{|c|c|c|c|c|}
\hline 
\textbf{Course ID} & \textbf{Subject} & \textbf{Teacher ID} & \textbf{Teacher Name} & \textbf{Room Number}\tabularnewline
\hline 
\hline 
934 & Geography & 334 & Mansoor & 12\tabularnewline
\hline 
926 & Maths & 451 & Yang & 16\tabularnewline
\hline 
882 & Physics & 628 & Lee & 12\tabularnewline
\hline 
425 & Computing & 329 & James & 14\tabularnewline
\hline 
927 & Maths & 723 & Morris & 8\tabularnewline
\hline 
883 & Physics & 534 & Weston & 10\tabularnewline
\hline 
586 & French & 271 & Dubois & 16\tabularnewline
\hline 
\end{tabular}

IsTaking

\begin{tabular}{|c|c|}
\hline 
\textbf{Student ID} & \textbf{Course ID}\tabularnewline
\hline 
\hline 
1279 & 934\tabularnewline
\hline 
1279 & 926\tabularnewline
\hline 
1279 & 882\tabularnewline
\hline 
1279 & 425\tabularnewline
\hline 
1395 & 934\tabularnewline
\hline 
1395 & 927\tabularnewline
\hline 
1395 & 883\tabularnewline
\hline 
1395 & 586\tabularnewline
\hline 
2883 & 425\tabularnewline
\hline 
2883 & 882\tabularnewline
\hline 
2883 & 934\tabularnewline
\hline 
2883 & 586\tabularnewline
\hline 
\end{tabular}

\newpage
\begin{enumerate}
\item[\textbf{(b)}] Give suitable primary keys for each of the following three tables. 
\begin{enumerate}
\item Student \hfill{}{[}1{]}
\item Course \hfill{} {[}1{]}
\item lsTaking \hfill{}{[}1{]}
\end{enumerate}
\item[\textbf{(c)}] Create an entity-relationship (ER) diagram showing the degree of
all relations. \hfill{} {[}3{]}
\item[\textbf{(d)}] Explain why table Course is not in third normal form (3NF). \hfill{}
{[}2{]}
\item[\textbf{(e)}] A table description can be expressed as: 

\texttt{TableName (Attributel, Attribute2, Attribute3 , ...) }

The primary key is indicated by underlining one or more attributes.
Foreign keys are indicated by using a dashed underline. Write table
descriptions for the required tables in the database so they are in
third nennai form (3NF). \hfill{}{[}7{]}
\item[\textbf{(e)}] Explain the reasons for reducing data redundancy in a relational
database. \hfill{} {[}2{]}
\item[\textbf{(f)}] Write an SQL query to output the subjects, teacher names and room
numbers for the courses taken by the student with Student ID of 1395. 

The output is to be in alphabetical order of subject. \hfill{}{[}5{]}
\end{enumerate}
{[}SPLIT\_HERE{]}
\begin{onehalfspace}
\item \textbf{{[}ALVL/9569/2020/P2/Q1{]} }
\end{onehalfspace}

\begin{onehalfspace}
\noindent Name your Jupyter Notebook as 

\noindent \texttt{TASK1\_<your name>\_<centre number>\_<index number>.ipynb}

\noindent The task is to implement a hashing function using the modulus
function and ASCII codes.

\noindent The hash is implemented with the following pseudocode, acting
on a string \texttt{string\_value}, which returns an integer, \texttt{h},
representing the hash.

\noindent \texttt{h $\leftarrow$ 0}

\noindent \texttt{FOR i $\leftarrow$ 0 TO length(string\_value)-1}

\noindent \texttt{\quad{}~~val $\leftarrow$ 33 {*} (ASCII value
of string\_value{[}i{]})}

\noindent \texttt{\quad{}~~h $\leftarrow$ (h+val) \% 1024}

\noindent \texttt{NEXT i}

\noindent \texttt{RETURN h}

\noindent For each of the sub-tasks, add a comment statement at the
beginning of the code, using the hash symbol '\#' to indicate the
sub-task the program code belongs to, for example:

\noindent %
\begin{tabular}{l|>{\raggedright}p{0.77\textwidth}|}
\cline{2-2} 
\texttt{In{[}1{]}:} & \texttt{\emph{\# Task 1.1}}\tabularnewline
 & \texttt{\emph{Program code}}\tabularnewline
\cline{2-2} 
\multicolumn{1}{l}{} & \multicolumn{1}{>{\raggedright}p{0.77\textwidth}}{\texttt{Output :}}\tabularnewline
\end{tabular}
\end{onehalfspace}
\begin{onehalfspace}

\subsubsection*{Task 1.1}
\end{onehalfspace}

\begin{onehalfspace}
\noindent Write a function \texttt{task1\_1(string\_value)} that:
\end{onehalfspace}
\begin{itemize}
\begin{onehalfspace}
\item takes a string value \texttt{string\_value}
\item implements the hash algorithm to produce an integer value
\item returns that integer value. \hfill{}{[}5{]}
\end{onehalfspace}
\end{itemize}
\begin{onehalfspace}
\noindent Test your function using the following \textbf{three} calls:
\end{onehalfspace}
\begin{itemize}
\begin{onehalfspace}
\item \texttt{task1\_1('Hello')}
\item \texttt{task1\_1('Hallo')}
\item \texttt{task1\_1('Hullo')}\hfill{}{[}5{]}
\end{onehalfspace}
\end{itemize}
\begin{onehalfspace}

\subsubsection*{Task 1.2}
\end{onehalfspace}

\begin{onehalfspace}
\noindent Strings are often combined with an original value (known
as the seed) before their hash is calculated. This makes it harder
to use reverse engineering to retrieve the original string.

\noindent Write a function task1\_2(seed, string\_value) that:
\end{onehalfspace}
\begin{itemize}
\begin{onehalfspace}
\item takes two string values seed and string\_value
\item concatenates these two string values together
\item uses the function in Task 1.1 to return the hash generated for the
concatenated value. \hfill{}{[}2{]}
\end{onehalfspace}
\end{itemize}
\begin{onehalfspace}
\noindent Test your function with the following three calls:
\end{onehalfspace}
\begin{itemize}
\begin{onehalfspace}
\item \texttt{task1\_2('seed-one','Hello')}
\item \texttt{task1\_2('seed-two','Hello')}
\item \texttt{task1\_2('seed-three','Hello')}\hfill{}{[}3{]}
\end{onehalfspace}
\end{itemize}
\begin{onehalfspace}
\noindent Save your Jupyter Notebook for Task 1. 
\end{onehalfspace}

{[}SPLIT\_HERE{]}
\begin{onehalfspace}
\item \textbf{{[}ALVL/9569/2020/P2/Q2{]} }
\end{onehalfspace}

\begin{onehalfspace}
\noindent Name your Jupyter Notebook as 

\noindent \texttt{TASK2\_<your name>\_<centre number>\_<index number>.ipynb}

\noindent The task is to:
\end{onehalfspace}
\begin{itemize}
\begin{onehalfspace}
\item generate a list of random integers
\item write this list to a file
\item read the list from the file
\item sort the list using a merge sort
\item write the sorted list to a second file.
\end{onehalfspace}
\end{itemize}
\begin{onehalfspace}
\noindent For each of the sub-tasks, add a comment statement, at the
beginning of the code using the hash symbol '\#' to indicate the sub-task
the program code belongs to, for example:

\noindent %
\begin{tabular}{l|>{\raggedright}p{0.77\textwidth}|}
\cline{2-2} 
\texttt{In{[}1{]}:} & \texttt{\emph{\# Task 2.1}}\tabularnewline
 & \texttt{\emph{Program code}}\tabularnewline
\cline{2-2} 
\multicolumn{1}{l}{} & \multicolumn{1}{>{\raggedright}p{0.77\textwidth}}{\texttt{Output :}}\tabularnewline
\end{tabular}
\end{onehalfspace}
\begin{onehalfspace}

\subsubsection*{Task 2.1}
\end{onehalfspace}

\begin{onehalfspace}
\noindent Write a function \texttt{task2\_1 (filename, quantity, maximum)}
that:
\end{onehalfspace}
\begin{itemize}
\begin{onehalfspace}
\item accepts three parameters:
\end{onehalfspace}
\begin{itemize}
\begin{onehalfspace}
\item \texttt{filename}, a string representing the name of a file
\item \texttt{quantity,} an integer representing the number of random integers
ot generate
\item \texttt{maximum}, representing the largest value that a random integer
can take
\end{onehalfspace}
\end{itemize}
\begin{onehalfspace}
\item generates \texttt{quantity} random numbers between 0 and \texttt{maximum}
(inclusive)
\item writes those values, on per line, to a file named \texttt{filename}.\hfill{}{[}4{]}
\end{onehalfspace}
\end{itemize}
\begin{onehalfspace}
\noindent Generate 1000 random numbers between 0 and 5000 (inclusive)
and save them to a file called 

\noindent \texttt{randomnumbers\_<your name>\_<centre number>\_<index
number>.txt}\hfill{}{[}2{]}
\end{onehalfspace}
\begin{onehalfspace}

\subsubsection*{Task 2.2}
\end{onehalfspace}

\begin{onehalfspace}
\noindent Write a function \texttt{task2\_2(list\_of\_integers)} that:
\end{onehalfspace}
\begin{itemize}
\begin{onehalfspace}
\item takes a list of integers, \texttt{list\_of\_integers}
\item sorts them into ascending order using merge sort
\item returns to the sorted list\hfill{}{[}7{]}
\end{onehalfspace}
\end{itemize}
\begin{onehalfspace}
\noindent Use the list\texttt{ {[}56,25,4,98,0,18,4,5,7,0{]}} to test
your function.

\noindent For example, the condition

\noindent \texttt{task2\_2({[}56,25,4,98,0,18,4,5,7,0{]})=={[}0,0,4,4,5,7,18,25,56,98{]}}

\noindent should return \texttt{True}.\hfill{}{[}2{]}
\end{onehalfspace}
\begin{onehalfspace}

\subsubsection*{Task 2.3}
\end{onehalfspace}

\begin{onehalfspace}
\noindent Write a function \texttt{task2\_3(filename\_in, filename\_out)}
that:
\end{onehalfspace}
\begin{itemize}
\begin{onehalfspace}
\item accepts two parameters:
\end{onehalfspace}
\begin{itemize}
\begin{onehalfspace}
\item \texttt{filename\_in} represents the input file name
\item \texttt{filename\_out} represents the output file name
\end{onehalfspace}
\end{itemize}
\begin{onehalfspace}
\item reads the integers from the input file
\item uses your \texttt{task2\_2} function to sort the integers
\item writes the integers to the output file.\hfill{}{[}5{]}
\end{onehalfspace}
\end{itemize}
\begin{onehalfspace}
\noindent The function should read the random numbers from

\noindent \texttt{randomnumbers\_<your name>\_<centre number>\_<index
number>.txt}

\noindent and then write the sorted integers to

\noindent \texttt{sortednumbers\_<your name>\_<centre number>\_<index
number>.txt}\hfill{}{[}3{]}

\noindent Save your Jupyter Notebook for Task 2.
\end{onehalfspace}

{[}SPLIT\_HERE{]}
\begin{onehalfspace}
\item \textbf{{[}ALVL/9569/2020/P2/Q3{]} }
\end{onehalfspace}

\begin{onehalfspace}
\noindent Name your Jupyter Notebook as

\noindent \texttt{TASK3\_<your name>\_<centre number>\_<index number>.ipynb}

\noindent The task is to write a function that takes a sequence of
characters that represents a quantity of data and unit, and translates
this quantity to a different unit.

\noindent The basic unit of data is the byte (B):
\end{onehalfspace}
\begin{itemize}
\begin{onehalfspace}
\item A kilobyte (KB) is $10^{3}$ bytes
\item A megabyte (MB) is $10^{6}$ bytes
\item A gigabyte (GB) is $10^{9}$ bytes
\item A terabyte (TB) is $10^{12}$ bytes
\end{onehalfspace}
\end{itemize}
\begin{onehalfspace}
\noindent For example, 8KB has 8000 bytes.

\noindent For each of the sub-tasks, add a comment statement at the
beginning of the code using the hash symbol '\#' to indicate the sub-task
the program code belongs to, for example:

\noindent %
\begin{tabular}{l|>{\raggedright}p{0.77\textwidth}|}
\cline{2-2} 
\texttt{In{[}1{]}:} & \texttt{\emph{\# Task 3.1}}\tabularnewline
 & \texttt{\emph{Program code}}\tabularnewline
\cline{2-2} 
\multicolumn{1}{l}{} & \multicolumn{1}{>{\raggedright}p{0.77\textwidth}}{\texttt{Output :}}\tabularnewline
\end{tabular}
\end{onehalfspace}
\begin{onehalfspace}

\subsubsection*{Task 3.1}
\end{onehalfspace}

\begin{onehalfspace}
\noindent Write a function called \texttt{task3\_1 (quantity\_of\_data)}
that:
\end{onehalfspace}
\begin{itemize}
\begin{onehalfspace}
\item takes a string, \texttt{quantity\_of\_data}
\item tests that the given string is a sequence of digits followed by one
of the four approved units shown above (KB, MB, GB, TB).
\item returns and displayes either:
\end{onehalfspace}
\begin{itemize}
\begin{onehalfspace}
\item the actual number of bytes represented by the input string\\
or
\item the error message, \texttt{``invalid data''}.\hfill{}{[}5{]}
\end{onehalfspace}
\end{itemize}
\end{itemize}
\begin{onehalfspace}
\noindent Test the function fully with suitable test data, including
all four approved units.

\noindent For example,

\noindent \texttt{task3\_1(``8KB'')}

\noindent should return and display \texttt{8000}.\hfill{}{[}3{]}
\end{onehalfspace}
\begin{onehalfspace}

\subsubsection*{Task 3.2}
\end{onehalfspace}

\begin{onehalfspace}
\noindent Companion units are also defined in terms of powers of 2.
These have similar abbreviations, as shown:
\end{onehalfspace}
\begin{itemize}
\begin{onehalfspace}
\item A kibibyte (KiB) is $2^{10}$ bytes
\item A mebibyte (MiB) is $2^{20}$ bytes
\item A gibibyte (GiB) is $2^{30}$ bytes
\item A tebibyte (TiB) is $2^{40}$ bytes
\end{onehalfspace}
\end{itemize}
\begin{onehalfspace}
\noindent Write a second function \texttt{task3\_2(quantity\_of\_data)}
that:
\end{onehalfspace}
\begin{itemize}
\begin{onehalfspace}
\item takes a string, \texttt{quantity\_of\_data}
\item tests that the given string is a sequence of digits followed by one
of the eight approved units (KB, KiB, MB, MiB, GB, GiB, TB, TiB)
\item returns and displays either:
\end{onehalfspace}
\begin{itemize}
\begin{onehalfspace}
\item the number of bytes represented by the input string\\
or
\item the error message, ``\texttt{invalid data}''\hfill{}{[}5{]}
\end{onehalfspace}
\end{itemize}
\end{itemize}
\begin{onehalfspace}
\noindent Test the function fully with suitable test data, including
all eight approved units.

\noindent For example,

\noindent \texttt{task3\_2('2MiB')}

\noindent should return and display \texttt{20197152}.\hfill{}{[}3{]}
\end{onehalfspace}
\begin{onehalfspace}

\subsubsection*{Task 3.2}
\end{onehalfspace}

\begin{onehalfspace}
\noindent Write a third function, \texttt{task3\_3(quantity\_of\_data,
target\_unit)} that:
\end{onehalfspace}
\begin{itemize}
\begin{onehalfspace}
\item takes two strings, \texttt{quantity\_of\_data} and \texttt{target\_unit}
\item tests that \texttt{target\_unit} is one of the eight approved units
from task 3.2
\item uses your function \texttt{task3\_2} to generate the actual number
of bytes represented by \texttt{quantity\_of\_data}
\item converts the generated number of bytes in \texttt{target\_unit}
\item returns and displays either:
\end{onehalfspace}
\begin{itemize}
\begin{onehalfspace}
\item the \texttt{quantity\_of\_data} in terms of the \texttt{target\_unit}
\item the error message, ``invalid data''\hfill{}{[}4{]}
\end{onehalfspace}
\end{itemize}
\end{itemize}
\begin{onehalfspace}
\noindent Test the function with three suitable set of values.

\noindent For example,

\noindent \texttt{task3\_3(``512MiB'',''GiB'')}

\noindent should return and display 0.5\hfill{}{[}3{]}
\end{onehalfspace}

{[}SPLIT\_HERE{]}
\begin{onehalfspace}
\item \textbf{{[}ALVL/9569/2020/P2/Q4{]} }
\end{onehalfspace}

\begin{onehalfspace}
\noindent A school has usd a text file to store data collected about
people who work at the school and students who attend the school.
People who have a teaching role at the school are referred to as 'staff'.
The school decides to transfer this information into a database.

\noindent A web page will then be used to summarise the data. Different
information will be visible on the web page, depending on the type
of person displayed.
\end{onehalfspace}
\begin{onehalfspace}

\subsubsection*{Task 4.1}
\end{onehalfspace}

\begin{onehalfspace}
\noindent Create an SQL file called \texttt{TASK4\_1\_<your name>\_<center
number>\_<index number>.sql} to show the SQL code to create database
\texttt{school.db} with the single table, \texttt{People}.

\noindent The table will have the following fields of the given SQLite
types:
\end{onehalfspace}
\begin{itemize}
\begin{onehalfspace}
\item \texttt{PersonID} - primary key, an auto-incremented integer
\item \texttt{FullName} - the full name of the person, text
\item \texttt{DateOfBirth} - the person's date of birth, text
\item \texttt{ScreenName} - the person's screen name, text
\item \texttt{IsAdult} - a Boolean using 0 for False adn 1 for True, integer.
\end{onehalfspace}
\end{itemize}
\begin{onehalfspace}
\noindent Save your SQL code as

\noindent \texttt{TASK4\_1\_<your name>\_<center number>\_<index number>.sql}\hfill{}{[}4{]}
\end{onehalfspace}
\begin{onehalfspace}

\subsubsection*{Task 4.2}
\end{onehalfspace}

\begin{onehalfspace}
\noindent The school wants to use the Python programming language
and object-oriented programming to help publish the database content
on a web page.

\noindent The class \texttt{Person} will store the following data:
\end{onehalfspace}
\begin{itemize}
\begin{onehalfspace}
\item \texttt{full\_name} - stored as a string
\item \texttt{date\_of\_birth} - initiliased with a string with the format
YYYY-MM--DD
\end{onehalfspace}
\end{itemize}
\begin{onehalfspace}
\noindent The class has two methods defined on it:
\end{onehalfspace}
\begin{itemize}
\begin{onehalfspace}
\item \texttt{is\_adult()} - returns a Boolean value to indicate whether
the person is an adult or not. It:
\end{onehalfspace}
\begin{itemize}
\begin{onehalfspace}
\item subtracts the year of the \texttt{date\_of\_birth} from the year of
today's date
\item returns True if the result is greater than 18, otherwise returns False.
\end{onehalfspace}
\end{itemize}
\begin{onehalfspace}
\item \texttt{screen\_name()} - returns a string which creates an identifier
to be used as a screen name, which should be construted as follows:
\end{onehalfspace}
\begin{itemize}
\begin{onehalfspace}
\item the full name with all spaces and punctuation removed
\item followed by the two-digit month of their birth
\item then the two-digit day of their birth.
\end{onehalfspace}
\end{itemize}
\begin{onehalfspace}
\noindent For example, John Tan, born on the $1^{\text{st}}$ of June
2000 (``\texttt{2000-06-01}''), would have the screen name ``\texttt{JohnTan0601}''
\end{onehalfspace}
\end{itemize}
\begin{onehalfspace}
\noindent Save your program code as

\noindent \texttt{TASK4\_2\_<your name>\_<centre number>\_<index number>.py}\hfill{}{[}7{]}

\noindent The \texttt{Staff} class inherits from \texttt{Person},
such that:
\end{onehalfspace}
\begin{itemize}
\begin{onehalfspace}
\item \texttt{screen\_name()} should be the name followed by ``\texttt{Staff}''
\item \texttt{is\_adult()} always returns True.
\end{onehalfspace}
\end{itemize}
\begin{onehalfspace}
\noindent The \texttt{Student} class inherits from \texttt{Person},
such that the \texttt{is\_adult()} method always returns False.

\noindent Add your program code to

\noindent \texttt{TASK4\_2\_<your name>\_<centre number>\_<index number>.py}\hfill{}{[}4{]}

\noindent The text file, \texttt{people.txt}, contains data items
for a number of people. Each data item is separated by a comma, with
each person's data on a new line as follows:
\end{onehalfspace}
\begin{itemize}
\begin{onehalfspace}
\item full name
\item date of birth in the form YYYY-MM-DD
\item a string indicating whether the person is ``\texttt{Staff}'', ``\texttt{Student}''
or ``\texttt{Person}''.
\end{onehalfspace}
\end{itemize}
\begin{onehalfspace}
\noindent Write program code to read in the information from the text
file, \texttt{people.txt}, creating instance of the appropriate class
for each person (either \texttt{Staff}, \texttt{Student} or \texttt{Person}).\hfill{}{[}4{]}

\noindent Write program code to insert all information from the file
into the \texttt{school.db} database.

\noindent Run the program.

\noindent Add your program code to

\noindent \texttt{TASK4\_2\_<your name>\_<centre number>\_<index number>.py}\hfill{}{[}8{]}
\end{onehalfspace}
\begin{onehalfspace}

\subsubsection*{Task 4.3}
\end{onehalfspace}

The screen names of the people in the text file, \texttt{people.txt},
are to be displayed in a web browser.

Write a Python program and the necessary files to create a web application
that enables the list of people to be displayed.

For each record the web page should include the:
\begin{itemize}
\item full name
\item screen name
\item identity as student, staff or person.
\end{itemize}
Save your program as

\begin{onehalfspace}
\noindent \texttt{TASK4\_3\_<your name>\_<centre number>\_<index number>.py}
\end{onehalfspace}

with any additional files / sub-folders as needed in a folder named

\begin{onehalfspace}
\noindent \texttt{TASK4\_3\_<your name>\_<centre number>\_<index number>}\hfill{}{[}5{]}
\end{onehalfspace}

Run the web application and save the output of the program as 

\begin{onehalfspace}
\noindent \texttt{TASK4\_2\_<your name>\_<centre number>\_<index number>.html}\hfill{}{[}3{]}
\end{onehalfspace}

{[}SPLIT\_HERE{]}
\end{enumerate}

\end{document}
