%% LyX 2.3.6.1 created this file.  For more info, see http://www.lyx.org/.
%% Do not edit unless you really know what you are doing.
\documentclass[english]{article}
\usepackage[T1]{fontenc}
\usepackage[latin9]{inputenc}
\usepackage{geometry}
\geometry{verbose,tmargin=2.5cm,bmargin=2.5cm,lmargin=2.5cm,rmargin=2.5cm}
\PassOptionsToPackage{normalem}{ulem}
\usepackage{ulem}

\makeatletter

%%%%%%%%%%%%%%%%%%%%%%%%%%%%%% LyX specific LaTeX commands.
%% Because html converters don't know tabularnewline
\providecommand{\tabularnewline}{\\}

\makeatother

\usepackage{babel}
\begin{document}
{[}SPLIT\_HERE{]}
\begin{enumerate}
\item \textbf{{[}ALVL/9597/2019/P1/Q1{]} }

A text file, \texttt{TIDES.TXT}. contains the low and high tide information
for a coastal location tor each day of a month. Each line contains
tab-delimited data that shows the date, the time. whether the tide
is high or low and the tide height in metres. 

Each line is in the format: 
\begin{center}
\texttt{YYYY-MM-DD\textbackslash tHH:mm\textbackslash tTIDE\textbackslash tHEIGHT\textbackslash n }
\par\end{center}
\begin{itemize}
\item The date is in the term YYYY-MM-DD, for example. 2019-08-03 is 3rd
August. 2019 
\item The time is in the form HH:mm, for example. 13:47 
\item TIDE is either HIGH or LOW 
\item HEIGHT is a positive number shown to one decimal place 
\item \texttt{\textbackslash t} represents the tab character
\item \texttt{\textbackslash n} represents the newline character 
\end{itemize}
The text file is stored in ascending order of date and time.

\subsubsection*{Task 1.1}

Write program code to:
\begin{itemize}
\item read the tide data from a text file
\item find the highest high tide and print this value
\item find the lowest low tide and print this value.
\end{itemize}
Use \texttt{TIDES.TXT} to test your program code.

\subsubsection*{Evidence 1}

Your program code.

Screenshot of your output. \hfill{}{[}9{]} 

\subsubsection*{Task 1.2}

The tidal range is the difference between the heights of successive
tides; from a high tide to the following low tide or from a low tide
to the following high tide.

Amend your program code to:
\begin{itemize}
\item output the largest tidal range and the date on which the second tide
occurs
\item output the smallest tidal range and the date on which the second tide
occurs.
\end{itemize}
Use \texttt{TIDES.TXT} to test your program code.

\subsubsection*{Evidence 2}

Your program code.

Screenshot of your output. \hfill{}{[}9{]}

{[}SPLIT\_HERE{]}
\item \textbf{{[}ALVL/9597/2019/P1/Q2{]} }

Characters are numerically encoded using ASCII codes.
\begin{itemize}
\item 'A' has the denary value 65; \textquoteleft B' has the denary value
66 and so on.
\item 'a' has the denary value 97; 'b' has the denary value 98 and so on.
\end{itemize}
The ROT-13 encoding function replaces a letter with the letter that
is 13 positions after it in the alphabet. Characters that are not
letters remain unchanged. .

The function wraps around from the end of the alphabet back to the
beginning. The case of the coded letter should match the case of the
original letter.

For example:
\begin{itemize}
\item 'A' is replaced with 'N'; 'a' is replaced with 'n'
\item 'B' is replaced with '0\textquoteleft ; 'b' is replaced with 'o' 
\item 'Z' is replaced with 'M'; 'z\textquoteleft{} is replaced with 'm'
\end{itemize}

\subsubsection*{Task 2.1}

Write program code that:
\begin{itemize}
\item reads a string of characters as input
\item encodes the string in ROT-13 form
\item outputs the encoded string.
\end{itemize}
Run the program \textbf{three} times with the inputs: 

\noindent\begin{minipage}[t]{1\columnwidth}%
\texttt{This is a word.}

\texttt{ALL \&\&\&\& CAPITALS}

\texttt{UpperCamelCase12()}%
\end{minipage}

\subsubsection*{Evidence 3}

Your program code.

Screenshots of your outputs. \hfill{}{[}9{]}

\subsubsection*{Task 2.2}

A string is encoded using ROT-13. The resulting string is then encoded
using ROT-13. The output of the second encoding should be identical
to the original string.

Amend your program code to apply HOT-13 twice, in the method described.
Show that the resulting string is identical to the original string.

\subsubsection*{Evidence 4}

Your program code.

Screenshot of the output from one of the given inputs. \hfill{}{[}3{]}

{[}SPLIT\_HERE{]}
\item \textbf{{[}ALVL/9597/2019/P1/Q3{]} }

A program is to be written to implement a to-do list using object-oriented
programming (OOP). The list shows tasks that need to be done.

Each task is given a category and a description.

The base class will be called \texttt{ToDo} and is designed as follows: 
\begin{center}
\begin{tabular}{|l|}
\hline 
\texttt{\hspace{0.25\columnwidth}ToDo}\tabularnewline
\hline 
\texttt{category : STRING }\tabularnewline
\texttt{description : STRING}\tabularnewline
\hline 
\texttt{constructor(c : STRING, d : STRING) }\tabularnewline
\texttt{set\_category(s : STRING)}\tabularnewline
\texttt{set\_description(s : STRING) }\tabularnewline
\texttt{get\_category() : STRING }\tabularnewline
\texttt{get:\_description() : STRING }\tabularnewline
\texttt{summary() : STRING}\tabularnewline
\hline 
\end{tabular}
\par\end{center}

The \texttt{summary()} method returns the category and description
as a single string.

\subsubsection*{Task 3.1}

Write program code to define the class \texttt{ToDo}.

\subsubsection*{Evidence 5}

Your program code. \hfill{}{[}5{]}

Tasks should be sorted alphabetically by category. Within each category.
tasks should be sorted alphabetically by description. 

A task to be added to the list is compared to the tasks already in
the list to determine its correct position in the list. If the list
is empty. it is added to the beginning of the list. 

This comparison will use an additional member method,
\begin{center}
\texttt{compare\_with(td : ToDo) : INTEGER }
\par\end{center}

This function compares the instance (the item in the list) and the
\texttt{ToDo} object passed to it, returning one of three values:

\texttt{\qquad{}}-1 if the instance is before the given \texttt{ToDo}

\texttt{\qquad{}}0 if the two are equal

\texttt{\qquad{}}+1 if the instance is after the given \texttt{ToDo}

\subsubsection*{Task 3.2}

There are four objects defined in the text file \texttt{TASK3\_2.TXT}.

Write program code to:
\begin{itemize}
\item implement the \texttt{compare\_with()} method
\item create an empty list of \texttt{ToDo} objects
\item add each of the four objects in the text file \texttt{TASK3\_2.TXT}
to its appropriate place in the list
\item printout the list contents using the \texttt{summary()} method.
\end{itemize}

\subsubsection*{Evidence 6}

Your program code.

Screenshot of test run. \hfill{}{[}13{]}

The to-do list can have items with extra information. One such item
has a date by which the task should be completed. 

The \texttt{DatedToDo} class inherits from the \texttt{ToDo} class,
extending it to have a \texttt{due\_date}. designed as follows: 
\begin{center}
\begin{tabular}{|l|}
\hline 
\hspace{0.25\columnwidth}DatedToDo : ToDo\tabularnewline
\hline 
\texttt{due\_date : DATE}\tabularnewline
\hline 
\texttt{constructor(dt : DATE, 0 : STRING, d : STRING)}\tabularnewline
\texttt{set\_due\_date(d : DATE)}\tabularnewline
\texttt{get\_due\_date() : DATE }\tabularnewline
\hline 
\end{tabular}
\par\end{center}

The \texttt{DatedToDo} class should extend the \texttt{compare\_with()}
method to ensure that tasks are ordered by ascending \texttt{due\_date},
and then by the ordering used by the base \texttt{compare\_with()}
method. The \texttt{summary()} method should also be extended to return
the \texttt{due\_date} and the return values of the base \texttt{summary()}
method.

\subsubsection*{Task 3.3}

There are seven objects defined in the text file \texttt{TASK3\_3.TXT}.

Amend your program code to:
\begin{itemize}
\item implement the \texttt{DatedToDo} class, with \texttt{constructor},
\texttt{get\_due\_date} and \texttt{set\_due\_date}
\item implement the extended \texttt{compare\_with()} method
\item implement the extended \texttt{summary()} method
\item ensure all seven objects in the text file \texttt{TASK3\_3.TXT} are
added to the list
\item print out the list contents using the \texttt{summary()} method.
\end{itemize}

\subsubsection*{Evidence 7}

Your program code. 

Screenshot of the test run.\hfill{}{[}5{]}

\noindent When a task in the to-do list has been completed, it should
be removed.

\subsubsection*{Task 3.4}

There are four completed tasks defined in the text file \texttt{TASK3\_4.TXT}.

If any of the four tasks exists in the list, it should be removed.

Amend your program to: 
\begin{itemize}
\item recreate the list of seven tasks from Task 3.3
\item check if each of the four completed tasks in the text file \texttt{TASK3\_4.TXT}
exists in the list and: 
\begin{itemize}
\item remove it from the list if it does or 
\item print a warning message it the completed task does not exist
\end{itemize}
\item print out the list after all four objects have been processed. 
\end{itemize}

\subsubsection*{Evidence 8}

Your amended code.

Screenshot of the test run.\hfill{}{[}10{]}

{[}SPLIT\_HERE{]}
\item \textbf{{[}ALVL/9597/2019/P1/Q4{]} }

A stack is used to store characters.

\subsubsection*{Task 4.1}

Write program code to implement the stack and the operations specified.

Your code should allow operations to: 
\begin{itemize}
\item push an item on to the stack
\item pop an item off the stack
\item determine the size of the stack. A size of zero indicates that the
stack is empty.
\end{itemize}

\subsubsection*{Evidence 9}

Your program code for the stack.\hfill{}{[}10{]}

The stack is to be used to identify it an arithmetic expression is
balanced. 

An expression is balanced if each opening bracket has a corresponding
closing bracket. 

Different pairs of brackets can be used. These are: {[}{]}, () or
\{\}. 

This is an example of an expression that is balanced.

\texttt{\qquad{}({[}8-1{]}/(5{*}7)) }

This is an example of an expression that is not balanced. 

\texttt{\qquad{}{[}(8-1{]}/(5{*}7)) }

Note the change in the order of the first two open bracket symbols.
The first closing bracket should be a closing bracket ')' to match
the previous opening bracket \textquoteleft ('. 

Note that an expression is not balanced if the order of the brackets
is incorrect, even if there are the same number of opening and closing
brackets of each bracket type. 

An expression is checked by iterating over it: 
\begin{itemize}
\item if a non-bracket symbol is found, continue to the next character. 
\item If an opening symbol is found, push it on to the stack and continue
to the next character. 
\item If a closing bracket is encountered: 
\begin{itemize}
\item If the stack is empty, return an error (because there is no corresponding
opening bracket) 
\item else pop the symbol from the top of the stack and compare it to the
current closing symbol to see if they make a matching pair 
\item If they do match continue to the next character 
\item else return an error (pairs of brackets must match). 
\end{itemize}
\item When the last symbol is encountered: 
\begin{itemize}
\item return an error if the stack is not empty (too many opening symbols) 
\item else return a success message.
\end{itemize}
\end{itemize}

\subsubsection*{Task 4.2}

Add \textbf{five} other suitable test cases and a reason for choosing
each test case.
\begin{center}
\begin{tabular}{|l|l|l|}
\hline 
\hspace{0.05\columnwidth}\textbf{Test case} & \textbf{Reason for choice} & \textbf{Expected value}\tabularnewline
\hline 
\texttt{({[}8-1{]}/(5{*}7))} & Provided & Succeeds\tabularnewline
\hline 
\texttt{{[}(8-1{]}/(5{*}7))} & Provided & Fails\tabularnewline
\hline 
 &  & Succeeds\tabularnewline
\hline 
 &  & Succeeds\tabularnewline
\hline 
 &  & Fails\tabularnewline
\hline 
 &  & Fails\tabularnewline
\hline 
 &  & Fails\tabularnewline
\hline 
\end{tabular}
\par\end{center}

\subsubsection*{Evidence 10}

The completed table with all seven test cases and a reason for choosing
each test case.\hfill{}{[}6{]}

\subsubsection*{Task 4.3}

Write program code that checks expressions using the given algorithm. 

Use all \textbf{seven} test cases to verify It.

\subsubsection*{Evidence 11}

Your program code for the stack.

Screenshots for each test data run. \hfill{}{[}19{]}

{[}SPLIT\_HERE{]}
\item \textbf{{[}ALVL/9597/2019/P2/Q1{]} }

Pharmacists working in a group of pharmacies, dispense medicine to
patients who present to them a prescription written by a doctor. A
new system is to be built to allow a doctor to send prescription data
electronically to a pharmacy of the patient's choice. Patients will
either collect the medicine, or have the pharmacy deliver it to them. 

A project proposal is written and sent to doctors and pharmacy staff,
inviting each to respond within a given time. 
\begin{enumerate}
\item Give a reason why the project proposal is sent to: 
\begin{enumerate}
\item Doctors \hfill{}{[}1{]}
\item Pharmacy staff \hfill{}{[}1{]}
\end{enumerate}
\end{enumerate}
The responses from the doctors and pharmacy staff are reviewed. Invitations
are sent to doctors to find out whether they are willing to take part
in a pilot scheme. The project proposal is sent to prospective software
developers. Some of the activities involved in the project are listed
in the following table. 
\begin{center}
\begin{tabular}{|c|l|c|}
\hline 
Label & \hspace{0.25\columnwidth}Activity & Duration(Weeks)\tabularnewline
\hline 
\textbf{A} & Send project proposal  & 4\tabularnewline
\hline 
\textbf{B} & Send project proposal to pharmacy staff & 2\tabularnewline
\hline 
\textbf{C} & Discuss all the responses from A and B. and revise the proposal if
2 required & 2\tabularnewline
\hline 
\textbf{D} & Send project proposal to prospective software developers & 3\tabularnewline
\hline 
\textbf{E} & Invite doctors to be part of a pilot scheme & 2\tabularnewline
\hline 
\textbf{F} & Request quotations of cost and development time from software developers & 3\tabularnewline
\hline 
\textbf{G} & Select a software developer  & 1\tabularnewline
\hline 
\end{tabular}
\par\end{center}
\begin{enumerate}
\item[(b)] {}
\begin{enumerate}
\item Draw a Gantt chart for the activities labelled \textbf{A} to \textbf{G}.
\hfill{}{[}6{]}
\item State the estimated time taken to complete activities \textbf{A} to
\textbf{G}. \hfill{}{[}1{]}
\end{enumerate}
\end{enumerate}
In activities \textbf{A} and \textbf{B}, doctors and pharmacy staff
identified ethical and security issues that would need to be addressed. 
\begin{enumerate}
\item[(c)] {}
\begin{enumerate}
\item Describe one security issue. \hfill{}{[}2{]}
\item Describe one ethical issue. \hfill{}{[}2{]}
\end{enumerate}
\end{enumerate}
In activity \textbf{F}, quotations are received from software developers.
The lowest cost is from a developer who works alone, but demonstrates
a number of successful projects. Other software developers that employ
many staff submit more expensive quotations. 
\begin{enumerate}
\item[(d)] Explain why the group of pharmacies may decide against the single
developer. \hfill{}{[}2{]}
\end{enumerate}
An analyst from the chosen software developer reviews the current
system. 
\begin{enumerate}
\item[(e)] Give \textbf{four} methods available to the analyst to find out how
a system operates. \hfill{}{[}4{]}
\end{enumerate}
The analyst proposes that the doctors and pharmacy staff interact
with a web-based system. 
\begin{enumerate}
\item[(f)] tate the software that will be needed on the devices used by the
doctors and pharmacy staff. other than the operating system. \hfill{}{[}1{]}
\end{enumerate}
The alternative to a web-based system would be to write and install
purpose-built application software for each computer used by a doctor
or member of the pharmacy staff. 
\begin{enumerate}
\item[(g)] Describe \textbf{two} advantages to the software developer of a web-based
solution over purpose- built software running on each user's computer.
\hfill{}{[}4{]}
\end{enumerate}
Doctors may wish to write prescriptions when they visit patients in
their own home. 
\begin{enumerate}
\item[(h)] Explain \textbf{one} benefit of a web-based solution in this situation.
\hfill{}{[}2{]}
\end{enumerate}
The computers used by the doctors and pharmacy staff are clients of
the server operated by the pharmacy. Some validation is provided by
client-side scripting.
\begin{enumerate}
\item[(i)] Give \textbf{two} advantages of using this type of scripting. \hfill{}{[}2{]}
\end{enumerate}
The new system is designed. coded and tested as a number of modules.
A tester performing black-box testing on a module would need its specification. 
\begin{enumerate}
\item[(j)] Explain why the tester would not need access to the source code.
\hfill{}{[}2{]}
\item[(k)] Explain why someone designing a test strategy for white box testing
would need access to the source code. \hfill{}{[}2{]}
\item[(l)] Alpha testing is performed on the system. 

Explain the purpose of alpha testing. \hfill{} {[}2{]}
\item[(m)] The group of pharmacies is responsible for the security and integrity
of the stored data. 
\begin{enumerate}
\item Give \textbf{two }methods that could be used to ensure security of
the stored data. \hfill{}{[}2{]}
\item Give \textbf{two} methods that could be used to ensure the integrity
of the stored data. \hfill{}{[}2{]}
\end{enumerate}
\item[(n)] The group considers using either the cloud or its own server to store
data needed by the proposed system. 

Give \textbf{one} advantage and \textbf{one} disadvantage of storing
the data in the cloud. \hfill{}{[}2{]}
\end{enumerate}
{[}SPLIT\_HERE{]}
\item \textbf{{[}ALVL/9597/2019/P2/Q2{]} }

A bakery bakes bread and cakes to sell in its own shop and to other
shops throughout a city. 

Its drivers visit every shop each day, delivering that day\textquoteright s
order and collecting the order for the next day. 

Order forms are pre-printed with the name of each shop and every item
that the bakery bakes. The manager of each shop writes onto the form
the quantity of each item required. When the drivers return to the
bakery. the data from the order forms are collated to give the bakers
the total of each item to bake. 

Copies of the order forms are made and used as delivery notes for
the next day\textquoteright s deliveries. The accounts department
use the original order forms to prepare a weekly invoice for each
shop.

The bakery wants the shops to submit their orders online. 

A program is needed to determine the number of each item needed and
produce the weekly invoice for each shop.

The new program will use a relational database with three tables:
Product, Shop and Order. 

Each product has a description. price. and a unique product ID number. 

Each shop has a name. an address, telephone number. manager's name,
and a unique shop iD number. 

Each order has a product lD, a quantity, a shop ID and a date for
delivery. 
\begin{enumerate}
\item Draw an Entity-Relationship (E-R) diagram showing the three tables
and the relationships between them. \hfill{}{[}5{]}
\item A table description can be expressed as: 

\texttt{TableName (}\texttt{\uline{Attribute1}}\texttt{, Attribute2,
Attribute3, ...) }

The primary key is indicated by underlining one or more attributes. 

Write table descriptions for the three tables. \hfill{}{[}4{]}
\end{enumerate}
The bakery can change the price of an item at any time. Validation
ensures that the new price is within specified limits and is more
likely to be correct.
\begin{enumerate}
\item[(c)] {}
\begin{enumerate}
\item Explain why this could still result in incorrect weekly invoices,
assuming that the new price input is correct. \hfill{}{[}2{]}
\item Describe changes to the database and draw a modified E-R diagram to
ensure correct invoices are created. \hfill{}{[}4{]}
\end{enumerate}
\end{enumerate}
{[}SPLIT\_HERE{]}
\item \textbf{{[}ALVL/9597/2019/P2/Q3{]} }

A programmer is asked to write a program to store names in alphabetical
order.

The program needs to:
\begin{itemize}
\item add and remove names
\item search for the presence of a specific name
\item output all the names in alphabetical order. 
\end{itemize}
The programmer considers two options: an array and a linked list.
\begin{enumerate}
\item {}
\begin{enumerate}
\item Explain why an array allows for more efficient searching. \hfill{}{[}2{]}
\item State why this advantage becomes more significant as the number of
names becomes much larger. \hfill{}{[}1{]}
\end{enumerate}
\item {}
\begin{enumerate}
\item Give \textbf{one} disadvantage of using an \textbf{array} to store
the names in alphabetical order. \hfill{}{[}2{]}
\item Give \textbf{one} advantage of using a \textbf{linked list} to store
the names in alphabetical order. \hfill{}{[}2{]}
\end{enumerate}
\end{enumerate}
A third option is to store the names in a binary tree.
\begin{enumerate}
\item[(c)] Explain how a binary tree provides some of the advantages of both
an array and a linked list when storing sorted data. \hfill{}{[}2{]}
\item[(d)] State why a binary tree may need to be re-created with exactly the
same data items. \hfill{}{[}1{]}
\end{enumerate}
{[}SPLIT\_HERE{]}
\item \textbf{{[}ALVL/9597/2019/P2/Q4{]} }

A company operates a multi-storey car park. All parking bays are identified
by a letter. indicating the floor. and a number indicating the position
of the bay on that floor (for example. C34 indicates bay 34 on floor
C). 

The entrance to the car park is controlled by a barrier. Before the
barrier lifts to allow a car to enter, the driver must press a button
to indicate if they need a standard bay or a special bay. 

Special bay users must present a card to a card reader at the barrier. 

The car park has an additional third type of bay that has a charging
point for electric vehicles. The hourly rate for these bays is not
the same as standard bays. The cost of using this type of bay additionally
depends on the cost of the electricity used. This is monitored by
the charging device and stored.

A camera captures the vehicle registration number. A ticket is printed
showing:
\begin{itemize}
\item current time
\item vehicle registration number 
\item floor letter 
\item position number of a suitable bay where the car must be parked 
\item the card number for the special bay, if a card had been presented
at the barrier. 
\end{itemize}
When the driver takes the ticket from the printer. the entrance barrier
lifts. Before a car is allowed to leave, the ticket must be presented
and a charge paid. The charge is determined by the length of stay
and type of bay. The hourly rate for a standard bay is not the same
as that for a special bay. As a car approaches the exit barrier a
camera captures the vehicle registration. The barrier only lifts if
the charge for this vehicle has been paid. 

This system is to be implemented using object-oriented programming
(OOP). 

The base class PARKING\_BAY has a property to store whether or not
a bay is occupied. 
\begin{enumerate}
\item Draw a class diagram, showing:
\begin{itemize}
\item any derived classes and inheritance from the base class
\item the properties needed in the base, and any derived classes
\item suitable methods to support the system with at least one getter and
one setter method. \hfill{}{[}8{]}
\end{itemize}
\item Add a class, CAR\_PARK. thathas properties to store:
\begin{itemize}
\item a list of all bays
\item the number of unoccupied bays. \hfill{}{[}3{]}
\end{itemize}
\item Explain why polymorphism is useful in object-oriented programming.
\hfill{}{[}2{]}
\item Explain the purpose of making the attributes of an object private.
\hfill{}{[}2{]}
\end{enumerate}
{[}SPLIT\_HERE{]}
\item \textbf{{[}ALVL/9597/2019/P2/Q5{]} }

The function \texttt{z} takes three integer parameters, \texttt{low},
\texttt{high}, \texttt{seek} and returns an integer value. It operates
on the values in the elements of the array \texttt{A}. 

\noindent %
\noindent\begin{minipage}[t]{1\columnwidth}%
\texttt{01 FUNCTION Z(low, high, seek, A) RETURNS INTEGER }

\texttt{02 \qquad{}IF low > high THEN }

\texttt{03 \qquad{}\qquad{}RETURN \textemdash 1 }

\texttt{04 \qquad{}ENDIF }

\texttt{05 \qquad{}mid <- low + INT( (high \textemdash{} low) /2)}

\texttt{06 \qquad{}IF seek = A{[}mid{]} THEN}

\texttt{07 \qquad{}\qquad{}RETURN mid }

\texttt{08 \qquad{}ELSE }

\texttt{09 \qquad{}\qquad{}IF seek < A{[}mid{]} THEN }

\texttt{10 \qquad{}\qquad{}\qquad{}RETURN Z(low, mid - 1, seek,
A) }

\texttt{11 \qquad{}\qquad{}ELSE }

\texttt{12 \qquad{}\qquad{}\qquad{}RETURN Z(mid + 1, high, seek,
A) . }

\texttt{13 \qquad{}\qquad{}ENDIF }

\texttt{14 \qquad{}ENDIF }

\texttt{15 ENDFUNCTION}%
\end{minipage}
\begin{enumerate}
\item {}
\begin{enumerate}
\item State what lines \texttt{10} and \texttt{12} tell you about the function.
\hfill{}{[}1{]}
\item State the purpose for the \texttt{RETURN} statements in lines \texttt{03}
and \texttt{07} of function \texttt{z}. \hfill{} {[}1{]}
\end{enumerate}
\end{enumerate}
The values in each of the eight elements of the array \texttt{A} are:
\begin{center}
\begin{tabular}{|l|c|c|c|c|c|c|c|c|}
\multicolumn{1}{l}{\textbf{Element}} & \multicolumn{1}{c}{0} & \multicolumn{1}{c}{1} & \multicolumn{1}{c}{2} & \multicolumn{1}{c}{3} & \multicolumn{1}{c}{4} & \multicolumn{1}{c}{5} & \multicolumn{1}{c}{6} & \multicolumn{1}{c}{7}\tabularnewline
\hline 
\textbf{Value} & -3 & 8 & 14 & 15 & 96 & 101 & 412 & 500\tabularnewline
\hline 
\end{tabular}
\par\end{center}
\begin{enumerate}
\item Copy and then complete the trace table for the instruction: 

\texttt{OUTPUT z(0, 7, 103, A)}
\begin{center}
\begin{tabular}{|c|c|c|c|c|c|c|}
\hline 
Function call & \texttt{\textbf{low}} & \texttt{\textbf{high}} & \texttt{\textbf{seek}} & \texttt{\textbf{mid}} & \texttt{\textbf{A{[}mid{]}}} & \texttt{\textbf{OUTPUT}}\tabularnewline
\hline 
\texttt{1} & \texttt{0} & \texttt{7} & \texttt{103} &  &  & \tabularnewline
\hline 
 &  &  &  &  &  & \tabularnewline
\hline 
 &  &  &  &  &  & \tabularnewline
\hline 
 &  &  &  &  &  & \tabularnewline
\hline 
 &  &  &  &  &  & \tabularnewline
\hline 
 &  &  &  &  &  & \tabularnewline
\hline 
\end{tabular}
\par\end{center}

\hfill{}{[}4{]}
\item Function \texttt{z} can return two different types of value.

Explain what these represent. \hfill{}{[}2{]}
\item The number of elements in array \texttt{A} may be very large. 

Explain why a programmer might prefer to use an iterative approach
rather than the one used in function \texttt{z}. \hfill{} {[}2{]}
\end{enumerate}
{[}SPLIT\_HERE{]}
\item \textbf{{[}ALVL/9597/2019/P2/Q6{]} }

Data communication networks can use circuit switching or packet switching. 
\begin{enumerate}
\item {}
\begin{enumerate}
\item Give \textbf{two} advantages of packet switching over circuit switching.
\hfill{}{[}2{]}
\item Give \textbf{one} advantage of circuit switching over packet switching.
\hfill{}{[}1{]}
\end{enumerate}
\item {} 
\begin{enumerate}
\item State \textbf{one} reason for using either a parity check or a checksum.
\hfill{}{[}1{]}
\item Give \textbf{one} example of an error that a parity check cannot detect.
\hfill{}{[}1{]}
\end{enumerate}
\item Switches and routers are common devices used in networking. 

Explain the most significant differences between a switch and a router.
\hfill{}{[}2{]}
\item Explain the purpose of a bridge in a network.\hfill{}{[}1{]}
\item A local area network (LAN) can be set up as either client-server or
peer-to-peer. 

Give \textbf{two} advantages in storing shared data on a client-server
network rather than on a peer-to-peer network. \hfill{}{[}2{]}
\end{enumerate}
{[}SPLIT\_HERE{]}
\end{enumerate}

\end{document}
