\item \textbf{{[}ACJC/PROMO/9758/2021/Q8{]}}

The figure below shows a cross-section $OBCE$ of a car headlight
whose reflective surface is modelled in suitable units by the curve
with parametric equations

\[
x=a(\theta-\sin\theta),y=a(1-\cos\theta)
\]
 for $0\le\theta\le2\pi$ , where $a$ is a positive constant.
\noindent \begin{center}
<INSERT IMAGE HERE>
\par\end{center}
\begin{enumerate}
\item[(i)]  Find in terms of $a$ 
\begin{enumerate}
\item[(a)]  the length of $OE$, \hfill{} {[}2{]}
\item[(b)]  the maximum height of the curve $OBCE$. \hfill{}{[}1{]}
\end{enumerate}
\item[(ii)]  Show that $\frac{\text{d}y}{\text{d}x}=\cot\frac{\theta}{2}$. \hfill{}
{[}3{]}
\end{enumerate}
Point $B$ lies on the curve and has parameter $\beta$. $TS$ is
tangential to the curve at $B$ and $BC$ is parallel to the $x$-axis.
Given that $\angle TBC=\frac{\pi}{6}$, 
\begin{enumerate}
\item[(iii)]  show that $\beta=\frac{2\pi}{3}$. \hfill{}{[}2{]}
\item[(iv)]  Show that the equation of normal to the curve at the point $B$
is 

\[
ky=-k^{2}x+2\pi a,
\]
 

where $k$ is an exact constant to be determined. \hfill{}{[}3{]}
\end{enumerate}